\begin{figure}[h]
\begin{center}
\begin{pspicture}(-5,-5)(5,5)%\psgrid
%\psset{unit=0.29cm}
\psset{unit=0.15cm}
\psset{linecolor=black,linewidth=1pt,arrowsize=8pt}

% Axes
\psaxes[Dx=5,Dy=5]{<->}(0,0)(-14.5,-14.5)(14.5,14.5)
\psset{linewidth=.4pt}
\psaxes[Dx=1,Dy=1,labels=none,ticksize=1.5pt](0,0)(-13,-13)(13,13)
\uput{10pt}[0](0,14){\psscalebox{1.25}{\it y-axis}}
\uput{10pt}[70](14,0){\psscalebox{1.25}{\it x-axis}}

% Origin
\pnode(0,0){O}
\uput{35pt}[310](0,0){\rnode{Ot}{\psscalebox{1.1}{
        \vbox{\halign{\hfil#\hfil\cr$(0,0)$\cr\it origin\cr}}}}}
\nccurve[angleA=315,angleB=145,arrowsize=4pt,
        nodesepA=2pt,nodesepB=-3pt,linecolor=gray]{<-}{O}{Ot}

% Point
\psdots[dotstyle=*,dotscale=1.2](3,5)
\psline[linestyle=dotted,linewidth=.8pt](3,0)(3,5)(0,5)
\uput{2.5pt}[53](3,5){\psscalebox{1.1}{$P(3,5)$}}


% Quadrants
\uput{3cm}[45](0,0){\psscalebox{1.6}{\bf I}}
\uput{3cm}[135](0,0){\psscalebox{1.6}{\bf II}}
\uput{3cm}[225](0,0){\psscalebox{1.6}{\bf III}}
\uput{3cm}[315](0,0){\psscalebox{1.6}{\bf IV}}
\end{pspicture}
\caption{Representación genérica del Producto Cartesiano mediante Sistema Cartesiano.}
\end{center}
\end{figure}




