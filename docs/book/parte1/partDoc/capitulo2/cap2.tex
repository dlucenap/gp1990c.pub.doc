%%%%%%%%%%%%%%
% Chapter 2: Proyecto gp1900c
%%%%%%%%%%%%%%

\label{chap:cap2}

\section{Objetivos}

El Proyecto Fin de Carrera \index{gp1990c}{\texttt{gp1990c}} gira en torno a dos conceptos:

\begin{enumerate}[i.]

\item \textbf{Desarrollar el estándar ISO Pascal 
7185:1990\endnote{\href{http://www.moorecad.com/standardpascal/}{http://www.moorecad.com/standardpascal/}}
además de la construcción de un prototipo} para su parte léxica (basada en 
\index{Flex}Flex) y su
parte sintáctica (basada en \index{Bison}Bison).

\item \textbf{Síntesis y Lenguaje Matemático propio de la Teoría de Lenguajes de
Programación} así como su evolución e influencias históricas.

\end{enumerate}

\section{Descripción general del proyecto}

El Software estará compuesto por dos elementos atómicos desde el punto de
vista funcional pero que se interconectan para constituir, como hemos dicho el
analizador.

Brevemente enumeraremos sus partes:

\begin{enumerate}[i.]
\item \toIndex{Analizador Léxico} ({\tt gp1990la}): Es el elemento encargado de verificar
que el conjunto de palabras de código fuente pertenecen al lenguaje. Genera un 
fichero \texttt{lex.yy.c}.
\item \toIndex{Analizador Sintáctico} \endnote{En inglés se denomina 
``parser''}.  ({\tt gp1990sa}): Es
el elemento encargado de
comprobar que el orden de esas palabras corresponde a la propia sintaxis 
(Reglas)
del lenguaje. Genera un fichero \texttt{y.tab.c}
\end{enumerate}

\paragraph*{Compilación:} El proceso para crear un ejecutable a partir de código Flex/Bison\endnote{Compatible con Lex/Yacc.} sería:

\begin{verbatim}
$ bison -yd gp1990sa.y
$ flex gp1990la.l
$ gcc y.tab.c lex.yy.c –lfl –o programa
\end{verbatim}

\begin{figure}[h]
\begin{center}
\begin{pspicture}(10,4)%\psgrid

\psline[linecolor=black,linewidth=1pt]{<-}(0.4,3)(3.1,3)

\pspolygon[fillstyle=solid,fillcolor=white](3.2,3.4)(6.8,3.4)(6.8,2.7)(3.2,2.7)

\rput(5,3){\texttt{gp1990sa}}

\pspolygon[fillstyle=solid,fillcolor=white](3.2,1.4)(6.8,1.4)(6.8,0.7)(3.2,0.7)

\rput(5,1){\texttt{gp1990la}}

\psline[linecolor=black,linewidth=1pt]{->}(0.4,1)(3.1,1)

% \pscurve[linecolor=black,linewidth=1pt]{->}(6.9,5)(7.4,6.8)(6.9,7)

\pscurve[linecolor=black,linewidth=1pt]{->}(6.9,1)(7.4,2.8)(6.9,3)

\rput(1.8,1.3){Input}
\rput(1.8,3.3){Output}

%\pscurve[linecolor=black,linewidth=1pt]{->}(3,1)(1.8,1.4)(2,3.3)
%\pspolygon[fillstyle=solid,fillcolor=white](0.2,4.6)(3.8,4.6)(3.8,3.4)(0.2,3.4)
%\pspolygon[fillstyle=solid,fillcolor=white](0.4,4.4)(3.6,4.4)(3.6,3.6)(0.4,3.6)


% %\psline[linecolor=black,linewidth=1pt]{->}(15,0.5)
% \pspolygon[fillstyle=solid,fillcolor=white](0.6,2)(2.4,2)(2.4,1)(0.6,1)
% \pspolygon[fillstyle=solid,fillcolor=white](0.8,1.8)(2.2,1.8)(2.2,1.2)(0.8,1.2)
% \rput(1.5,1.5){Input}
% 
% \psline[linecolor=black,linewidth=1pt]{->}(2.5,1.5)(3.9,1.5)
% 
% \pspolygon[fillstyle=solid,fillcolor=white](4,1.8)(6,1.8)(6,1.2)(4,1.2)
% \rput(5,1.5){{\tt gp1990la}}
% 
% \pspolygon[fillstyle=solid,fillcolor=white](8,1.8)(10,1.8)(10,1.2)(8,1.2)
% \rput(9,1.5){{\tt gp1990la}}
% 
% \psline[linecolor=black,linewidth=1pt]{<->}(7.9,1.5)(6.1,1.5)
% 
% \psline[linecolor=black,linewidth=1pt]{->}(10.1,1.5)(12.5,1.5)
% 
% \pspolygon[fillstyle=solid,fillcolor=white](12.6,2)(14.4,2)(14.4,1)(12.6,1)
% \pspolygon[fillstyle=solid,fillcolor=white](12.8,1.8)(14.2,1.8)(14.2,1.2)(12.8,1.2)
% 
% \rput(13.5,1.5){Output}
% 
% % \pscurve[linecolor=black,linewidth=1pt]{->}(2,4.7)(2,7.8)(3,8)
% % \pscurve[linecolor=black,linewidth=1pt]{->}(2,4.2)(2,1.2)(3,1)
% % \pscurve[linestyle=dotted, 
% % linecolor=black,linewidth=1pt]{->}(4.4,7.8)(5,4.8)(7.4,4.5)
% % 
% % \rput(6.4,4.8){{\tt \$Pipe\$}}
% % \pscurve[linecolor=black,linewidth=1pt]{-}(5,8)(6,7.2)(7.5,4.5)
% % \pscurve[linecolor=black,linewidth=1pt]{-}(5,1)(6,1.8)(7.5,4.5)
% % \psline[linecolor=black,linewidth=1pt]{->}(7.5,4.5)(8.2,4.5)
% % 
% % \rput(9,4.5){Output}
\end{pspicture}
\caption{Síntesis y partes de {\tt gp1990c}.}
\end{center}
\end{figure}

\section{Transformación de una Expresión Regular en
Software}

Dicho proceso comprende una serie de etapas:

\begin{figure}[h]
\begin{center}
\begin{pspicture}(10,8)%\psgrid


\pscurve[linecolor=black,linewidth=1pt]{->}(7,7)(8,6.6)(8,4.6)
\pspolygon[fillstyle=solid,fillcolor=white](3.2,7.4)(6.8,7.4)(6.8,6.7)(3.2,6.7)
\rput(5,7){Expresión Regular}


\pscurve[linecolor=black,linewidth=1pt]{->}(8,3.4)(8,1.4)(7,1)
\pspolygon[fillstyle=solid,fillcolor=white](6,4.4)(10,4.4)(10,3.6)(6,3.6)
\rput(8,4){Aut. Finito No Det.}


\pspolygon[fillstyle=solid,fillcolor=white](3.2,1.4)(6.8,1.4)(6.8,0.7)(3.2,0.7)
\rput(5,1){Aut. Finito Det.}


\pscurve[linecolor=black,linewidth=1pt]{->}(3,1)(1.8,1.4)(2,3.3)
\pspolygon[fillstyle=solid,fillcolor=white](0.2,4.6)(3.8,4.6)(3.8,3.4)(0.2,3.4)
\pspolygon[fillstyle=solid,fillcolor=white](0.4,4.4)(3.6,4.4)(3.6,3.6)(0.4,3.6)
\rput(2,4){Implementación}

\end{pspicture}
\caption{Transformación desde una Expresión Regular a su Implementación.}
\end{center}
\end{figure}


\defn \index{Expresión Regular}Expresión Regular: Se trata de una simplificación de una cadena de caracteres (Ver Apartado \ref{sec:expresionesRegulares}).
\defn AFND (\index{Autómata Finito no Determinista}Autómata Finito no Determinista): Reconocedor de expresiones
regulares con transiciones $\delta$ del tipo (Ver Apartado \ref{sec:nda}):

\begin{equation}\label{deltaAFND}
\delta(i,o) \rightarrow q; \diagup\ e\in Q \wedge s \in \Sigma \cup \{\lambda\} \wedge q
\subset Q
\end{equation}

\defn AFD (\index{Autómata Finitio Determinista}Autómata Finitio Determinista): Reconocedor de expresiones
regulares con transiciones $\delta$ del tipo (Ver Apartado \ref{sec:dfa}):

\begin{equation}\label{deltaAFD}
\delta(i,o) \rightarrow q_i; \diagup\ e\in Q \wedge s \in \Sigma \cup \{\lambda\} \wedge q_{i}
\subset Q
\end{equation}
\defn Minimización de los estados: Dicho proceso se basa en el siguiente teorema:

{\thm Para cualquier Autómata Finito, existe un Autómata Finito Mínimo
equivalente (Ver Apartado \ref{algolAFnD2AFD}).}

\paragraph*{Implementación:} La implementación y desarrollo de un analizador 
depende en gran medida de tipo de lenguaje base. Existen la siguiente 
clasificación para el análisis de Gramáticas Libres de 
Contexto\endnote{Gramática Chomskiana de Tipo 2: P = $\{(S \rightarrow \lambda) 
\vee (A \rightarrow p_2)\ |\ p_2 \in \Sigma^+; A
\in N\}$}

\section{Breve descripción de {\tt gp1990la}}

\defn \index{gp1990la}\textbf{{\tt gp1990la}} en un Analizador Léxico para ISO Pascal 7185:1990.

\paragraph*{Implementación:} Hay tres tipos de implementaciones para un Analizador Léxico:

\begin{enumerate}[i.]

\item Implementación Software con Lenguaje de Alto Nivel: Se programa el 
analizador con un lenguaje que permita rutinas de bajo nivel, normalmente 
\index{Lenguajes C y C++}Lenguajes C y C++.
\item Implementación en \index{Lenguaje Ensamblador}Lenguaje Ensamblador: Código ``a priori'' nativo para
una determinada arquitectura. 
\item \index{Lex}Lex: Se trata de un programa
que se adapta a
las necesidades de un alfabeto y es capaz de reconocer y ordenar tokens. 

\end{enumerate}

\begin{table}[h]

\begin{center}

\begin{tabular}{|l|l|l|l|}\hline
\textbf{Implementación} & \textbf{Eficiencia} & \textbf{Velocidad} & 
\textbf{Portabilidad} \\ \hline
\hline
Aplicación Lex & Regular & Regular & Óptima \\ \hline
Código C & Buena/Muy Buena & Buena/Muy Buena & Óptima \\ \hline
Código C++ & Buena & Buena/Muy Buena & Buena/Muy Buena \\ \hline
Código Ensamblador & Muy Buena & Óptima & Muy Mala \\ \hline
\end{tabular}

\caption{Comparativa para los distintos tipos de implementaciones para un AL.}

\end{center}

\end{table}

\paragraph*{Nota:} Las formalidades que describen a un Analizador Léxico se
tratan con más detalle en el Capítulo \ref{chap:lexAnalycer}.

\section{Breve descripción de {\tt gp1990sa}}

\defn \index{gp1990sa}{\textbf{{\tt gp1990sa}}} en un Analizador Sintáctico para ISO Pascal 
7185:1990.

% \defn La gramática de ISO Pascal 7185:1990 se presenta en notación {Backus-Naur 
% Form} (BNF) que se
% describe como $G = \{\sigma_ T, \sigma_ N, S, P)$
% 
% \paragraph*{Dónde:}
% 
% \begin{enumerate}[i.]
% 
% \item $\sigma_T$ es el conjunto de los símbolos terminales.
% 
% \item $\sigma_N$ es el conjunto de los símbolos no terminales.
% 
% \item $S$ es el símbolo inicial de la gramática.
% 
% \item $P$ es el conjunto de las reglas de producción. 
% 
% \end{enumerate}

% \defn El \toIndex{Análisis Sintáctico Descendente} (ASD) trata de encontrar en las
% producciones de una gramática determinada la derivación por la izquierda del
% símbolo inicial para un conjunto de caracteres de entrada.

\defn La finalidad de {\tt p1990sa} o Analizador Sintáctico es la de certificar que las
palabras del lenguaje se organizan de acuerdo a la estructura del lenguaje.

% {\cor Nuestro análisis será predictivo}
% 
% 
% \defn El analizador predictivo se encarga de anticipar la regla a aplicar
% tomando como entrada el primer símbolo de una estructura.
% 
% 
% {\cor Con ello conseguimos una complejidad lineal: $\theta n$.} Ver Capítulo (\ref{chap:capitulo0}).
% 
% \defn Las gramáticas que son analizadas sintácticamente de forma descendente
% predictivo con un único símbolo de entrada pertenecen al grupo LL(1). 

\paragraph*{Implementación:} Existen tres tipos de implementaciones para un Analizador Sintático:

\begin{enumerate}[i.]

\item Analizador Sintáctico Descendente (\index{Top-Down-Parser}Top-Down-Parser): Se basa en analizar 
una gramática a partir de cada símbolo no terminal (es una desarrollo de arriba 
hacía abajo). Son formalmente denominados Analizadores LL.

\item Analizador Sintáctico Ascendente (\index{Bottom-Up-Parser}Bottom-Up-Parser): Analizan la gramática 
a partir de los símbolos terminales (por ello es un desarrollo de abajo 
hacía arriba). Son formalmente denominados Analizadores LR.

\item \index{Yacc}Yacc: A partir de una gramática no ambigua (aunque también acepta 
gramáticas ambiguas con resolución de problemas) genera un autómata tipo LALR 
(analizador sintáctico LR con lectura anticipada). 

\end{enumerate}


\section{Métodos y Fases de desarrollo}

El proyecto se construirá siguiendo el modelo clásico de \textbf{Ciclo de 
desarrollo en Cascada}\endnote{Debido al contexto del proyecto se omite la fase 
de Mantenimiento.}: Análisis, Diseño, Codificación y Pruebas.

\begin{enumerate}[I.]

\item Análisis: Consistirá en un estudio sobre los fundamentos matemáticos de
los compiladores (con especial énfasis en el Lenguaje de Programación Pascal)
además de una contextualización y evolución de los compiladores.

\item Diseño: Sobre la base teórica antes descrita, se hará un estudio
teórico-práctico sobre las herramientas Lex/Yacc cara a la especificación de los prototipos:
\texttt{gp1990la} y \texttt{gp1990sa}.

\item Codificación: Para las herramientas Flex/Bison:

\begin{enumerate}[i.]

\item \texttt{gp1990la.l}: Fichero de especificación léxica. Será un modelo 
funcional que
incluirá todo lo necesario para realizar programas sencillos.

\item \texttt{gp1990sa.y}: Fichero de especificación de las reglas sintácticas.

\item GNU Build System 
(Autoconf\endnote{\href{http://www.gnu.org/software/autoconf/}{http://www.gnu.org/software/autoconf/}}): Ficheros \texttt{makefile.am} y 
\texttt{configure.ac} con el objetivo mejorar la
compatibilidad de la herramientas con otras familias UNIX (principalmente
GNU/Linux y BSD) además de ser una potente ayuda para futuras correcciones y
mejoras
\end{enumerate}


\item Pruebas: Usando GNU
Pascal Compiler\endnote{{\textbf{\index{GNU
Pascal Compiler}GNU Pascal Compiler}}: {
\begin{enumerate}[i.]
\item Desarrollador: GNU Pascal Development Team.
\item Última versión estable: 2.1
\item Tipo de sistema base: UNIX y clones.
\item Licencia: GPL.
\item Página Web: \href{http://www.gnu-pascal.de/}{http://www.gnu-pascal.de/}
\end{enumerate}
}
}(\texttt{gpc}) y Free
Pascal\endnote{{\textbf{\index{Free
Pascal}Free Pascal}}: {
\begin{enumerate}[i.]
\item Desarrollador: Free Pascal Team. 
\item Última versión estable: 2.6.0
\item Tipo de sistema base: Multiplataforma.
\item Licencia: GPL.
\item Página Web: \href{http://www.freepascal.org/}{http://www.freepascal.org/}
\end{enumerate}
}
}(\texttt{fpc}) se realizará una batería de pruebas sobre las partes léxica y
sintáctica basadas en algoritmos clásicos:

\begin{enumerate}[i.]

\item \index{Algoritmos de Ordenación}Algoritmos de Ordenación: Selección Directa, Inserción Directa, 
Intercambio Directo, Ordenación Rápida (\textit{Quick Sort}) y Ordenación por 
Mezcla (\textit{Merge Sort}), 

\item \index{Algoritmos de Búsqueda}Algoritmos de Búsqueda: Búsqueda Secuencial, Búsqueda Secuencial Ordenada 
y Búsqueda Binaria. 

\end{enumerate}

\end{enumerate}

% \begin{table}[h]
% \begin{center}
% \begin{tabular}{|l|l|}\hline
% \textbf{Fase} & \textbf{Tiempo (Meses)} \\ \hline
% \hline
% Análisis & 3 Meses\\ \hline
% Diseño & 4 Meses \\ \hline
% Codificación & 2 Meses \\ \hline
% Pruebas & 2 Meses \\ \hline
% \end{tabular}
% \caption{Relación de desarrollo del proyecto en meses.}
% \end{center}
% \end{table}

%%%%%%%%%%
% Gráfico: Evolución del Proyecto
%%%%%%%%%%

\begin{landscape}
\pagestyle{empty}
\begin{figure}
\begin{center}
\begin{pspicture}(20,15)%\psgrid
\psline[linecolor=black,linewidth=1pt]{->}(0,0.52)(21,0.52)
\psline[linecolor=black,linewidth=1pt]{-}(0,0.5)(0,12.5)




\pspolygon[fillstyle=solid,fillcolor=white](1,11)(1,12)(5,12)(5,11)
\rput(3,11.5){Análisis}

\psline[linestyle=dotted, linecolor=black,linewidth=1pt]{-}(1,11)(1,0.5)
\psline[linestyle=dotted, linecolor=black,linewidth=1pt]{-}(5,10)(5,0.5)

\pspolygon[fillstyle=solid,fillcolor=white](5,10)(5,11)(12,11)(12,10)
\rput(8.5,10.5){Diseño}

\psline[linestyle=dotted, linecolor=black,linewidth=1pt]{-}(12,9)(12,0.5)

\pspolygon[fillstyle=solid,fillcolor=white](12,9)(12,10)(16,10)(16,9)
\rput(14,9.5){Codificación}

\psline[linestyle=dotted, linecolor=black,linewidth=1pt]{-}(16,8)(16,0.5)

\pspolygon[fillstyle=solid,fillcolor=white](16,8)(16,9)(20,9)(20,8)
\rput(18,8.5){Pruebas}

\psline[linestyle=dotted, linecolor=black,linewidth=1pt]{-}(20,8)(20,0.5)

\rput(1,0){Mes 1}
\rput(5,0){Mes 3}
\rput(12,0){Mes 7}
\rput(16,0){Mes 9}
\rput(20,0){Mes 11}

\end{pspicture}
\caption{Diagrama de Gantt para desarrollo de \texttt{gp1990c}.}
\end{center}
\end{figure}
\end{landscape}


\section{Entorno de desarrollo}

\begin{enumerate}[I.]

\item GNU/Linux: Sistema Operativo base (Gentoo 
GNU/Linux\endnote{{\textbf{Gentoo GNU/Linux}}: {
\begin{enumerate}[i.]
\item Desarrollador: Comunidad Gentoo GNU/Linux.
\item Última versión estable: 12.1
\item Tipo de sistema base: Monolítico.
\item Licencia: GPL y otras Licencias Libres.
\item Página Web: \href{http://www.gentoo.org/}{http://www.gentoo.org/}
\end{enumerate}
}
} para 
el desarrollo del Software y la documentación. La elección de GNU/Linux se debe 
principalmente a la plena compatibilidad con las herramientas de desarrollo tanto del Software como de la documentación (escrita con \LaTeXe). También es 
resaltable el hecho de que es compatible con otras familias UNIX como BSD.

\item BSD\endnote{{\textbf{FreeBSD (Free Berkeley Software Distribution)}}: {
\begin{enumerate}[i.]
\item Desarrollador: Comunidad FreeBSD.
\item Última versión estable: 9.1
\item Tipo de sistema base: Monolítico.
\item Licencia: Licencia BSD.
\item Página Web: \href{http://www.freebsd.org/}{http://www.freebsd.org/}
\end{enumerate}
}
}: Principalmente usaremos la versión 
FreeBSD (derivado de BSD-Lite 4.4) para mejorar la compatibilidad del Software. Se usará 
especialmente para configurar y ajustar las herramientas GNU Build System así 
como la pruebas de estabilidad y optimización del código fuente.

\item GCC\endnote{{\textbf{\index{GCC (GNU Compiler Collection)}GCC (GNU Compiler Collection)}}: {
\begin{enumerate}[i.]
\item Desarrollador: Proyecto GNU.
\item Última versión estable: 4.8.1
\item Tipo de sistema base: UNIX y clones.
\item Licencia: Licencia GPLv3.
\item Página Web: \href{http://gcc.gnu.org/}{http://gcc.gnu.org/}
\end{enumerate}
}
}: 
Metacompilador que nos servirá para generar programas ejecutables.

\item GNU Build System: Conjunto de herramientas:

\begin{enumerate}[i]

\item GNU Autoconf: Se trata de una herramienta de propósito general para 
generar ficheros ejecutables para distintas versiones de UNIX. Usa: 
\texttt{configure.ac} y \texttt{makefile.in} para generar \texttt{makefile} 
sobre el entorno.

\item GNU Automake: Genera el fichero \texttt{makefile.in} a partir de las 
especificaciones de \texttt{makefile.am} necesario para Autoconf.

\item GNU Libtool: Se trata de una herramienta que genera bibliotecas estáticas 
y dinámicas para las distintas versiones de UNIX.

\end{enumerate}

\item Flex\endnote{\textbf{Flex (Fast Lexical Analyzer Generator)}: {
\begin{enumerate}[i.]
\item Desarrollador: Vern Paxson.
\item Última versión estable: 2.5.37 (3 de Agosto de 2012)
\item Tipo de sistema base: UNIX y clones.
\item Licencia: Licencia BSD.
\item Página Web: \href{http://flex.sourceforge.net/}{http://flex.sourceforge.net/}
\end{enumerate}
}
}: Flex (Fast Lexical Analyzer Generator) se trata de un programa para 
el análisis léxico de Lenguajes Regulares (versión GNU de Lex). Internamente es 
un Autómata Finito Determinista (AFD).

\item Bison\endnote{\textbf{Bison (GNU Bison)}: {
\begin{enumerate}[i.]
\item Desarrollador: Proyecto GNU.
\item Última versión estable: 3.0 (26 de Julio de 2013)
\item Tipo de sistema base: UNIX y clones.
\item Licencia: Licencia GPL.
\item Página Web: \href{http://www.gnu.org/software/bison/}{http://www.gnu.org/software/bison/}
\end{enumerate}
}
}: Se trata de un analizador sintáctico (versión GNU de Yacc) para 
Gramáticas Libres de Contexto (también es capaz de generar código para algunos 
tipos de Gramáticas Ambiguas). Se trata de una analizador que genera un autómata LALR para los Lenguajes C, C++ y Java (principalmente).

\item \TeX\ Live 2011\endnote{\textbf{\TeX\ Live 2011}: {
\begin{enumerate}[i.]
\item Desarrollador: TeX Users Group.
\item Última versión estable: 2013.
\item Tipo de sistema base: Familia UNIX, Familia GNU/Linux y Familia Win2k.
\item Licencia: LaTeX Project Public License (LPPL), GPLv2.
\item Página Web: \href{http://www.tug.org/texlive/}{http://www.tug.org/texlive/}
\end{enumerate}
}
}: Es la Metadistribución de \TeX\ común para sistemas GNU. 
Contiene todos los paquetes oficiales propuestos por \TeX\ Users Group. Se 
usarán además los entornos PStricks y MetaPost para la generación de gráficos 
vectoriales.

\end{enumerate}

%\nocite{aalp, cptt, article/iso/piso7185}