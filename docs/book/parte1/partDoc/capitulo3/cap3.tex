%%%%%%%%%%%%%%
% Chapter 3: El Lenguaje de Programación Pascal
%%%%%%%%%%%%%%

\label{chap:cap3}

\section{Introducción}

\textit{A programming language called Pascal is described which was developed
on the basis of Algol 60. Compared to Algol 60, its rango of applicability is
considerably increased due to a variety of data structuring facilities. In view
of its intended usage both as convenient basis to teach programming and as an
efficient tool to write large programs, emphasis was placed on keeping the
number of fundamental concepts reasonably small, on a simple and systematic
language structure, and on efficient implementability.A one-pass compiler has
been contructed for the CDC 6000 computer family; it is expressed entirely in
terms of Pascal itself.} \endnote{\textit{El Lenguaje de Programación Pascal es 
descrito a partir del desarrollo
de una versión de Algol 60. 
Comparado con Algol 60, el rango de aplicación es considerablemente mayor
gracias a la variedad de estructura de datos.
En principio es un intento para fundamentar las bases para enseñar programación
con una herramienta eficiente para escribir grandes programas
enfatizando en usar conceptos razonable sencillos, sistemáticos a la
programación estructurada, y la eficiencia de la implementación.
El compilador de es una sola pasada y ha sido construido para la familia CDC
6000.}} \cite{article/pascal/tplp}


\textbf{El Lenguaje de Programación \index{Pascal}Pascal fue creado por el profesor
\index{Niklaus Wirth}Niklaus Wirth}\endnote{El profesor Niklaus Wirth nació en Winterthur
Suiza el 15 de febrero de 1934.

Se gradúa en 1959 como Ingeniero en Electrónica por la Escuela Politécnica
Federal de Zúrich (ETH) en Suiza. Un año más tarde, se doctora (Ph.D.) en la
Universidad de
Berkley, California.

Trabajó a mediados de la década de los sesenta del siglo XX en la Universidad de
Stanford y en la Universidad de Zúrich. Finalmente en 1968 se convierte
en profesor de Informática en la ETH en Suiza.} a finales de la década de los
sesenta del siglo
XX. En 1970 fue
finalmente publicado,
fijando dos objetivos en su diseño arquitectónico:

\begin{enumerate}[i.] 
\item Crear un \textbf{lenguaje claro y natural orientado a la enseñanza} de los
fundamentos de la programación de computadores. Por ello se estructuran los
módulos como funciones y procedimientos.
\item Definir un lenguaje que \textbf{permita realizar programas lo más
eficientes
posibles}. El tipado de datos es explícito.
\end{enumerate}

Pascal recibe su nombre en honor al matemático francés \index{Blaise Pascal}Blaise Pascal (ver Anexo \ref{chap:blaisePascal}).

\begin{figure}[h]
\begin{center}
\begin{pspicture}(15,10)%\psgrid
\psline[linecolor=black,linewidth=1pt]{->}(0,0.52)(15,0.52)
\psline[linecolor=black,linewidth=1pt]{->}(0,0.5)(0,10)
\rput(1.5,1.5){\texttt{FORTRAN}}
\psline[linecolor=black,linewidth=1pt]{<->}(2.5,1.5)(9.7,1.5)
\rput(11,1.5){\texttt{FORTRAN 77}}
\psline[linecolor=black,linewidth=1pt]{<->}(12.2,1.5)(14.6,1.5)


\rput(2,5){\texttt{ALGOL 58}}
\psline[linecolor=black,linewidth=1pt]{<->}(3,5)(4,5)
\rput(5,5){\texttt{ALGOL 60}}
\pscurve[linecolor=black,linewidth=1pt]{<->}(5,5.3)(5,7.4)(7.1,8)

\pscurve[linecolor=black,linewidth=1pt]{->}(6.2,5)(8.8,5.5)(9.2,6)(9,6.7)
\rput(8,8){\texttt{ALGOL W}}
\pscurve[linecolor=black,linewidth=1pt]{->}(8.1,8.4)(8.1,8.9)(9.2,9)
\rput(9,7){\texttt{ALGOL 68}}
\pscurve[linestyle=dotted, linecolor=black,linewidth=1pt]{->}(9,7.3)(9,7.7)(9.3,7.9)(10.1,8.3)(10,8.7)
\rput(10,9){\texttt{PASCAL}}
\psline[linecolor=black,linewidth=1pt]{->}(10.8,9)(12.2,9)
\rput(13,9){\texttt{MODULA}}
\pscurve[linecolor=black,linewidth=1pt]{->}(13,8.7)(14.1,7.8)(14,6.3)
\rput(14,6){\texttt{ADA}}

\rput(1.5,0){1957}
\psline[linecolor=black,linewidth=0.8pt]{-}(1.5,0.3)(1.5,0.7)
\rput(5,0){1960}
\psline[linecolor=black,linewidth=0.8pt]{-}(5,0.3)(5,0.7)
\rput(8,0){1966}
\psline[linecolor=black,linewidth=0.8pt]{-}(8,0.3)(8,0.7)
\rput(9,0){1968}
\psline[linecolor=black,linewidth=0.8pt]{-}(9,0.3)(9,0.7)
\rput(10,0){1971}
\psline[linecolor=black,linewidth=0.8pt]{-}(10,0.3)(10,0.7)
\rput(11,0){1977}
\psline[linecolor=black,linewidth=0.8pt]{-}(11,0.3)(11,0.7)
\rput(12.5,0){1978}
\psline[linecolor=black,linewidth=0.8pt]{-}(12.5,0.3)(12.5,0.7)
\rput(14,0){1983}
\psline[linecolor=black,linewidth=0.8pt]{-}(14,0.3)(14,0.7)


\pspolygon[fillstyle=solid,fillcolor=white](0.4,8.6)(0.4,10)(3.5,10)(3.5,8.6)

\psline[linecolor=black,linewidth=1pt]{->}(2.6,9.8)(3.4,9.8)
\rput(1.45,9.8){{\scriptsize{\textit{Evo. Directa:}}}}
\psline[linecolor=black,linewidth=1pt]{<->}(2.6,9.3)(3.4,9.3)
\rput(1.5,9.3){{\scriptsize{\textit{Evo. Concept:}}}}
\psline[linestyle=dotted, linecolor=black,linewidth=1pt]{->}(2.6,8.8)(3.4,8.8)
\rput(1.25,8.8){{\scriptsize{\textit{Influencia:}}}}
\end{pspicture}
\caption{Relaciones entre los primeros Lenguajes de Programación.}
\end{center}
\end{figure}



\section{Influencias del Lenguaje Pascal}

\subsection{Fortran (The IBM Mathematical Formula Translating System)}

Fortran\index{Fortran}, inicialmente conocido como \textbf{FORTRAN es el acrónimo de
\textit{The IBM
Mathematical Formula Translating System}}. 

\textbf{Fortan se trata del primer lenguaje de alto nivel}. Es multipropósito y
se basa
en el paradigma de la programación estructurada.

Su origen tiene que ver con la necesidad de crear aplicaciones científicas de
manera más sencilla y lógica para el entendimiento humano.

\textit{The FORTRAN language is intended to be capable of expressing any
problem of
numerical computation. In particular, it deals easily with problems containing
large sets of formulae and many variables, and it permits any variable to have
up to three independent subscripts. However, for problems in which machine words
have a logical rather than a numerical meaning it is less satisfactory, and it
may fail entirely to express some such problems. Nevertheless, many logical
operations not directly expressable in the FORTRAN language can be obtained by
making use of provisions for incorporating library routines.}
\endnote{\textit{El
lenguaje 
de programación FORTRAN intenta hacer posible la expresión de cualquier problema
numérico.
En particular, es ideal para formular conjuntos de múltiples variables, 
permitiendo que cualquier variable sea tratada desde un plano libre de contexto.
Sin embargo, este tipo de ejercicios presenta inconsistencias con el repertorio 
de palabras de cada máquina de computo y su lógica numérica, lo que puede llevar
a 
problemas a la hora de expresar algunos problemas numéricos.
Mucha de la lógica que emplea FORTRAN no es directamente expresable por lo que
se 
puede obtener incorporando bibliotecas.}}
\cite{article/iso/fortran66}

El primer proyecto de compilador de FORTRAN fue un \index{Milestone}Milestone que
ocupaba
\texttt{15KB} aproximadamente. Era muy rudimentario y funcionaba con rutinas muy
primitivas de los SSOO de la época, prácticamente era código ensamblador.

El compilador oficial de FORTRAN fue escrito entre 1954 y 1957 a cargo de John
W. Backus y grandes programadores como: Sheldon F. Best, Harlan Herrick, Peter
Sheridan, Roy Nutt, Robert Nelson, Irving Ziller, Richard Goldberg, Lois Haibt
and David Sayre. La primera ejecución del compilador se realizó sobre una
máquina \index{IBM 704}IBM 704.
 
Su primeros programas fueron para control energético de reactores nucleares.
Demostraba ser mucho más rápido que otras soluciones tradicionales sobre
Lenguaje Ensamblador.

%%%%%%%%%%
% Gráfico: Evolución de Fortran
%%%%%%%%%%

\begin{figure}[h]
\begin{center}
\begin{pspicture}(15,3)%\psgrid
\psline[linecolor=black,linewidth=1pt]{->}(0,0.5)(15,0.5)
\rput(1.5,1){\texttt{FORTRAN 66}}
\rput(1.5,0){1966}
\psline[linecolor=black,linewidth=0.8pt]{-}(1.5,0.3)(1.5,0.7)

\rput(4.5,1){\texttt{FORTRAN 77}}
\rput(4.5,0){1977}
\psline[linecolor=black,linewidth=0.8pt]{-}(4.5,0.3)(4.5,0.7)

\rput(7.5,1){\texttt{Fortran 90}}
\rput(7.5,0){1991}
\psline[linecolor=black,linewidth=0.8pt]{-}(7.5,0.3)(7.5,0.7)

\pscurve[linecolor=black,linewidth=1pt]{->}(6.6,1.3)(6.4,1.8)(7.2,2)


\rput(8.5,2){\texttt{Fortran 95}}
\rput(8.5,0){1997}
\psline[linecolor=black,linewidth=0.8pt]{-}(8.5,0.3)(8.5,0.7)

\rput(10.5,1){\texttt{Fortran 2003}}
\rput(10.5,0){2003}
\psline[linecolor=black,linewidth=0.8pt]{-}(10,0.3)(10,0.7)

\rput(14,1){\texttt{Fortran 2008}}
\rput(14,0){2007}
\psline[linecolor=black,linewidth=0.8pt]{-}(14,0.3)(14,0.7)



\end{pspicture}
\caption{Evolución del Lenguaje Fortran.}
\end{center}
\end{figure}



\textbf{El Lenguaje Fortran ha sido parte de seis estandarizaciones}:

\begin{enumerate}[I.]
\item FORTRAN o \index{FORTRAN 66}FORTRAN 66: La característica más destacada es la separación de
las fases de compilación, además de la posibilidad de enlazar con rutinas de
lenguaje ensamblador.
\item \index{FORTRAN 77}FORTRAN 77: Entre sus características destacan:
\begin{enumerate}[i.]
\item Bucles \texttt{DO} con variable índice de incremento y decremento.
\item Bloque de secuencias: \texttt{\{IF...THEN...ELSE...ENDIF.\}}
\item Pruebas antes de compilación de bucles \texttt{\{DO\}}.
\item Tipo de dato \texttt{CHARACTER}.
\item El símbolo apostrofe (\texttt{\textquotesingle}) como delimitador de
conjuntos de caracteres.
\item Final de un programa sin necesidad de usar la palabra \texttt{\{STOP\}}.
\end{enumerate}
\item \index{Fortran 90}Fortran 90:  Sus principales novedades son:
\begin{enumerate}[i.]
\item Nuevas estructuras de flujo: \texttt{\{CASE \& DO WHILE\}}.
\item Estructuras de datos tipo \texttt{RECORD}.
\item Mejora en el manejo de \texttt{ARRAY} (nuevos operadores).
\item Memoria dinámica.
\item Sobrecarga de operadores.
\item Paso de argumentos por referencia.
\item Control de precisión y rango.
\item Módulos (paquetes de código).
\end{enumerate}
\item \index{Fortran 95}Fortran 95: 
\begin{enumerate}[i.]
\item Construcciones \texttt{\{FORALL\}}.
\item Procedimientos \texttt{PURE} y \texttt{ELEMENTAL}.
\item Mejoras en la inicialización de objetos.
\item Sentencia \texttt{\{DO\}} para tipos de datos: \texttt{REAL} y
\texttt{DOUBLE PRECISION}.
\item Sentencia \texttt{\{END IF\}} para terminar bloque.
\item Sentencia \texttt{\{PAUSE\}}.
\item Incorporación de \texttt{ISO/IEC 1539-1:1997} que incluye dos tipos de
módulos opcionales:

\begin{enumerate}[a.]
\item \texttt{STRINGS} dinámicos \texttt{ISO/IEC 1539-2:2000}.
\item Compilación condicional \texttt{ISO/IEC 1539-3:1998}.
\end{enumerate}

\end{enumerate}
\item \index{Fortran 2003}Fortran 2003:
\begin{enumerate}[i.]
\item Soporte de Programación Orientada a Objetos: Extensión de tipos,
Polimorfismo y completo soporte para TADS (Tipos Abstractos de Datos) entre otras
características.
\item Mejora en la manipulación de memoria: Valores por referencia, atributo
\texttt{VOLATILE}, especificación explícita de constructores para \texttt{ARRAY}
y sentencia \texttt{\{POINTER\}}.
\item Mejoras en Entrada/Salida: Transferencia asíncrona, acceso por flujo
\texttt{(STEAM)}, especificación de operaciones de transferencia, sentencias de
control y de redondeo para conversiones y sentencia \texttt{\{FLUSH\}}.
\item Soporte para aritmética flotante de \texttt{IEEE}.
\item Interoperabilidad con el Lenguaje de Programación C.
\item Internacionalización \texttt{ISO 1064}.
\item Mejora en la integración son SSOO anfitrión: Acceso a línea de comandos,
variables de sistema, procesos y mensajes de error.
\end{enumerate}
\item \index{Fortran 2008}Fortran 2008:
\begin{enumerate}[i.]
\item Submódulos \texttt{ISO/IEC TR 19767:2005}.
\item Modelos de \texttt{ARRAY} para ejecución en paralelo.
\item Construcción \texttt{\{DO CONCURRENT\}}.
\item Atributo \texttt{CONTIGUOUS}.
\item Construcciones de tipo \texttt{BLOCK}.
\item Componentes recursivos dinámicos.
\end{enumerate}
\end{enumerate}

\subsubsection{Análisis de Fortran}

\paragraph*{Nota:} Basado en: \texttt{Fortran ISO 2003}.

\begin{enumerate}[I.]

\item Alfabeto:

\begin{enumerate}[i.]

\item 26 letras\endnote{La relación entre mayúsculas (\index{Upper-Case}) y minúsculas
(\index{Lower-Case}) se delega en el fabricantes del compilador.}:
\texttt{\textquoteleft a\textquoteright\ | \textquoteleft b\textquoteright\  |
\textquoteleft c\textquoteright\  | \textquoteleft d\textquoteright\  |
\textquoteleft e\textquoteright\  | \textquoteleft f\textquoteright\  |
\textquoteleft g\textquoteright\  | \textquoteleft h\textquoteright\  |
\textquoteleft i\textquoteright\ | \textquoteleft j\textquoteright\ 
          | \textquoteleft k\textquoteright\ | \textquoteleft l\textquoteright\
| \textquoteleft m\textquoteright\ | \textquoteleft n\textquoteright\ |
\textquoteleft o' | \textquoteleft p' | \textquoteleft q\textquoteright\ |
\textquoteleft r\textquoteright\ | \textquoteleft s\textquoteright\ |
\textquoteleft t\textquoteright
          | \textquoteleft u\textquoteright\ | \textquoteleft v\textquoteright\
| \textquoteleft w\textquoteright\ | \textquoteleft x\textquoteright\ |
\textquoteleft y\textquoteright\ | \textquoteleft z\textquoteright\ }

\item 10 dígitos: \texttt{\textquoteleft 0\textquoteright\ | \textquoteleft
1\textquoteright\ | \textquoteleft 2\textquoteright\ | \textquoteleft
3\textquoteright\ | \textquoteleft 4\textquoteright\ | \textquoteleft
5\textquoteright\ | \textquoteleft 6\textquoteright\ | \textquoteleft
7\textquoteright\ | \textquoteleft 8\textquoteright\ | \textquoteleft
9\textquoteright\ }

\item Carácter \textit{Underscore}: \texttt{'\_'}

\item Símbolos especiales: ver Figura (\ref{fig:ssFortran2003}).


\begin{figure}

\begin{flushleft}
\begin{verbatim}
------------------------------------------------------------------------------
| Carácter  | Nombre                  || Carácter  | Nombre                  |
------------------------------------------------------------------------------
|           | Blank                   || ;         | Semicolon               |
| =         | Equals                  || !         | Exclamation point       |
| +         | Plus                    || "         | Quotation mark or quote |
| -         | Minus                   || %         | Percent                 |
| *         | Asterisk                || &         | Ampersand               |
| /         | Slash                   || ~         | Tilde                   |
| \         | Backslash               || <         | Less than               |
| (         | Left parenthesis        || >         | Greater than            |
| )         | Right parenthesis       || ?         | Question mark           |
| [         | Left square bracket     || ’         | Apostrophe              |
| ]         | Right square bracket    || `         | Grave accent            |
| {         | Left curly bracket      || ^         | Circumflex accent       |
| }         | Right curly bracket     || |         | Vertical line           |
| ,         | Comma                   || $         | Currency symbol         |
| .         | Decimal point or period || #         | Number sign             |
| :         | Colon                   || @         | Commercial at           |
------------------------------------------------------------------------------
\end{verbatim}              
\end{flushleft}
\caption{Símbolos especiales de \texttt{Fortran 2003}.}\label{fig:ssFortran2003}
\end{figure}


\item Otros símbolos: Dichos símbolos pueden ser representables pero solamente
aparecen en: comentarios, caracteres de constantes, registros de entrada/salida
y descripciones.

%\item Separadores: \texttt{[( ... ) | / ... / | [ ... ] | (/ ... /)]}

\end{enumerate}

\item Gramática: Ver bibliografía capitular \cite{article/iso/fortran2003}.

\end{enumerate}

\prog \texttt{helloProgrammer.f95}

\lstinputlisting[language=Fortran]{
./programms/programms13/Fortran/helloProgrammer.f95}

\paragraph*{Notas sobre compilación:} Para compilar el archivo fuente
\texttt{helloProgrammer.f95} sobre GNU, usaremos el compilador \index{GNU
Fortran}GNU Fortran\endnote{\href{http://gcc.gnu.org/fortran/}{http://gcc.gnu.org/fortran/}
}. 

Las ordenes para compilarlo y ejecutarlo son las siguientes:

\begin{verbatim}
$ gfortran -o helloProgrammer helloProgrammer.f95 
$ ./helloProgrammer 
 Hello programmer!
\end{verbatim}

\subsection{ALGOL (ALGOrithmic Language)}

\subsubsection{Definiciones}

\defn \index{ALGOL}ALGOL inicialmente recibió el nombre de IAL \textit{Internationa
Alogrithmic Language}.

\defn ALGOL se trata de una familia de Lenguajes de Programación basados todos
ellos en la primera versión del Lenguaje base \index{ALGOL 58}ALGOL 58.

{
\defn Diseñado entre 1957 y 1960 por un comité de científicos europeos y
americanos que se basaban en dos ideas principales:

\begin{enumerate}[i.]

\item Mejorar las deficiencias estructurales de FORTRAN (todavía sin estándar
pero ampliamente usado).

\item Crear un lenguaje altamente expresivo que sea capaz de dar una respuesta
común a todos los científicos.

\end{enumerate}
}

{\cor Unos de sus notables avances fue el de limitar unidades de código
(sentencias) en bloques} \texttt{\{BEGIN...END.\}}

\subsubsection{Historia}

\textit{The purpose of the algorithmic language is to describe computational
processes.
The basic concept used for the description of calculating rules is the well
known
arithmetic expression containing as constituents numbers, variables, and functions. From such expressions are compounded, by applying rules of arithmetic
composition, self-contained units of the language--explicit formulae--called
assignment statements.} \endnote{\textit{El propósito de ALGOrithmic Lenguaje es
la 
de describir procesos computacionales.
El concepto básico usado en la descripción de reglas de cálculo es conocido 
expresiones aritméticas constituidas por: números, variables y funciones.
Las expresiones son compuestas aplicando reglas de composición aritmética, 
constituyendo unidades del lenguaje, explícitamente formuladas,
llamadas asignación de recursos.}} \cite{article/algol/ralgol60}

Como hemos dicho anteriormente, ALGOL tiene su primera especificación formal en
el año 1958. Este este documento base (ALGOL 58) fue oficialmente presentado en
tres formatos:

\begin{enumerate} [i.]

\item \textit{Reference Language} (\index{Lenguaje de Referencia}Lenguaje de Referencia): Es el documento donde se recoge íntegramente el trabajo del Comité (Enero de 1960). En el mismo, se define el lenguaje basándose en notación matemática. Así mismo, es la referencia básica de ALGOL.

\item \textit{Publications Language} (\index{Lenguaje de Publicaciones}Lenguaje de Publicaciones): Frente a \textit{Reference Language} permite variaciones en la simbología del documento para que pueda ser publicado y distribuido internacionalmente.

\item \textit{Hardware Representations} (\index{Representaciones Hardware}Representaciones Hardware): Se trata de consideraciones sobre \textit{Reference Language} con el objetivo de limitar la especificación al Hardware de la época.

\begin{table}[h]

\begin{center}

\begin{tabular}{|l|l|}\hline
\textbf{Símbolo} & \textbf{Descripción} \\ \hline
\hline
Corchetes $[\ ]$ & Determinar la existencia o no de espacio `\ '\ y otros caracteres. \\ \hline
Exponenciación $\uparrow$ & Determinar la operación Exponente. \\ \hline
Paréntesis $(\ )$ & Usado como: Paréntesis o Corchete. \\ \hline
Base 10: $a_{10}$ & Determinar la notación de Base 10 en operaciones matemáticas. \\ \hline
\end{tabular}

\caption{Convención entre \textit{Reference Language} y otras publiaciones de ALGOL.}

\end{center}

\end{table}

\end{enumerate}

ALGOL desde su comienzo tuvo un importante nicho entre científico europeos y
americanos. De igual manera, ALGOL introduce en su especificación formal la
notación \index{Backus-Naur Form}Backus-Naur Form que ha sido
utilizada desde entonces como método descriptivo de los Lenguajes de
Programación. También es notable el hecho de que ALGOL es el primer lenguaje
que combina el flujo imperativo con \textit{Lambda-Calculus}.

La primera versión estandarizada de ALGOL es ALGOL 58 que finalmente fue
mejorado y actualizado con la nueva versión ALGOL 60.


\index{ALGOL 60}ALGOL 60 es un uno de los estándares más usados y marco de referencia básica
para la creación y especificación de otros lenguajes. Ha sido por ello, base de lenguajes tan
importantes como: BCPC, B, Pascal, Simula o C.

El problema que intentó solucionar esta versión fue la de hacer de ALGOL un
lenguaje con aspiraciones comerciales. Por ello, se trabajó intensamente en
mejorar la Entrada/Salida y la relación con el entorno de ejecución (SSOO).

Partiendo de este estándar conceptual y muy avanzado surgieron dos nuevas
propuestas:

\begin{enumerate}[i.]

\item ALGOL 68: Sobre \index{ALGOL 68}ALGOL 68 destacar que añade gran
cantidad de utilidades que eran comúnmente utilizadas por programadores de la época, entre
ellas: declaración de tipos, estructuras de unión y modelos de variables por
referencia.

La especificación de esta nueva revisión adoptó la notación de \index{Adriann van Wijgaarden}Adriann van Wijgaarden, que usaba gramáticas libres de contexto para generar infinitos conjuntos de producción.

\item ALGOL W: \index{ALGOL W}ALGOL W fue un proyecto encargado al profesor Nicklaus Wirth que tras la publicación de ALGOL 68 y las enormes quejas que despertó, trataba de actualizas ALGOL 60 intentando conservar la esencia y cultura del
lenguaje. 

\end{enumerate}


%%%%%%%%%%
% Gráfico: Evolución de ALGOL
%%%%%%%%%%

\begin{figure}[h]
\begin{center}
\begin{pspicture}(12,3)%\psgrid
\psline[linecolor=black,linewidth=1pt]{->}(0,0.5)(12,0.5)
\rput(1.5,1){\texttt{ALGOL 58}}
\rput(1.5,0){1958}
\psline[linecolor=black,linewidth=0.8pt]{-}(1.5,0.3)(1.5,0.7)

\rput(4.5,1){\texttt{ALGOL 60}}
\rput(4.5,0){1960}
\psline[linecolor=black,linewidth=0.8pt]{-}(4.5,0.3)(4.5,0.7)

%\pscurve[linestyle=dotted, linecolor=black,linewidth=1pt]{->}(4.5,1.2)(5.25,1.5)(6,1.6)(6.75,1.5)(7.5,1.2)

%\pscurve[linestyle=dotted, linecolor=black,linewidth=1pt]{->}(4.5,1.2)(6,2.2)(7.5,2.5)(9,2.2)(10.5,1.2)

\rput(7.5,1){\texttt{ALGOL W}}
\rput(7.5,0){1966}
\psline[linecolor=black,linewidth=0.8pt]{-}(7.5,0.3)(7.5,0.7)

\rput(10.5,1){\texttt{ALGOL 68}}
\rput(10.5,0){[1968 - 1973]}
\psline[linecolor=black,linewidth=0.8pt]{-}(10.5,0.3)(10.5,0.7)

\end{pspicture}
\caption{Evolución del Lenguaje ALGOL.}
\end{center}
\end{figure}



% \subsubsection{ALGOL 58}
% 
% \paragraph*{Análisis}
% 
% \paragraph*{Nota:} 
% 
% \begin{enumerate}[I.]
% 
% \item Alfabeto:
% 
% \begin{enumerate}[i.]
% 
% \item asdfg
% 
% \end{enumerate}
% 
% \item Gramática: Ver bibliografía capitular .
% 
% \end{enumerate}


\subsubsection{ALGOL 60}

\begin{enumerate}[I.]

\item Alfabeto:

\begin{enumerate}[i.]

\item Letras:

\begin{verbatim}
<letter> ::= a | b | c | d | e | f | g | h | i | j | k | l |
        m | n | o | p | q | r | s | t | u | v | w | x | y | z | A |
        B | C | D | E | F | G | H | I | J | K | L | M | N | O | P |
        Q | R | S | T | U | V | W | X | Y | Z
\end{verbatim}

\item 10 dígitos: \texttt{<digit> ::= 0 | 1 | 2 | 3 | 4 | 5 | 6 | 7 | 8 | 9 }

\item Valores lógicos: \texttt{<logical value> ::= true | false}

\item Delimitadores:
\begin{verbatim}
<delimiter> ::= <operator> | <separator> | <bracket> |
        <declarator> | <specificator>

<operator> ::= <arithmetic operator> | <relational operator> |
        <logical operator> | <sequential operator>

<arithmetic operator> ::= + | - | TIMES | / | ÷ | POWER

<relational operator> ::= < | NOTGREATER | = | NOTLESS | > | NOTEQUAL

<logical operator> ::= EQUIVALENCE | IMPLICATION | OR | AND | ¬

<sequential operator> ::= goto | if | then |
        else | for | do (2)

<separator> ::= , | . | 10 | : | ; | := | BLANK | step |
        until | while | comment

<bracket> ::= ( | ) | [ | ] | ` | ' | begin | end

<declarator> ::= own | Boolean | integer |
        real | array | switch |
        procedure

<specificator> ::= string | label |
        value
\end{verbatim}

\end{enumerate}

\item Gramática: Ver bibliografía capitular .

\end{enumerate}

%
% PROGRAMA helloProgrammer! para Algol 60
%

\prog \texttt{helloProgrammer.a60}

\lstinputlisting[language=Pascal]{./programms/programms13/Algol/Algol60/helloProgrammer.a60}

\paragraph*{Notas sobre compilación:} Para compilar el archivo fuente
\texttt{helloProgrammer.a60} sobre GNU, usaremos traductor de ALGOL a Lenguaje C Marst \endnote{\href{http://www.gnu.org/software/marst/}{http://www.gnu.org/software/marst/}}. 

Las ordenes para compilarlo y ejecutarlo son las siguientes:

\begin{verbatim}
marst helloProgrammer.a60 -o helloProgrammer.a60
cc helloProgrammer.a60 -lalgol -lm -o ./helloProgrammer
./helloProgrammer

Hello Programmer!

\end{verbatim}

% \subsubsection{ALGOL 68}
% 
% \paragraph*{Análisis}
% 
% \paragraph*{Nota:} 
% 
% \begin{enumerate}[I.]
% 
% \item Alfabeto:
% 
% \begin{enumerate}[i.]
% 
% \item asdfg
% 
% \end{enumerate}
% 
% \item Gramática: Ver bibliografía capitular .
% 
% \end{enumerate}
% 
% %
% % PROGRAMA helloProgrammer! para Algol 68
% %
% 
% \prog \texttt{helloProgrammer.a68}
% 
% %\lstinputlisting[language=Algol]{
% %./programms/programms13/Algol68/helloProgrammer.a68}
% %\lstinputlisting[]{./programms/programms13/Algol/Algol68/helloProgrammer.a68}
% 
% \paragraph*{Notas sobre compilación:} Para compilar el archivo fuente
% \texttt{helloProgrammer.a68} sobre GNU, usaremos el compilador \texttt{Algol 68
% Genie}\endnote{\href{jmvdveer.home.xs4all.nl}{jmvdveer.home.xs4all.nl}}. 
% 
% Las ordenes para compilarlo y ejecutarlo son las siguientes:
% 
% \begin{verbatim}
% $ a68g  --compile helloProgrammer.a68
% $ ./helloProgrammer.sh
% Hello Programmer!
% 
% \end{verbatim}

\subsubsection{ALGOL W}

ALGOL W, inicialmente denominado \index{ALGOL X}ALGOL X, fue desarrollado por
Nicklaus Wirth y C.A.R como sucesor directo de ALGOL 60 a propuesta en
IFIP Working Group. La especificación del lenguaje se vio insuficiente
y fue finalmente publicada como: \textit{``A contribution to the development of
ALGOL''}. 

Dicha especificación incluía mejoras que en ningún momento querían romper la
armonía original de ALGOL 60. Se trataba de una revisión conservadora. Entre estas mejoras destacan:

\begin{enumerate}[i.]

\item Tipo de datos \texttt{STRING}.

\item Incorporación de Números Complejos.

\item Llamada por referencia de tipos de datos \texttt{RECORD}.

\item Añade la estructura \texttt{WHILE}.

\item Reemplazo de la estructura \texttt{SWITCH} por \texttt{CASE}.

\end{enumerate}

\section{El Lenguaje Pascal} 

Pascal se sustenta sobre dos poderosos lenguajes del ámbito científico:
FORTRAN y
ALGOL, de los que anteriormente hemos hablado.


\textbf{Lo que trataba de hacer Wirth era actualizar ALGOL 60 en un nuevo
lenguaje de propósito mucho más general.
}

El primer prototipo de esta idea fue ALGOL W programado sobre una computadora
\index{IBM 360}IBM 360. Fue una versión bastante conservadora del lenguaje lo que dio
fuerza a la idea de que el nuevo lenguaje que deseaba construir Wirth tenia que
soportar un repertorio de mucho más amplio.

Wirth además \textbf{quería que dicho lenguaje tuviera fines educativos}, es
decir,
\textbf{que enseñase la cultura de ``la buena programación''}. Para ello tomo
como
referencia a FORTRAN y al Lenguaje Ensamblador.


En 1968 cuando empezó a implementar estos hitos en el futuro lenguaje.


Había igualmente una alta competencia con las soluciones de compilación que
ofrecía FORTRAN, por ello decidió que su compilador sería de una sola
pasada\endnote{Single-Pass.}
basada en el diseño ``\index{Top-Down}Top-Down''.


El compilador se completó finalmente a mediados de 1970.



Pascal después llego a ser un lenguaje muy popular en círculos
universitarios durante la década de los ochenta y noventa del siglo XX debido
principalmente a la venta de compilares muy económicos, y un IDE de
propósito general que se basaba en el mismo, hablamos de Turbo Pascal.

\subsection{Pascal ISO 7185:1990}
En 1977 BSI\endnote{\textit{British Standards Institution}:
\href{http://www.bsigroup.com/}{http://www.bsigroup.com/}}
\cite{article/iso/piso7185}
produjo el estándar del Lenguaje de
Programación Pascal, publicado en 1979. Ese mismo año, el
organismo
BSI propuso
que Pascal fuese parte del programa ISO. Fue aceptado con
denominación \texttt{ISO/TC97/SC5/WG4}. 

En los Estados Unidos de América, \index{IEEE}IEEE aprobó el 10978 del proyecto 770
(Pascal).


En Diciembre, \texttt{178 X3J9} convino el resultado de
\texttt{SPARCH}\endnote{\textit{Standards
Planning and
Requirements Committee}} para la resolución de US
TAG\endnote{\textit{Technical Advisory
Group}:
\href{http://technicaladvisorygroup.com/}{http://technicaladvisorygroup.com/}}
para la \texttt{ISO Pascal}.

En Febrero de 1979, representantes de \texttt{IEEE} combinaron los proyectos
\texttt{X3} y \texttt{IEEE 770}
bajo el comité \texttt{X3J9/IEEE-P770 Pascal Standards (JPC)}.

La resolución de \texttt{JFC} fue avalada en \texttt{NSI/IEEE770X3 .97-1983} por
ANSI
bajo \texttt{American National Standard Pascal Computer Programming Language}.

Las especificaciones de BSI se hicieron públicas en 1982, internacionalmente conocido como
\index{Standard 7185}Standard 7185.

\subsubsection{Alfabeto}

\begin{enumerate}[I.]

\item Unidades:

\begin{enumerate}[i.]

\item \texttt{letter = \textquoteleft a\textquoteright\ | \textquoteleft
b\textquoteright\  | \textquoteleft c\textquoteright\  | \textquoteleft
d\textquoteright\  | \textquoteleft e\textquoteright\  | \textquoteleft
f\textquoteright\  | \textquoteleft g\textquoteright\  | \textquoteleft
h\textquoteright\  | \textquoteleft i\textquoteright\ | \textquoteleft
j\textquoteright\ 
          | \textquoteleft k\textquoteright\ | \textquoteleft l\textquoteright\
| \textquoteleft m\textquoteright\ | \textquoteleft n\textquoteright\ |
\textquoteleft o' | \textquoteleft p' | \textquoteleft q\textquoteright\ |
\textquoteleft r\textquoteright\ | \textquoteleft s\textquoteright\ |
\textquoteleft t\textquoteright
          | \textquoteleft u\textquoteright\ | \textquoteleft v\textquoteright\
| \textquoteleft w\textquoteright\ | \textquoteleft x\textquoteright\ |
\textquoteleft y\textquoteright\ | \textquoteleft z\textquoteright\ .}

\item \texttt{digit = \textquoteleft 0\textquoteright\ | \textquoteleft
1\textquoteright\ | \textquoteleft 2\textquoteright\ | \textquoteleft
3\textquoteright\ | \textquoteleft 4\textquoteright\ | \textquoteleft
5\textquoteright\ | \textquoteleft 6\textquoteright\ | \textquoteleft
7\textquoteright\ | \textquoteleft 8\textquoteright\ | \textquoteleft
9\textquoteright\ .}

\end{enumerate}

\item Símbolos:

\begin{enumerate}[i.]

\item \texttt{special-symbol = \textquoteleft +\textquoteright\ | \textquoteleft
-\textquoteright\ | \textquoteleft *\textquoteright\ | \textquoteleft
/\textquoteright\ |\textquoteleft =\textquoteright\ | \textquoteleft
$<$\textquoteright\ | \textquoteleft $>$\textquoteright\ | \textquoteleft
[\textquoteright\ | \textquoteleft ]\textquoteright
                  | \textquoteleft .\textquoteright\ |  \textquoteleft
,\textquoteright\ | \textquoteleft :\textquoteright\ | \textquoteleft
;\textquoteright\ | '\^\textquoteright\ | \textquoteleft (\textquoteright\ |
\textquoteleft )\textquoteright
                  | \textquoteleft $< >$\textquoteright\ |\textquoteleft $<
=$\textquoteright\ | \textquoteleft $> =$\textquoteright\ | \textquoteleft
:=\textquoteright\ | \textquoteleft ..\textquoteright\ | word-symbol .}
               

\item \texttt{word-symbol = \textquoteleft and\textquoteright\ | \textquoteleft
array\textquoteright\ | \textquoteleft begin\textquoteright\ | \textquoteleft
case\textquoteright\ | \textquoteleft const\textquoteright\ | \textquoteleft
div\textquoteright
               |\textquoteleft do\textquoteright\ | \textquoteleft
downto\textquoteright\ | \textquoteleft else\textquoteright\ | \textquoteleft
end\textquoteright\ | \textquoteleft file\textquoteright\ | \textquoteleft
for\textquoteright
               |\textquoteleft function\textquoteright\ | \textquoteleft
goto\textquoteright\ | \textquoteleft if\textquoteright\ |\textquoteleft
in\textquoteright\ | \textquoteleft label\textquoteright\ | \textquoteleft
mod\textquoteright
               |\textquoteleft nil\textquoteright\ | \textquoteleft
no\textquoteright\ | \textquoteleft of\textquoteright\ | \textquoteleft
or\textquoteright\ | \textquoteleft packed\textquoteright\ | \textquoteleft
procedure\textquoteright
               | \textquoteleft program\textquoteright\ | \textquoteleft
record\textquoteright\ | \textquoteleft repeat\textquoteright\ | \textquoteleft
set\textquoteright\ | \textquoteleft then\textquoteright
               |\textquoteleft to\textquoteright\ | \textquoteleft
type\textquoteright\ | \textquoteleft until\textquoteright\ | \textquoteleft
var\textquoteright\ | \textquoteleft while\textquoteright\ | \textquoteleft
with\textquoteright\ .}
               
\end{enumerate}

\item Identificadores: \texttt{identifier = letter { letter | digit }.}

\item Directivas: \texttt{directive = letter { letter | digit }.}

\item Números:

\begin{enumerate}[i.]

\item \texttt{signed-number = signed-integer | signed-real .}

\item \texttt{signed-real = [ sign ] unsigned-real .}

\item \texttt{signed-integer = [ sign ] unsigned-integer .}

\item \texttt{unsigned-number = unsigned-integer | unsigned-real .}

\item \texttt{sign = \textquoteleft +\textquoteright\ | \textquoteleft
-\textquoteright\ .}

\item \texttt{unsigned-real = digit-sequence \textquoteleft .\textquoteright\
fractional-part [ \textquoteleft e\textquoteright\ scale-factor
                              | digit-sequence \textquoteleft e\textquoteright\
scale-factor .}
\item \texttt{unsigned-integer = digit-sequence }

\item \texttt{fractional-part = digit-sequence .}

\item \texttt{scale-factor = [ sign ] digit-sequence .}

\item \texttt{digit-sequence = digit { digit }}

\end{enumerate}

\item Etiquetas: \texttt{label = digit-sequence .}

\item Cadenas de caracteres:

\begin{enumerate}[i.]

\item \texttt{character-string = \textquoteleft \textquotesingle
\textquoteright\ string-element { string-element } \textquoteleft
\textquotesingle \textquoteright\ .}

\item \texttt{string-element = apostrophe-image | string-character .}

\item \texttt{apostrophe-image = \textquoteleft \textquotesingle\textquotesingle
\textquoteright\ .}

\item \texttt{string-character =
one-of-a-set-of-implementation-defined-characters .}

\end{enumerate}

\item Separadores: \texttt{( \textquoteleft \{\textquoteright\ | \textquoteleft
(*\textquoteright\ ) commentary ( \textquoteleft *)\textquoteright\ |
\textquoteleft \}\textquoteright\ )}

\end{enumerate}

\subsubsection{Tipos de Datos}

\begin{enumerate}[I.]

\item Datos Simples: Se trata de los tipos de datos base del propio lenguaje.

{

\begin{enumerate}[i.]
 
\item Entero (\index{Integer}Integer): Es un tipo de dato ordinal y representa en Conjunto de los Números Enteros.
{
\ejem \texttt{\{+5;-4,7\}}
}
\item Reales (\index{Real}Real): Representa el Conjunto de los Números Reales. Siendo los propios Reales de naturaleza matemática infinita, la precisión del número vendrá dada por la el tamaño del bloque en memoria asignado al dato.
{
\ejem \texttt{\{+5.5;-4.2;7.5\}}
}
\item Booleano (\index{Boolean}Boolean): Se trata de un conjunto ordinal de datos con dos posibles valores.
{
\form \texttt{\{TRUE; FALSE\}} $\equiv \{0,1\}$
}
\item Carácter (\index{Char}Char): Depende de la naturaleza del valor, puede ser o no un tipo de dato ordinal. Se establece como un Subconjunto de:
{
\begin{enumerate}

\item El Conjunto de valores para dígito: $[0,9]$

\item El Conjunto de valores para letra: $[a-z,A-Z]$
\end{enumerate}
}
\end{enumerate}

}


\item Datos Estructurados: Se trata de estructuras de datos complejas, definidas a partir de datos simples. {

\begin{enumerate}[i.]

\item \index{Tipo Enumerado}Tipo Enumerado: {Se trata de una lista de datos y valores hermética. Constituye un tipo ordinario finito y explícito en su declaración.
a su declaración.

\ejem \texttt{(red, green, blue, yellow)}
}

\item \index{Tipo Subrango}Tipo Subrango: { Se trata de una subconjunto enumerado a partir otro meta-conjunto dónde sus valores son implícitos 

\ejem \texttt{\{1..80; -15..+15\}}
}

\item \index{Tipo Array}Tipo Array: {Array se trata de una estructura de datos indexada con espacio en memoria estático. El ``tipado'' de la estructura Array viene determinado por la naturaleza de cada uno de sus datos (siendo estos obligatoriamente de la misma naturaleza).

\ejem \texttt{\{ARRAY [1..1000] OF REAL; ARRAY [1..100] OF INTEGER\}}
}

\item \index{Tipo Registro}Tipo Registro: {


\lstinputlisting[language=Pascal]{
./programms/programms13/Pascal/template/typeRecordBlock.txt}

\ejem Registro: 
{
\lstinputlisting[language=Pascal]{
./programms/programms13/Pascal/template/exampTypeRecord.txt}
}

}

\item \index{Tipo Conjunto}Tipo Conjunto: {Estructura heredada de la Teoría de Conjuntos (ver Apartado \ref{setsTheory}). Su declaración viene dada por: \texttt{SET OF 'base-type'}

\ejem \texttt{\{SET OF CHAR; SET OF (red, green, blue, yellow)\}}
}

\item \index{Tipo Fichero}Tipo Fichero: {Se trata de una estructura de ``flujo'' de valores. El tamaño de la misma viene dado por el conjunto de datos que serán leídos o escritos.

\ejem \texttt{\{FILE OF REAL; FILE OF INTEGER\}}
}

\item \index{Tipo Puntero}Tipo Puntero: {La estructura de Puntero tiene dos posibles valores:

\begin{enumerate}[a.]
\item \texttt{Null}: Constituye el índice natural para las estructuras en memoria dinámica.
\item Identificador de Valores: Se trata de un valor con naturaleza en memoria dinámico. A medida que se constituye como valor se reserva el espacio del tipo base. Dicho proceso se realiza a través del procedimiento \texttt{NEW()}. De igual manera, para ``liberar'' los recursos en memoria es necesario utilizar el procedimiento \texttt{DISPOSE()}.
\end{enumerate}
\ejem Puntero:
{
\lstinputlisting[language=Pascal]{
./programms/programms13/Pascal/template/exampTypePointer.txt}
}


}

\end{enumerate}

}

\end{enumerate}


% Tipos de Datos$
% 	\begin{cases}
% 	\T{- Simple (Scalar)}
% 	\begin{cases}
% 	\T{- Real}\\	
% 	\T{- Ordinal}
%         \begin{cases}
% 	\T{- Estandard}
% 	\begin{cases}
% 	\T{- Byte}\\
% 	\T{- Integer}\\
% 	\T{- Char}\\
% 	\T{- Boolean}
% 	\end{cases}
% \\
% 	\T{- User Defined}
% 	\begin{cases}
% 	\T{- Enumerated}\\
% 	\T{- Subrange}
% 	\end{cases}
% 	\end{cases}
% 	\end{cases}
% \\
% 	\T{- String}\\
%         \T{- Estructured}
% \begin{cases}
% 	\T{- Set}\\
% 	\T{- Array}\\
% 	\T{- Record}\\
% 	\T{- File}
% 	\begin{cases}
% 	\T{- Text}\\
% 	\T{- Binary}
% 	\end{cases}
% \\
% 	\T{- Object}
% 	\end{cases}
% \\
%         \T{- Pointer}
% 	\end{cases}
% $

\subsubsection{Biblioteca}\label{sec:pascalISOLibrary}

\begin{enumerate}[I.]
\item \index{Procedimientos}Procedimientos:

\begin{enumerate}[i.]
\item \texttt{PROCEDURE REWRITE(\textit{f})} $\rightarrow$ Crea un fichero en
modo escritura. En caso de existir el propio fichero es sobreescrito.

\begin{enumerate}

\item Precondición: \texttt{True}

\item Postcondición: \texttt{(f$\uparrow$)} está indefinido.

\end{enumerate}

\item \texttt{PROCEDURE PUT(\textit{f})} $\rightarrow$ Añade el valor del buffer del \texttt{(f$\uparrow$)} al propio fichero.

\begin{enumerate}

\item Precondición: \texttt{(f$\uparrow$)} está definido.

\item Postcondición: \texttt{(f$\uparrow$)} está indefinido.

\end{enumerate}

\item \texttt{PROCEDURE RESET(\textit{f})} $\rightarrow$ Abre un fichero en modo
lectura con el puntero de fichero sobre el comienzo del mismo.

\begin{enumerate}

\item Precondición: \texttt{(f$\uparrow$)} está definido.

\item Postcondición: \texttt{(f$\uparrow$)} está indefinido.

\end{enumerate}

\item \texttt{PROCEDURE GET(\textit{f})} $\rightarrow$ Avanza de descriptor de fichero y asigna el valor del buffer de al \texttt{(f$\uparrow$)}.

\begin{enumerate}

\item Precondición: \texttt{(f$\uparrow$)} está definido.

\item Postcondición: \texttt{(f$\uparrow$)} está indefinido.

\end{enumerate}

\item \texttt{PROCEDURE READ(\textit{f})} $\rightarrow$ Se encarga de leer el
fichero (\texttt{var F: tipodeFichero}) y de asignar sus datos al conjunto de
variables (\texttt{lista de
variables}).

%\textit{idem} respecto a la operación a realizar con el procedimiento anterior,
con la salvedad de que el (\texttt{var F: tipodeFichero}) es opcional, y en el
caso de no especificarse explícitamente como parámetro se lee el fichero por
defecto \texttt{input}.

\paragraph*{Nota:} \texttt{(f)} es equivalente: \texttt{begin read(ff,$v_1$);
begin read(ff,$v_2$, $\ldots$, $v_n$) end}

\item \texttt{PROCEDURE WRITE(\textit{f})} $\rightarrow$ Se encarga de escribir
el fichero (\texttt{var F: tipodeFichero}) y de escribir en el mismo los datos
de (\texttt{lista de
variables}).

% \textit{idem} respecto a la operación a realizar con el procedimiento
anterior, con la salvedad de que el (\texttt{var F: tipodeFichero}) es opcional,
y en el caso de no especificarse explícitamente como parámetro se lee el fichero
por defecto \texttt{input}.

\paragraph*{Nota:} \texttt{(f)} es equivalente: \texttt{begin write(ff,$v_1$);
begin write(ff,$v_2$, $\ldots$, $v_n$) end}

\item \texttt{PROCEDURE NEW(\textit{p})} $\rightarrow$ Reserva una variable $v$ en memoria y asigna el puntero de $v$ a \texttt{p}. El tipado de $v$ viene dado explícitamente en la declaración de \texttt{p}.
\item \texttt{PROCEDURE DISPOSE(\textit{p})} $\rightarrow$ Libera el registro de memoria $v$ asociado a \texttt{p}.

\end{enumerate}

\item Funciones:

\begin{enumerate}[i.]

\item \index{Funciones Aritméticas}Funciones Aritméticas:
\begin{enumerate}[a.]

\item \texttt{FUNCTION ABS(\textit{x})}{ $\rightarrow$ Se trata de un operador
genérico para tipos de datos Entero y Real que a partir del parámetro
\texttt(x:tipo) devuelve el valor absoluto de \texttt{x}.

\form \texttt{FUNCTION ABS(\textit{x:tipo}): tipo;} $\equiv |x|$
}
\item \texttt{FUNCTION SQR(\textit{x})}{ $\rightarrow$Para el parámetro
\texttt{x} de tipo INTEGER o REAL devuelve el valor de $x^2$.

\form \texttt{FUNCTION SQR(\textit{x:tipo}): tipo;}  $\equiv x^2$
}
\item \texttt{FUNCTION SIN(\textit{x})}{ $\rightarrow$ Para el tipo de datos
REAL, devuelve el valor del seno del parámetro \texttt{x}.

\form \texttt{FUNCTION SIN(\textit{x:REAL}): REAL;}  $\equiv sen(x)$
}
\item \texttt{FUNCTION COS(\textit{x})}{ $\rightarrow$ Para el tipo de datos
REAL devuelve el coseno de \texttt{x}.

\form \texttt{FUNCTION COS(\textit{x:REAL}): REAL;} $\equiv cos(x)$
}
\item \texttt{FUNCTION EXP(\textit{x})}{ $\rightarrow$ Para el tipo de datos
REAL devuelve el valor de $e^x$, siendo \texttt{x} el parámetro.

\form \texttt{FUNCTION EXP(\textit{x:REAL}): REAL;}  $\equiv e^x$
}
\item \texttt{FUNCTION LN(\textit{x})}{ $\rightarrow$ Para el tipo de datos REAL
devuelve $Ln(x)$, siendo \texttt{x} el parámetro.

\form \texttt{FUNCTION LN(\textit{x:REAL}): REAL;}  $\equiv Ln(x)$
}
\item \texttt{FUNCTION SQRT(\textit{x})}{ $\rightarrow$ Para el parámetro
\texttt{x} de tipo REAL devuelve el valor de $\sqrt{x}$.

\form \texttt{FUNCTION SQRT(\textit{x:REAL}): REAL;} $\equiv \sqrt{x}$
}
\item \texttt{FUNCTION ARCTAN(\textit{x})}{ $\rightarrow$ Para el tipo de dato
Real devuelve el valor del arcotangente \texttt{x} en radianes.

\form \texttt{FUNCTION ARCTAN(\textit{x:REAL}): REAL;} $\equiv arctg(x)$
}
\end{enumerate}

\item \index{Funciones de Transferencia}Funciones de Transferencia:

\begin{enumerate}[a.]
\item \texttt{FUNCTION TRUNC(\textit{x})}{ $\rightarrow$ Para el tipo de dato
REAL, obtiene la parte entera del parámetro \texttt{x}.

\form \texttt{FUNCTION TRUNC(\textit{x:REAL}): LONGINT;} $\equiv TRUNC(a,b) = a$
}
\item \texttt{FUNCTION ROUND(\textit{x})} $\rightarrow$ Para el tipo de datos
REAL, redondea el parámetro \texttt{x} al valor entero mas próximo.

\end{enumerate}

\item \index{Funciones Ordinales}Funciones Ordinales:

\begin{enumerate}[a.]
\item \texttt{FUNCTION ORD(\textit{x})} $\rightarrow$ Para el tipo de datos
LONGINT devuelve \texttt{true} en caso de que el parámetro \texttt{x} sea par.
Siendo impar devuelve \texttt{false}.
\item \texttt{FUNCTION CHR(\textit{x})} $\rightarrow$ Para el tipo de datos BYTE
devuelve el carácter (ver tabla ASCII, Apéndice ...) del valor ordinal
\texttt{x}. 
\item \texttt{FUNCTION SUCC(\textit{x})} $\rightarrow$ Para un valor ordinal
devuelve el sucesor del parámetro \texttt{x}.
\item \texttt{FUNCTION PRED(\textit{x})}  $\rightarrow$ Para un valor ordinal
devuelve el predecesor del parámetro \texttt{x}.
\end{enumerate}

\item \index{Funciones Booleanas}Funciones Booleanas:

\begin{enumerate}[a.]
\item \texttt{FUNCTION ODD(\textit{x})}{ $\rightarrow$ Devuelve si el valor \textit{x} es par (TRUE) o IMPAR (FALSE).

\form \texttt{FUNCTION ODD(\textit{x}:INTEGER): BOOLEAN;} $\equiv (abs(x)\ mod\ 2 = 1)$
}

\item \texttt{FUNCTION EOF(\textit{f})} $\rightarrow$ Devuelve el valor
\texttt{true} en caso de que sea final de fichero (el puntero de
lectura/escritura se encuentra en el carácter de final de fichero). Caso
contrario \texttt{false}.  
\item \texttt{FUNCTION EOLN(\textit{f})} $\rightarrow$ Devuelve el valor
\texttt{true} en caso de que sea final de línea (el puntero de lectura/escritura
se encuentra en el carácter de final de línea). Caso contrario \texttt{false}. 
\end{enumerate}

\end{enumerate}

\end{enumerate}

\subsubsection{Estructura de un programa}

Un programa en Pascal se divide en tres partes bien diferenciadas:

\prog \texttt{Plantilla de programa en Pascal}

\lstinputlisting[language=Pascal]{
./programms/programms13/Pascal/template/genericProgramm2.txt}

\begin{enumerate}[I.]
{
\item \index{Program Heading}Program Heading (Cabecera del Programa): Se trata de una estructura clásica de Entrada/Salida del programa. Compuesto a su vez por:
{
\begin{enumerate}[i.]

\item \texttt{program name}: Nombre principal y raíz del programa. 

\item \texttt{(file variables)} Lista de parámetros para enviar/recibir flujos de datos hacia en entorno de ejecución.

\end{enumerate}
}
\item \index{Declaration Part}Declaration Part (Apartado de Declaraciones): El bloque de declaración de programa es la estructura central aplicada y repetida a su vez, en lo bloques PROCEDURE Y FUNCTION y se divide en los siguientes apartados:
{
\begin{enumerate}[i.]

\item \texttt{laber declaration} (sección de etiquetas).

\item \texttt{(constant declaration)} (sección de constantes).

\item \texttt{(type declaration)} (sección de estructuras de datos).

\item \texttt{(variable declaration)} (sección de variables).

\item \texttt{(PROCEDURE | FUNCTION declaration)} (Declaración de Procedimientos o Funciones).

\end{enumerate}
}

\item \index{Statement Part}Statement Part (Apartado de Sentencias): Se trata del bloque del programa principal que contiene el conjunto de sentencias y estructuras de control.

\lstinputlisting[language=Pascal]{
./programms/programms13/Pascal/template/programmBlock.txt}

}
\end{enumerate}

\prog \texttt{helloProgrammer.pas}

\lstinputlisting[language=bash]{
./programms/programms13/Pascal/helloProgrammer/helloProgrammer.pas}

\paragraph*{Notas sobre compilación:} Para compilar el archivo fuente
\texttt{helloProgrammer.pas} sobre GNU, usaremos el compilador \texttt{GNU
Pascal
Compiler}\endnote{\href{http://www.gnu-pascal.de/gpc/}{
http://www.gnu-pascal.de/gpc/}}. 

Las ordenes para compilarlo y ejecutarlo son las siguientes:

\begin{verbatim}
$ gpc -o helloProgrammer helloProgrammer.pas 
$ ./helloProgrammer 
Hello Programmer!
\end{verbatim}


\section{Evoluciones del Lenguaje Pascal}

\subsection{Modula/Modula-2}

\textit{Modula-2 grew out of a practical need for a general, efficiently
implementable
systems programming language for minicomputers. Its ancestors are Pascal and
Modula. From the latter it has inherited the name, the important module concept,
and a systematic, modem syntax, from Pascal most of the rest. This includes in
particular the data structures, i.e. arrays, records, variant records, sets, and
pointers. Structured statements include the familiar if, case, repeat, while,
for, and with statements. Their syntax is such that every structure ends with an
explicit termination symbol.}\endnote{Modula-2 en un lenguaje de programación
de propósito general,
eficientemente implementado para para minicomputadoras. Sus antecesores son
Pascal y Modula. 
Suscrito a su nombre está el concepto de módulo y el trato sistemático con una
sintaxis moderna, por ejemplo:
matrices, registros, registros variables, conjuntos y punteros. Sus estructuras
incluye los familiares: 
if, case, repeat, while y símbolos de terminación explicita.} \cite{article/modula/modula2}

\index{Modula}Modula/Modula-2 es un lenguaje multipropósito pensado originalmente para ser un
lenguaje mucho más eficiente que Pascal. \textbf{De alguna manera Wirth ha buscado
siempre la mejora en el rendimiento de todos sus lenguajes.}

Se presenta como un lenguaje basado en importantes conceptos de \index{Programación
Orientada a Objetos}Programación
Orientada a Objetos (POO), aunque no los implementa todos. La idea más destacada
de POO en Modula/Modula-2 es la de encapsulación. Como sabemos crea una
estructura de datos modular en base a unas propiedades y métodos (operaciones
sobre los datos). La idea de cápsula\endnote{El concepto de cápsula como unidad estructural se de debe a David Lorge Parnas (Canada, 10 de Febrero de 1941). 

David Lorge Parnas es una de las figuras más representativas de la Ingeniería del Software. Graduado en Ingeniería (especialidad en Electricidad) a lo largo de su carrera ha sido profesor en Universidades como: Universidad de Carolina del Norte (EEUU), Universidad de Victoria (Canada), Universidad de Limerick (República de Irlanda) entre otras. 

De todas sus aportaciones a las Ciencias de la Computación destaca su Diseño Modular donde establece el concepto de ``Cápsula de Datos'' como abstracción de un objeto en la vida real con una serie de propiedades y acciones.} deriva de que por sí las propiedades del
módulo nunca son accesibles directamente como ocurre en la programación
estructurada. Por contra, son los métodos (las acciones) las que dan acceso al
contenido de estas variables bien sea para su lectura, escritura o ambas.

Como decimos, este es el aspecto más destacado de Modula/Modula-2 ya que,
conceptos universales como la Herencia\index{Herencia} son inexistentes dentro del lenguaje.
Dado que su propósito era más general y profesional que Pascal, Modula/Modula-2
tuvo cierto auge en entornos profesionales en los años ochenta del siglo XX. Una
vez más sus ``limitaciones por definición'' lo han convertido en
un ``lenguaje para aprender a programar''.

\subsubsection{Símbolos y Gramática} 

\begin{enumerate}

\item Tokens:

\begin{verbatim}
   letter = 'a' | 'b' | 'c' | 'd' | 'e' | 'f' | 'g' | 'h' | 'i' | 'j'
          | 'k' | 'l' | 'm' | 'n' | 'o' | 'p' | 'q' | 'r' | 's' | 't'
          | 'u' | 'v' | 'w' | 'x' | 'y' | 'z' .

   digit = '0' | '1' | '2' | '3' | '4' | '5' | '6' | '7' | '8' | '9' .
\end{verbatim}

\item Símbolos especiales:

\begin{verbatim}
   special-symbol = '+' | '-' | '*' | '/' | '=' | '<' | '>' | '[' | ']'
                  | '.' | ',' | ':' | ';' | '^' | '(' | ')'
                  | '<>' | '<=' | '>=' | ':=' | '..' | word-symbol .

   word-symbol = 'and' | 'array' | 'begin' | 'case' | 'const' | 'div'
               |'do' | 'downto' | 'else' | 'end' | 'file' | 'for'
               |'function' | 'goto' | 'if' | 'in' | 'label' | 'mod'
               |'nil' | 'not' | 'of' | 'or' | 'packed' | 'procedure'
               |'program' | 'record' | 'repeat' | 'set' | 'then'
               |'to' | 'type' | 'until' | 'var' | 'while' | 'with' .
\end{verbatim}

\item Gramatica: (ver Anexo \ref{chap:grammars} sección 2.)

\end{enumerate}

\prog \texttt{helloProgrammer.mod}

\lstinputlisting[language=Modula-2]{
./programms/programms13/Modula/helloProgrammer.mod}

\paragraph*{Notas sobre compilación:} Para compilar el archivo fuente
\texttt{helloProgrammer.adb} sobre GNU, usaremos el compilador \index{GNU Modula-2}GNU Modula-2\endnote{\href{http://www.nongnu.org/gm2/homepage.html}{http://www.nongnu.org/gm2/homepage.html}}. 

Las ordenes para compilarlo y ejecutarlo son las siguientes:

\begin{verbatim}
$ gm2 -o helloProgrammer helloProgrammer.mod
$ ./helloProgrammer 
Hello Programmer!
\end{verbatim}

\subsection{Ada}

La idea conceptual del lenguaje de programación \index{Ada}Ada nace por los
enormes gastos que generaban compiladores, editores y otras herramientas de
sistemas embebidos en el Departamento de Defensa de EEUU. En 1974 da comienzo un
estudio que desvela el gran problema de tener un sistema de desarrollo muy
caótico dónde eran frecuentes aplicaciones sobre un lenguaje determinado para un
tipo de sistema concreto. Se llegó a la conclusión de que era necesaria una
estandarización de desarrollo (lo que incluía un nuevo lenguaje general para
estas ideas).


\textbf{En 1975 se elaboró un documento técnico (\texttt{Strawman}) con las primeras
especificaciones}. Dicho documento sufrió distintas modificaciones gracias a la
participación y comentarios de muchos desarrolladores. En el año 1976 se tenía
una versión muy robusta del lenguaje que querían a nivel conceptual. Se hizo
un concurso público para que distintas empresas dieran una forma Software a dichas
especificaciones. Finalmente y tras un duro proceso de selección en 1979 se
público ganadora la empresa \textit{CII Honeywell Bull}. Al mismo tiempo se dio
nombre al lenguaje que empezaba a ser una realidad. Se denomino Ada en
honor a la primera programadora de la historia, \index{Augusta Ada King}Augusta Ada King (Ada
Lovelace)\endnote{Augusta Ada King (Londres 10 de Diciembre de 1815, Londres, 27 de Noviembre 1852) es considerada la primera programadora de máquinas de la historia. 

Hija del famosos poeta George Byron, su formación giró en torno a las Matemáticas y la Lógica. A pesar de ello, tuvo igualmente un interés notorio en las disciplinas humanísticas. Entro en el mundo de la programación gracias a su compañero y amigo Charles Babbage (creador de la Máquina Analítica Babbage). 

En su obra \textit{Notas} describe la Máquina Analítica y desarrolla un conjunto de instrucciones para realizar cálculos. Igualmente estableció las tarjetas perforadas como método para almacenar datos.

Anotar finalmente, que la Máquina Analítica de Charles Babbage nunca llegó a construirse por lo que Ada realizó todo su trabajo desde un punto de vista formal y lógico.},
asistente y mecenas de \index{Charles Babbage}Charles Babbage a su vez, creador de la primera máquina
analítica.  


En 1980 se publica la versión definitiva del lenguaje y es propuesta para su
estandarización en ANSI. \textbf{En 1893 se tuvo la primera versión de Ada
estándar (\texttt{ANSI/MIL-STD 1815A}) conocido como Ada 83}. El modelo
se perfecciono y por fin fue publicado por ISO (\texttt{ISO-8652:1987}).


El mismo Ada 83 desde el principio tuvo deficiencias prácticas por lo que se
trabajo en una nueva versión (\texttt{Ada 9X}) que entre otras ideas,
incorporaba el mecanismo de herencia. La nueva versión se llamó Ada 95
(\texttt{ISO-8652:1995}) y es uno de los estándares más usados hoy en día.

Ya en el año 2012, en el Congreso Ada-Europe celebrado en Stockholm, los
organismos: Ada Resource Association (ARA) y Ada-Europe anunciaron el
diseño de una nueva versión pendiente actualmente de aprobación por parte de
ISO/IEC.

\begin{figure}[h]
\begin{center}
\begin{pspicture}(9,3)%\psgrid
\psline[linecolor=black,linewidth=1pt]{->}(0,0.5)(9,0.5)
\rput(1.5,1){\texttt{Ada Standar}}
\rput(1.5,0){1983}
\psline[linecolor=black,linewidth=0.8pt]{-}(1.5,0.3)(1.5,0.7)

\rput(4.5,1){\texttt{Ada 83}}
\rput(4.5,0){1985}
\psline[linecolor=black,linewidth=0.8pt]{-}(4.5,0.3)(4.5,0.7)

\rput(7.5,1){\texttt{Ada 95}}
\rput(7.5,0){1995}
\psline[linecolor=black,linewidth=0.8pt]{-}(7.5,0.3)(7.5,0.7)

\end{pspicture}
\caption{Evolución del Lenguaje Ada.}
\end{center}
\end{figure}



El lenguaje de programación Ada destaca sobre otros por lo siguiente:

\begin{enumerate}[i.]

\item Legibilidad: Ada en mayor medida que Pascal insiste en que los programas
deben ser legibles(cualquier programador puede ser capaz de comprender el código
fuente de un programa). Por ejemplo, un valor en aritmética flotante es expresado como: \texttt{(PART INT).(PART DECIMAL)}.

\item Tipificación fuerte: Es necesario y sabido que cualquier estado de datos
(variable o propiedad) debe estar perfectamente tipificado es decir, desde el
principio se debe aclarar a que familia de datos pertenece 

\item Programación en gran escala: Es parte del concepto de ADA es ser un
lenguaje que atienda a necesidades de computación masiva. Por ello se utilizan
técnicas de encapsuláción, unidades lógicas, parametrización de procesos con el
objetivo final de que el programa sea siempre un conjunto de partes que se puedan
auditar individualmente.

\item Manejo de excepciones: Las excepciones son imprescindibles para este tipo de lenguajes que puesto que trabajan muy cercanos al
Hardware. ADA es un lenguaje ampliamente usado aplicaciones de alto rendimiento y es
por esto, que hace muy necesario conocer como se comporta el programa en tiempo de 
ejecución.

\item Unidades genéricas: Lo hemos comentado anteriormente y es que, la lógica
de programación nos dice \index{Divide and Conquer}Divide and Conquer (D\&C)\endnote{Divide y
Vencerás.} Tiende siempre a unidades lógicas (cápsulas) y funcionales. Con
esto ADA siempre ha pretendido que los errores se corrijan desde lo más básico y
procurando que afecte lo menos posible al resto de módulos.

\end{enumerate}

\prog \texttt{helloProgrammer.adb}

\lstinputlisting[language=Ada]{./programms/programms13/Ada/helloprogrammer.adb}

\paragraph*{Notas sobre compilación:} Para compilar el archivo fuente
\texttt{helloProgrammer.adb} sobre GNU, usaremos el compilador GNAT (\index{GNU NYU Ada
Translator}GNU NYU Ada
Translator)\endnote{\href{http://www.adacore.com/}{http://www.adacore.com/}}. 

Las ordenes para compilarlo y ejecutarlo son las siguientes:

\begin{verbatim}
$ gnatmake helloprogrammer.adb 
$ ./helloprogrammer
Hello programmer!\begin{verbatim}

\end{verbatim}

\subsection{Oberon}

\textit{Oberon is a general-purpose programming language that evolved from
Modula-2. Its principal new
feature is the concept of type extension. It permits the construction of new
data types on the basis
of existing ones and to relate them.}\endnote{Oberon is un lenguaje de 
propósito general basado en Modula-2. Es principalmente nueva la
característica de extensión de tipo.- Esto permite
la construcción de nuevos tipos de datos basados en otros existentes. } \cite{book/oberon/tplo}

\index{Oberon}Oberon es un lenguaje de programación orientado a objetos y procedimental creado por Niklaus Wirth (autor también de Pascal, Modula y Modula-2) y sus colaboradores del ETHZ (Suiza).

Oberon puede considerarse una evolución de Modula-2 con un soporte completo de orientación a objetos. De este lenguaje y de sus antecesores hereda buena parte de la sintaxis y de la filosofía. Wirth siempre ha intentado simplificar los lenguajes sin que por ello se pierda en potencia. También está diseñado con la seguridad en mente: tiene chequeos de rango en arrays, recolector de basura y es fuertemente tipado. Sin embargo, por su intento de simplicidad carece de enumeraciones y enteros restringidos en rango, los cuales pueden implementarse como objetos.

La sintaxis de orientación a objetos de Oberon no se parece a la de otros lenguajes más populares como C++ o Java, pero sí guarda similitud con la de Ada 95.

Oberon es también el nombre de un sistema operativo,escrito con y para este lenguaje. Oberon se ha portado a otros sistemas (incluyendo a Windows y sistemas Unix) e incluso se puede compilar en \index{bytecodes}bytecodes para la máquina virtual de \index{Java}Java. También existe un proyecto para crear un compilador para la plataforma .NET.

\subsubsection{Símbolos y Gramática} 

\begin{enumerate}

\item Tokens:

\item Símbolos especiales:

\item Gramática: (ver Anexo \ref{chap:grammars} sección 3.)

\end{enumerate}

\prog \texttt{helloProgrammer.mod}

\lstinputlisting[language=Ada]{./programms/programms13/Oberon/helloProgrammer.mod}

\paragraph*{Notas sobre compilación:} Para compilar el archivo fuente
\texttt{helloProgrammer.adb} sobre GNU, usaremos el compilador \texttt{Oberon for GNU/Linux}\endnote{\href{http://olymp.idle.at/tanis/oberon.linux.html}{http://olymp.idle.at/tanis/oberon.linux.html}}. 

Las ordenes para compilarlo y ejecutarlo son las siguientes:

\begin{verbatim}
$  helloprogrammer.mod 
$ ./helloprogrammer
Hello programmer!
\end{verbatim}