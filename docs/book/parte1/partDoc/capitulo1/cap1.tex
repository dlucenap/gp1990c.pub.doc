%%%%%%%%%%%%%%
% Chapter 1: Formalismos
%%%%%%%%%%%%%%

\label{chap:cap1}

\section{Introducción a la Teoría de Conjuntos}\label{setsTheory}

\subsection{Definiciones}

\defn Se conoce por Conjunto\index{Conjunto} \endnote{La Teoría de Conjuntos se 
trata de la síntesis de siglos de trabajo con el objetivo de llegar a una 
descripción formal de un grupo o elementos relacionados. La figura que 
finalmente dio forma a estos grupos de elementos es Georg Ferdinand Ludwig 
Philipp Cantor, nacido el 3 de Marzo de 1845 en San Petersburgo (Rusia) y 
fallecido el 6 de Enero de 1918 en Halle, Alemania.}
a una estructura finita de elementos que guardan
una relación entre si.

\defn Los elementos que componen un conjunto reciben también el nombre de
\textbf{objetos}.

{\cor Existen dos métodos para describir un conjunto: conjuntos por extensión y 
conjuntos por compresión.}

\begin{enumerate}[i.]
\item {Conjunto por Extensión\index{Conjunto por Extensión}: Se dice que un conjunto está 
descrito por extensión cuando
todos los elementos que lo componen se puede enumerar. Normalmente se denotan 
los elementos
entre corchetes:

\ejem El conjunto $O$, contiene los dígitos de 0 a 9:

\begin{equation}
O = \{0,1,2,3,4,5,6,7,8,9\}
\end{equation}

}

\item {Conjunto por Compresión\index{Conjunto por Compresión}: Se dice que un conjunto está 
descrito por compresión
cuando sus elementos se describen a través de una propiedad.

\ejem El conjunto $P$, contiene los número pares de 0 a 9:

\begin{equation}
P = \{x\ \diagup\ (x \% 2 = 0) \wedge (x >=0\ \&\ x <10)\}
\end{equation}

}
\end{enumerate}

\defn Dos conjuntos son iguales si y sólo si contienen los mismos elementos
(incluyendo los repetidos).

\ejem Son iguales los siguientes conjuntos:

\begin{equation}
A = B \Rightarrow \{0,1,2,3,4,5\} = \{0,0,4,4,4,5,3,3,2,1,1,1\}
\end{equation}

\defn Se dice que un conjunto $O$ es un subconjunto de $P$, si todos los
elementos de $O$ forman parte de $P$.

\ejem Para los conjuntos: $O = \{1,2,3,5,7,9\}$ y $P = \{0,1,2,3,4,5,6,7,8,9\}$ 
decimos:

\begin{equation}
O \subseteq P
\end{equation}
 
\defn Se denomina \textbf{\index{Conjunto Universal}Conjunto Universal} al conjunto origen a partir
del cual derivan otros conjuntos. Se denota como
$U$.

\ejem El conjunto universal $U$ contiene a todos los números naturales:

\begin{equation}
U = \{1,2,\ldots,n,\ldots\}
\end{equation}

Luego $P$ es subconjunto de $U$ porque $P$ se describe como el \textbf{conjunto 
de los
números naturales primos}:

\begin{equation}
P = \{1,2,3,5,\ldots,n,\ldots\} \subseteq U
\end{equation}

\defn Se denomina \textbf{\index{Conjunto Vacío}Conjunto Vacío} al conjunto que no contiene ningún 
elemento. Se
denota como: $\varnothing$.

\ejem Se puede decir formalmente que el conjunto vacío:

\begin{equation}
\varnothing \equiv \{\ \} \equiv \{ \varnothing \}
\end{equation}

\subsection{Operaciones}

\begin{enumerate}[I.]
\item Unión:
{
\defn Dados los conjuntos $O$ y $P$ se tiene por \textbf{\index{Unión}Unión} de ambos 
(denotado
mediante el signo $\cup$) $O \cup P$, a otro conjunto que contiene los elementos
de: $O$ y $P$ y ambos.

\ejem Sea $O = \{v,o,c,a,l,e,s\}$ y $ P = \{a,e,i,o,u\}$. Se tiene:

\begin{equation}
O \cup P = \{a,e,i,o,u,v,c,l,s\}
\end{equation}

\paragraph*{Propiedades:}

\begin{enumerate}[i.]
\item Propiedad Conmutativa:

\begin{equation}
O \cup P \equiv P \cup O = \{a,e,i,o,u,v,c,l,s\}
\end{equation}
 
\item Propiedad Asociativa: Para $Q = \{a,b,c,d\}$

\begin{equation}
(O \cup P) \cup Q \equiv O \cup (P \cup Q) = \{a,e,i,o,u,v,c,l,s,b,d\}
\end{equation}

\item Propiedad de Absorción:

\begin{equation}
O \cup U = U
\end{equation}

\item Propiedad de Idempotencia:

\begin{equation}
O \cup O \equiv O = \{v,o,c,a,l,e,s\}
\end{equation}

\item Propiedad de Neutralidad:

\begin{equation}
O \cup \varnothing \equiv O = \{v,o,c,a,l,e,s\}
\end{equation}

\end{enumerate}

\begin{figure}[h]
\begin{center}
\begin{pspicture}(-5,-3)(5,3)%\psgrid
\psset{unit=2cm}




\pscircle[fillstyle=hlines,fillcolor=white](-0.5,0){1}
\pscircle[fillstyle=hlines,fillcolor=white](0.5,0){1}



\pspolygon(-2,-1.5)(2,-1.5)(2,1.5)(-2,1.5)

\pspolygon[fillstyle=solid,fillcolor=white](-1,-0.2)(-1,0.2)(-0.6,0.2)(-0.6,-0.2)
\rput(-0.8,0){$A$}

\pspolygon[fillstyle=solid,fillcolor=white](0.6,-0.2)(0.6,0.2)(1,0.2)(1,-0.2)
\rput(0.8,0){$B$}

\rput(2.2,1.4){$U$}


\end{pspicture}
\caption{Relación de Unión.}
\end{center}
\end{figure}
}
\item Intersección:{
\defn Dados los conjuntos $O$ y $P$ se tiene por \textbf{\index{Intersección}Intersección} de ambos 
(denotado
mediante el signo $\cap$) $O \cap P$, a otro conjunto que contiene los elementos
comunes a $O$ y $P$.

\ejem Sea $O = \{v,o,c,a,l,e,s\}$ y $ P = \{a,e,i,o,u\}$. Se tiene:

\begin{equation}
O \cap P = \{o,a,e\}
\end{equation}

\paragraph*{Propiedades:}

\begin{enumerate}[i]
\item Propiedad Conmutativa:

\begin{equation}
O \cap P \equiv P \cap O = \{o,a,e\}
\end{equation}
 
\item Propiedad Asociativa: $Q = \{a,b,c,d\}$

\begin{equation}
(O \cap P) \cap Q \equiv O \cap (P \cap Q) = \{a\}
\end{equation}

\item Propiedad de Absorción:

\begin{equation}
\varnothing \cap O \equiv \varnothing = \varnothing
\end{equation}

\item Propiedad de Idempotencia:

\begin{equation}
O \cap O \equiv O = \{v,o,c,a,l,e,s\}
\end{equation}

\item Propiedad de Neutralidad:

\begin{equation}
O \cap U \equiv O = \{v,o,c,a,l,e,s\}
\end{equation}

\end{enumerate}

\begin{figure}[h]
\begin{center}
\begin{pspicture}(-5,-3)(5,3)%\psgrid
\psset{unit=2cm}

\pscircle[fillstyle=solid,fillcolor=white](0.5,0){1}
\pscircle[fillstyle=solid,fillcolor=white](-0.5,0){1}
\pscurve[fillstyle=hlines,fillcolor=white]{-}(0,0.85)(0.2,0.7)(0.4,0.4)(0.48,0)(0.4,-0.4)(0.2,-0.7)(0,-0.85)
\pscurve[fillstyle=hlines,fillcolor=white]{-}(0,0.85)(-0.2,0.7)(-0.4,0.4)(-0.48,0)(-0.4,-0.4)(-0.2,-0.7)(0,-0.85)

\pspolygon(-2,-1.5)(2,-1.5)(2,1.5)(-2,1.5)

%\pspolygon[fillstyle=solid,fillcolor=white](-0.9,-0.1)(-0.9,0.1)(-0.7,0.1)(-0.7,-0.1)
\rput(-0.8,0){$A$}

%\pspolygon[fillstyle=solid,fillcolor=white](0.7,-0.1)(0.7,0.1)(0.9,0.1)(0.9,-0.1)
\rput(0.8,0){$B$}

\rput(2.2,1.4){$U$}



\end{pspicture}
\caption{Relación de Intersección entre dos conjutos genéricos.}
\end{center}
\end{figure}






}\item Leyes de De Morgan: Nos permiten establecer equivalencias entre los 
operadores antes vistos (Unión e Intersección){

\defn Primera Ley:

\begin{equation}
\overline{O \cup P} = \bar{O} \cap \bar{P}
\end{equation}

\begin{equation}
\overline{O \cup P} = \{v,c,l,s\} \equiv \bar{O} \cap \bar{P} = \{v,c,l,s\}
\end{equation}


\defn Segunda Ley:

\begin{equation}
\overline{O \cap P} = \bar{O} \cup \bar{P}
\end{equation}

}\item Resta de Conjuntos:{

\defn Dados los conjuntos $O$ y $P$ se tiene por \textbf{\index{Resta}Resta} de ambos 
(denotado
mediante el signo $-$) $O - P$, a aquellos elementos de $O$ que no estén en $P$

\begin{equation}
O \cap P = \{v,c,l,s\}
\end{equation}


\begin{figure}[h]
\begin{center}
\begin{pspicture}(-5,-3)(5,3)%\psgrid
\psset{unit=2cm}

\pscircle[fillstyle=solid,fillcolor=white](0.5,0){1}
\pscircle[fillstyle=hlines,fillcolor=white](-0.5,0){1}
\pscurve[fillstyle=solid,fillcolor=white]{-}(0,0.85)(0.2,0.7)(0.4,0.4)(0.48,0)(0.4,-0.4)(0.2,-0.7)(0,-0.85)
\pscurve[fillstyle=solid,fillcolor=white]{-}(0,0.85)(-0.2,0.7)(-0.4,0.4)(-0.48,0)(-0.4,-0.4)(-0.2,-0.7)(0,-0.85)


\pspolygon(-2,-1.5)(2,-1.5)(2,1.5)(-2,1.5)

\pspolygon[fillstyle=solid,fillcolor=white](-1,-0.2)(-1,0.2)(-0.6,0.2)(-0.6,-0.2)
\rput(-0.8,0){$A$}

%\pspolygon[fillstyle=solid,fillcolor=white](0.7,-0.1)(0.7,0.1)(0.9,0.1)(0.9,-0.1)
\rput(0.8,0){$B$}

\rput(2.2,1.4){$U$}

\end{pspicture}
\caption{Relación de Resta entre dos conjuntos genéricos.}
\end{center}
\end{figure}






}\item Disjunción:{

\defn Dos conjuntos son \textbf{\index{Disjuntos}Disjuntos} cuando su intersección es vacía.

\ejem Dados los conjuntos: $P = \{a,e,i,o,u\}$ y $Q = \{0,1,2,3,4,5,6,7,8,9\}$

\begin{equation}
P \cap Q = \varnothing
\end{equation}

\begin{figure}[h]
\begin{center}
\begin{pspicture}(-7,-3)(7,3)%\psgrid

\psset{unit=2cm}


\pscircle[fillstyle=solid,fillcolor=white](1.5,0){1}
\pscircle[fillstyle=solid,fillcolor=white](-1.5,0){1}


\pspolygon(-3,-1.5)(3,-1.5)(3,1.5)(-3,1.5)

\rput[d](-1.5,0){$A$}
\rput[d](1.5,0){$B$}
\rput(3.2,1.4){$U$}

\end{pspicture}
\caption{Relación de Disjunción.}
\end{center}
\end{figure}






}\item Diferencia Simétrica: {

\defn Dados los conjuntos $O$ y $P$ se entiende por \textbf{\index{Diferencia Simétrica}Diferencia Simétrica},
denotado como $\oplus$, a todos los elementos que están en $O$ y no en $P$ u
todos los elementos que están en $P$ y no están en el conjunto $O$.

\form $O \oplus P = (O - P) \cup (P - O)$

\ejem Sea $O = \{a,b,c,d,e,f,g,i\}$ y $P = \{a,e,i,o,u\}$ se tiene:

\begin{equation}
O - P = \{b,c,d,f,g\} \wedge P - O = \{o,u\} \Rightarrow O \oplus P =
\{b,c,d,f,g,o,u\}
\end{equation}

\begin{figure}[h]
\begin{center}
\begin{pspicture}(-5,-3)(5,3)%\psgrid
\psset{unit=2cm}


\pscircle[fillstyle=hlines,fillcolor=white](0.5,0){1}
\pscircle[fillstyle=hlines,fillcolor=white](-0.5,0){1}
\pscurve[fillstyle=solid,fillcolor=white]{-}(0,0.85)(0.2,0.7)(0.4,0.4)(0.48,0)(0.4,-0.4)(0.2,-0.7)(0,-0.85)
\pscurve[fillstyle=solid,fillcolor=white]{-}(0,0.85)(-0.2,0.7)(-0.4,0.4)(-0.48,0)(-0.4,-0.4)(-0.2,-0.7)(0,-0.85)

\pspolygon(-2,-1.5)(2,-1.5)(2,1.5)(-2,1.5)

\pspolygon[fillstyle=solid,fillcolor=white](-1,-0.2)(-1,0.2)(-0.6,0.2)(-0.6,-0.2)
\rput(-0.8,0){$A$}

\pspolygon[fillstyle=solid,fillcolor=white](0.6,-0.2)(0.6,0.2)(1,0.2)(1,-0.2)
\rput(0.8,0){$B$}

\rput(2.2,1.4){$U$}


\end{pspicture}
\caption{Relación de Diferencia Simétrica entre dos conjutos genéricos.}
\end{center}
\end{figure}






}\item Complemento: {

\defn Dado el conjunto $P$ se tiene por \textbf{\index{Complementario}Complementario} (denotado
mediante el signo $\bar{P}$), a aquellos elementos de $U$ que no están en $P$.

\form $\bar{P} = U - P$

\ejem En nuestro caso siendo $U$ el alfabeto castellano y $P = \{a,e,i,o,u\}$ se tiene:

\begin{equation}
\bar{P} = \{b,c,d,\ldots,x,y,z\}
\end{equation}

\begin{figure}[h]
\begin{center}
\begin{pspicture}(-3.5,-3)(3.5,3)%\psgrid

\psset{unit=2cm}

\pspolygon[fillstyle=hlines,fillcolor=white](-1.5,-1.5)(1.5,-1.5)(1.5,1.5)(-1.5,1.5)
\pscircle[fillstyle=solid,fillcolor=white](0,0){1}

\rput(0,0){$A$}

\pspolygon[fillstyle=solid,fillcolor=white](1.1,1.1)(1.1,1.5)(1.5,1.5)(1.5,1.1)
\rput(1.3,1.3){$\overline{A}$}


\rput(1.7,1.4){$U$}

\end{pspicture}
\caption{Relación de Complemento.}
\end{center}
\end{figure}






}%Finish Operations of Subnets

\end{enumerate}

%%%%%%%%%%%%%%%%%%%%%%%%%%%%%%%%%%%%%

\section{Relaciones}

\subsection{Definiciones}

\defn Denominamos \textbf{\index{Par}Par} a \textbf{todo conjunto finito de dos elementos}:

\begin{equation}
P = (a,b)
\end{equation}

de modo que:

\begin{enumerate}[i.]

\item $a$ es la \textbf{primera coordenada} o primer elemento.

\item $b$ de manera análoga, es la \textbf{segunda coordenada} o segundo 
elemento.

\end{enumerate}

\defn Dos pares: $(a,b)$ y $(c,d)$ \textbf{son iguales} si:

\begin{equation}
a \div c \wedge b \div d
\end{equation}

\defn Un Par es \textbf{idéntico} si:

\begin{equation}
a \div b
\end{equation}

\defn El Par \textbf{recíproco} a $(a,b)$ es:

\begin{equation}
(b,a)
\end{equation}

\subsection{Producto Cartesiano}

\defn Formalmente diremos que el \textbf{\index{Producto Cartesiano}Producto Cartesiano} para $A,B$ es:

\begin{equation}
A \times B = \{(a,b)\ \diagup\ (a \in A)\ \wedge\ (b \in B)\}
\end{equation}

\ejem Para los conjuntos: $O=\{1,3,6\}$ y $P=\{2,4\}$ tenemos:\label{ejemOxP}

\begin{equation}
O \times P = \{(1,2),(1,4),(3,2),(3,4),(6,2),(6,4)\}
\end{equation}

\paragraph*{Propiedades:}

\begin{enumerate}[i.]

\item \textbf{No Conmutativo}: Dados los conjuntos: $R = (a)$ y $S = (b)$ 
tenemos:

\begin{equation}
R \times S = \{(a,b) \diagup (a \in R) \wedge (b \in S)\} 
\end{equation}

Por contra:

\begin{equation}
S \times R = \{(b,a) \diagup (b \in S) \wedge (a \in R)\} 
\end{equation}

\item \textbf{Asociativo}: Dados los conjuntos: $R = (a)$ , $S = (b)$ y $T = 
(c)$ tenemos:

\begin{equation}
R \times S \times T = (R \times S) \times T = R \times (S \times T) 
\end{equation}


\item \textbf{Distributivo}:

\begin{equation}
R \times (S \cap T) = (R \times S) \cap (R \times T) 
\end{equation}

\end{enumerate}

\subsection{Representaciones}\label{sec:procCart}
{
\begin{enumerate}[I.]

\item \textbf{Representación Mediante Tabla}: Para los conjuntos $O = (o_1, 
o_2, 
\ldots , o_m)$ y $P = (p_1, p_2, \ldots , p_n)$, cada elemento de $O$ sería el 
índice de cada columna y, de manera análoga cada elemento de $P$ constituiría 
el 
índice de una fila. \textbf{La intersección representa el Par resultado}.

\item \textbf{Representación Cartesiana}: Se representa mediante dos ejes. El 
eje horizontal corresponde al conjunto $O$ y, el eje vertical corresponde al 
conjunto $P$. \textbf{La intersección de ambos (un Punto) es un Par producto}.

\begin{figure}[h]
\centering
\begin{subtable}[A]{0.3\textwidth}
\centering
\begin{tabular}{|c|c|c|c|}\hline
$\varnothing$ & $O_1$ & $\ldots$ & $O_m$\\ \hline
\hline
$P_1$ & $O \times P_{11}$ & $\ldots$ & $O \times P_{1m}$ \\ \hline
$P_2$ & $O \times P_{21}$ & $\ldots$ & $O \times P_{2m}$ \\ \hline
$\vdots$ & $\ldots$ & $\ldots$ & $\ldots$ \\ \hline
$P_n$ & $O \times P_{n1}$& $\ldots$ & $O \times P_{nm}$ \\ \hline
\end{tabular}

\caption{Representación mediante tabla.}
\end{subtable}%
\quad
\begin{subfigure}[B]{0.3\textwidth}
\begin{pspicture}(-3,-3)(3,3)%\psgrid
%\psset{unit=0.29cm}
\psset{unit=0.15cm}
\psset{linecolor=black,linewidth=1pt,arrowsize=8pt}

% Axes
\psaxes[Dx=5,Dy=5]{<->}(0,0)(-14.5,-14.5)(14.5,14.5)
\psset{linewidth=.4pt}
\psaxes[Dx=1,Dy=1,labels=none,ticksize=1.5pt](0,0)(-13,-13)(13,13)
\uput{10pt}[0](0,14){\psscalebox{1.25}{\it eje-y}}
\uput{10pt}[70](14,0){\psscalebox{1.25}{\it eje-x}}

% Origin
\pnode(0,0){O}
\uput{35pt}[310](0,0){\rnode{Ot}{\psscalebox{1.1}{
        \vbox{\halign{\hfil#\hfil\cr$(0,0)$\cr\it origen\cr}}}}}
\nccurve[angleA=315,angleB=145,arrowsize=4pt,
        nodesepA=2pt,nodesepB=-3pt]{<-}{O}{Ot}

% Point
\psdots[dotstyle=*,dotscale=1.2](3,5)
\psline[linestyle=dotted,linewidth=.8pt](3,0)(3,5)(0,5)
\uput{2.5pt}[53](3,5){\psscalebox{1.1}{$P(3,5)$}}


% Quadrants
%\uput{3cm}[45](0,0){\psscalebox{1.6}{\bf I}}
%\uput{3cm}[135](0,0){\psscalebox{1.6}{\bf II}}
%\uput{3cm}[225](0,0){\psscalebox{1.6}{\bf III}}
%\uput{3cm}[315](0,0){\psscalebox{1.6}{\bf IV}}
\end{pspicture}
\caption{Representación mediante Sistema Cartesiano.}

\end{subfigure}

\caption{Representaciones genéricas del Producto Cartesiano.}

\end{figure}



\end{enumerate}

\paragraph*{Nota:} Para el Ejemplo (\ref{ejemOxP}):

\begin{figure}[h]
\centering
\begin{subfigure}[A]{0.3\textwidth}
\centering
\begin{tabular}{|c|c|c|c|}\hline
$\varnothing$ & $1$ & $3$ & $6$ \\ \hline
\hline
$2$ & $(1,2)$ & $(3,2)$ & $(6,2)$ \\ \hline
$4$ & $(1,4)$ & $(3,4)$ & $(6,4)$ \\ \hline
\end{tabular}
\caption{Representación Mediante Tabla.}

\end{subfigure}%
\quad
\begin{subfigure}[B]{0.3\textwidth}
\centering

\begin{pspicture}(-3,-3)(3,3)%\psgrid
%\psset{unit=0.29cm}
\psset{unit=0.15cm}
\psset{linecolor=black,linewidth=1pt,arrowsize=8pt}

% Axes
\psaxes[Dx=5,Dy=5]{<->}(0,0)(-14.5,-14.5)(14.5,14.5)
\psset{linewidth=.4pt}
\psaxes[Dx=1,Dy=1,labels=none,ticksize=1.5pt](0,0)(-13,-13)(13,13)
% \uput{10pt}[0](0,14){\psscalebox{1.25}{\it eje-y}}
% \uput{10pt}[70](14,0){\psscalebox{1.25}{\it eje-x}}

% Origin
%\pnode(0,0){O}
%\uput{35pt}[310](0,0){\rnode{Ot}{\psscalebox{1.1}{
%        \vbox{\halign{\hfil#\hfil\cr$(0,0)$\cr\it origen\cr}}}}}
%\nccurve[angleA=315,angleB=145,arrowsize=4pt,
%        nodesepA=2pt,nodesepB=-3pt]{<-}{O}{Ot}

% Point
%% (1,2)
\psdots[dotstyle=*,dotscale=1.2](1,2)
\psline[linestyle=dotted,linewidth=.8pt](1,0)(1,2)(0,2)
%\uput{2.5pt}[53](1,2){\psscalebox{1.1}{$(1,2)$}}

%% (1,4)
\psdots[dotstyle=*,dotscale=1.2](1,4)
\psline[linestyle=dotted,linewidth=.8pt](1,0)(1,4)(0,4)
%\uput{2.5pt}[53](1,2){\psscalebox{1.1}{$(1,2)$}}

%% (3,2)
\psdots[dotstyle=*,dotscale=1.2](3,2)
\psline[linestyle=dotted,linewidth=.8pt](3,0)(3,2)(0,2)
%\uput{2.5pt}[53](1,2){\psscalebox{1.1}{$(1,2)$}}

%% (3,4)
\psdots[dotstyle=*,dotscale=1.2](3,4)
\psline[linestyle=dotted,linewidth=.8pt](3,0)(3,4)(0,4)
%\uput{2.5pt}[53](1,2){\psscalebox{1.1}{$(1,2)$}}

%% (6,2)
\psdots[dotstyle=*,dotscale=1.2](6,2)
\psline[linestyle=dotted,linewidth=.8pt](6,0)(6,2)(0,2)
%\uput{2.5pt}[53](1,2){\psscalebox{1.1}{$(1,2)$}}

%% (6,4)
\psdots[dotstyle=*,dotscale=1.2](6,4)
\psline[linestyle=dotted,linewidth=.8pt](6,0)(6,4)(0,4)
%\uput{2.5pt}[53](1,2){\psscalebox{1.1}{$(1,2)$}}

% Quadrants
%\uput{3cm}[45](0,0){\psscalebox{1.6}{\bf I}}
%\uput{3cm}[135](0,0){\psscalebox{1.6}{\bf II}}
%\uput{3cm}[225](0,0){\psscalebox{1.6}{\bf III}}
%\uput{3cm}[315](0,0){\psscalebox{1.6}{\bf IV}}
\end{pspicture}

\caption{Representación mediante Sistema Cartesiano.}

\end{subfigure}

\caption{Representaciones del Producto Cartesiano; $O \times P$.}

\end{figure}



}%% Producto Cartesiano

\subsection{Tipos}

\begin{enumerate}[I.]

\item \textbf{Relaciones Binarias:}{

\defn Para dos conjuntos dados $A$ y $B$ y la relación $\Re$ decimos que: 
\textit{``La \index{Relación Binaria}Relación Binaria de $A$ hacia $B$ es de la forma''}:

\begin{equation}
\Re = \{(a,b) \diagup (((a,b) \in A \times B) \wedge (a\ \Re\ b))
\end{equation}

\ejem Para el conjunto: $O=\{1,2,3\}$; $o_i\ \Re\ o_j \Leftrightarrow o_i \cdot 
o_j$ \textit{es número par}. Por ello tenemos:\label{ejemoxopar}

\begin{equation}
o_i\ \Re\ o_j = \{(1,2),(2,1),(2,2),(3,2)\}
\end{equation}

\paragraph*{Representaciones de las Relaciones Binarias:}
{
\paragraph*{Nota:} Para el Ejemplo (\ref{ejemoxopar}):

\begin{enumerate}[i.]

\item Representación Cartesiana: Partiendo de la definición 
(\ref{sec:procCart}), \textbf{la intersección de los conjuntos es un Par de la 
Relación}.

\item Representación Sagital: Partiendo de la definición (\ref{sec:procCart}), 
\textbf{el punto de intersección es un Par de la Relación}.

\item Representación Matricial: Se trata de las transcripción directa de la 
\textbf{Representación Cartesiana} a Matriz donde, \textbf{cada $a_{ij}$ 
representa un Par de la Relación}.

\begin{figure}[h]

\begin{center}
\[
A=\left(
\begin{array}{l c c l}
a_{11} & a_{12} & \ldots & a_{1n}\\
a_{21} & a_{22} & \ldots & a_{2n}\\
\vdots & \vdots & \ddots & \vdots\\
a_{m1} & a_{m2} & \ldots & a_{mn}\\
\end{array}
\right)
\]

\caption{Representación genérica de una Relación Binaria mediante una Matriz.}

\end{center}
 
\end{figure}


\[
\begin{pmatrix}
0 & 1 & 0 \\
1 & 1 & 1 \\
0 & 0 & 0
\end{pmatrix}
\]



\end{enumerate}

}

}

\item \textbf{Relación Inversa:}{

\defn Definimos \textbf{\index{Relación Inversa}Relación Inversa} (denotada como $\Re^{-1}$) a aquella 
relación entre pares que establece:

\begin{equation}
\Re^{-1} = \{(a,b)|(b,a) \in \Re\}
\end{equation}

\ejem Para el Ejemplo (\ref{ejemoxopar}):

\begin{equation}
o_i\ \Re^{-1} o_j = \{(2,1),(1,2),(2,2),(2,3)\}
\end{equation}

}
\item \textbf{Relación Complementaria:}{

\defn Definimos \textbf{\index{Relación Complementaria}Relación Complementaria} (denotada como 
$\overline{\Re}$) a aquella relación entre pares que establece:


\begin{equation}
\forall\ a \in A, b \in B;\ a\ \overline{\Re}\ b \Leftrightarrow a\ \Re\ b 
\notin (A\times B)
\end{equation}

\ejem Para el Ejemplo (\ref{ejemoxopar}):

\begin{equation}
o_i\ \overline{\Re}\ o_j = \{(1,1),(1,3),(2,3),(3,1),(3,3)\}
\end{equation}

}
\item \textbf{Relaciones Transitivas:}{

\defn Definimos \textbf{\index{Relación Transitiva}Relación Transitiva} a aquella que cumple:

\begin{equation}
\forall\ a \in A, b \in B, c \in C; a\ \Re\ b \wedge b\ \Re\ c \Rightarrow a\ 
\Re\ b
\end{equation}

\ejem Para el Ejemplo (\ref{ejemoxopar}) y el conjunto $P = \{4,5,6\} 
\Rightarrow p_i\ \Re\ p_j$ \textit{es número par} =

\begin{equation}
o_i\ \Re\ p_j =
\end{equation}
}
\item \textbf{Relación Compuesta:}{

\defn Definimos \textbf{\index{Relación Compuesta}Relación Compuesta} a aquella relación (en nuestro 
caso, 
con tres conjuntos origen) que se establece $\forall a \in A, b \in B, c \in C; 
A \subseteq B \wedge\ a\ \Re\ b$ y $B\ \subseteq\ C\ \wedge b\ \Im\ c$:

\begin{equation}
\Re \circ \Im = \{(a,c) \diagup \exists\ b \in B \Leftrightarrow\ a\ \Re\ b 
\wedge\ b\ \Im\ c\}
\end{equation}

\ejem Para el Ejemplo (\ref{ejemoxopar}) y la relación en el conjunto $P$ $p_i\ 
\Im\ p_j \Leftrightarrow o_i + o_j \%\ 3 = 0$

\begin{equation}
\Re \circ \Im = \{(1,2),(2,1)\}
\end{equation}

}

\end{enumerate}

%%%%%%%%%%%%%%%%%%%%%%%%%%%%%%%%%%%%%

\section{Funciones}

\defn De manera somera podemos decir que una \textbf{\index{Función}Función} \endnote{\textbf{El concepto 
de Función como unidad estructural del Cálculo se debe} al intenso trabajo de: 
\textbf{René Descartes, Isaac Newton y Gottfried Leibniz} siendo este último, el 
estableció términos como: función, variable, constante y parámetro. \textbf{Gottfried 
Leibniz} nacido el 1 de Julio de 1646 en el Electorado de Sajonia y fallecido el 
14 de Noviembre de 1716 en Hannover, Electorado de Brunswick-Lüneburg, \textbf{fue un 
importante filósofo y matemático del siglo XVII padre del Cálculo Infinitesimal} 
(desde una perspectiva matemática junto a Isaac Newton (desde un principio 
físico). \textbf{Igualmente inventó el Sistema Binario} que actualmente es la lógica base 
de cualquier computadora digital.} es una regla que 
transforma un conjunto (\textbf{Conjunto Inicial} o \textit{Dominio}) 
\textbf{en otro 
nuevo conjunto} (\textbf{Conjunto Imagen} o \textit{Recorrido}). Si establecemos 
el 
Conjunto Origen como $D_1$ y el Conjunto Imagen como $R_1$ tenemos la relación:
\begin{equation}
f: D_1 \longrightarrow R_1
\end{equation}

\ejem Tenemos la función $f(x+5)$ y el Conjunto Origen $U = \{0,1,2,3,4,5\}$ 
por 
lo que:

\begin{equation}
f(O + 5) = \{5,6,7,8,9,10\}
\end{equation}

\begin{figure}[h]
\centering
\begin{subfigure}[A]{0.3\textwidth}
\centering
\begin{tabular}{c|c}
$Dominio$ & $Recorrido$ \\ \hline
\hline
0 & 5 \\ \hline
1 & 6 \\ \hline
2 & 7 \\ \hline
3 & 8 \\ \hline
4 & 9 \\ \hline
5 & 10 \\ \hline
\end{tabular} 
\caption{Representación mediante tabla.}

\end{subfigure}%
\quad
\begin{subfigure}[B]{0.3\textwidth}
\centering

% GNUPLOT: LaTeX picture using PSTRICKS macros
% Define new PST objects, if not already defined
\ifx\PSTloaded\undefined
\def\PSTloaded{t}
\psset{arrowsize=.01 3.2 1.4 .3}
\psset{dotsize=.01}
\catcode`@=11

\newpsobject{PST@Border}{psline}{linewidth=.0015,linestyle=solid}
\newpsobject{PST@Axes}{psline}{linewidth=.0015,linestyle=dotted,dotsep=.004}
\newpsobject{PST@Solid}{psline}{linewidth=.0015,linestyle=solid}
\newpsobject{PST@Dashed}{psline}{linewidth=.0015,linestyle=dashed,dash=.01 .01}
\newpsobject{PST@Dotted}{psline}{linewidth=.0025,linestyle=dotted,dotsep=.008}
\newpsobject{PST@LongDash}{psline}{linewidth=.0015,linestyle=dashed,dash=.02 .01}
\newpsobject{PST@Diamond}{psdots}{linewidth=.001,linestyle=solid,dotstyle=square,dotangle=45}
\newpsobject{PST@Filldiamond}{psdots}{linewidth=.001,linestyle=solid,dotstyle=square*,dotangle=45}
\newpsobject{PST@Cross}{psdots}{linewidth=.001,linestyle=solid,dotstyle=+,dotangle=45}
\newpsobject{PST@Plus}{psdots}{linewidth=.001,linestyle=solid,dotstyle=+}
\newpsobject{PST@Square}{psdots}{linewidth=.001,linestyle=solid,dotstyle=square}
\newpsobject{PST@Circle}{psdots}{linewidth=.001,linestyle=solid,dotstyle=o}
\newpsobject{PST@Triangle}{psdots}{linewidth=.001,linestyle=solid,dotstyle=triangle}
\newpsobject{PST@Pentagon}{psdots}{linewidth=.001,linestyle=solid,dotstyle=pentagon}
\newpsobject{PST@Fillsquare}{psdots}{linewidth=.001,linestyle=solid,dotstyle=square*}
\newpsobject{PST@Fillcircle}{psdots}{linewidth=.001,linestyle=solid,dotstyle=*}
\newpsobject{PST@Filltriangle}{psdots}{linewidth=.001,linestyle=solid,dotstyle=triangle*}
\newpsobject{PST@Fillpentagon}{psdots}{linewidth=.001,linestyle=solid,dotstyle=pentagon*}
\newpsobject{PST@Arrow}{psline}{linewidth=.001,linestyle=solid}
\catcode`@=12

\fi
\psset{unit=5.0in,xunit=5.0in,yunit=3.0in}
\pspicture(0.000000,0.000000)(0.500000,0.500000)
\ifx\nofigs\undefined
\catcode`@=11

\PST@Border(0.1259,0.0840)
(0.1409,0.0840)

\PST@Border(0.4140,0.0840)
(0.3990,0.0840)

\rput[r](0.1099,0.0840){ 0}
\PST@Border(0.1259,0.1608)
(0.1409,0.1608)

\PST@Border(0.4140,0.1608)
(0.3990,0.1608)

\rput[r](0.1099,0.1608){ 2}
\PST@Border(0.1259,0.2376)
(0.1409,0.2376)

\PST@Border(0.4140,0.2376)
(0.3990,0.2376)

\rput[r](0.1099,0.2376){ 4}
\PST@Border(0.1259,0.3144)
(0.1409,0.3144)

\PST@Border(0.4140,0.3144)
(0.3990,0.3144)

\rput[r](0.1099,0.3144){ 6}
\PST@Border(0.1259,0.3912)
(0.1409,0.3912)

\PST@Border(0.4140,0.3912)
(0.3990,0.3912)

\rput[r](0.1099,0.3912){ 8}
\PST@Border(0.1259,0.4680)
(0.1409,0.4680)

\PST@Border(0.4140,0.4680)
(0.3990,0.4680)

\rput[r](0.1099,0.4680){ 10}
\PST@Border(0.1547,0.0840)
(0.1547,0.1040)

\PST@Border(0.1547,0.4680)
(0.1547,0.4480)

\rput(0.1547,0.0420){-4}
\PST@Border(0.2123,0.0840)
(0.2123,0.1040)

\PST@Border(0.2123,0.4680)
(0.2123,0.4480)

\rput(0.2123,0.0420){-2}
\PST@Border(0.2700,0.0840)
(0.2700,0.1040)

\PST@Border(0.2700,0.4680)
(0.2700,0.4480)

\rput(0.2700,0.0420){ 0}
\PST@Border(0.3276,0.0840)
(0.3276,0.1040)

\PST@Border(0.3276,0.4680)
(0.3276,0.4480)

\rput(0.3276,0.0420){ 2}
\PST@Border(0.3852,0.0840)
(0.3852,0.1040)

\PST@Border(0.3852,0.4680)
(0.3852,0.4480)

\rput(0.3852,0.0420){ 4}
\PST@Axes(0.1259,0.0840)
(0.4140,0.0840)

\PST@Axes(0.2700,0.0840)
(0.2700,0.4680)

\PST@Solid(0.1259,0.0840)
(0.1259,0.0840)
(0.1288,0.0879)
(0.1317,0.0918)
(0.1346,0.0956)
(0.1375,0.0995)
(0.1405,0.1034)
(0.1434,0.1073)
(0.1463,0.1112)
(0.1492,0.1150)
(0.1521,0.1189)
(0.1550,0.1228)
(0.1579,0.1267)
(0.1608,0.1305)
(0.1637,0.1344)
(0.1666,0.1383)
(0.1696,0.1422)
(0.1725,0.1461)
(0.1754,0.1499)
(0.1783,0.1538)
(0.1812,0.1577)
(0.1841,0.1616)
(0.1870,0.1655)
(0.1899,0.1693)
(0.1928,0.1732)
(0.1957,0.1771)
(0.1987,0.1810)
(0.2016,0.1848)
(0.2045,0.1887)
(0.2074,0.1926)
(0.2103,0.1965)
(0.2132,0.2004)
(0.2161,0.2042)
(0.2190,0.2081)
(0.2219,0.2120)
(0.2248,0.2159)
(0.2278,0.2198)
(0.2307,0.2236)
(0.2336,0.2275)
(0.2365,0.2314)
(0.2394,0.2353)
(0.2423,0.2392)
(0.2452,0.2430)
(0.2481,0.2469)
(0.2510,0.2508)
(0.2539,0.2547)
(0.2569,0.2585)
(0.2598,0.2624)
(0.2627,0.2663)
(0.2656,0.2702)
(0.2685,0.2741)
(0.2714,0.2779)
(0.2743,0.2818)
(0.2772,0.2857)
(0.2801,0.2896)
(0.2830,0.2935)
(0.2860,0.2973)
(0.2889,0.3012)
(0.2918,0.3051)
(0.2947,0.3090)
(0.2976,0.3128)
(0.3005,0.3167)
(0.3034,0.3206)
(0.3063,0.3245)
(0.3092,0.3284)
(0.3121,0.3322)
(0.3151,0.3361)
(0.3180,0.3400)
(0.3209,0.3439)
(0.3238,0.3478)
(0.3267,0.3516)
(0.3296,0.3555)
(0.3325,0.3594)
(0.3354,0.3633)
(0.3383,0.3672)
(0.3412,0.3710)
(0.3442,0.3749)
(0.3471,0.3788)
(0.3500,0.3827)
(0.3529,0.3865)
(0.3558,0.3904)
(0.3587,0.3943)
(0.3616,0.3982)
(0.3645,0.4021)
(0.3674,0.4059)
(0.3703,0.4098)
(0.3733,0.4137)
(0.3762,0.4176)
(0.3791,0.4215)
(0.3820,0.4253)
(0.3849,0.4292)
(0.3878,0.4331)
(0.3907,0.4370)
(0.3936,0.4408)
(0.3965,0.4447)
(0.3994,0.4486)
(0.4024,0.4525)
(0.4053,0.4564)
(0.4082,0.4602)
(0.4111,0.4641)
(0.4140,0.4680)

\catcode`@=12
\fi
\endpspicture


\caption{Representación gráfica.}

\end{subfigure}

\caption{Representaciones para la función: $f(x+5)$.}

\end{figure}



\defn Formalmente \textbf{una función para una variable} (tomaremos $x$ por 
convención) que pertenece al conjunto $Dominio(x)$ le corresponden uno o varios 
valores en $y$ que a su vez pertenece al conjunto $Recorrido(x)$

\begin{equation}
y = f(x)
\end{equation}

\begin{figure}[h]
\begin{center}
\begin{pspicture}(-5,-4)(5,4)%\psgrid
%\psset{unit=2cm}

% \psset{unit=.7cm}\rput(-2,0){ \pscircle(-2,0){2}
% \psellipse[fillcolor=white, fillstyle=solid](6.5,0)(5,2.2)
% \pscircle[fillcolor=white, fillstyle=solid](6,0){2}

\pspolygon[fillstyle=solid,fillcolor=white](-4.5,-3.5)(4.5,-3.5)(4.5,3.5)(-4.5,3.5)

%[showpoints=true
\psccurve[fillstyle=solid,fillcolor=white](-3,-2.2)(-3.6,0.1)(-2.4,2.2)(-0.7,1.9)(0.2,0.8)(-1,-0.5)(-2,-1.8)
\psccurve[fillstyle=hlines,fillcolor=white](-0.8,-2.8)(-0.2,-1.1)(1,0.5)(0.8,2.4)(3.2,1.2)(3.4,-0.8)(1.5,-1.8)

\psccurve[fillstyle=solid,fillcolor=white](0.2,-1.8)(1.2,0.4)(1.6,1.8)(3,0.8)(3.2,-0.4)(2.4,-0.8)

\pspolygon[fillstyle=solid,fillcolor=white](-2.4,0.6)(-1.6,0.6)(-1.6,1.4)(-2.4,1.4)
\rput(-2,1){$X$}
\pspolygon[fillstyle=solid,fillcolor=white](1.6,-0.4)(2.4,-0.4)(2.4,0.4)(1.6,0.4)
\rput(2,0){$Y$}

\psdot*(-2,-1)
\psdot*(-1.8,-0.8)
\psdot*(-1.5,-0.5)
\psdot*(-1.3,-0.3)
\psdot*(-1,0)
\psdot*(-0.8,0.2)

\psdot*(0.4,-1.4)
\psdot*(0.8,-0.9)
\psdot*(1.2,-0.4)

\psdot*(0.8,2)
\psdot*(0.95,1.6)
\psdot*(1.1,1.2)

\psline[linecolor=black,linewidth=1pt]{->}(-2,-1)(0.4,-1.4)

\psline[linecolor=black,linewidth=1pt]{->}(-1.5,-0.5)(0.8,-0.9)
\psline[linecolor=black,linewidth=1pt]{->}(-1,0)(1.2,-0.4)

\rput(-3,2.8){Dominio}
\pscurve[linecolor=black,linewidth=1pt]{->}(-2.2,2.7)(-1.4,2.5)(-1,2)(-1,1.2)

\rput(2.2,-2.8){Recorrido}
\pscurve[linecolor=black,linewidth=1pt]{->}(3.1,-2.8)(3.4,0.3)(2.8,0.2)
\rput(-1.1,-1.6){$f(x)$}

\rput(4.8,3.4){$U$}



% \psline[linewidth=.7pt, labels=none,showpoints=true]{->}(-1.8,0)(5.9,0) 
% \psline[linewidth=.7pt,labels=none, showpoints=true]{->}(-1.8,.5)(5.9,.5)
% \psline[linewidth=.7pt, labels=none,showpoints=true]{->}(-1.8,-.5)(5.9,-.5)
% 
% \rput(-1.9,1){$X$}%
% \rput(6,1){$Y$}%
% \rput(0.8,1){$f(x)$}%
% 
% \rput(9.6,-0.2){$Imagen$}%
% 
% \psdots*[linewidth=.5pt](-2,0)(-2, .5)(-2, -.5)(6,0)(6,.5)(6,-.5)(8.3,1)(8.3,.5)}
\end{pspicture}
\caption{Relaciones entre los principales elementos de una función.}
\end{center}
\end{figure}



\subsection{Propiedades}

Para la relación: $f:D_1 \longrightarrow R_1$ tenemos la siguientes propiedades:
\begin{enumerate}[I.]
\item \textbf{Representación Gráfica}: el conjunto $D_1$ es un subconjunto del 
Producto Cartesiano $D_1 \times R_1$

\item \textbf{Imagen}: Establecemos que la Imagen de $X$ como $X^\prime$ por lo 
que:

\begin{equation}
X^\prime \subset X: f(X^\prime) = \{f(x^\prime)\ \diagup\ x^\prime \in X^\prime\}
\end{equation}

\item \textbf{Imagen Recíproca}: Es la función inversa del Conjunto Imagen es 
decir:

\begin{equation}
f^{-1}:y \in Y = \{x \in X\ \diagup\ f(x) = y\}
\end{equation}

% \item Imagen Recíproca: Establecemos que la Imagen Recíproca de $Y$ como 
% $Y\backprime$ por lo que:
% 
% \begin{equation}
% Y^\prime \subset Y: f^{-1}(Y^\prime) = \{f(x\in\ X\ \diagup\ f(x) \in Y^\prime\}
% \end{equation}

\item \textbf{Restricción} de $f$ sobre $U \subset X$:

\begin{equation}
f: U \longrightarrow Y\ \diagup\ \{u_i \in U, u_i \in X, y \in Y\}
\end{equation}

\end{enumerate}

\subsection{Tipos}

Se conocen tres tipos de funciones dada por la relación entre los valores del 
Conjunto Inicial y los valores del Conjunto Imagen: $f:X\longrightarrow Y$

\begin{enumerate}[I.]

\item \textbf{\index{Funciones Exhaustivas o Suprayectiva}Funciones Exhaustivas o Suprayectiva}: Una \textbf{Función es 
Exhaustiva} si para cada elemento de $X$ existe al menos un elemento en $Y$

\begin{equation}
\forall\ y\ \in\ Y\ \exists\ x\ \in\ X\ \diagup\ f(x) = y
\end{equation}

\item \textbf{\index{Funciones Inyectivas}Funciones Inyectivas}: Una \textbf{Función es Inyectiva} si para 
cada elemento de $Y$ existe como máximo un elemento en $X$

\begin{equation}
\forall\ x\ \in\ X\ \exists\ y\ \in\ Y\ \diagup\ f(x) = y
\end{equation}

\item \textbf{\index{Funciones Biyectivas}Funciones Biyectivas}: Una \textbf{Función es Biyectiva} si para 
cada elemento de $X$ existe un único elemento de $Y$

\begin{equation}
\exists\ x\ \in\ X,\ \exists\ y\ \in\ Y\ \diagup\ f(x) = y
\end{equation}

\begin{figure}[h]
\centering
\begin{subfigure}[A]{0.3\textwidth}
\centering
\begin{pspicture}(-2,-2)(3,2)%\psgrid
\psset{unit=1.2pc} \rput(-1,-.5){\rput(1.5, .8){
\pspolygon[fillstyle=solid,fillcolor=white](-4.5,-3.5)(4.5,-3.5)(4.5,3.5)(-4.5,3.5)

\psellipse(-2,0)(1.5,2.5)
\psellipse(2,0)(1.5,2.5)
\pspolygon[fillstyle=solid,fillcolor=white](-2.4,-0.4)(-1.6,-0.4)(-1.6,0.4)(-2.4,0.4)

\rput(-2,0){$X$}
\psdot*(-2,1)
\psdot*(-1,0)
\psdot*(-2,-1)

\pspolygon[fillstyle=solid,fillcolor=white](1.6,-0.4)(2.4,-0.4)(2.4,0.4)(1.6,0.4)
\rput(2,0){$Y$}
\psdot*(2,1)
\psdot*(2,-1)

\psline[linecolor=black,linewidth=1pt]{->}(-2,1)(2,1)
\psline[linecolor=black,linewidth=1pt]{->}(-1,0)(2,1)
\psline[linecolor=black,linewidth=1pt]{->}(-2,-1)(2,-1)

}
}
\end{pspicture}
\caption{Función Suprayectiva $\alpha$}

\end{subfigure}%
\quad
\begin{subfigure}[B]{0.3\textwidth}
\centering
\begin{pspicture}(-2,-2)(3,2)%\psgrid
\psset{unit=1.2pc} \rput(-1,-.5){\rput(1.5, .8){
\pspolygon[fillstyle=solid,fillcolor=white](-4.5,-3.5)(4.5,-3.5)(4.5,3.5)(-4.5,3.5)

\psellipse(-2,0)(1.5,2.5)
\psellipse(2,0)(1.5,2.5)
\pspolygon[fillstyle=solid,fillcolor=white](-2.4,-0.4)(-1.6,-0.4)(-1.6,0.4)(-2.4,0.4)

\rput(-2,0){$X$}
\psdot*(-2,1)

\psdot*(-2,-1)

\pspolygon[fillstyle=solid,fillcolor=white](1.6,-0.4)(2.4,-0.4)(2.4,0.4)(1.6,0.4)
\rput(2,0){$Y$}
\psdot*(2,1)
\psdot*(1,0)
\psdot*(2,-1)

\psline[linecolor=black,linewidth=1pt]{->}(-2,1)(2,1)

\psline[linecolor=black,linewidth=1pt]{->}(-2,-1)(2,-1)

}
}
\end{pspicture}
\caption{Función Inyectiva $\beta$}

\end{subfigure}%
\quad
\begin{subfigure}[C]{0.3\textwidth}
\centering
\begin{pspicture}(-2,-2)(3,2)%\psgrid
\psset{unit=1.2pc} \rput(-1,-.5){\rput(1.5, .8){
\pspolygon[fillstyle=solid,fillcolor=white](-4.5,-3.5)(4.5,-3.5)(4.5,3.5)(-4.5,3.5)

\psellipse(-2,0)(1.5,2.5)
\psellipse(2,0)(1.5,2.5)
\pspolygon[fillstyle=solid,fillcolor=white](-2.4,-0.4)(-1.6,-0.4)(-1.6,0.4)(-2.4,0.4)

\rput(-2,0){$X$}
\psdot*(-2,1)
\psdot*(-1,0)
\psdot*(-2,-1)

\pspolygon[fillstyle=solid,fillcolor=white](1.6,-0.4)(2.4,-0.4)(2.4,0.4)(1.6,0.4)
\rput(2,0){$Y$}
\psdot*(2,1)
\psdot*(1,0)
\psdot*(2,-1)

\psline[linecolor=black,linewidth=1pt]{->}(-2,1)(2,1)
\psline[linecolor=black,linewidth=1pt]{->}(-1,0)(1,0)
\psline[linecolor=black,linewidth=1pt]{->}(-2,-1)(2,-1)

}
}
\end{pspicture}

\caption{Función Biyectiva $\gamma$}

\end{subfigure}

\caption{Tipos de funciones basadas en la relación de Dominio y Recorrido.}

\end{figure}



\end{enumerate}

% Operaciones sobre Funciones

\subsection{Operaciones}

\paragraph*{Nota:} Usaremos las funciones genéricas: $F(x) = (f_1, f_2 x, 
\ldots, f_n x^{n-1})$ y $G(x) = (g_1, g_2 x , \ldots, g_n x^{n-1})$ con $\{n 
\in 
\mathbb{N}\}$. A modo de ejemplos tendremos las funciones: $U(x) = \frac{3x+5}{2}$ y 
$V(x) = 4x^{2} -1$.

\begin{enumerate}[I.]

{
\item \textbf{Suma}:

\begin{equation}
F(x) + G(x) = (f_1 + g_1, f_2 x + g_2 x. \ldots, f_n x^{n-1} + g_n x^{n-1})\ 
\diagup\ \{n \in \mathbb{N}\}
\end{equation}

\ejem

\begin{equation}
U(x) + V(x) = \frac{8 x^{2} + 3 x + 3}{2}
\end{equation}

\begin{figure}[h]
\centering
\begin{subfigure}[A]{0.3\textwidth}
\centering
\begin{tabular}{c|c}
$Dominio$ & $Recorrido$ \\ \hline
\hline
$0$ & $\frac{3}{2}$ \\ \hline
$1$ & $7$ \\ \hline
$\vdots$ & $\vdots$ \\ \hline
$n$& $\frac{8 n^{2} + 3 n + 3}{2}$ \\ \hline
\end{tabular} 
\caption{Representación mediante tabla.}

\end{subfigure}%
\quad
\begin{subfigure}[B]{0.3\textwidth}
\centering

\begin{figure}[h]
\centering
\begin{subfigure}[A]{0.3\textwidth}
\centering
\begin{tabular}{c|c}
$Dominio$ & $Recorrido$ \\ \hline
\hline
$0$ & $\frac{3}{2}$ \\ \hline
$1$ & $7$ \\ \hline
$\vdots$ & $\vdots$ \\ \hline
$n$& $\frac{8 n^{2} + 3 n + 3}{2}$ \\ \hline
\end{tabular} 
\caption{Representación mediante tabla.}

\end{subfigure}%
\quad
\begin{subfigure}[B]{0.3\textwidth}
\centering

\begin{figure}[h]
\centering
\begin{subfigure}[A]{0.3\textwidth}
\centering
\begin{tabular}{c|c}
$Dominio$ & $Recorrido$ \\ \hline
\hline
$0$ & $\frac{3}{2}$ \\ \hline
$1$ & $7$ \\ \hline
$\vdots$ & $\vdots$ \\ \hline
$n$& $\frac{8 n^{2} + 3 n + 3}{2}$ \\ \hline
\end{tabular} 
\caption{Representación mediante tabla.}

\end{subfigure}%
\quad
\begin{subfigure}[B]{0.3\textwidth}
\centering

\input{./gnuplot/gnuplot11/exampleSUMFUNCTIONS}

\caption{Representación gráfica.}

\end{subfigure}

\caption{Representaciones para la Función: $\frac{8 x^{2} + 3 x + 3}{2}$.}

\end{figure}



\caption{Representación gráfica.}

\end{subfigure}

\caption{Representaciones para la Función: $\frac{8 x^{2} + 3 x + 3}{2}$.}

\end{figure}



\caption{Representación gráfica.}

\end{subfigure}

\caption{Representaciones para la Función: $\frac{8 x^{2} + 3 x + 3}{2}$.}

\end{figure}


}

{
\item \textbf{Resta}:

\begin{equation}
F(x) - G(x) = (f_1 - g_1, f_2 x - g_2 x. \ldots, f_n x^{n-1} - g_n x^{n-1})\ 
\diagup\ \{n \in \mathbb{N}\}
\end{equation}

\ejem

\begin{equation}
U(x) - V(x) = \frac{-8 x^{2} + 3 x + 7}{2}
\end{equation}

\begin{figure}[h]
\centering
\begin{subfigure}[A]{0.3\textwidth}
\centering
\begin{tabular}{c|c}
$Dominio$ & $Recorrido$ \\ \hline
\hline
$0$ & $\frac{3}{2}$ \\ \hline
$1$ & $1$ \\ \hline
$\vdots$ & $\vdots$ \\ \hline
$n$& $\frac{-8 n^{2} + 3 n + 7}{2}$ \\ \hline
\end{tabular} 
\caption{Representación mediante tabla.}

\end{subfigure}%
\quad
\begin{subfigure}[B]{0.3\textwidth}
\centering

\begin{figure}[h]
\centering
\begin{subfigure}[A]{0.3\textwidth}
\centering
\begin{tabular}{c|c}
$Dominio$ & $Recorrido$ \\ \hline
\hline
$0$ & $\frac{3}{2}$ \\ \hline
$1$ & $1$ \\ \hline
$\vdots$ & $\vdots$ \\ \hline
$n$& $\frac{-8 n^{2} + 3 n + 7}{2}$ \\ \hline
\end{tabular} 
\caption{Representación mediante tabla.}

\end{subfigure}%
\quad
\begin{subfigure}[B]{0.3\textwidth}
\centering

\begin{figure}[h]
\centering
\begin{subfigure}[A]{0.3\textwidth}
\centering
\begin{tabular}{c|c}
$Dominio$ & $Recorrido$ \\ \hline
\hline
$0$ & $\frac{3}{2}$ \\ \hline
$1$ & $1$ \\ \hline
$\vdots$ & $\vdots$ \\ \hline
$n$& $\frac{-8 n^{2} + 3 n + 7}{2}$ \\ \hline
\end{tabular} 
\caption{Representación mediante tabla.}

\end{subfigure}%
\quad
\begin{subfigure}[B]{0.3\textwidth}
\centering

\input{./gnuplot/gnuplot11/exampleMINUSFUNCTIONS}

\caption{Representación gráfica.}

\end{subfigure}

\caption{Representaciones para la Función: $\frac{-8 x^{2} + 3 x + 7}{2}$.}

\end{figure}



\caption{Representación gráfica.}

\end{subfigure}

\caption{Representaciones para la Función: $\frac{-8 x^{2} + 3 x + 7}{2}$.}

\end{figure}



\caption{Representación gráfica.}

\end{subfigure}

\caption{Representaciones para la Función: $\frac{-8 x^{2} + 3 x + 7}{2}$.}

\end{figure}


}

{
\item \textbf{Producto}:

\begin{equation}
F(x) \cdot G(x) = \sum\limits_{i=1}^{i=n} \prod\limits_{j=1}^{j=n} f_i 
\cdot 
g_j\ \diagup\ \{i,j \leqslant n\}\ e\ \{i,j \in \mathbb{N}\}
\end{equation}

\ejem

\begin{equation}
U(x) \cdot V(x) = \frac{13 x^3 + 20 x^2 -3 x -5}{2}
\end{equation}

\begin{figure}[h]
\centering
\begin{subfigure}[A]{0.3\textwidth}
\centering
\begin{tabular}{c|c}
$Dominio$ & $Recorrido$ \\ \hline
\hline
$0$ & $\frac{-5}{2}$ \\ \hline
$1$ & $\frac{25}{2}$ \\ \hline
$\vdots$ & $\vdots$ \\ \hline
$n$& $\frac{13 n^3 + 20 n^2 -3 n -5}{2}$ \\ \hline
\end{tabular} 
\caption{Representación mediante tabla.}

\end{subfigure}%
\quad
\begin{subfigure}[B]{0.3\textwidth}
\centering

\begin{figure}[h]
\centering
\begin{subfigure}[A]{0.3\textwidth}
\centering
\begin{tabular}{c|c}
$Dominio$ & $Recorrido$ \\ \hline
\hline
$0$ & $\frac{-5}{2}$ \\ \hline
$1$ & $\frac{25}{2}$ \\ \hline
$\vdots$ & $\vdots$ \\ \hline
$n$& $\frac{13 n^3 + 20 n^2 -3 n -5}{2}$ \\ \hline
\end{tabular} 
\caption{Representación mediante tabla.}

\end{subfigure}%
\quad
\begin{subfigure}[B]{0.3\textwidth}
\centering

\begin{figure}[h]
\centering
\begin{subfigure}[A]{0.3\textwidth}
\centering
\begin{tabular}{c|c}
$Dominio$ & $Recorrido$ \\ \hline
\hline
$0$ & $\frac{-5}{2}$ \\ \hline
$1$ & $\frac{25}{2}$ \\ \hline
$\vdots$ & $\vdots$ \\ \hline
$n$& $\frac{13 n^3 + 20 n^2 -3 n -5}{2}$ \\ \hline
\end{tabular} 
\caption{Representación mediante tabla.}

\end{subfigure}%
\quad
\begin{subfigure}[B]{0.3\textwidth}
\centering

\input{./gnuplot/gnuplot11/examplePRODFUNCTIONS}

\caption{Representación gráfica.}

\end{subfigure}

\caption{Representaciones para la Función: $\frac{13 x^3 + 20 x^2 -3 x -5}{2}$.}

\end{figure}



\caption{Representación gráfica.}

\end{subfigure}

\caption{Representaciones para la Función: $\frac{13 x^3 + 20 x^2 -3 x -5}{2}$.}

\end{figure}



\caption{Representación gráfica.}

\end{subfigure}

\caption{Representaciones para la Función: $\frac{13 x^3 + 20 x^2 -3 x -5}{2}$.}

\end{figure}


}

{
\item \textbf{División}:

\begin{equation}
\frac{F(x)}{G(x)} = \frac{(f_1, f_2 x, \ldots, f_n x^{n-1})}{(g_1, g_2 x , 
\ldots, g_n x^{n-1})} = \frac{f_1}{g_1} + \frac{f_2 x}{g_2 x} + \ldots, 
\frac{f_n x^{n-1}}{g_n x^{n-1}}\ \diagup\ \{n \in \mathbb{N}\}
\end{equation}

\ejem

\begin{equation}
\frac{U(x)}{V(x)} = \frac{3 x + 5}{8 x^2 - 2}
\end{equation}

\begin{figure}[h]
\centering
\begin{subfigure}[A]{0.3\textwidth}
\centering
\begin{tabular}{c|c}
$Dominio$ & $Recorrido$ \\ \hline
\hline
$0$ & $\varnothing$ \\ \hline
$1$ & $\varnothing$ \\ \hline
$\vdots$ & $\vdots$ \\ \hline
$n$& $\frac{3 n + 5}{8n^2 - 2}$ \\ \hline
\end{tabular} 
\caption{Representación mediante tabla.}

\end{subfigure}%
\quad
\begin{subfigure}[B]{0.3\textwidth}
\centering

\begin{figure}[h]
\centering
\begin{subfigure}[A]{0.3\textwidth}
\centering
\begin{tabular}{c|c}
$Dominio$ & $Recorrido$ \\ \hline
\hline
$0$ & $\varnothing$ \\ \hline
$1$ & $\varnothing$ \\ \hline
$\vdots$ & $\vdots$ \\ \hline
$n$& $\frac{3 n + 5}{8n^2 - 2}$ \\ \hline
\end{tabular} 
\caption{Representación mediante tabla.}

\end{subfigure}%
\quad
\begin{subfigure}[B]{0.3\textwidth}
\centering

\begin{figure}[h]
\centering
\begin{subfigure}[A]{0.3\textwidth}
\centering
\begin{tabular}{c|c}
$Dominio$ & $Recorrido$ \\ \hline
\hline
$0$ & $\varnothing$ \\ \hline
$1$ & $\varnothing$ \\ \hline
$\vdots$ & $\vdots$ \\ \hline
$n$& $\frac{3 n + 5}{8n^2 - 2}$ \\ \hline
\end{tabular} 
\caption{Representación mediante tabla.}

\end{subfigure}%
\quad
\begin{subfigure}[B]{0.3\textwidth}
\centering

\input{./gnuplot/gnuplot11/exampleDIVFUNCTIONS}

\caption{Representación gráfica.}

\end{subfigure}

\caption{Representaciones para la Función: $\frac{3 x + 5}{8 x^2 - 2}$.}

\end{figure}



\caption{Representación gráfica.}

\end{subfigure}

\caption{Representaciones para la Función: $\frac{3 x + 5}{8 x^2 - 2}$.}

\end{figure}



\caption{Representación gráfica.}

\end{subfigure}

\caption{Representaciones para la Función: $\frac{3 x + 5}{8 x^2 - 2}$.}

\end{figure}


}

{
\item \textbf{Composición}:

\begin{equation}
F(x) \circ G(x) = F(G(x)) = (f_1, f_2(g(x)), \ldots f_n(g(x))) \diagup\ \{n \in 
\mathbb{N}\}
\end{equation}

\ejem

\begin{equation}
U(x) \circ V(x) = U(V(x)) = \frac{3(4x^2 -1) + }{2} = \frac{12x^2 + 2}{2} = 6x^2 
+1 
\end{equation}

\begin{figure}[h]
\centering
\begin{subfigure}[A]{0.3\textwidth}
\centering
\begin{tabular}{c|c}
$Dominio$ & $Recorrido$ \\ \hline
\hline
$0$ & $1$ \\ \hline
$1$ & $7$ \\ \hline
$\vdots$ & $\vdots$ \\ \hline
$n$& $6n^2 +1$ \\ \hline
\end{tabular} 
\caption{Representación mediante tabla.}

\end{subfigure}%
\quad
\begin{subfigure}[B]{0.3\textwidth}
\centering

\begin{figure}[h]
\centering
\begin{subfigure}[A]{0.3\textwidth}
\centering
\begin{tabular}{c|c}
$Dominio$ & $Recorrido$ \\ \hline
\hline
$0$ & $1$ \\ \hline
$1$ & $7$ \\ \hline
$\vdots$ & $\vdots$ \\ \hline
$n$& $6n^2 +1$ \\ \hline
\end{tabular} 
\caption{Representación mediante tabla.}

\end{subfigure}%
\quad
\begin{subfigure}[B]{0.3\textwidth}
\centering

\begin{figure}[h]
\centering
\begin{subfigure}[A]{0.3\textwidth}
\centering
\begin{tabular}{c|c}
$Dominio$ & $Recorrido$ \\ \hline
\hline
$0$ & $1$ \\ \hline
$1$ & $7$ \\ \hline
$\vdots$ & $\vdots$ \\ \hline
$n$& $6n^2 +1$ \\ \hline
\end{tabular} 
\caption{Representación mediante tabla.}

\end{subfigure}%
\quad
\begin{subfigure}[B]{0.3\textwidth}
\centering

\input{./gnuplot/gnuplot11/exampleCOMPFUNCTIONS}

\caption{Representación gráfica.}

\end{subfigure}

\caption{Representaciones para la función: $6x^2 +1$.}

\end{figure}



\caption{Representación gráfica.}

\end{subfigure}

\caption{Representaciones para la función: $6x^2 +1$.}

\end{figure}



\caption{Representación gráfica.}

\end{subfigure}

\caption{Representaciones para la función: $6x^2 +1$.}

\end{figure}


}
\end{enumerate}


\section{Álgebra de Boole}

\subsection{Generalidades}

\defn Un \textbf{\index{\'Algebra de Boole}Álgebra de Boole} se define como una tupla de cuatro elementos 
(también denomina retícula booleana):

\begin{equation}
(\mathfrak{B}, \sim, \oplus, \odot)
\end{equation}

Dónde:

\begin{enumerate}[i.]

\item $\mathfrak{B}$: Se trata del \textbf{\index{Conjunto de Variables Booleanas}Conjunto de Variables Booleanas}.
 
\item $\sim$: Se trata de una \textbf{operación interna unitaria} 
($\mathfrak{B} 
\rightarrow \mathfrak{B}$) que cumple:

\begin{equation}
a \rightarrow b = \sim a\ \diagup\ \{a,b \in \mathfrak{B}\}
\end{equation}

\item $\oplus$: Se trata de una \textbf{operación binaria interna} 
($\mathfrak{B} \times \mathfrak{B} \rightarrow \mathfrak{B}$: ) que cumple:

\begin{equation}
(a,b) \rightarrow c = a \oplus b\ \diagup\ \{a,b,c \in \mathfrak{B}\}
\end{equation}

\item $\odot$: Se trata de una \textbf{operación binaria interna} 
($\mathfrak{B} 
\times \mathfrak{B} \rightarrow \mathfrak{B}$: ) que cumple:

\begin{equation}
(a,b) \rightarrow c = a \odot b\ \diagup\ \{a,b,c \in \mathfrak{B}\}
\end{equation}

\end{enumerate}

Siendo las condiciones necesarias:

\begin{enumerate}[i.]

\item $a \oplus b = b$
 
\item $a \odot b = a$

\item $\sim a \oplus b = U$

\item $a \odot \sim b = \varnothing$

\end{enumerate}

\defn Se establece una \textbf{relación directa entre el Álgebra de Boole y la 
Lógica Binaria} de manera que:

\begin{equation}
(\mathfrak{B}, \sim, \oplus, \odot) \equiv (\{0,1\}, \bar{ }, +, \cdot)
\end{equation}

\subsection{Lógica Binaria}

\defn Decimos que $x,y$ son \textbf{\index{Variables Booleanas Binarias}Variables Booleanas Binarias} si:

\begin{equation}
x,y \in (\{0,1\}, \bar{ }, +, \cdot)
\end{equation}

por lo que cumplen:

\begin{enumerate}[I.]

\item \textbf{Operación Complemento} también denominada Operación NOT (ver Figura \ref{fig:not}):

\begin{figure}[h]
\centering
\begin{subfigure}[A]{0.3\textwidth}
\centering
\begin{tabular}{c|c|c|c}
$x$ & $\bar{x}$ & $x + \bar{x}$ & $x \cdot \bar{x}$\\ \hline
\hline
0 & 1 & 1 & 0 \\ \hline
1 & 0 & 1 & 0 \\ \hline
\end{tabular} 
\caption{Tabla de la Verdad.}

\end{subfigure}%
\quad
\begin{subfigure}[B]{0.3\textwidth}
\centering
\begin{pspicture}(0,0)(8,2)%\psgrid
\logicnot[invertoutput=true](0,0){}
\logicnot[invertoutput=true,iec=true,iecinvert=true](4,0){}

\rput(3,1.5){IEEE}
\rput(7.5,1.5){IEC}
\end{pspicture}
\caption{Puertas Lógicas para IEEE e IEC.}

\end{subfigure}

\caption{Representaciones comunes del Operador Booleano NOT.}\label{fig:not}

\end{figure}




\item \textbf{Operación de Suma} también denominada Operación OR (ver Figura \ref{fig:or}):

\begin{figure}[h]
\centering
\begin{subfigure}[A]{0.3\textwidth}
\centering
\begin{tabular}{c|c|c}
$x$ & $y$ & $x + y$\\ \hline
\hline
0 & 0 & 0 \\ \hline
0 & 1 & 1 \\ \hline
1 & 0 & 1 \\ \hline
1 & 1 & 1 \\ \hline
\end{tabular}
\caption{Tabla de la Verdad.}

\end{subfigure}%
\quad
\begin{subfigure}[B]{0.3\textwidth}
\centering
\begin{pspicture}(0,0)(8,2)%\psgrid
\logicor[invertoutput=false](0,0){}
\logicor[invertoutput=false,iec=true,iecinvert=false](4,0){}

\rput(4,2){IEEE}
\rput(7.5,1.5){IEC}
\end{pspicture}
\caption{Puertas Lógicas para IEEE e IEC.}

\end{subfigure}

\caption{Representaciones comunes del Operador Booleano OR.}\label{fig:or}

\end{figure}




\item \textbf{Operación de Producto} también denominada Operación AND (ver Figura \ref{fig:and}):

\begin{figure}[h]
\centering
\begin{subfigure}[A]{0.3\textwidth}
\centering
 \begin{tabular}{c|c|c}
$x$ & $y$ & $x \cdot y$\\ \hline
\hline
0 & 0 & 0 \\ \hline
0 & 1 & 0 \\ \hline
1 & 0 & 0 \\ \hline
1 & 1 & 1 \\ \hline
\end{tabular} 
\caption{Tabla de la Verdad.}

\end{subfigure}%
\quad
\begin{subfigure}[B]{0.3\textwidth}
\centering
\begin{pspicture}(0,0)(8,2)%\psgrid
\logicand[invertoutput=false](0,0){}
\logicand[invertoutput=false,iec=true,iecinvert=false](4,0){}

\rput(4,2){IEEE}
\rput(7.5,1.5){IEC}
\end{pspicture}
\caption{Puertas Lógicas para IEEE e IEC.}

\end{subfigure}

\caption{Representaciones comunes del Operador Booleano AND.}\label{fig:and}

\end{figure}


\item \textbf{Operación de Suma Exclusiva} también denominada Operación XOR (ver Figura \ref{fig:xor}):

\begin{figure}[h]
\centering
\begin{subfigure}[A]{0.3\textwidth}
\centering
 \begin{tabular}{c|c|c}
$x$ & $y$ & $x \oplus y$\\ \hline
\hline
0 & 0 & 0 \\ \hline
0 & 1 & 1 \\ \hline
1 & 0 & 1 \\ \hline
1 & 1 & 0 \\ \hline
\end{tabular} 
\caption{Tabla de la Verdad.}

\end{subfigure}%
\quad
\begin{subfigure}[B]{0.3\textwidth}
\centering
\begin{pspicture}(0,0)(8,2)%\psgrid
\logicxor[ninputs=2]{0}(0,0){}
\logicxor[ninputs=2,iec=true]{0}(4,0){}

\rput(4,2){IEEE}
\rput(7.5,1.5){IEC}
\end{pspicture}
\caption{Puertas Lógicas para IEEE e IEC.}

\end{subfigure}

\caption{Representaciones comunes del Operador Booleano XOR.}\label{fig:xor}

\end{figure}


\end{enumerate}

\subsection{Funciones Booleanas}

\defn Decimos que $O$ es una \textbf{Función Booleana}:

\begin{equation}
O = (u_1,u_2,\ldots,u_n) \Rightarrow u_i \in (\mathfrak{B}, \sim, \oplus, \odot)
\end{equation}

de igual manera decimos que $O$ en una \textbf{Función Booleana Binaria} si:

\begin{equation}
O = (u_1,u_2,\ldots,u_n) \Rightarrow u_i \in (\{0,1\}, \bar{ }, +, \cdot)
\end{equation}

Para el Álgebra de Boole tenemos dos operaciones fundamentales:

\begin{enumerate}[I.]

\item \textbf{Operación de Suma} de Funciones Booleanas Binarias para $O$ y $P$:

\begin{equation}
O + P = (u_1,u_2,\ldots,u_n) + (v_1,v_2,\ldots,v_n) = (u_1 + v_1,u_2 + 
v_2,\ldots,u_n + v_n)
\end{equation}

\item \textbf{Operación de Producto} de Funciones Booleanas Binarias sobre $O$ 
y 
$P$:

\begin{equation}
U \cdot V = (u_1,u_2,\ldots,u_n) \cdot (v_1,v_2,\ldots,v_n) = (u_1 \cdot 
v_1,u_2 
\cdot v_2,\ldots,u_n \cdot v_n)
\end{equation}

\end{enumerate}

%%%%%%%%%%%%%%%%%%%%%%%%%
% Grafos
%%%%%%%%%%%%%%%%%%%%%%%%%

\section{Nociones sobre Grafos}

\subsection{Definiciones}
\defn Un \index{Grafo}Grafo $G$ \textbf{esta compuesto por tres conjuntos finitos} y 
necesariamente 
uno de ellos no vacío:

\begin{enumerate}[i.]
\item Conjunto $V$: El Conjunto de sus \textbf{Vértices} (\textbf{no puede ser 
vacío}).
{
\item Conjunto $E$: El Conjunto de sus \textbf{Aristas}.
\begin{equation}
V\ x\ V \rightarrow E
\end{equation}
}
{
\item Conjunto $p$: El Conjunto de los \textbf{Pesos} o \textbf{Etiquetas} por 
aristas.

\begin{equation}
p:E
\end{equation}
}

\end{enumerate}

Por ello establecemos la siguiente notación para describir un Grafo:

\begin{equation}
G = (V,E,p)
\end{equation}

\defn Para una arista dada: $a_\lambda = (v_1, v_2)$ decimos que:
{
\begin{enumerate}[i.]

\item $v_1$ es el \textbf{origen}.

\item $v_2$ es el \textbf{destino} o \textbf{final}.

\end{enumerate}
}

\ejem Dado el siguiente grafo $G_3$ (ver Figura \ref{fig:generalGraphs}):

\begin{enumerate}[i.]
\item $V = \{a,b,c\}$
\item $E = \{\{a,b\},\{b,c\},\{c,a\}\}$
\end{enumerate}

\defn Decimos que una arista $a_{\lambda}$ es \textbf{incidente} para dos 
vértices $v_1, v_2$ si une dichos vértices:

\begin{equation}
a_{\lambda} = (v_1, v_2)\ \diagup\ v_1, v_2 \in V,\ a_{\lambda} \in E
\end{equation}

\defn Un vértice $v_1$ es \textbf{adyacente} sobre otros vértices $v_{\lambda}$ 
si dicho vértice forma parte de la relación:

\begin{equation}
a_{\lambda} = (v_1, v_{\lambda})\ \diagup\ v_{\lambda} \in V,\ a_{\lambda} \in E
\end{equation}


\defn Una arista del tipo $a_1 = (v_1, v_1)$ se denomina \textbf{bucle} puesto 
que el vértice origen y destino son el mismo.

\defn Si $(a_1, a_2)$ inciden sobre el mismo vértice, se dice que son 
\textbf{aristas paralelas}.


\defn Un vértice $v_\mu$ es un \textbf{vértice aislado} si para el 
conjunto $E$ no existe ningún par o relación.

\begin{figure}[h]
\centering

\begin{subfigure}[A]{0.4\textwidth}
\centering
\begin{pspicture}(0,-1)(6,4)%\psgrid


\psline[linecolor=black,linewidth=1pt]{-}(1,0)(5,0)
\psline[linecolor=black,linewidth=1pt]{-}(1,0)(1,3)
\psline[linecolor=black,linewidth=1pt]{-}(1,3)(5,0)

\psdot[dotsize=5pt,dotstyle=o](1,0)
\psdot[dotsize=5pt,dotstyle=o](5,0)
\psdot[dotsize=5pt,dotstyle=o](1,3)


\rput(0.6,3){$v_1$}
\rput(0.6,0){$v_2$}
\rput(5.4,0){$v_3$}


\end{pspicture}
\caption{Grafo $G_1$.}
\end{subfigure}%
\quad
\begin{subfigure}[B]{0.4\textwidth}
\centering
\begin{pspicture}(0,-1)(6,4)%\psgrid


\psline[linecolor=black,linewidth=1pt]{-}(1,0)(5,0)
\psline[linecolor=black,linewidth=1pt]{-}(1,0)(1,3)
\psline[linecolor=black,linewidth=1pt]{-}(1,3)(5,3)
\psline[linecolor=black,linewidth=1pt]{-}(5,3)(5,0)

\psdot[dotsize=5pt,dotstyle=o](1,0)
\psdot[dotsize=5pt,dotstyle=o](5,0)
\psdot[dotsize=5pt,dotstyle=o](1,3)
\psdot[dotsize=5pt,dotstyle=o](5,3)

\rput(0.6,3){$v_1$}
\rput(5.4,3){$v_2$}
\rput(0.6,0){$v_3$}
\rput(5.4,0){$v_4$}

\end{pspicture}
\caption{Grafo $G_2$.}
\end{subfigure}%
\quad
\begin{subfigure}[C]{0.4\textwidth}
\centering
\begin{pspicture}(0,-1)(6,4)%\psgrid

\psline[linecolor=black,linewidth=1pt]{-}(3,3)(1,0)
\psline[linecolor=black,linewidth=1pt]{-}(3,3)(5,0)
\psline[linecolor=black,linewidth=1pt]{-}(1,0)(5,0)


\psdot[dotsize=5pt,dotstyle=o](3,3)
\psdot[dotsize=5pt,dotstyle=o](1,0)
\psdot[dotsize=5pt,dotstyle=o](5,0)

\rput(3,3.3){$v_1$}
\rput(0.6,0){$v_2$}
\rput(5.4,0){$v_3$}
\end{pspicture}
\caption{Grafo $G_3$.}
\end{subfigure}

\caption{Ejemplos de Grafos.}\label{fig:generalGraphs}

\end{figure}


\subsection{Clasificación}
\label{sec:clasificacionGrafos}

\defn Un grafo $G = (V,E,p=\emptyset)$ que contiene más de un par de aristas para uno de 
sus vértices en un \textbf{\index{Grafo Multigrafo}Grafo Multigrafo}.

\defn Un grafo $G = (V,E,p=\emptyset)$ que contiene al menos un bucle y ningún conjunto de 
aristas paralelas es lo que convencionalmente denominamos \textbf{Grafo}.


\defn Para un grafo $G = (V,E,p)$, si $p = \emptyset$ y no existen bucles, se 
dice que es 
un \textbf{\index{Grafo Simple}Grafo Simple}.

{\cor Si $p = \emptyset$ y existen bucles, el grafo se denomina \textbf{\index{Grafo no Simple}Grafo no Simple} .}

\defn Para un grafo G de tipo simple, si $p\neq \emptyset$ entonces se denomina 
\textbf{\index{Grafo Dirigido}Grafo Dirigido}.

\begin{figure}[h]
\centering
\begin{subfigure}[A]{0.4\textwidth}
\centering
\begin{pspicture}(0,-1)(6,4)%\psgrid

\psline[linecolor=black,linewidth=1pt]{-}(3,3)(1,0)
\psline[linecolor=black,linewidth=1pt]{-}(3,3)(5,0)
\pscurve[linecolor=black,linewidth=1pt]{-}(1,0)(3,0.5)(5,0)
\pscurve[linecolor=black,linewidth=1pt]{-}(1,0)(3,-0.5)(5,0)

\psdot[dotsize=5pt,dotstyle=o](3,3)
\psdot[dotsize=5pt,dotstyle=o](1,0)
\psdot[dotsize=5pt,dotstyle=o](5,0)

\rput(3,3.3){$v_1$}
\rput(0.6,0){$v_2$}
\rput(5.4,0){$v_3$}

\end{pspicture}
\caption{Multigrafo $G_4$.}
\end{subfigure}%
\quad
\begin{subfigure}[B]{0.4\textwidth}
\centering
\begin{pspicture}(0,-1)(6,4)%\psgrid

\psline[linecolor=black,linewidth=1pt]{-}(3,3)(1,0)
\psline[linecolor=black,linewidth=1pt]{-}(3,3)(5,0)
\psline[linecolor=black,linewidth=1pt]{-}(1,0)(5,0)
\pscircle[linecolor=black,linewidth=1pt](3,3.5){0.5}

\psdot[dotsize=5pt,dotstyle=o](3,3)
\psdot[dotsize=5pt,dotstyle=o](1,0)
\psdot[dotsize=5pt,dotstyle=o](5,0)

\rput(3.6,2.8){$v_1$}
\rput(0.6,0){$v_2$}
\rput(5.4,0){$v_3$}
\end{pspicture}
\caption{Grafo no simple $G_5$.}
\end{subfigure}

\quad
\begin{subfigure}[C]{0.4\textwidth}
\centering
\begin{pspicture}(0,-1)(6,4)%\psgrid


\psline[linecolor=black,linewidth=1pt]{<-}(1,0)(5,0)
\psline[linecolor=black,linewidth=1pt]{->}(1,0)(1,3)
\psline[linecolor=black,linewidth=1pt]{->}(1,3)(5,3)
\psline[linecolor=black,linewidth=1pt]{->}(5,3)(5,0)

\psline[linecolor=black,linewidth=1pt]{->}(1,3)(5,0)

\psdot[dotsize=5pt,dotstyle=o](1,0)
\psdot[dotsize=5pt,dotstyle=o](5,0)
\psdot[dotsize=5pt,dotstyle=o](1,3)
\psdot[dotsize=5pt,dotstyle=o](5,3)

\rput(0.6,3){$v_1$}
\rput(5.4,3){$v_2$}
\rput(0.6,0){$v_3$}
\rput(5.4,0){$v_4$}

\rput(3,3.3){$a$}
\rput(0.7,1.5){$b$}
\rput(5.3,1.5){$c$}
\rput(3,-0.3){$d$}
\rput(3,2){$e$}

\end{pspicture}
\caption{Grafo dirigido $G_6$.}
\end{subfigure}

\caption{Ejemplo de Multigrafo y Grafo no simple y Grafo dirigido.}

\end{figure}







\defn Dos grafos $G_1$ y $G_2$ son \textbf{\index{Isomorfos}Isomorfos} si existe una 
\textbf{Biyección} entre ellos 
$\alpha$.

\begin{equation}
V_{G_1} = \{a,b,c\} \equiv \alpha V_{G_1} = V_{G_2} = \{\alpha(a)= d,\alpha(b)= 
e,\alpha(c)= f\}
\end{equation}

\ejem Para $G_9 \Rightarrow V(G_9) = \alpha\ \{v_1, v_2, v_3, v_4\} \in V(G_8)  = \{\alpha(v_1) = v_1^\prime,\ \alpha (v_2) = v_2^\prime,\ \alpha (v_3) = v_3^\prime,\ \alpha (v_4) = v_4^\prime\}$

\begin{figure}[h]
\centering
\begin{subfigure}[A]{0.4\textwidth}
\centering
\begin{pspicture}(0,-0.5)(6,4.5)%\psgrid

\psline[linecolor=black,linewidth=1pt]{-}(1,0)(5,0)
\psline[linecolor=black,linewidth=1pt]{-}(1,0)(1,3)
\psline[linecolor=black,linewidth=1pt]{-}(1,3)(5,3)
\psline[linecolor=black,linewidth=1pt]{-}(5,3)(5,0)

\psline[linecolor=black,linewidth=1pt]{-}(1,0)(5,3)

\psdot[dotsize=5pt,dotstyle=o](1,0)
\psdot[dotsize=5pt,dotstyle=o](5,0)
\psdot[dotsize=5pt,dotstyle=o](1,3)
\psdot[dotsize=5pt,dotstyle=o](5,3)

\rput(0.6,3){$v_1$}
\rput(5.4,3){$v_2$}
\rput(0.6,0){$v_3$}
\rput(5.4,0){$v_4$}

\end{pspicture}
\caption{Grafo $G_8$.}
\end{subfigure}%
\quad
\begin{subfigure}[B]{0.4\textwidth}
\centering
\begin{pspicture}(0,-0.5)(6,4.5)%\psgrid
\psline[linecolor=black,linewidth=1pt]{-}(1,0)(5,0)

\psline[linecolor=black,linewidth=1pt]{-}(1,0)(3,1.5)
\psline[linecolor=black,linewidth=1pt]{-}(5,0)(3,1.5)

\psline[linecolor=black,linewidth=1pt]{-}(1,0)(3,4)
\psline[linecolor=black,linewidth=1pt]{-}(5,0)(3,4)




\psdot[dotsize=5pt,dotstyle=o](1,0)
\psdot[dotsize=5pt,dotstyle=o](5,0)
\psdot[dotsize=5pt,dotstyle=o](3,1.5)
\psdot[dotsize=5pt,dotstyle=o](3,4)

\rput(3,4.3){$v_1$}
\rput(3,1.8){$v_2$}
\rput(0.6,0){$v_3$}
\rput(5.4,0){$v_4$}

\end{pspicture}
\caption{Grafo $G_9\equiv \alpha G_8$.}
\end{subfigure}

\caption{Ejemplo de Grafos Isomorfos.}

\end{figure}







\defn Denominamos grado de un vértice $v_\lambda$ para un grafo $G = (V,E,p)$ 
al 
numero de artistas del grafo $G$ en dicho vértice.

\begin{equation}
\delta(v)\ \diagup\ v \in V = \sum\ e_i\ \diagup\ \{e \in E, v \in e\}
\end{equation}

\ejem Para $G_4 \Rightarrow \delta(v_1) = 2;\ \delta(v_2) = 2;\ \delta(v_3) = 
2;$

{\thm La suma de los grados de los vértices de un \textbf{grafo no dirigido} es 
igual al doble del 
número de aristas.}

\begin{equation}
\sum\limits_{i = 0}^{i= n} \delta(v_i) = 2 \cdot |E|
\end{equation}

\ejem Para $G_5 \Rightarrow \delta(v_1) = 3;\ \delta(v_2) = 2;\ \delta(v_3) = 
2;\ \delta(v_4) = 3; \equiv 2\cdot |E| = 2 \cdot 4 = 8$

{\thm Para un \textbf{grafo dirigido} $G = (V,E,p)$ se cumple:}

\begin{equation}
\sum\limits_{i = 0}^{i= n} \delta(v_i)^+ = \sum\limits_{j = 0}^{j= n} 
\delta(v_j)^- = |E|
\end{equation}

\paragraph*{Dónde:}

\begin{enumerate}[i.]

\item $\delta(v)^+$: Es el número de aristas que se dirigen a $v$.

\item $\delta(v)^-$: Es el número de aristas que parten de $v$.

\end{enumerate}

\ejem Para $G_6$:

\begin{enumerate}[i.]

\item $\delta^+ \Rightarrow \delta(v_1)^+ = 2;\ \delta(v_2)^+ = 1;\ 
\delta(v_3)^+ = 1;\ \delta(v_4)^+ = 1$

\item $\delta^+ \Rightarrow \delta(v_1)^- = 1;\ \delta(v_2)^- = 1;\ 
\delta(v_3)^-= 1;\ \delta(v_4)^- = 2$

\item $E = 5$

\end{enumerate}

\subsection{Tipos}

% Completo

\defn Se denomina \textbf{\index{Grafo Completo}Grafo Completo} a aquel grafo simple de $n$ vértices 
que tiene una sola arista entre cada  par de vértices. Se denotan como $K_n$.

% Regular

\defn Se denomina \textbf{\index{Grafo Regular}Grafo Regular} a aquel que tiene en mismo grado en 
todos sus vértices.

% Bipartito

\defn Se dice que un grafo es \textbf{\index{Bipartito}Bipartito} si su número de vértices se 
pueden dividir en dos conjuntos $G = G_1 \cup G_2$ disjuntos $G_1 \cap G_2 = 
\emptyset$.

\begin{figure}[h]
\centering
\begin{subfigure}[A]{0.4\textwidth}
\centering
\begin{pspicture}(0,-1)(6,4)%\psgrid

\psline[linecolor=black,linewidth=1pt]{-}(1,0)(5,3)
\psline[linecolor=black,linewidth=1pt]{-}(1,0)(5,0)
\psline[linecolor=black,linewidth=1pt]{-}(5,3)(5,0)

\psdot[dotsize=5pt,dotstyle=o](5,3)
\psdot[dotsize=5pt,dotstyle=o](1,0)
\psdot[dotsize=5pt,dotstyle=o](5,0)

\rput(5.4,3){$v_1$}
\rput(0.6,0){$v_2$}
\rput(5.4,0){$v_3$}

\end{pspicture}
\caption{Grafo Completo $G_{10}$.}
\end{subfigure}%
\quad
\begin{subfigure}[B]{0.4\textwidth}
\centering
\begin{pspicture}(0,-1)(6,4)%\psgrid


\psline[linecolor=black,linewidth=1pt]{-}(1,0)(5,0)
\psline[linecolor=black,linewidth=1pt]{-}(1,0)(1,3)
\psline[linecolor=black,linewidth=1pt]{-}(1,3)(5,3)
\psline[linecolor=black,linewidth=1pt]{-}(5,3)(5,0)

\psline[linecolor=black,linewidth=1pt]{-}(2,1.5)(4,1.5)

\psline[linecolor=black,linewidth=1pt]{-}(1,3)(2,1.5)
\psline[linecolor=black,linewidth=1pt]{-}(1,0)(2,1.5)

\psline[linecolor=black,linewidth=1pt]{-}(5,3)(4,1.5)
\psline[linecolor=black,linewidth=1pt]{-}(5,0)(4,1.5)


\psdot[dotsize=5pt,dotstyle=o](1,0)
\psdot[dotsize=5pt,dotstyle=o](5,0)
\psdot[dotsize=5pt,dotstyle=o](1,3)
\psdot[dotsize=5pt,dotstyle=o](5,3)
\psdot[dotsize=5pt,dotstyle=o](2,1.5)
\psdot[dotsize=5pt,dotstyle=o](4,1.5)

\rput(0.6,3){$v_1$}
\rput(5.4,3){$v_2$}
\rput(0.6,0){$v_3$}
\rput(5.4,0){$v_4$}

\end{pspicture}
\caption{Grafo Regular $G_{11}$.}
\end{subfigure}%
\quad
\begin{subfigure}[C]{0.4\textwidth}
\centering
\begin{pspicture}(0,-1)(6,4)%\psgrid

\psline[linecolor=black,linewidth=1pt]{-}(3,3)(1,0)
\psline[linecolor=black,linewidth=1pt]{-}(3,3)(5,0)
\psline[linecolor=black,linewidth=1pt]{-}(1,0)(5,0)



\psdot[dotsize=5pt,dotstyle=o](3,3)
\psdot[dotsize=5pt,dotstyle=o](1,0)
\psdot[dotsize=5pt,dotstyle=o](5,0)


\psline[linecolor=black,linewidth=1pt]{-}(1,2)(5,2)

\psdot[dotsize=5pt,dotstyle=o](1,2)
\psdot[dotsize=5pt,dotstyle=o](5,2)


\rput(3,3.3){$v_1$}
\rput(0.6,0){$v_2$}
\rput(5.4,0){$v_3$}

\rput(0.6,2){$v_4$}
\rput(5.4,2){$v_5$}
\end{pspicture}
\caption{Grafo Bipartito $G_{12}$.}
\end{subfigure}
\caption{Ejemplo de Grafos: Completo, Regular y Bipartito.}

\end{figure}






\subsection{Circuitos y Ciclos}

\begin{enumerate}[I.]

\item Recorrido y \index{Circuito Eureliano}Circuito Eureliano:
{
\defn Un grafo $G = (V,E,p)$ (o multigrafo sin vértices asilados) contiene un 
camino simple (\textbf{Camino Simple de Euler o Recorrido Eureliano}) que parte 
de $v_0$ 
hasta $v_n$ y que pasa una sola vez por cada uno de los vértices.

\defn Recibe el nombre de Circuito de Eurler a todo camino que pase una sola 
vez 
por todos los lados de un grafo $G$.

{\cor Si un grafo $G = (V,E)$ tiene un Circuito de Euler\endnote{El origen de la 
\textbf{Teoría de Grafos parte de la famosa publicación 
\href{http://math.dartmouth.edu/~euler/docs/originals/E053.pdf}{``Los siete 
puentes de Königsberg''}} dónde su autor \textbf{Leonhard Euler}, nacido el 15 de Abril de 
1707 en Basilea (Suiza) y fallecido el 18 de Septiembre de 1783 en San 
Petersburgo (Rusia), se preguntaba como en la propia cuidad de Königsberg 
(actual Kalingrad) era posible cruzar los siete puentes una sola vez del río 
Pregel iniciando y finalizado el trayecto en el mismo punto. Para ello determinó 
un modelo: \begin{figure}[h]

\begin{center}
\begin{pspicture}(0,1)(6,3.5)%\psgrid

\psline[linecolor=black,linewidth=1pt]{-}(3,3)(5,2)
\psline[linecolor=black,linewidth=1pt]{-}(3,1)(5,2)
\psline[linecolor=black,linewidth=1pt]{-}(1,2)(5,2)
\pscurve[linecolor=black,linewidth=1pt]{-}(1,2)(2,2.2)(3,3)
\pscurve[linecolor=black,linewidth=1pt]{-}(1,2)(2,1.8)(3,1)
\pscurve[linecolor=black,linewidth=1pt]{-}(1,2)(2,2.8)(3,3)
\pscurve[linecolor=black,linewidth=1pt]{-}(1,2)(2,1.2)(3,1)
\psdot[dotsize=10pt,dotstyle=o](1,2)
\psdot[dotsize=10pt,dotstyle=o](3,1)
\psdot[dotsize=10pt,dotstyle=o](3,3)
\psdot[dotsize=10pt,dotstyle=o](5,2)
\end{pspicture} 
\caption{Grafo de Königsberg.}
\end{center}


\end{figure}


 y postuló su 
famoso Teorema. \textit{El Teorema de Euler dice que para un grafo $G = 
(V,E,p)$) o multigrafo (no 
digrafo) sin vértices aislados, $G$ posee un Circuito de Euler si y sólo si $G$ 
es conexo y cada vértice tiene grado par.}}, es un grafo 
Eureliano.}

{\thm El Teorema de Euler dice que para un grafo $G = (V,E,p)$ o multigrafo 
(no 
digrafo) sin vértices aislados, $G$ posee un Circuito de Euler si y sólo si $G$ 
es conexo y cada vértice tiene grado par.}
}
\item Recorrido y \index{Ciclo Hamiltoniano}Ciclo Hamiltoniano:
{
\defn Para un grafo $G = (V,E,p)$ con $|V| \geq 3$ sin vértices aislados. $G$ 
tiene un \textbf{Camino Hamiltoniano} natural que recorre todos sus vértices.

\defn Un grafo $G = (V,E,p)$ tiene un \textbf{Ciclo Hamiltoniano\endnote{La 
figura de \textbf{Sir William Rowan Hamilton} nacido el 4 de Agosto de 1805 en Dublin 
(Irlanda) y fallecido en 1865 en Dublin, reformó el trabajo previo sobre la 
Teoría de Grafos llegando a la conclusión de que \textbf{en ciertas condiciones es 
posible recorrer un grafo con el mismo punto de origen y destino pasando por 
todas sus aristas una sola vez}. Tras este trabajo postuló su Teorema: \textit{Si 
un grafo $G = (V,E,p)$  tiene $|V| \geq 3$ y $\delta(v_i) \geq 2$, 
entonces $G$ es hamiltoniano}}} si existe un ciclo para 
todos los vértices de V.

{\cor Si un grafo tiene un Ciclo Hamiltoniano se dice que es un grafo 
hamiltoniano.}

{\thm Si un grafo $G = (V,E,p)$  tiene $|V| \geq 3$ y $\delta(v_i) \geq 2$, 
entonces $G$ es hamiltoniano.}

}
\end{enumerate}

\subsection{Árboles}

\subsubsection{Generalidades}

{\defn Un \index{\'Arbol}Árbol se define como un grafo conectado sin ciclos.}

{\thm Dado un grafo T = (V,E) decimos que se trata de un árbol sí:}


\begin{enumerate}[i.]
 
\item $T$ es un grafo acíclico.

\item $T$ está tiene un número de vértices $n$ y de arista que las 
interconectan 
$(n-1)$.

\item Cada par de vértices está conectado únicamente por una arista.

\end{enumerate}

{\cor Si para un árbol $T$ eliminamos una arista de un par de vértices (u,v) el 
grafo resultante $T^{^\prime}$ no tiene estructura de árbol.}


%% Árboles Generadores

\subsubsection{Árboles Generadores}

\defn Para una grafo simple $G = (V,E,p)$, existe un \index{\'Arbol Generador}Árbol Generador $T$ si y 
sólo si: $T(E) = G(E)$

{\cor Un Árbol Generador Mínimo de un Grafo Ponderado es un árbol en el que la suma de sus aristas el la mínima posible.}

%%%%%%%%%%%%%%
% Algoritmo de Prim
%%%%%%%%%%%%%%
\paragraph{{\index{Algoritmo de Prim}Algoritmo de Prim}}

\prog Pseudocódigo del \index{Algoritmo de Prim}Algoritmo de Prim:

\lstinputlisting{./algorithms/algorithms11/primsAlgorithm.tex}

\algo Para un grafo $G = (V,E,p)$

\begin{enumerate}[i.]
\item Seleccionar un vértice $v_0$ aleatorio de: $G(E)$ 
 
\item Establecer para $v_0$ las aristas que lo conectan. Si existen dos aristas con idéntico peso, seleccionar cualquiera de ellas.

\item Seleccionar la nueva arista con peso mínimo.

\item Añadir el vértice y el lado que lo interconecta al conjunto $T$ como resultado.

\item Iterar pasos $ii \dots iv$.
\end{enumerate}

\begin{figure}[h]
\begin{center}
\begin{pspicture}(0,0)(13,5.5)%\psgrid

\psline[linecolor=black,linewidth=0.8pt]{-}(1,0.5)(12,1.5)
\psline[linecolor=black,linewidth=0.8pt]{-}(1,0.5)(1,4.5)
\psline[linecolor=black,linewidth=1.4pt]{-}(1,4.5)(4,3)
\psline[linecolor=black,linewidth=1.4pt]{-}(4,3)(12,1.5)

\psline[linecolor=black,linewidth=1.4pt]{-}(4,3)(8,4.8)
\psline[linecolor=black,linewidth=0.8pt]{-}(8,4.8)(12,1.5)

\psline[linecolor=black,linewidth=1.4pt]{-}(1,0.5)(3,2.2)
\psline[linecolor=black,linewidth=1.4pt]{-}(1,4.5)(3,2.2)


%\pscircle[linecolor=black,linewidth=1pt]


\psdot[dotsize=5pt,dotstyle=o](1,0.5)
\psdot[dotsize=5pt,dotstyle=o](1,4.5)
\psdot[dotsize=5pt,dotstyle=o](4,3)
\psdot[dotsize=5pt,dotstyle=o](12,1.5)


\psdot[dotsize=5pt,dotstyle=o](8,4.8)
\psdot[dotsize=5pt,dotstyle=o](3,2.2)

% \psdot[dotsize=5pt,dotstyle=o](5,0)

\rput(0.7,0.5){$e$}
\rput(0.7,4.5){$a$}
\rput(3.3,2.2){$f$}
\rput(4,3.4){$b$}
\rput(8,5.1){$c$}
\rput(12.3,1.6){$d$}

\rput(0.7,2.5){$6$}
\rput(2,2.8){$4$}
\rput(2,1.7){$3$}
%\rput(5.4,0){$c$}

\rput(3,3.8){$4$}

\rput(7,0.8){$4$}

\rput(8.5,2.5){$5$}
\rput(6,4.2){$4$}
\rput(9.5,3.9){$8$}


\end{pspicture}

\caption{Grafo origen para Algoritmo de Prim.\label{fig:prim}}

\end{center}
\end{figure}







Dónde:

\begin{enumerate}[i.]

\item $V = \{a,b,c,d,e,f\}$ 

\item $E = \{ab,bc,cd,de,db,ef,af\}$ 

\end{enumerate}

\ejem Aplicar el Algoritmo de Prim al siguiente Grafo y encontrar su Árbol Recubridor Mínimo (Figura \ref{fig:prim})

\begin{enumerate}[i.]
\item Seleccionamos el vértice $\{d\}$ como origen.

\item Calculamos sobre los pesos de las arístas conexas: $\{dc = 8 , de = 7, db = 5\}$

\item Tomamos como vértice de resultado $b$; $T(E) = \{d,b\}$

\item Continuamos iterando para obtener el Árbol Recubrido Ḿínimo: $T(E) = \{d,b,c,a,f,e\}$
 
\end{enumerate}

%%%%%%%%%%%%%%
% Algoritmo de Kruskal
%%%%%%%%%%%%%%

\paragraph{{\index{Algoritmo de Kruskal}Algoritmo de Kruskal}}

\prog Pseudocódigo del Algoritmo de Kruskal:

\lstinputlisting{./algorithms/algorithms11/kruskalsAlgorithm.tex}

\algo Para un grafo $G = (V,E,p)$

\begin{enumerate}[i.]
\item Clasificar las arístas de: $G(E)$ en orden creciente.
 
\item Añadir a $T(E)$ cualquiera de los los lados de $G$ con menor peso y que no formen ciclo con otros lados. 

\item Iterar el paso $ii$ desde: $i=1$ hasta $G(E) -1$.
\end{enumerate}

\begin{figure}[h]
\begin{center}
\begin{pspicture}(0,0)(18,5.5)%\psgrid

\psline[linecolor=black,linewidth=0.8pt]{-}(1,0.5)(12,1.5)
\psline[linecolor=black,linewidth=0.8pt]{-}(1,0.5)(1,4.5)
\psline[linecolor=black,linewidth=1.4pt]{-}(1,4.5)(4,3)
\psline[linecolor=black,linewidth=1.4pt]{-}(4,3)(12,1.5)

\psline[linecolor=black,linewidth=1.4pt]{-}(4,3)(8,4.8)
\psline[linecolor=black,linewidth=0.8pt]{-}(8,4.8)(12,1.5)

\psline[linecolor=black,linewidth=1.4pt]{-}(1,0.5)(3,2.2)
\psline[linecolor=black,linewidth=1.4pt]{-}(1,4.5)(3,2.2)

\psline[linecolor=black,linewidth=1.4pt]{-}(12,1.5)(14,2.7)
\psline[linecolor=black,linewidth=1.4pt]{-}(12,1.5)(14,0.4)

\psline[linecolor=black,linewidth=1.4pt]{-}(14,2.7)(16,1.5)


\psline[linecolor=black,linewidth=0.8pt]{-}(14,0.4)(16,1.5)
%\pscircle[linecolor=black,linewidth=1pt]


\psdot[dotsize=5pt,dotstyle=o](1,0.5)
\psdot[dotsize=5pt,dotstyle=o](1,4.5)
\psdot[dotsize=5pt,dotstyle=o](4,3)



\psdot[dotsize=5pt,dotstyle=o](8,4.8)
\psdot[dotsize=5pt,dotstyle=o](3,2.2)

% \psdot[dotsize=5pt,dotstyle=o](5,0)

\rput(0.7,0.5){$i$}
\rput(0.7,4.5){$a$}
\rput(3.3,2.2){$d$}
\rput(4,3.4){$e$}
\rput(8,5.2){$b$}
\rput(12,1.1){$f$}
\psdot[dotsize=5pt,dotstyle=o](12,1.5)

\rput(14,0.1){$g$}
\psdot[dotsize=5pt,dotstyle=o](14,0.4)

\rput(16.3,1.5){$h$}
\psdot[dotsize=5pt,dotstyle=o](16,1.5)

\rput(14,3){$c$}
\psdot[dotsize=5pt,dotstyle=o](14,2.7)

\rput(0.7,2.5){$6$}
\rput(2,2.8){$4$}
\rput(2,1.7){$3$}
%\rput(5.4,0){$c$}

\rput(3,3.8){$4$}

\rput(7,0.8){$4$}

\rput(8.5,2.5){$5$}
\rput(6,4.2){$4$}
\rput(9.5,3.9){$8$}


\end{pspicture}

\caption{Grafo origen para Algoritmo de Kruskal.\label{fig:kruskal}}

\end{center}
\end{figure}







Dónde:

\begin{enumerate}[i.]

\item $V = \{a,b,c,d,e,f,...\}$ 

\item $E = \{ab,bc,cd,de,db,ef,af,...\}$ 

\end{enumerate}

\ejem Aplicar el Algoritmo de Kruskal al siguiente Grafo y encontrar su Árbol Recubridor Mínimo (Figura \ref{fig:kruskal})

\begin{enumerate}[i.]

\item Clasificamos las arístas de en orden creciente (Tabla \ref{tab:tabKruskal})

\begin{table}[h]

\begin{center}

\begin{tabular}{|l|l|l|l|l|l|l|l|l|l|l|l|l|}\hline
\textbf{Lados} & \textit{ai} & \textit{ad} & \textit{fg} & \textit{ae} & \textit{eb} & \textit{fc} & \textit{ch} & \textit{ef} & \textit{ai} & \textit{gh} & \textit{bf} & \textit{if} \\ \hline
\hline
\textbf{Pesos} & 1 & 1 & 1 & 2 & 2 & 4 & 4 & 4 & 6 & 6 & 7 & 7\\ \hline
\textbf{¿Añadir?} & Si & Si & Si & Si & Si & Si & Si & Si & No & No & No & No\\ \hline
\end{tabular}

\caption{Tabla de Pesos Crecientes para Figura \ref{fig:kruskal}}\label{tab:tabKruskal}

%\ref{tab:exampleKruskal}

\end{center}

\end{table}

\item Añadir a $T(E)$; $T(E) = \{a\}$; $T(E) = \{a,e\}$ $\dots$

\item Obtenemos el Árbol Recubrido Ḿínimo: $T(E) = \{a,e,b,c,h,d,f,i\}$

\end{enumerate}

%%%%%%%%%%%%%%
% Árboles m-arios
%%%%%%%%%%%%%%

\subsubsection{Árboles \textit{m-arios}}
\label{subsec:mTrees}

\defn Definimos un \textbf{\index{\'Arbol con Raíz}Árbol con Raíz} si uno de sus vértices se nombra 
de esta manera (vértice Raíz $R$).

\begin{figure}[h]
\centering
\begin{subfigure}[B]{0.4\textwidth}
\centering
\begin{pspicture}(-3,-5)(4,1)%\psgrid
\psset{unit=0.5cm}
\pstree[treesep=2cm,levelsep=2cm]{\Tc{2pt}~[tnpos=a,tndepth=0pt,radius=4pt]{$r$}}%
{%
\Tdot
\pstree{\Tdot}%
	       {
               \Tdot
               \Tdot
               \Tdot
               }%
\Tdot
\pstree{\Tdot}%
               {
               \Tdot
               \Tdot
               }%
}
\end{pspicture}

\caption{Ejemplo de Árbol Raíz \textit{m-ario}.}

\end{subfigure}%
\quad
\begin{subfigure}[B]{0.4\textwidth}
\centering
\begin{pspicture}(-3,-5)(4,1)%\psgrid
\psset{unit=0.5cm}
\pstree[treesep=2cm,levelsep=2cm]{\Tc{2pt}~[tnpos=a,tndepth=0pt,radius=4pt]{$r$}}%
{%
\pstree{\Tdot}%
               {
               %\Tdot
               \Tdot
               }%
\pstree{\Tdot}%
               {
               \Tdot
               \Tdot
               }%
}
\end{pspicture}

\caption{Ejemplo de Árbol Raíz Binario.}

\end{subfigure}

\caption{Ejemplo de Árboles \textit{m-arios}.}

\end{figure}



\defn Un \textbf{Árbol con Raíz es \textit{m-ario}} (con $m \geq 2$) si de 
designamos al ńumero máximo de hijos por cada nodo con $m$.

\defn Un Árbol es \textit{m-ario} completo si por cada vértice tiene \textit{m} hijos o ninguno.

\paragraph*{Recorridos:}

Existen tres tipos de algoritmos para recorrer Árboles \textit{m-arios}:  

\begin{figure}[h]
\begin{center}
\begin{pspicture}(-4,-5.2)(9,1.3)%\psgrid

%\pspolygon[fillstyle=solid,fillcolor=white](-1,1)(9.5,1)(9.5,-5)(-1,-5)

\pstree[treesep=2cm,levelsep=2cm]{\Tcircle{$a$}}%
{%
\Tcircle{$b$}
\pstree{\Tcircle{$c$}}%
               {
               \Tcircle{$f$}
               \Tcircle{$g$}
               \Tcircle{$h$}
               }%
\Tcircle{$d$}
\Tcircle{$e$}
}

%\pscurve[linestyle=dotted, linecolor=black,linewidth=1pt]{->}(-0.4,0)(-3.4,-2)(-2.9,-2.4)(-1.6,-1.6)(-3.4,-4)(-2.8,-4.4)(-2.2,-3.6)(-1.2,-4.4)(-0.8,-4.4)(0,-3.6)(0.8,-4.4)(1.6,-4)(0.4,-2.2)(1,-2.4)(2,-1.8)(3,-2.4)(3.2,-2)(3,-1.6)
\end{pspicture}
\caption{Ejemplo de Árbol con Raíz.\label{fig:expRootTrees}}
\end{center}
\end{figure}


\paragraph*{Nota:} Siendo $R_1$, $R_2$, $\ldots$, $R_n$ subárboles de $R$ de Izquierda a Derecha.

\begin{enumerate}[I.]

\item{ \textbf{Preordén} (Raíz, Izquierda, Derecha): Parte de la raíz $r$ para recorrer los vértices de: 
$R_1$, 
$R_2$, $\ldots$, $R_n$ en Preordén. 

\ejem En el caso de la Figura \ref{fig:expRootTrees}: $\{a,b,c,f,g,h,d,e\}$
}
\item{ \textbf{Postordén} (Izquierda, Derecha, Raíz): Recorre los vértices: $R_1$, $R_2$, $\ldots$, $R_n$ 
en 
Postordén para terminar finalmente en $r$.

\ejem En el caso de la Figura \ref{fig:expRootTrees}: $\{b,f,g,h,c,d,e,a\}$
}
\item{ \textbf{Inordén} (Izquierda, Raíz, Derecha): Si $r$ contiene: $R_1$, $R_2$, $\ldots$, $R_n$ entonces recorre los nodos de 
izquierda a derecha $R_i$ para volver a $r$ y recorrer en Inordén $R_{i+1}$ 
hasta finalizar en $R_{n}$.

\ejem En el caso de la Figura \ref{fig:expRootTrees}: $\{b,a,f,c,g,h,d,e\}$
}

\end{enumerate}

%\begin{figure}[h]
\centering
\begin{subfigure}[A]{1\textwidth}
\centering
\begin{pspicture}(-4,-5.2)(9,1.3)%\psgrid

\pspolygon[fillstyle=solid,fillcolor=white](-1,1)(9.5,1)(9.5,-5)(-1,-5)
\psline[linestyle=dotted, linecolor=black,linewidth=1pt]{-}(4,0)(0.8,-1.6)
\pscurve[linestyle=dotted, linecolor=black,linewidth=1pt]{->}(0.8,-1.6)(0,-1.6)(0.1,-2.4)(2.4,-1.6)(-0.4,-4.2)(0.5,-4.8)(1.7,-3.8)(2.3,-4)(3,-4.5)(4,-4)(4.7,-3.8)(5.3,-4.5)(6.2,-4.5)(6.4,-3.6)(5,-3)(6,-2.4)(7,-1.8)(8,-2.8)(8.8,-2.2)(8.6,-1.6)(8.2,-1.4)

\pstree[treesep=2cm,levelsep=2cm]{\Tcircle{$a$}}%
{%
\Tcircle{$b$}
\pstree{\Tcircle{$c$}}%
               {
               \Tcircle{$f$}
               \Tcircle{$g$}
               \Tcircle{$h$}
               }%
\Tcircle{$d$}
\Tcircle{$e$}
}


\end{pspicture}

\caption{Recorrido en Preordén.}

\end{subfigure}%
\quad
\begin{subfigure}[B]{1\textwidth}
\centering
\begin{pspicture}(-4,-5.2)(9,1.3)%\psgrid

\pspolygon[fillstyle=solid,fillcolor=white](-1,1)(9.5,1)(9.5,-5)(-1,-5)

\pscurve[linestyle=dotted, linecolor=black,linewidth=1pt]{-}(0.8,-1.4,)(0,-1.6)(0.1,-2.7)(-0.4,-4.2)(0.5,-4.8)(1.7,-3.8)(2.3,-4)(3,-4.5)(4,-4)(4.7,-3.8)(5.3,-4.5)(6.2,-4.5)(6.4,-3.6)(5,-3)(3.7,-2)(4.5,-1.2)(5,-2.4)(7,-2.4)(8,-2.8)(8.8,-2.2)(8.6,-1.6)(8.2,-1.4)(7.6,-1.5)
\psline[linestyle=dotted, linecolor=black,linewidth=1pt]{->}(7.6,-1.5)(4.7,0)
\pstree[treesep=2cm,levelsep=2cm]{\Tcircle{$a$}}%
{%
\Tcircle{$b$}
\pstree{\Tcircle{$c$}}%
               {
               \Tcircle{$f$}
               \Tcircle{$g$}
               \Tcircle{$h$}
               }%
\Tcircle{$d$}
\Tcircle{$e$}
}

\end{pspicture}

\caption{Recorrido en Postordén.}

\end{subfigure}
\quad
\begin{subfigure}[C]{1\textwidth}
\centering
\begin{pspicture}(-4,-5.2)(9,1.3)\psgrid

\pspolygon[fillstyle=solid,fillcolor=white](-1,1)(9.5,1)(9.5,-5)(-1,-5)

\pscurve[linestyle=dotted, linecolor=black,linewidth=1pt]{->}(0.8,-1.4,)(0,-1.6)(0.1,-2.4)(2.4,-1.6)(-0.4,-4.2)(0.5,-4.8)(1.7,-3.8)(2.3,-4)(3,-4.5)(4,-4)(4.7,-3.8)(5.3,-4.5)(6.2,-4.5)(6.4,-3.6)(5,-3)(6,-2.4)(7,-1.8)(8,-2.8)(8.8,-2.2)(8.6,-1.6)(8.2,-1.4)

\pstree[treesep=2cm,levelsep=2cm]{\Tcircle{$a$}}%
{%
\Tcircle{$b$}
\pstree{\Tcircle{$c$}}%
               {
               \Tcircle{$f$}
               \Tcircle{$g$}
               \Tcircle{$h$}
               }%
\Tcircle{$d$}
\Tcircle{$e$}
}

\end{pspicture}

\caption{Recorrido en Inordén.}

\end{subfigure}
\caption{Recorridos para Árboles \textit{m-arios}.}

\end{figure}


{\defn Los \index{\'Arboles Binarios}Árboles Binarios son de tipo \textit{2-ario}, es decir: $m=2$.}


%\nocite{book/es/matematicaDiscreta/fMerayo, book/es/calculo/jBurgos, book/mdysa, book/alygc, dpp}