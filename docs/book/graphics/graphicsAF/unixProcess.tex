\begin{figure}[h]
\centering
\begin{pspicture}(4,-11)%\psgrid
\psline[linecolor=black,linewidth=1pt]{-}(0,0)(4,0)
\psline[linecolor=black,linewidth=1pt]{-}(0,0)(0,-11)
\psline[linecolor=black,linewidth=1pt]{-}(4,0)(4,-11)
\rput(2,-0.5){Área de Sistema}
\psline[linecolor=black,linewidth=1pt]{-}(0,-1)(4,-1)
\psline[linecolor=black,linewidth=1pt,linestyle=dashed](-1,-1)(5,-1)
\rput(6,-1.5){Área de Usuario}
\rput(2,-1.5){Bibliotecas dinámicas}
\psline[linecolor=black,linewidth=1pt]{-}(0,-2)(4,-2)
\rput(2,-2.5){Cabecera}
\psline[linecolor=black,linewidth=1pt]{-}(0,-3)(4,-3)
\rput(2,-3.5){Código}
\psline[linecolor=black,linewidth=1pt]{-}(0,-4)(4,-4)
\rput(2,-4.5){Datos}
\psline[linecolor=black,linewidth=1pt]{-}(0,-5)(4,-5)
\rput(2,-5.5){Pila}
\psline[linecolor=black,linewidth=1pt]{-}(0,-6)(4,-6)
\rput(2,-6.5){Puntero a contexto}
\psline[linecolor=black,linewidth=1pt]{-}(0,-7)(4,-7)
\rput(2,-7.5){Inf. dinámica}
\psline[linecolor=black,linewidth=1pt]{-}(0,-8)(4,-8)
\rput(2,-8.5){Argumentos}
\psline[linecolor=black,linewidth=1pt]{-}(0,-9)(4,-9)
\rput(2,-9.5){Entorno}
\psline[linecolor=black,linewidth=1pt]{-}(0,-10)(4,-10)
\rput(2,-10.5){Nombre}
\psline[linecolor=black,linewidth=1pt]{-}(0,-11)(4,-11)
\end{pspicture} 
\caption{Modelo de proceso en UNIX.}
\end{figure}