\section{Introducción a los Lenguajes Formales (LFs)}

\subsection{Definiciones}

\defn Un Lenguaje Formal se compone de un conjunto de signos finitos y unas
leyes para operar con ellos.

\defn Al conjunto de símbolos de un lenguaje se les denomina \textit{Alfabeto},
denotado como $\Sigma$.

\defn Al conjunto de leyes que describen al lenguaje se les denomina
\textit{sintaxis}.

{\cor Por tanto una palabra derivada de un alfabeto pertenecerá (será propio del
lenguaje) si cumple las leyes formales del mismo.}

\defn Para todos los lenguajes existe la palabra vacía, que se denota en este
texto mediante el símbolo $\lambda$.

{\cor Por lo tanto:

\begin{equation}
|\lambda| = 0
\end{equation}

}

\begin{figure}[h]
\begin{center}
\begin{pspicture}(-1,0)(7,4)%\psgrid
\rput(3,1){Teoremas}
\rput(3,2){Lenguajes\ Formales\ (LFs)}
\rput(3,3){Cadenas\ de\ Caracteres}

\pspolygon(1.8,0.6)(4.2,0.6)(4.2,1.4)(1.8,1.4)

\pspolygon(0.2,0.3)(5.8,0.3)(5.8,2.4)(0.2,2.4)

\pspolygon(-0.2,0)(6.2,0)(6.2,3.6)(-0.2,3.6)

\end{pspicture}
\caption{Relación entre: Teoremas, LFs y Cadenas de Caracteres.}
\end{center}
\end{figure}


\ejem Para el alfabeto $O = \{0,1\}$ y la palabra $p$, se dice
que dicha palabra pertenece al alfabeto si cumple con la sintaxis:

\begin{equation}
p\ \subset\ O\ \backslash\ p_0 = \lambda,\ p_1 =[01],\ p_2=[0101],\ \ldots,\ p_n
= {CONCAT}_{i = 0}^{i = n}{{[01]}_i} \equiv {[01]}^*
\end{equation}


\subsection{Especificación de los LFs}

Los Lenguajes Formales se pueden describir por diversos métodos, sobre los que
destacan:

\begin{enumerate}[i.]
\item Mediante cadenas producidas por una gramática de Chomsky. Ver sección
(\ref{sec:jerarquiaChomsky})
\item Por medio de una Expresión Regular. Ver sección
(\ref{sec:expresionesRegulares})
\item Por cadenas aceptadas por un Autómata. Ver sección
(\ref{sec:aut})
\end{enumerate}

\subsection{¿Qué diferencia a un Lenguaje Natural (Humano) de un LF?}

Para responder a esta pregunta, debemos aclarar que entendemos por Lenguaje
Natural. Los Lenguajes Naturales tienen estructuras básicas en común con los
Lenguajes Formales (de hecho la especificación formal se basa en el Lenguaje
Humano).

El denominador común es la palabra como unidad estructural para construir
oraciones. Por ello se tiene un alfabeto $\Sigma$ para los Lenguajes Naturales, que es
finito. La diferencia real entre estas dos formas de lenguajes radica en la
polisemia (distintos significados) que tiene una palabra dentro de una oración
(semántica) es decir, el significado varía según su posición y el contexto en el
que se formula.

\ejem Dados las siguientes palabras: 

\begin{equation}
\{Javier,compr\acute{o},una,casa\} 
\end{equation}

Se puede construir la frase:

\begin{equation}
Javier\ compr\acute{o}\ una\ casa 
\end{equation} 

que sintáctica y semánticamente es correcta, pero la oración:

\begin{equation}
Una\ casa\ compr\acute{o}\ Javier 
\end{equation}

es sintácticamente correcta pero no semánticamente.


Por supuesto otra característica que diferencia a estos dos lenguajes es que un
Lenguaje Formal como el Castellano ha sido perfeccionado a lo largo del tiempo.
Con esto decimos que los Lenguajes Naturales evolucionan y están directamente
relacionados con el tiempo.


\section{Gramáticas Independientes de Contexto}

\defn Las Gramáticas Independientes de Contexto (GIC) se describen mediante una Tupla de 4 elementos:

\begin{equation}
G= (T, E, S, P)
\end{equation}

\paragraph*{Donde:}

\begin{enumerate}[i.]

\item $T$: Se trata de un alfabeto compuesto de símbolos no terminales.

\item $E$: {Se trata de un alfabeto compuesto de símbolos terminales.

\form $T \cap E = \varnothing$ 

}
\item $S$: Variable símbolo inicial de la gramática $S \in T$.

\item $P$: Se trata del conjunto de reglas de producción del tipo:

\begin{equation}
A\ {\longrightarrow}\ w
\end{equation}

\paragraph*{Donde:}

\begin{enumerate}

\item $A$ es la cabeza de la producción.

\item $w$ es el cuerpo de la producción.

\end{enumerate}



\end{enumerate}

\defn El Lenguaje generado por una gramática se denota como:

\begin{equation}
L(G): \{w \in \Sigma^*: S \stackrel{+}{\longrightarrow} w\}
\end{equation}

\ejem Gramática para $L(G) = a^*$ \\

T = \{a\} \\

E = \{S\} \\

S = \{$\lambda$\} \\

$ $P$\ = \begin{cases}

$S$\ {\longrightarrow}\ $aS$ \\

$aS$\ \stackrel{*}{\longrightarrow}\ $a$\ldots $aS$ \\

$aS$\ {\longrightarrow}\ $a$\ldots $a$ \\

\end{cases}$
     
%\subsection{Axiomas}

\subsection{Derivaciones}

\defn Definimos derivación como el conjunto de producciones o generaciones donde una regla $w_i$ iteran $[0,\infty]$ veces sobre si misma $w_n$

\form Existen dos tipos de producciones:

\begin{enumerate}[i.]

\item $w \Longrightarrow^+ w^*$: Derivaciones de $[1,\infty]$ veces.

\item $w \Longrightarrow^* w^*$: Derivaciones de $[0,\infty]$ veces.

\end{enumerate}

% 
% {\cor Las derivaciones pueden se en cero o más pasos: $\{DERIVACION^*\}$.
% 
% {\cor Las derivaciones se aplican al analisis sintáctico.}
% 
% \defn Se tiene como \textit{Derivación por la Izquierda} para un símbolo
% $\sigma_N$ cuando este es reemplazado siempre y se situa en la parte izquierda
% de la regla de producción.
% 
% \ejem Para la siguiente gramática:


\subsubsection{Representación mediante Árboles}
\defn Cualquier tipo de derivación puede ser representada gráficamente mediante
un Árbol de Derivación.

\defn Dicho Árbol se construye de la siguiente manera: 

\begin{enumerate}[i.]

\item La raíz se etiqueta con el símbolo inicial de la gramática $S$.

\item Cada nodo se etiqueta con un símbolo no terminal $E$.

\item Las hojas del árbol son etiquetadas con un símbolo terminal o $\lambda$.

\item Si la derivación es del tipo: $A \rightarrow s_1, s_2, \ldots, s_k$ para $s_i \in (V \subset \Sigma^*)$:

\begin{equation}
A\ \text{tiene}\ k\ \text{descendientes escritos de izquierda a derecha.} 
\end{equation}

 



\end{enumerate}


\begin{figure}[h]
\begin{center}
\begin{pspicture}(-4,-5.5)(4,0.5)%\psgrid
\pstree[treesep=2cm,levelsep=2cm]{\Tcircle{$A$}}%
{%
\pstree{\Tcircle{$B$}}%
               {
               \Tcircle{$d$}
               \Tcircle{$e$}
               }%
\pstree{\Tcircle{$C$}}%
               {
               \Tcircle{$f$}
               \Tcircle{$g$}
               \Tcircle{$h$}
               }%
}
\end{pspicture}
\caption{Ejemplo genérico de Árbol de Derivación.}
\end{center}
\end{figure}

\ejem Para la siguiente gramática de la Figura (\ref{fig:exampleDerGra}) obtener la cadena $abba$

\begin{equation}
S \rightarrow AB \rightarrow AbBa \rightarrow abBa \rightarrow abba
\end{equation}


\begin{figure}[h]
\centering
\begin{subfigure}[A]{0.4\textwidth}
\vspace{48 pt}

T = \{a,b\} \\
 
E = \{S,A,B\} \\

S = \{$\lambda$\} \\

$ $P$\ = \begin{cases}

  $S$\ {\longrightarrow}\ $AB$\ |\ $AaB$ \\
  
  $A$\ {\longrightarrow}\ $aA$\ |\ $a$ \\

  $B$\ {\longrightarrow}\ $bBa$\ |\ $b$ \\

\end{cases}$

\vspace{30 pt}   

\caption{Gramática.}
\end{subfigure}%
\quad
\begin{subfigure}[B]{0.3\textwidth}
\centering
\begin{pspicture}(0,-7)(4,0.5)%\psgrid
\pstree[treesep=2cm,levelsep=2cm]{\Tcircle{$S$}}%
{%
\pstree{\Tcircle{$A$}}%
               {
               \Tcircle{$a$}
               }%
\pstree{\Tcircle{$B$}}%
               {
               \Tcircle{$b$}
\pstree{\Tcircle{$B$}}%
               {
               \Tcircle{$b$}
               }%
               \Tcircle{$a$}
               }%
}
\end{pspicture}

\caption{Árbol de Derivación.}

\end{subfigure}

\caption{Ejemplo Árbol de Derivación para obtener: $abba$}\label{fig:exampleDerGra}

\end{figure}


\section{Jerarquía de Chomsky (JC)}\label{sec:jerarquiaChomsky}

Noam Avram Chomsky propuso en 1956 la formalidad de las gramáticas. A través de las reglas de producción en las que se basan las mismas, estableció un orden o jerarquía de las gramáticas y los lenguajes asociados.

\paragraph*{Nota:} Dichas gramáticas son subconjuntos unas de otras de modo que, el universo de las gramáticas son las de Tipo 0.

\begin{equation}
\text{T}_3 \subset \text{T}_2 \subset \text{T}_1 \subset \text{T}_0
\end{equation}


\begin{table}[h]
\begin{center}

\begin{tabular}{|c|l|l|}\hline
\textbf{Nivel} & \textbf{Lenguaje} & \textbf{Autómata} \\ \hline
\hline
0 & Recursivamente Enumerable (LRE) & Máquina de Turing (MT) \\ \hline
1 & Dependiente del Contexto (LSC) & Autómata Linealmente Acotado \\ \hline
2 & Independiente del Contexto (LLC) & Autómata a Pila \\ \hline
3 & Regular (LR) & Autómata Finito \\ \hline
\end{tabular}

\caption{Relación entre: Nivel, Lenguaje y Autómata en la JC.}

\end{center}
\end{table}

\subsection{Niveles}

La Jerarquía de Chomsky\endnote{\textbf{Avram Noam Chomsky} es unos de los
mayores
lingüistas del siglo XX. Nació en Filadelfia el 7 de diciembre de 1928. A través
de sus estudios sobre la formalidad de los lenguajes enuncia su teoría sobre
``La adquisición individual'' dónde intenta dar explicación a las formalidades
de los lenguajes naturales a través de representaciones formales.} contiene los
siguientes niveles.

\begin{enumerate}[I.]

\item Gramáticas de Tipo 0 (No Restrictivas): Estas gramáticas generan Lenguajes sin Restricciones. 
{
\paragraph*{Reglas de producción:}

\begin{equation}
u\ {\longrightarrow}\ v 
\end{equation}

\paragraph*{Donde:}

\begin{enumerate}[i.]

\item $u \in \Sigma^+$ 

\item $v \in \Sigma^*$

\item $u= xAy$

\begin{enumerate}

\item $x,y \in \Sigma^*$

\item $A \in E$

\end{enumerate}

\end{enumerate}

\paragraph*{Notas:}

\begin{enumerate}[i.]

\item La cabeza de la producción no puede ser palabra vacía $\lambda$.

\item La cabeza de la regla debe contener al menos un símbolo no terminal.

\end{enumerate}

\ejem Sea la gramática $G$: \\

T = \{a,b\} \\
 
E = \{A,B,C\} \\

S = \{A\} \\

$ $P$\ = \begin{cases}

$A$\  {\longrightarrow}\ $aABC$\ |\ $abC$ \\
  
$CB$\ {\longrightarrow}\ $BC$ \\

$bB$\ {\longrightarrow}\ $bb$ \\

$bC$\ {\longrightarrow}\ $b$ \\
  
\end{cases}$
}
\item Gramáticas de Tipo 1 (Sensibles al Contexto): Dichas gramáticas generan Lenguajes dependientes de Contexto.
{
\paragraph*{Reglas de producción:}

\begin{equation}
xAy\ {\longrightarrow}\ xvy 
\end{equation}

\paragraph*{Donde:}

\begin{enumerate}[i.]

\item $v \in \Sigma^+$ 

\item $x,y \in \Sigma^*$

\end{enumerate}

\paragraph*{Nota:} Se admite $S\ {\longrightarrow}\ \lambda$

\ejem Sea la gramática $G$: \\

T = \{S,B,C\} \\
 
E = \{a,b,b\} \\

S = \{$\lambda$\} \\

$ $P$\ = \begin{cases}

$S$\  {\longrightarrow}\ $aSBc$\ |\ $aBC$ \\
  
$bB$\ {\longrightarrow}\ $bb$ \\

$bC$\ {\longrightarrow}\ $bc$ \\

$CB$\ {\longrightarrow}\ $BC$ \\
  
$cC$\ {\longrightarrow}\ $cc$ \\

$aB$\ {\longrightarrow}\ $ab$ \\

\end{cases}$
}
\item Gramáticas de Tipo 2 (Libres de Contexto): Las gramáticas de Tipo 2 generan Lenguajes independientes de Contexto.
{
\paragraph*{Reglas de producción:}

\begin{equation}
A\ {\longrightarrow}\ v 
\end{equation}

\paragraph*{Donde:}

\begin{enumerate}[i.]

\item $v \in \Sigma^*$

\item $A \in E$

\end{enumerate}

\paragraph*{Nota:} La mayor parte de los Lenguajes de Programación pueden describirse a través de esta tipología.

\ejem Ver Figura (\ref{fig:exampleDerGra}). 
}
\item Gramáticas de Tipo 3 (Regulares): Estas gramáticas generan Lenguajes Regulares.
{
\paragraph*{Reglas de producción:} Dichas gramáticas se clasifican en dos grupos:

\begin{enumerate}[i.]

\item Gramáticas Lineales por la Izquierda: \\
{ \\
$ $P$\ = \begin{cases}

$A$\  {\longrightarrow}\ $a$ \\
  
$A$\  {\longrightarrow}\ $Va$ \\

$S$\  {\longrightarrow}\ \lambda \\
  
\end{cases}$
}
\item Gramáticas Lineales por la Derecha: \\
{ \\
$ $P$\ = \begin{cases}

$A$\  {\longrightarrow}\ $a$ \\
  
$A$\  {\longrightarrow}\ $aV$ \\

$S$\  {\longrightarrow}\ \lambda \\
  
\end{cases}$
}

\end{enumerate}

\paragraph*{Donde:}

\begin{enumerate}[i.]

\item $a \in T$

\item $A,V \in E$

\end{enumerate}

\ejem Para $G$ sobre:

\begin{enumerate}[i.]

\item Gramáticas Lineales por la Izquierda: \\
{ \\
T = \{0,1\} \\
 
E = \{A,B\} \\

S = \{A\} \\

$ $P$\ = \begin{cases}

$A$\  {\longrightarrow}\ $B1$ \\
  
$B$\  {\longrightarrow}\ $A0$ \\

\end{cases}$
}
\item Gramáticas Lineales por la Derecha: \\ 
{ \\
T = \{0,1\} \\
 
E = \{A,B\} \\

S = \{A\} \\

$ $P$\ = \begin{cases}

$A$\  {\longrightarrow}\ $1B$ \\
  
$B$\  {\longrightarrow}\ $0A$ \\

\end{cases}$
}
\end{enumerate}
}
\end{enumerate}

\section{Descripción de Gramáticas Formales}

\subsection{Backus-Naur Form}

Backus\endnote{\textbf{John
Backus} fue un
importante científico de computación nacido en el estado de Filadelfia (EEUU),
el 3 de diciembre de 1924. Es prestigioso ganador del
Premio Turing en el año 1977 debido en gran parte a sus trabajos sobre
especificación de lenguajes de alto nivel.


Backus estuvo dentro del primer proyecto de FORTRAN, el primer lenguaje de alto
nivel en la historia de la computación. Además su notación sobre gramáticas sentó
las bases para ALGOL. 

Su famosa notación es Backus Naur Form (BNF) que describe un autómata a partir
del un conjunto de símbolos.}-Naur\endnote{\textbf{Peter Naur} es un prestigio científico
danés nacido el 25
de octubre de 1928 ganador del Premio Turing en 2005.
Su trabajo más representativo consiste en sentar junto a John Backus la notación
para especificación de autómatas para lenguajes formales.} Form se trata una de las dos notaciones más importantes para Gramáticas Libres de Contexto.

John Backus, diseñador de lenguajes en IBM propuso para el Lenguaje de Programación IAL (conocido como ALGOL 58) un meta-lenguaje. Posteriormente con la publicación de ALGOL 60 la fórmula BNF se simplificó y perfeccionó. 

BNF se trata de un conjunto de reglas derivativas del tipo: \texttt{<symbol> ::= \_expression\_}

\paragraph*{Donde:} 

\begin{enumerate}[i.]

\item \texttt{symbol}: Es un símbolo No Terminal. 

\item \texttt{\_expression\_}: Consiste en un conjunto de símbolos o de secuencias (separadas por el carácter '|') donde el símbolo ``más a la izquierda'' es el Terminal.

\item \texttt{'::='}: Operador de asignación. Indica que el símbolo de la izquierda es sustituido por la expresión de la derecha.

\end{enumerate}


\begin{verbatim}
<syntax>         ::= <rule> | <rule> <syntax>
<rule>           ::= <opt-whitespace> "<" <rule-name> ">" <opt-whitespace> "::=" 
					  <opt-whitespace> <expression> <line-end>
<opt-whitespace> ::= " " <opt-whitespace> | ""
<expression>     ::= <list> | <list> "|" <expression>
<line-end>       ::= <opt-whitespace> <EOL> | <line-end> <line-end>
<list>           ::= <term> | <term> <opt-whitespace> <list>
<term>           ::= <literal> | "<" <rule-name> ">"
<literal>        ::= '"' <text> '"' | "'" <text> "'" 
\end{verbatim}

\paragraph*{EBNF:} Existen distintas variantes sobre BNF. Las más popular es Extended Backus-Naur Form (EBNF) que incorpora operadores de Expresiones Regulares como:

\begin{enumerate}[i.]

\item $a^+$: Repetir $a$ de $[1,\infty]$ veces.

\item $a^*$: Repetir $a$ de $[0,\infty]$ veces.

\end{enumerate}


\subsection{Wijngaarden Form}

Var Wijngaarden Form (también conocida como vW-grammar p W-grammar) se trata de una técnica para definir Gramáticas Libres de Contexto en un número finito de reglas.

Las W-grammars se basan en la idea de que los símbolos No Terminales intercambian información entre los nodos y el árbol de ``parseo''.

El primer uso de estas gramáticas fue en ALGOL 68.

\section{Analizadores Sintácticos}

\defn La función del Analizador Sintáctico es la de relacionar el flujo de \textit{tokens} elaborada por el Analizador Léxico y comprobar que la secuencia de estos \textit{tokens} se corresponde con los patrones sintácticos (las reglas) del lenguaje.

{\cor El Analizador Sintáctico es el encargado de elaborar el árbol de análisis del código fuente sobre el que trabajaran el resto de fases del compilador.}

\defn El Analizador Sintáctico es capaz de detectar errores en segunda fase, es decir, en la correspondencia entre \textit{token} y \textit{patrón sintáctico}.

{\cor Al contrario que ocurre con los errores léxicos, los errores sintácticos tienen una gran consistencia\endnote{Están perfectamente definidos en el Lenguaje
de Programación.}.}

\begin{figure}[h]
\begin{center}
\begin{pspicture}(14,5)%\psgrid
\pspolygon[fillstyle=solid,fillcolor=white](0.5,3.7)(3.5,3.7)(3.5,4.4)(0.5,4.4)
\rput(2,4){{Código Fuente}}
\pspolygon[fillstyle=solid,fillcolor=white](5.6,3.7)(8.4,3.7)(8.4,4.4)(5.6,4.4)
\rput(7,4){{\tt gp1990la}}
\pspolygon[fillstyle=solid,fillcolor=white](5.2,0.7)(8.8,0.7)(8.8,1.4)(5.2,1.4)
\rput(7,1){{Tabla de Símbolos}}
\psline[linecolor=black,linewidth=1pt]{->}(3.6,4)(5.5,4)
\psline[linecolor=black,linewidth=1pt]{<->}(7,1.5)(7,3.6)
\pspolygon[fillstyle=solid,fillcolor=white](10.6,3.7)(13.4,3.7)(13.4,4.4)(10.6,4.4)
\rput(12,4){{\tt gp1990sa}}
\psline[linecolor=black,linewidth=1pt]{->}(8.6,4)(10.5,4)

\pscurve[linecolor=black,linewidth=1pt]{<->}(9,1)(11.6,1.5)(12,3.6)

\end{pspicture}
\caption{Relación entre el Analizador Léxico, Analizador Sintáctico y el Programa Fuente.}
\end{center}
\end{figure}


\section{Análisis Sintáctico Descendente}

\defn Los Analizadores de Tipo Descendente (\textit{Top-Down-Parser}) generan un árbol sintáctico a partir de una de entra $w \in L$. 

{\cor Se trata de un recorrido en Preorden.} (Var Apartado \ref{subsec:mTrees})

{\cor Este análisis es análogo al proceso de derivación de una cadena por la izquierda.}

\paragraph*{Tipos:}


\begin{enumerate}

\item ASDR (Analizadores Sintácticos Descendentes Recursivos): Son analizadores ASD que basan su lógica en la recursión (Ver Apartado \ref{subsec:LL1})

\item ASDnR (Analizadores Sintácticos Descendentes no Recursivos): Se trata de analizadores ASD dirigos por pila o tabla. Su lógica de derivación es de la forma:

\begin{equation}
S\ \stackrel{*}{\longrightarrow}\ w \alpha \\
\end{equation}

Donde: 

\begin{enumerate}[i.]

\item $w$: Es la cadena de entrada.

\item $\alpha$: Es la Tabla/Pila de símbolos gramáticales.

\end{enumerate}

\end{enumerate}

\subsection{Autómatas LL(1)}
\label{subsec:LL1}

\defn Se trata de ASD con $k=1$ tokens de predicción.
\\

Este tipo de análisis se divide en tres fases o etapas:

\begin{enumerate}[I.]

\item Conjunto de los PRIMEROS: {

\defn Si $\alpha$ es una forma sentencial compuesta por una concatenaci\'on de
s\'imbolos $PRIM(\alpha)$ es el conjunto de terminales o $\lambda$ que pueden
aparecer iniciando las cadenas que pueden derivar de $\alpha$.


\form $a \in PRIM(\alpha)$ si $a \in (T \cup \{\lambda\})\ \wedge \stackrel{*}{\longrightarrow} a\beta$

\regl Para calcular el conjunto de los PRIMEROS tenemos:

\begin{enumerate}[i.]

\item Si $\alpha \equiv \lambda \Rightarrow PRIM\{\lambda\} = \{\lambda\}$.

\item Si $\alpha \in (T \cup E)^+ \Rightarrow \alpha = a_1, a_2, \ldots, a_n$ demuestra:
 
\begin{enumerate}[a.]

\item Si $a_1 \equiv a \in T \Rightarrow PRIM(\alpha) = \{a\}$.  

\item Si $a_1 \equiv A \in E$ para:

\begin{enumerate}[1.]

\item $PRIM(A) = \cup^n_{i=1} PRIM(\alpha_i)\ \diagup\ \alpha_i \in P$

\item Si $PRIM(A)\ \diagup\ \lambda \in PRIM(A)\ \wedge\ A$ no es el último símbolo de $\alpha \Rightarrow PRIM(\alpha) = (PRIM(A)-\{\lambda\}) \cup PRIM (a_2, a_3, \ldots, a_n)$

\item Si $A$ es el último símbolo de $\alpha \vee \lambda \notin PRIM(A) \Rightarrow PRIM(\alpha) = PRIM(A)$

\end{enumerate}


\end{enumerate}


\end{enumerate}

\ejem Dada la gramática:
\label{eq:gramLL1}
\begin{equation}
G= (T, E, S, P)
\end{equation}

T = \{ +, -, *, $\diagup$, num, id\} \\

E = \{\ E, E$^\prime$, T, T$^\prime$, F\} \\

S = \{E\} \\

$ $P$\ = \begin{cases}

$E$\ {\longrightarrow}\ $T E$^\prime \\

$E$^\prime\ {\longrightarrow}\ +\ $T E$^\prime\ |\ -\ $T E$^\prime\  |\ \lambda\\

$T$\ {\longrightarrow}\ $F T$^\prime \\

$T$^\prime {\longrightarrow}\ *\ $F T$^\prime\ |\ \diagup\ $F T$^\prime\ |\ \lambda \\

$F$ {\longrightarrow}\ ($E$)\ |\ $num$\ |\ $id$ \\

\end{cases}$
\\

\begin{equation}
example
\end{equation}

}

\item Conjunto de los SIGUIENTES: {
\defn Si $A$ es un símbolo inicial no terminal de la gramática, $SIG(A)$ es el conjunto de terminales $ +\ \{\$\}$ que pueden aparecer a continuación de $A$ en alguna forma sentencial derivada del símbolo inicial.


\form $a \in SIG(A)$ si $a \in (T \cup \{\$\}) \wedge\ \exists\ \alpha, \beta\ \diagup\ S 
\Longrightarrow^{*} \alpha Aa \beta$

\regl Para calcular el conjunto de los siguientes tenemos:

\begin{enumerate}[i.]
\item Partimos de que: $SIG(A) = \varnothing$

\item Si $A$ es símbolo inicial $\Rightarrow SIG(A) = SIG(A) \cup \{\$\}$

\item Dada la regla: $B \rightarrow \alpha A \beta \Rightarrow SIG(A) = SIG(A) \cup (PRIM(\beta) - \{\lambda\})$

\item Dada la regla: $B \rightarrow \alpha A \vee B \rightarrow \alpha A \beta\ \diagup\ \lambda \in PRIM(\beta) \Rightarrow SIG(A) = SIG(A) \cup SIG(B)$

\item Repetir los pasos 3 y 4 hasta que no se puedan añadir más símbolos a $SIG(A)$

\end{enumerate}

\ejem Para la gramática (\ref{eq:gramLL1})

\begin{equation}
example
\end{equation}
}
\item Conjunto de PREDICCIÓN: {
\defn Para una gramática ASDP con símbolo no terminal $\sigma_N$ se debe consultar el símbolo de entrada y buscar en cada regla de dicho símbolo. Si los conjuntos de producción son disjuntos, el AS podrá construir una derivación hacia la izquierda de la cadena de entrada. 


\form $PRED(A \rightarrow \alpha) \Rightarrow$

\begin{enumerate}[i.]
\item Si $\alpha \in PRIM(\alpha) \Rightarrow (PRIM(\alpha) - \{\lambda\} \cup SIG(A)$
\item Si no $\Rightarrow PRIM(\alpha)$ 
\end{enumerate}
}
\end{enumerate}

\section{Análisis Sintáctico Ascendente}

\defn Los Analizadores de Tipo Ascendente (\textit{Bottom-Up-Parser}) generan un árbol desde la hojas a la raíz.

\paragraph*{Tipos:}

\begin{enumerate}

\item LR ():

\item LALR ():

\end{enumerate}

%\subsection{Autómatas LR}

\section{Yacc (Yet another compiler-compiler)}

Yacc se trata de un popular ``Front-End'' para construir compiladores a nivel sintáctico diseñado originalmente por S.C. Johnson en 1970. 

El análisis realizado por Yacc es del tipo LALR. 

El proceso para generar un ejecutable con Yacc es el que sigue:

\begin{enumerate}[i.]

\item Generar el fichero \texttt{source.y}

\item Compilar con Yacc \texttt{source.y} y generar por defecto \texttt{y.tab.c}

\item Compilar con \texttt{CC} (C Compiler) \texttt{y.tab.c} para obtener finalmente el ejecutable.

\end{enumerate}

La estructura de un fichero en Yacc es la que sigue:

\lstinputlisting[language=bash]{./programms/programms31/yaccSyntax}

\begin{enumerate}[I.]

\item Sección de Declaraciones: Dentro de este apartado existen a su vez, dos apartados:

\begin{enumerate}[i.]

\item Apartado de rutinas en C: Delimitada por los símbolos \{\% (apertura) \%\} (cierre) contiene las directivas del preprocesador además, de variables y definiciones necesarias para el resto del programa.

\item Apartado de Tokens: Establece los Tokens a utilizar en el programa. Son necesarios desde el punto de vista global del programa. A su vez, en relación con Lex, dichos Tokens son intercambiados entre ambos programas.

\end{enumerate}

\item {Sección de Reglas de Traducción: Se definen en el mismo, las acciones semánticas que se corresponde a su vez con instrucciones en Código C. Las reglas de producción son de la forma:

\begin{equation}
E\ \longrightarrow \ E\ +\ T\ |\ T
\end{equation}

Donde:

\begin{enumerate}[i.]

\item $E$: Es un símbolo No Terminal.

\item $T$: Es un símbolo Terminal.

\end{enumerate}

}
\item Apartado de Código en C: Se trata del conjunto de rutinas en C definidas por el desarrollador/programador. En el mismo apartado se establece o no la relación con LEX por medio de la función \texttt{yylex();}


Dicha relación se describe en la Figura (\ref{fig:relLexYacc})

\end{enumerate}

\begin{figure}
\begin{center}
\begin{pspicture}(0,0)(10,2)%\psgrid
\pspolygon[fillstyle=solid,fillcolor=white](0.5,0.7)(3.5,0.7)(3.5,1.4)(0.5,1.4)
\rput(2,1){Lex}
\pspolygon[fillstyle=solid,fillcolor=white](6.5,0.7)(9.5,0.7)(9.5,1.4)(6.5,1.4)
\rput(8,1){Yacc}

\psline[linecolor=black,linewidth=1pt]{->}(3.7,1.2)(6.3,1.2)
\psline[linecolor=black,linewidth=1pt]{<-}(3.7,0.8)(6.3,0.8)

\rput(5,1.5){\texttt{yylex();}}
\rput(5,0.5){\texttt{yyval}}
\end{pspicture}
\caption{Relación entre el primitivas de Lex y Yacc.}\label{fig:relLexYacc}
\end{center}
\end{figure}


%\section{Código fuente: gp1990sa.y}

%\nocite{book/gnu/gpc}