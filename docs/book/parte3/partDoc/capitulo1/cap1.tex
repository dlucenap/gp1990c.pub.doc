\section{Gramáticas}

%\subsection{Axiomas}

\subsection{Derivaciones}

\defn Se puede decribir la derivación de un símbolo no terminal $\sigma_N$ para
una cadena a través de las reglas de derivación sobre un símbolo inicial de dos
formas: \textit{Derivacion por la Izquierda} y \textit{Derivación por la
Derecha}.

% \form Si $A \textrightarrow \zeta \Rightarrow \alpha A \beta \wedge \alpha \zeta
% \beta$ son derivaciones.

{\cor Las derivaciones pueden se en cero o más pasos: $\{DERIVACION^*\}$.

{\cor Las derivaciones se aplican al analisis sintáctico.}

\defn Se tiene como \textit{Derivación por la Izquierda} para un símbolo
$\sigma_N$ cuando este es reemplazado siempre y se situa en la parte izquierda
de la regla de producción.

\ejem Para la siguiente gramática:

\defn Se tiene como \textit{Derivación por la Derecha} para un símbolo
$\sigma_N$ cuando este es reemplazado siempre y se situa en la parte derecha de
la regla de producción.

\ejem Para la siguiente gramática:

\subsubsection{Representación mediante Árboles}

\begin{figure}[h]
\begin{center}
\begin{pspicture}(0,-2)%\psgrid
\pstree[levelsep=1,nodesep=3pt]{\Tr[name=A]{A}}{%
	\pstree{\Tr[name=B]{B}}{%
	  \Tr[name=C]{C}
	  \Tr[name=D]{D}
	}
}
\end{pspicture}
\caption{Ejemplo genérico de \textit{Árbol de Derivación}.}
\end{center}
\end{figure}

\defn Cualquier tipo de derivación puede ser representada gráficamente mediante
un \textit{Árbol de Derivación}.

\ejem Para la siguiente gramática: $G = \{\sigma_T, \sigma_N, S, P)$

\vspace{5mm}

\hspace{2mm} \textbf{Dónde:}

\vspace{5mm}

\hspace{5mm}  %$\sigma_T \equiv Q$\newline

\hspace{5mm}  %$\sigma_N \equiv R$\newline

\hspace{5mm}  %$S \equiv Q$\newline

\hspace{5mm}  $P:$

\hspace{10mm} %$Q \rightarrow Q\ |\ QR$\newline

\hspace{10mm} %$R \rightarrow 0\ |\ 1\ |\ 2\ |\ 3\ |\ 4\ |\ 5\ |\ 6\ |\ 7\ |\ 8\
%|\ 9$

\vspace{5mm}

Para obtener el número 399: 

\begin{figure}[h]
\begin{center}
\begin{pspicture}(-4,-8.5)(4,0.5)%\psgrid
\pstree[treesep=2cm,levelsep=2cm]{\Tcircle[tnpos=a]{Q}}{%
   \pstree{\Tcircle[tnpos=l]{Q}}{%
      \pstree{\Tcircle[tnpos=l]{Q}}{%
      \pstree{\Tcircle[tnpos=r]{R}}{%
      \Tcircle{3}
      }
	  
      }
      \pstree{\Tcircle[tnpos=r]{R}}{%
      \Tcircle{9}
      }

  }
  \pstree{\Tcircle[tnpos=r]{R}}{%
  \Tcircle{9}
  }
  }
\end{pspicture}
\caption{\textit{Árbol de Derivación} para el Ejemplo:}
\end{center}
\end{figure}

\begin{equation}
Q \rightarrow QR \rightarrow QRR \rightarrow RRR \rightarrow 3RR \rightarrow 39R
\rightarrow 399
\end{equation}


\section{Introducción a los Lenguajes Formales (LFs)}

\subsection{Definiciones}

\defn Un Lenguaje Formal se compone de un conjunto de signos finitos y unas
leyes para operar con ellos.

\defn Al conjunto de símbolos de un lenguaje se les denomina \textit{Alfabeto},
denotado como $\Sigma$.

\defn Al conjunto de leyes que describen al lenguaje se les denomina
\textit{sintaxis}.

{\cor Por tanto una palabra derivada de un alfabeto pertenecerá (será propio del
lenguaje) si cumple las leyes formales del mismo.}

\defn Para todos los lenguajes existe la palabra vacía, que se denota en este
texto mediante el símbolo $\lambda$.

{\cor Por lo tanto:

\begin{equation}
|\lambda| = 0
\end{equation}

}

\begin{figure}[h]
\begin{center}
\begin{pspicture}(-1,0)(7,4)%\psgrid
\rput(3,1){Teoremas}
\rput(3,2){Lenguajes\ Formales\ (LFs)}
\rput(3,3){Cadenas\ de\ Caracteres}

\pspolygon(1.8,0.6)(4.2,0.6)(4.2,1.4)(1.8,1.4)

\pspolygon(0.2,0.3)(5.8,0.3)(5.8,2.4)(0.2,2.4)

\pspolygon(-0.2,0)(6.2,0)(6.2,3.6)(-0.2,3.6)

\end{pspicture}
\caption{Relación entre: Teoremas, LFs y Cadenas de Caracteres.}
\end{center}
\end{figure}


\ejem Para el alfabeto $O = \{0,1\}$ y la palabra $p$, se dice
que dicha palabra pertenece al alfabeto si cumple con la sintaxis:

\begin{equation}
p\ \subset\ O\ \backslash\ p_0 = \lambda,\ p_1 =[01],\ p_2=[0101],\ \ldots,\ p_n
= {CONCAT}_{i = 0}^{i = n}{{[01]}_i} \equiv {[01]}^*
\end{equation}


\subsection{Especificación de los LFs}

Los Lenguajes Formales se pueden describir por diversos métodos, sobre los que
destacan:

\begin{enumerate}
\item Mediante candenas producidas por una gramática de Chomsky. Ver sección
(\ref{sec:jerarquiaChomsky})
\item Por medio de una Expresión Regular. Ver sección
(\ref{sec:expresionesRegulares})
\item Por cadenas aceptadas por un Autómata.
\end{enumerate}

\subsection{¿Qué diferencia a un Lenguaje Natural (Humano) de un LF?}

Para responder a esta pregunta, debemos aclarar que entendemos por Lenguaje
Natural. Los Lenguajes Naturales tienen estructuras básicas en común con los
Lenguajes Formales (de hecho la especificación formal se basa en el Lenguaje
Humano).

El denominador común es la palabra como unidad estructural para construir
oraciones. Por ello se tiene un alfabeto $\Sigma$ para los Lenguajes Naturales, que es
finito. La diferencia real entre estas dos formas de lenguajes radica en la
polisemia (distintos significados) que tiene una palabra dentro de una oración
(semántica) es decir, el significado varía según su posición y el contexto en el
que se formula.

\ejem Dados las siguientes palabras: 

\begin{equation}
\{Javier,rompi\acute{o},la,ventana\} 
\end{equation}

Se puede construir la frase:

\begin{equation}
Javier\ rompi\acute{o}\ la\ ventana 
\end{equation} 

que sintáctica y semánticamente es correcta, pero la oración:

\begin{equation}
La\ ventana\ rompi\acute{o}\ Javier 
\end{equation}

es sintácticamente correcta pero no semánticamente.


Por supuesto otra característica que diferencia a estos dos lenguajes es que un
Lenguaje Formal como el Castellano ha sido perfeccionado a lo largo del tiempo.
Con esto decimos que los Lenguajes Naturales evolucionan y están directamente
relacionados con el tiempo.


\section{Jerarquía de Chomsky (JC)}\label{sec:jerarquiaChomsky}

\subsection{Definiciones}

\begin{table}[h]
\begin{center}

\begin{tabular}{|c|l|l|p{2,8cm}|}\hline
\textit{Nivel} & \textit{Lenguaje} & \textit{Autómata} & \textit{Producciones}
\\ \hline
\hline
0 & Recursivamente Enumerable (LRE) & Máquina de Turing (MT) & Sin restricciones
\\ \hline
1 & Dependiente del Contexto (LSC) & Autómata Linealmente Acotado & $\alpha A
\beta \rightarrow \alpha \gamma B$ \\ \hline
2 & Independiente del Contexto (LLC) & Autómata a Pila & $A \rightarrow \gamma$
\\ \hline
3 & Regular (LR) & Autómata Finito & $A \rightarrow aB$ \par $A \rightarrow a$
\\ \hline
\end{tabular}

\caption{Tabla de relaciona entre: Nivel, Lenguaje, Autómata y Producciones en
la JC.}

\end{center}
\end{table}

\subsection{Niveles}

La Jeraquía de Chomsky\endnote{\textbf{Avram Noam Chomsky} es unos de los
mayores
liguistas del siglo XX. Nació en Filadelfia el 7 de diciembre de 1928. A través
de sus estudios sobre la formalidad de los lenguajes enuncia su teoría sobre
``La adquisición individual'' dónde intenta dar explicación a las formalidades
de los lenguajes naturales a través de representaciones formales.} contiene los
siguientes niveles.

\paragraph*{Gramáticas de tipo 0 (No Restrictivas):}

\paragraph*{Reglas de producción:}

\begin{equation}
P = \{(p_1 \rightarrow p_2)\ |\ p_1 = xAy; p_1 \in \Sigma^+; p_2,x,y \in
\Sigma^*; A \in N\} 
\end{equation}

\ejem 

\begin{equation}
example
\end{equation}

\paragraph*{Gramáticas de tipo 1 (Sensibles al Contexto):}

\paragraph*{Reglas de producción:}

\begin{equation}
P = \{(S \rightarrow \lambda) \vee (xAy \rightarrow xp_2y)\ |\ p_2 \in \Sigma^+;
x,y \in \Sigma^*; A \in N\} 
\end{equation}

\ejem 

\begin{equation}
example
\end{equation}

\paragraph*{Gramáticas de tipo 2 (Libres de Contexto):}

\paragraph*{Reglas de producción:}

\begin{equation}
P = \{(S \rightarrow \lambda) \vee (A \rightarrow p_2)\ |\ p_2 \in \Sigma^+; A
\in N\} 
\end{equation}

\ejem 

\begin{equation}
example
\end{equation}

\paragraph*{Gramáticas de tipo 3 (Regulares):}

\paragraph*{Reglas de producción:}

\begin{equation}
P = \{(S \rightarrow \lambda) \vee (A \rightarrow aB) \vee (A \rightarrow a)\ |\
a \in T; A,B \in N\} 
\end{equation}

\begin{equation}
P = \{(S \rightarrow \lambda) \vee (A \rightarrow Ba) \vee (A \rightarrow a)\ |\
a \in T; A,B \in N\} 
\end{equation}

\ejem 

\begin{equation}
example
\end{equation}

\begin{table}[h]

\begin{center}

\begin{tabular}{|c|l|}\hline
\textit{Tipo} & \textit{Para producciones de tipo} $p_1 \longrightarrow p_2$ \\
\hline
\hline
0 & $(p_1 \rightarrow p_2)\ \equiv$ No restrictivas. \\ \hline
1 & $(S \rightarrow \lambda) \vee (xAy \rightarrow xp_2y) $ \\ \hline
2 & $(S \rightarrow \lambda) \vee (A \rightarrow v)$. \\ \hline
3 & $(S \rightarrow \lambda) \vee (A \rightarrow aB) \vee (A \rightarrow Ba)
\vee (A \rightarrow a)$ \\ \hline
\end{tabular}

\caption{Grados en la JC para producciones de tipo $p_1 \longrightarrow p_2$.}

\end{center}

\end{table}

\section{Descripción de Gramáticas Formales}

\subsection{Backus-Naur Form}

Backus\endnote{\textbf{John
Backus} fue un
importante científico de computación nacido en el estado de Filadelfia (EEUU),
el 3 de diciembre de 1924. Es prestigioso ganador del
Premio Turing en el año 1977 debido en gran parte a sus trabajos sobre
especificación de lenguajes de alto nivel.


Backus estuvo dentro del primer proyecto de FORTRAN, el primer lenguaje de alto
nivel en la hitoria de la computación. Además su notación sobre gramáticas sentó
las bases para ALGOL. 

Su famosa notación es Backus Naur Form (BNF) que describe un autómata a partir
del un conjunto de símbolos (ver Definiciones: [\ref{def:BNF},
\ref{def:RPBNF}]).}-Naur\endnote{\textbf{Peter Naur} es un prestigio científico
danés nacido el 25
de octubre de 1928 ganador del Premio Turing en 2005.
Su trabajo más representativo consiste en sentar junto a John Backus la notación
para especificación de autómatas para lenguajes formales.} Form se trata una de las dos notaciones más importantes para Gramáticas Libres de Contexto.

John Backus, diseñador de lenguajes en IBM propuso para el Lenguaje de Programación IAL (conocido como ALGOL 58) un meta-lenguaje. Posteriormente con la publicación de ALGOL 60 la fórmula BNF se simplificó y perfeccionó. 

BNF se trata de un conjunto de reglas derivativas del tipo: \texttt{<symbol> ::= \_expression\_}

\paragraph*{Dónde:} 

\begin{enumerate}

\item \texttt{symbol}: Es un símbolo No Terminal. 

\item \texttt{\_expression\_}: Consiste en un conjunto de símbolos o de secuencias (separadas por el carácter '|') donde el símbolo ``más a la izquierda'' es el Terminal.

\item \texttt{'::='}: Operador de asignación. Indica que el símbolo de la izquierda es sustituido por la expresión de la derecha.

\end{enumerate}


\begin{verbatim}
<syntax>         ::= <rule> | <rule> <syntax>
<rule>           ::= <opt-whitespace> "<" <rule-name> ">" <opt-whitespace> "::=" 
					  <opt-whitespace> <expression> <line-end>
<opt-whitespace> ::= " " <opt-whitespace> | ""
<expression>     ::= <list> | <list> "|" <expression>
<line-end>       ::= <opt-whitespace> <EOL> | <line-end> <line-end>
<list>           ::= <term> | <term> <opt-whitespace> <list>
<term>           ::= <literal> | "<" <rule-name> ">"
<literal>        ::= '"' <text> '"' | "'" <text> "'" 
\end{verbatim}

\paragraph*{EBNF:} Existen distintas variantes sobre BNF. Las más popular es Extended Backus-Naur Form (EBNF) que incopora operadores de Expresiones Regulares como:

\begin{enumerate}[i.]

\item $a^+$: Repetir $a$ de $1$ a $n$ veces. 

\item $a^*$: Repetir $a$ de $0$ a $n$ veces.

\end{enumerate}


\subsection{Wijngaarden Form}

Var Wijngaarden Form (también conocida como vW-grammar p W-grammar) se trata de una técnica para definir Gramáticas Libres de Contexto en un número finito de reglas.

Las W-grammars se basan en la idea de que los símbolos No Terminales intercambian información entre los nodos y el árbol de ``parseo''.

El primer uso de estas gramáticas fue en ALGOL 68.


\section{Especificaciones}

\section{Reconocimiento}

\section{Definiciones}

\defn La función de un \textit{analizador sintáctico} es la de relacionar la
cadena de \textit{tokens} elaborada por el \textit{analizador léxico} y la de
comprobar que la secuencia de estos \textit{tokens} se corresponden con patrones
sintácticos (las reglas) del lenguaje.


{\cor El analizador sintáctico es el encargado de elaborar el árbol de análisis
del código fuente sobre el que trabajaran el resto de fases de los
compiladores.}

\defn El \textit{analizador sintáctico} es capaz de detectar errores en segunda fase, es
decir, en la correspondencia entre \textit{token} y \textit{patrón sintáctico}.

{\cor Al contrario que ocurre con los errores léxicos, los errores sintácticos
tiene una gran cosistecia\footnote{Están perfectamente definidos en el Lenguaje
de Programación.}.}

\begin{figure}[h]
\begin{center}
\begin{pspicture}(14,5)%\psgrid
\rput(2,4){{código fuente}}
\rput(7,4){{\tt p1990la}}
\rput(12,4){{\tt p1990sa}}
\rput(7,1){{tabla de símbolos}}
\psline[linecolor=black,linewidth=1pt]{->}(3.4,4)(6,4)
\psline[linecolor=black,linewidth=1pt]{<->}(7,3.8)(7,1.2)
\psline[linecolor=black,linewidth=1pt]{<->}(7.2,1.2)(12,3.8)
\psline[linecolor=black,linewidth=1pt]{->}(8,4)(11,4)
\end{pspicture}
\caption{Relación entre el \textit{analizador léxico}, \textit{analizador sintáctico} y el programa fuente.}
\end{center}
\end{figure}

\subsection{Automátas LL(1)}

\subsection{Conjunto de los Primeros}

Partiendo de una gram\'actica: $G = (\Sigma_{T}, \Sigma_{N}, S, P)$


\defn Si $\alpha$ es una forma sentencial compuesta por una concatenaci\'on de
s\'imbolos $PRIM(\alpha)$ es el conjunto de terminales o $\lambda$ que pueden
aparacer iniciando las cadenas que pueden derivar de $alpha$.


\form $a \in PRIM(\alpha)$ si $a \in (\Sigma_{T} \cup \{\lambda\}) \wedge \alpha
\Longrightarrow^{*} a\beta$

\regl Para calcular el conjunto de los primeros tenemos:

\begin{enumerate}

\item Si $\alpha \equiv \lambda \Rightarrow PRIM\{\lambda\} = \{\lambda\}$.

\item Si $\alpha \in (\Sigma_T \cup \Sigma_N)^+ \Rightarrow \alpha = a_1, a_2, \ldots, a_n$ demuestra:
 
\begin{enumerate}

\item Si $a_1 \equiv a \in \Sigma_T \Rightarrow PRIM(\alpha) = \{a\}$.  

\item Si $a_1 \equiv A \in \Sigma_N$ para:

\begin{enumerate}

\item $PRIM(A) = \cup^n_{i=1} PRIM(\alpha_i) \backslash \alpha_i \in P$

\item Si $PRIM(A)\ \backslash\ \lambda \in PRIM(A)\ \wedge\ A$ no es el último símbolo de $\alpha \Rightarrow PRIM(\alpha) = (PRIM(A)-\{\lambda\}) \cup PRIM (a_2, a_3, \ldots, a_n)$

\item Si $A$ es el último símbolo de $\alpha \vee \lambda \notin PRIM(A) \Rightarrow PRIM(\alpha) = PRIM(A)$

\end{enumerate}


\end{enumerate}


\end{enumerate}

\ejem

\begin{equation}
example
\end{equation}

\ejem

\begin{equation}
example
\end{equation}

\subsection{Conjunto de los Siguientes}

\paragraph*{Nota:} Partiendo de la gramática (Referencia a la gramática 2.3.2).

\defn Si $A$ es un símbolo inicial no terminal de la gramática, $SIG(A)$ es el conjunto de terminales $ +\ \{\$\}$ que pueden aparecer a continuación de $A$ en alguna forma sentencial derivada del símbolo inicial.


\form $a \in SIG(A)$ si $a \in (\Sigma_{TN} \cup \{\$\}) \wedge \exists \alpha, \beta \backslash S 
\Longrightarrow^{*} \alpha Aa \beta$

\regl Para calcular el conjunto de los siguientes tenemos:

\begin{enumerate}
\item Partimos de que: $SIG(A) = \varnothing$

\item Si $A$ es símbolo inicial $\Rightarrow SIG(A) = SIG(A) \cup \{\$\}$

\item Dada la regla: $B \rightarrow \alpha A \beta \Rightarrow SIG(A) = SIG(A) \cup (PRIM(\beta) - \{\lambda\})$

\item Dada la regla: $B \rightarrow \alpha A \vee B \rightarrow \alpha A \beta\ \backslash\ \lambda \in PRIM(\beta) \Rightarrow SIG(A) = SIG(A) \cup SIG(B)$

\item Repetir los pasos 3 y 4 hasta que no se puedan añadir más símbolos a $SIG(A)$

\end{enumerate}

\ejem

\begin{equation}
example
\end{equation}

\subsection{Conjuntos de Predicción}

\defn Para una gramática ASDP con símbolo no terminal $\sigma_N$ se debe consultar el símbolo de entrada y buscar en cada regla de dicho símbolo. Si los conjuntos de producción son disjuntos, el AS podrá construir una derivación hacia la izquierda de la cadena de entrada. 


\form $PRED(A \rightarrow \alpha) \Rightarrow$

\begin{enumerate}
\item Si $\alpha \in PRIM(\alpha) \Rightarrow (PRIM(\alpha) - \{\lambda\} \cup SIG(A)$
\item Si no $\Rightarrow PRIM(\alpha)$ 
\end{enumerate}

\ejem

\begin{equation}
example
\end{equation}

\section{Código fuente: gp1990sa.y}

\nocite{book/gnu/gpc}