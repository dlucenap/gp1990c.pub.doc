%%%%%

El presente texto consta de tres apartados conceptuales y seis capítulos cuyo propósito es dar un sentido pedagógico a la exposición, partiendo de lo general para llegar a lo particular.

\begin{enumerate}[I.]
\item Introducción
{

Capítulo I: El primer capítulo prentende sintetizar la 
matemática necesaria y básica sobre la que se asientan los desarrollo de la 
Teoría de Compiladores y Analizadores Automáticos de Lenguajes. Por ello, se 
fundamenta en tres corrientes matemáticas:

\begin{enumerate}[i.]
\item Teoría de Conjuntos. 
 
\item Funciones.

\item Teoría de Grafos.
\end{enumerate}

Capítulo II: Este capítulo tiene el objetivo de sintetizar todo el trabajo sobre 
el que gira \texttt{gp1990c}:

\begin{enumerate}[i.]

\item Motivo del proyecto.

\item Síntesis, partes y desarrollo de las fases de Análisis Léxico (\texttt{gp199la}) y 
Análisis Sintáctico (\texttt{gp1990sa}) de Pascal ISO 1990 (el conjunto Software).

\end{enumerate}

Capítulo III: En este apartado se describe el Lenguaje Pascal y su evolución a los 
largo de los años. 

En el contexto de su nacimiento (principios de los años setenta del siglo XX) 
la Computación sufrió una intensa evolución desde una computación para grandes 
corporaciones y con altos costes de explotación, pasando por las primeras máquinas de Apple y PC de IBM, hasta lo que conocemos hoy día.

Por ello se describen los lenguajes sobre los que se baso Nicklaus Wirth para crear Pascal y como, su concepto de lenguaje con un 
repertorio discreto de instrucciones (frente a otros de la época como Fortran o 
Algol) además de su expresividad han sido fundamentales para la construcción de 
importantes lenguajes que hoy día y como es el caso de ADA, son lideres 
indiscutibles en la Computación a Tiempo Real.

Capítulo IV: Es un capítulo que precede al desarrollo del propio Analizador 
Léxico y que busca dar sentido y forma a la sucesión de compiladores de Pascal. 
Son destacables los hitos:

\begin{enumerate}[i.]

\item Creación de PUG (\texttt{Pascal Users Group}). 

\item \texttt{Pascal-P2} y \texttt{P4}.

\item \texttt{UCSD Pascal}: Concepto de p-systems e independencia en su ejecución.

\item \texttt{Borland Pascal}: Compilador asequible para estudiantes de programación en 
la era PC.

\item \texttt{PFC}: Implementación GPL de un potente y multiplataforma IDE de Pascal.
\end{enumerate}

}
\item {{\tt gp1990la} (Analizador Léxico)}
{

Capítulo V: Dicho Capítulo tiene como objetivo sintetizar los aspectos matemáticos de los Analizadores Léxicos. Por ello, en el siguiente orden, se estudian:

\begin{enumerate}[i.]

\item Lenguajes Formales.

\item Teoría de Autómatas.
\end{enumerate}

De igual manera se incluye una introducción al uso de LEX además del análisis de los elementos léxicos que forman parte del compilador:

\begin{enumerate}[i.]

\item Expresiones Regulares.

\item Conjunto de Tokens.

\end{enumerate}

}
\item {{\tt gp1990sa} (Analizador Sintáctico)}
{

Capítulo VI: El mismo se resumen los aspectos fundamentales de los Analizadores Sintácticos. Contiene los siguientes subapartados:

\begin{enumerate}[i.]

\item Lenguajes Formales (LFs)

\item Jerarquía de Chomsky (JC)

\item Gramáticas Formales

\item Analizadores Sintácticos

\item Yacc (Yet another compiler-compiler)

\end{enumerate}



}
% \item Pruebas
% {
% 
% Capítulo IX:
% 
% Capítulo X:
% 
% }
\item Anexos
{

Anexo A: Vida y obra de Blaise Pascal.

Anexo B: Código fuente: \texttt{gp1990sa.y}

Anexo C: Gramáticas de familia Pascal (Pascal; Modula, Oberon).

Anexo D: Cronología de publicaciones de \textit{Pascal Users group Newsletter}.

Anexo E: Introducción a UNIX y GNU.

Anexo F: Introducción a Linux.

Anexo G: Tabla ASCII  $[0, (2^8 - 1)]$

Anexo H: Manifiesto GPL v.2

}
\end{enumerate}
