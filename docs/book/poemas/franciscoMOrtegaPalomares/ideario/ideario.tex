\begin{verse}

\begin{center}

\underline{\bf Ideario}

Me da vértigo el punto muerto \\

y la marcha atrás, \\

vivir en los atascos, \\

los frenos automáticos y el olor a gasoil.

\par

Me angustia el cruce de miradas \\

la doble dirección de las palabras \\

y el obsceno guiñar de los semáforos.

\par

Me da pena la vida, los cambios de sentido, \\

las señales de stop y los pasos perdidos.

\par

Me agobian las medianas, \\

las frases que están hechas, \\

los que nunca saludan y los malos profetas.

\par

Me fatigan los dioses bajados del Olimpo \\

a conquistar la Tierra \\

y los necios de espíritu.

\par

Me entristecen quienes me venden clines \\

en los pasos de cebra, \\

los que enferman de cáncer \\

y los que sólo son simples marionetas.

\par

Me aplasta la hermosura \\

de los cuerpos perfectos, \\

las sirenas que ululan en las noches de fiesta, \\

los códigos de barras, \\

el baile de etiquetas.

\par

Me arruinan las prisas y las faltas de estilo, \\

el paso obligatorio, las tardes de domingo \\

y hasta la línea recta.

\par

Me enervan los que no tienen dudas \\

y aquellos que se aferran \\

a sus ideales sobre los de cualquiera.



Me cansa tanto tráfico \\

y tanto sinsentido, \\

parado frente al mar mientras que el mundo gira.

\par{\it Francisco M. Ortega Palomares}

\end{center}

\end{verse}           