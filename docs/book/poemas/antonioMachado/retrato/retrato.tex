\begin{verse}

\begin{center}

\underline{\bf Retrato}

Mi infancia son recuerdos de un patio de Sevilla, \\


y un huerto claro donde madura el limonero; \\


mi juventud, veinte años en tierras de Castilla; \\


mi historia, algunos casos que recordar no quiero.

\par

Ni un seductor Mañara, ni un Bradomín he sido \\


¿ya conocéis mi torpe aliño indumentario?, \\


más recibí la flecha que me asignó Cupido, \\


y amé cuanto ellas puedan tener de hospitalario.

\par

Hay en mis venas gotas de sangre jacobina, \\


pero mi verso brota de manantial sereno; \\


y, más que un hombre al uso que sabe su doctrina, \\


soy, en el buen sentido de la palabra, bueno.

\par

Adoro la hermosura, y en la moderna estética \\


corté las viejas rosas del huerto de Ronsard; \\


mas no amo los afeites de la actual cosmética, \\


ni soy un ave de esas del nuevo gay-trinar.

\par

Desdeño las romanzas de los tenores huecos \\


y el coro de los grillos que cantan a la luna. \\


A distinguir me paro las voces de los ecos, \\


y escucho solamente, entre las voces, una.

\par

¿Soy clásico o romántico? No sé. Dejar quisiera \\


mi verso, como deja el capitán su espada: \\


famosa por la mano viril que la blandiera, \\


no por el docto oficio del forjador preciada.

\par

Converso con el hombre que siempre va conmigo \\


¿quien habla solo espera hablar a Dios un día?; \\


mi soliloquio es plática con ese buen amigo \\


que me enseñó el secreto de la filantropía.

\par

Y al cabo, nada os debo; debéisme cuanto he escrito. \\


A mi trabajo acudo, con mi dinero pago \\


el traje que me cubre y la mansión que habito, \\


el pan que me alimenta y el lecho en donde yago.

\par

Y cuando llegue el día del último vïaje, \\


y esté al partir la nave que nunca ha de tornar, \\


me encontraréis a bordo ligero de equipaje, \\


casi desnudo, como los hijos de la mar.

\par{\it Antonio Machado}

\end{center}

\end{verse}      