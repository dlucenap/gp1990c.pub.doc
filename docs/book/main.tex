\documentclass[a4paper,12pt]{book}
%Paquetes
\usepackage[utf8x]{inputenc}
\usepackage[spanish,activeacute,es-lcroman,es-tabla]{babel}
%\usepackage[spanish, es-tabla]{babel}
\usepackage[spanish]{minitoc}
%\usepackage{mtcoff}
%\usepackage{anttor}
\usepackage{amsmath, amsthm, amssymb}
\usepackage{amsfonts}

\usepackage{yfonts}
\usepackage{wrapfig}

%\usepackage{lettrine}
\usepackage[T1]{fontenc}
%\usepackage{lmodern}

%texlive-fontsextra
\usepackage{dsfont}
\usepackage{graphs}
\usepackage{graphicx}
\usepackage{caption}
\usepackage{subcaption}

\usepackage{pstricks} % para color
\usepackage{pst-node} % para diagramas
\usepackage{pst-plot} % para representacion de datos
                      % funciones, etc
\usepackage{pst-text}
\usepackage{pst-tree}
\usepackage{pst-circ}
\usepackage{pst-poly}
\usepackage{pstricks-add}
%\usepackage{pst-coil}
\usepackage{pst-gantt}

\usepackage{hyperref}


\usepackage{graphicx}
\usepackage{textcomp}
\usepackage{color}
\usepackage{fancyvrb}
\usepackage{fancyhdr}
\usepackage{eso-pic,calc}                        
\usepackage{bibunits}
\usepackage{listings}
\usepackage{lscape}

\usepackage{ascii}

\usepackage{anysize}

\usepackage{endnotes}

\usepackage{enumerate}





\usepackage{ascii}
\usepackage{textcomp}
\usepackage{eurosym}
\usepackage{cclicenses}

\usepackage[spanish,intoc]{nomencl}
\makenomenclature

\usepackage{makeidx}

\usepackage{filecontents}


%\TPGrid[10mm,5mm]{26}{20} 

%\parindent=0pt\chapter*{}
%\parskip=0.5\baselineskip








\let\T=\text
\fancyhead[L]{\textbf{(\textit{preview} 1.0)}} 
%\fancyhead[RE]{}
%\fancyhead[RO,LE]{\thepage}

% Utilidades
%--------------------------------------------------------------------------
\newtheorem{thm}{Teorema}[section]
\newtheorem{cor}[thm]{Corolario}
\newtheorem{lem}[thm]{Lema}
\newtheorem{prop}[thm]{Proposici\'on}
\theoremstyle{definition}
\newtheorem{defn}[thm]{Definici\'on}
\newtheorem{form}[thm]{Formalidad}
\newtheorem{regl}[thm]{Reglas}
\newtheorem{ejem}[thm]{Ejemplo}
\newtheorem{prog}[thm]{Programa}
\newtheorem{algo}[thm]{Algoritmo}
\theoremstyle{remark}
\newtheorem{rem}[thm]{Observación}
\newtheorem{dhm}[thm]{Demostración}


\let\stdthebibliography\thebibliography
\renewcommand*{\thebibliography}{%
\let\section\subsection\stdthebibliography}

\renewcommand{\notesname}{Notas}


\newcommand{\dos}{\textsf{~dos/win~}}
\newcommand{\unix}{\UNIX}
% \pdfpagewidth 6\chapter{Formalismos}
% \pdfpageheight 9in
% \setlength\topmargin{0in}
% \setlength\headheight{0in}
% \setlength\headsep{0in}
% \setlength\textheight{7.7in}
% \setlength\textwidth{6.5in}
% \setlength\oddsidemargin{0in}
% \setlength\evensidemargin{0in}
% \setlength\headheight{25pt}
% \setlength\headsep{0.25in}

\marginsize{2cm}{2cm}{2cm}{2cm} 

% \makeatletter\renewcommand\theenumii{\@roman\c@enumiii}\makeatother
% \makeatletter\renewcommand\theenumii{\@roman\c@enumii}
% \renewcommand\labelenumii{\theenumii)}
% \makeatletter % for internal macros with @
% \renewcommand\theenumii{\@roman\c@enumii}
% \makeatother

\pagestyle{empty}
\frontmatter

\setcounter{secnumdepth}{5}
\setcounter{tocdepth}{5}

\setcounter{parttocdepth}{1}
\setcounter{minitocdepth}{1}
%\nomtcrule 

\renewcommand{\mtctitle}{Resumen:}
 
%\setlength{\mtcindent}{24pt}
%\renewcommand{\mtcfont}{\texttt}
%\renewcommand{\mtcSfont}{\small\bf}

%% for Greek Alphabet

\def\X#1{$#1$ &\tt\string#1}

\definecolor{gray}{rgb}{0.98,0.98,0.98}
\definecolor{black}{rgb}{0,0,0}

\lstset{showstringspaces=false,numbers=left,
        numberstyle=\footnotesize,backgroundcolor=\color{gray},
        rulesep=1pt, rulesepcolor=\color{black},frame=leftline,
        basicstyle=\ttfamily, mathescape=true}

        
%%% Commands@dlucenap
        
        
%%%%%%%%%%%%%%%%
% Command: Double Page with style Empty
%%%%%%%%%%%%%%%%

\global\def\doublePageEmpty{\newpage{\pagestyle{empty}\cleardoublepage}}        
        
        
%%%%%%%%%%%%%%%%
% Command: Enviroment for compile chapter
% #1 name of Chapter
% #2 is a Source of file
%%%%%%%%%%%%%%%%
   
\global\def\metaChapter#1#2
{
\doublePageEmpty

\pagestyle{fancy}
% 
% \renewcommand*{\bibname}{Bibliografía capitular}
% \bibliographyunit
% \doublePageEmpty
% \bibliographyunit[\chapter]
% \bibliographystyle*{alpha}
% \bibliography*{lib}

\chapter{#1}
\minitoc
\input{#2}

\doublePageEmpty
\addcontentsline{toc}{section}{Notas}

\pagestyle{plain}

\theendnotes

\doublePageEmpty
% \addcontentsline{toc}{section}{Bibliografía capitular}
% \putbib
% \bibliographyunit

}


%%%%%%%%%%%%%%%%
% Command: Enviroment for compile chapter
% #1 name of Chapter
% #2 is a Source of file
%%%%%%%%%%%%%%%%
   
\global\def\simpleChapter#1#2
{
\doublePageEmpty

\pagestyle{fancy}
% 
% \renewcommand*{\bibname}{Bibliografía capitular}
% \bibliographyunit
% \doublePageEmpty
% \bibliographyunit[\chapter]
% \bibliographystyle*{alpha}
% \bibliography*{lib}

\chapter{#1}
\minitoc
\input{#2}

\doublePageEmpty
% \addcontentsline{toc}{section}{Notas del capítulo}
% 
% \pagestyle{plain}
% 
% \theendnotes

% \doublePageEmpty
% \addcontentsline{toc}{section}{Bibliografía capitular}
% \putbib
% \bibliographyunit

}

%%%%%%%%%%%%%%%%
% Command: 
% #1 name of Chapter
% #2 is a Source of file
%%%%%%%%%%%%%%%%
   
\global\def\preChapter#1#2
{
\doublePageEmpty
\chapter*{#1}
\input{#2}
\doublePageEmpty
}

%%%%%%%%%%%%%%%%
% Command: 
% #1 name of word
%%%%%%%%%%%%%%%%
   
\global\def\toIndex#1
{#1\index{#1} }

% \begin{filecontents*}{myStyle.xdy}
% -----------------xindy style file-----------------------
% ;;; xindy style file for the VIBS book series
% 
% ;;; xindy style file
% (markup-locclass-list :open "\dotfill" :sep "")
% 
% (define-attributes (( "textbf" "default" )) )
% (markup-locref   :attr  "textbf"     :open "\textbf{\hyperpage{" :close "}}")
% (markup-locref   :attr  "textit"     :open "\textit{\hyperpage{" :close "}}")
% (markup-locref   :attr  "textttt"     :open "\textttt{\hyperpage{" :close "}}")
% (markup-locref   :attr  "texttsc"     :open "\texttsc{\hyperpage{" :close "}}")
% (markup-locref   :attr  "default"     :open "\hyperpage{" :close "}")
% 
% define-letter-groups
%   ("a" "b" "c" "d" "e" "f" "g" "h" "i" "j" "k" "l" "m"
%    "n" "o" "p" "q" "r" "s" "t" "u" "v" "w" "x" "y" "z"))
% 
% (markup-letter-group-list :sep "~n\indexspace")
% 
% ;; End
% 
% \end{filecontents*}

\makeatletter
\renewcommand\part{%
  \if@openright
    \cleardoublepage
  \else
    \clearpage
  \fi
  \thispagestyle{empty}%   % Original »plain« replaced by »emptyx
  \if@twocolumn
    \onecolumn
    \@tempswatrue
  \else
    \@tempswafalse
  \fi
  \null\vfil
  \secdef\@part\@spart}
\makeatother

\makeindex
%\makeindex[program=texindy,columns=1]

\begin{document}

\thispagestyle{empty}
\vspace{5cm}
\begin{center}

{\LARGE
Departamento de Automática\\
Escuela Politécnica Superior\\
Universidad de Alcalá\\
}

\vspace{2cm}


\includegraphics[width=6cm]{pictures/logo-uah.eps}

\vspace{1cm}

{\LARGE \textbf{Proyecto Fin de Carrera}}

\vspace{1cm}

{\LARGE \texttt{gp1990c (GNU Pascal 1990 Compiler)}}

\vspace{2cm}

\textbf{S\v{e}ptemb\v{e}r - MMXIV}

\begin{table}[h]
\centering
\begin{tabular}{r l}
\textbf{Autor:} & Diego Antonio Lucena Pumar \\
\textbf{Titulación:} & Ingeniería Técnica en Informática de Gestión
\end{tabular}
\end{table}

\end{center}
\doublePageEmpty

%%%%%%%%%%%%%%
% Acknowledgment
%%%%%%%%%%%%%%

\preChapter{Agradecimientos}{./agradecimientos}

\doublePageEmpty

\pagestyle{plain}


%%%%%%%%%%%%%%
% Introduction
%%%%%%%%%%%%%%

%\preChapter{Prólogo}{./prologo}

%%%%%%%%%%%%%%
% Preface
%%%%%%%%%%%%%%

\preChapter{Prefacio}{./prefacio}

\doparttoc
\dominitoc
%
{\tableofcontents}
\doublePageEmpty

%
{\listoffigures}
\doublePageEmpty

%
{\listoftables}
\doublePageEmpty

\pagestyle{empty}

\begin{center}
\underline{\bf Conjuntos de Números y Operadores}
\end{center}

\begin{table}[h]
\begin{center}
\begin{tabular}{r|l}
    $\mathbb{N}$ & Conjunto de los números Naturales. \\
    $\mathbb{Z}$ & Conjunto de los números Enteros. \\ 
    $\mathbb{Q}$ & Conjunto de los números Racionales. \\
    $\mathbb{R}$, $\mathbb{R}^+$, $\mathbb{R}^n$ & Conjunto de los números Reales. \\
    $\emptyset$ & Conjunto Vacío. \\
    $\in$ & Operador de Pertenecia. \\
    $\notin$ & Operador de no Pertenecia. \\
    $\subseteq$ & Subconjunto Exclusivo. \\
    $\cup$ & Operador de Unión. \\
    $\cap$ & Operador de Intersección. \\
    $\times$ & Producto Cartersiano. \\ 
    $\wedge$ & Operador de Conjunción. \\
    $\vee$ & Operador de Disjunción. \\
    $\oplus$ & Operador OR Exclusivo. \\
    $\rightarrow$ & Operador de Implicación. \\
    $\leftrightarrow$ & Operador de Equivalencia. \\
    $\Rightarrow$ & Implicación. \\
    $\Leftarrow$ & Implicación Inversa. \\
    $\Leftrightarrow$ & Equivalencia Lógica. \\
    $\circ$ & Operador de Concatenación. \\
    $\forall$ & Cuantificador Universal. \\
    $\exists$ & Cuantificador Existencial.          
\end{tabular}
\end{center}
\end{table}

\newpage

\begin{center}
\underline{\bf Alfabeto Griego}
\end{center}

\begin{table}[h]
\begin{center}
\begin{tabular}{c|c|l}
    A & $\alpha$ & alfa  \\
    B & $\beta$ & beta \\
    $\Gamma$ & $\gamma$ & gamma \\
    $\Delta$ & $\delta$ & delta \\
    E & $\epsilon$ & epsilón\\
    Z & $\zeta$ & dseta \\
    H & $\eta$ & eta \\
    $\Theta$ & $\theta$ & zeta \\
    I & $\iota$ & iota \\
    K & $\kappa$ & cappa \\
    $\Lambda$ & $\lambda$ & lambda \\
    M & $\mu$ & my \\
    N & $\nu$ & ny \\
    $\Xi$ & $\xi$ & xi \\
    O & $o$  & omicrón \\
    $\Pi$ & $\pi$ & pi \\
    P & $\rho$ & rho \\
    $\Sigma$ & $\sigma$ & sigma \\
    T & $\tau$ & tau \\
    $\Upsilon$ & $\upsilon$ & ypsilón \\
    $\Phi$ & $\phi$ & fi \\
    X & $\chi$ & ji \\
    $\Psi$ & $\psi$ & psi \\
    $\Omega$ & $\omega$ & omega \\
\end{tabular}
\end{center}
\end{table}

\doublePageEmpty

\mainmatter

%%%%%%%%%%%%%%
% PART ONE
%%%%%%%%%%%%%%

\part{Introducción}
\doublePageEmpty

% Poema: Ideario, Francisco M. Ortega Palomares.
\addstarredchapter{Francisco M. Ortega Palomares. \textit{Ideario}}
\include{./poemas/franciscoMOrtegaPalomares/ideario/ideario}
\doublePageEmpty

%%%%%%%%%%%%%%
% Chapter 1
%%%%%%%%%%%%%%

\metaChapter{Formalismos}{./parte1/capitulo1/cap1}

%%%%%%%%%%%%%%
% Chapter 2
%%%%%%%%%%%%%%

\metaChapter{Resumen: Proyecto {\tt gp1990c}}{./parte1/capitulo2/cap2}

%%%%%%%%%%%%%%
% Chapter 3
%%%%%%%%%%%%%%

\metaChapter{El Lenguaje de Programaci\'on Pascal}{./parte1/capitulo3/cap3}

%%%%%%%%%%%%%%
% Chapter 4
%%%%%%%%%%%%%%

\metaChapter{Compiladores del Lenguaje Pascal}{./parte1/capitulo4/cap4}

%%%%%%%%%%%%%%
% PART TWO: LEX
%%%%%%%%%%%%%%

\part{{\tt gp1990la} (Analizador Léxico)}
%% Poema: XI, Gustavo Adolfo Bécquer.
\pagestyle{empty}
\addstarredchapter{Gustavo Adolfo Bécquer. \textit{XI}}
\include{./poemas/gustavoAdolfoBequer/rimasyLeyendas/XI}
\doublePageEmpty

%%%%%%%%%%%%%%
% Chapter 5
%%%%%%%%%%%%%%

\metaChapter{Formalidades del Analizador L\'exico}{./parte2/capitulo1/cap1}

%%%%%%%%%%%%%%
% PART TREE: YACC
%%%%%%%%%%%%%%

\part{{\tt gp1990sa} (Analizador Sintáctico)}
%% Poema: Retrato, Antonio Machado.
\pagestyle{empty}
\addstarredchapter{Antonio Machado. \textit{Retrato}}
\include{./poemas/antonioMachado/retrato/retrato}
\doublePageEmpty

%%%%%%%%%%%%%%
% Chapter 9
%%%%%%%%%%%%%%

\metaChapter{Formalidades del Analizador Sintactico}{./parte3/capitulo1/cap1}

%%%%%%%%%%%%%%
% PART FOUR: TEST
%%%%%%%%%%%%%%

%\part{Pruebas}
%\doublePageEmpty

%%%%%%%%%%%%%%
% Chapter 11
%%%%%%%%%%%%%%

%\metaChapter{Eficiencia y Optimizaci\'on}{./parte4/capitulo1/cap1}

%%%%%%%%%%%%%%
% PART FIVE: APPENDS
%%%%%%%%%%%%%%

\part{Anexos y Formalidades}
%% Poema:.
\pagestyle{empty}
%\addstarredchapter{Antonio Machado. \textit{Retrato}}
%\include{./poemas/antonioMachado/retrato/retrato}
\doublePageEmpty

% Style
\appendix

%%%%%%%%%%%%%%
% APPEND A: Blaise Pascal
%%%%%%%%%%%%%%

\include{anexos/A/blaisePascal}
\doublePageEmpty

%%%%%%%%%%%%%%
% APPEND B: Grammars
%%%%%%%%%%%%%%

%// Generated by gramex V2.1 from 'iso_pascal_7185.txt' on Jan 25 2007 at 11:56:25
%// Command line: C:\elsie\lag\grammars\pascal\gramex.exe -p -c iso_pascal_7185.txt

\chapter{gp1990sa.y}
\label{chap:grammars}

\section{Yacc}

%// Generated by gramex V2.1 from 'iso_pascal_7185.txt' on Jan 25 2007 at 11:56:25
%// Command line: C:\elsie\lag\grammars\pascal\gramex.exe -p -c iso_pascal_7185.txt

\chapter{gp1990sa.y}
\label{chap:grammars}

\section{Yacc}

%// Generated by gramex V2.1 from 'iso_pascal_7185.txt' on Jan 25 2007 at 11:56:25
%// Command line: C:\elsie\lag\grammars\pascal\gramex.exe -p -c iso_pascal_7185.txt

\chapter{gp1990sa.y}
\label{chap:grammars}

\section{Yacc}

\input{./grammars/grammars31/gp1990sa}






\doublePageEmpty


%%%%%%%%%%%%%%
% APPEND B: Grammars
%%%%%%%%%%%%%%

%// Generated by gramex V2.1 from 'iso_pascal_7185.txt' on Jan 25 2007 at 11:56:25
%// Command line: C:\elsie\lag\grammars\pascal\gramex.exe -p -c iso_pascal_7185.txt

\chapter{Gramáticas}
\label{chap:grammars}

\section{Pascal ISO 1990:7185}

\input{./grammars/grammars13/gramPascal}

\section{Modula-2}

\renewcommand{\theFancyVerbLine}{%
{\small
ms.\arabic{FancyVerbLine}}}
 
\begin{Verbatim}[numbers=left]
ident = letter {letter | digit}.

number = integer | real.

integer = digit {digit} | octalDigit {octalDigit} ("B"|"C") |
          digit {hexDigit} "H".
          
real = digit {digit} "." {digit} {ScaleFactor}.

ScaleFactor = "E" ["+"|"-"] digit {digit}.

hexDigit = digit | "A" | "B" | "C" | "D" | "E" | "F".

digit = octalDigit | "8" | "9".

octalDigit = "0" | "1" | "2" | "3" | "4" | "5" | "6" | "7".

string = "'" {character} "'" | '"' {character} '"' .

qualident = ident {"." ident}.

ConstantDeclaration = ident "=" ConstExpression.

ConstExpression = expression.

TypeDeclaration = ident "=" type.

type = SimpleType | ArrayType | RecordType | SetType |
       PointerType | ProcedureType.
       
SimpleType = qualident | enumeration | SubrangeType.

enumeration = "(" IdentList ")".

IdentList = ident {"," ident}.

SubrangeType = [ident] "[" ConstExpression ".." ConstExpression "]".

ArrayType = ARRAY SimpleType {"," SimpleType} OF type.

RecordType = RECORD FieldListSequence END.

FieldListSequence = FieldList {";" FieldList}.

FieldList = [IdentList ":" type |
            CASE [ident] ":" qualident OF variant {"|" variant}
            [ELSE FieldListSequence] END].
            
variant = [CaseLabelList ":" FieldListSequence].

CaseLabelList = CaseLabels {"," CaseLabels}.

CaseLabels = ConstExpression [".." ConstExpression].

SetType = SET OF SimpleType.

PointerType = POINTER TO type.

ProcedureType = PROCEDURE [FormalTypeList].

FormalTypeList = "(" [ [VAR] FormalType
                 {"," [VAR] FormalType} ] ")" [":" qualident].
                 
VariableDeclaration = IdentList ":" type.

designator = qualident {"." ident | "[" ExpList "]" | "^"}.

ExpList = expression {"," expression}.

expression = SimpleExpression [relation SimpleExpression].

relation = "=" | "#" | "<" |"<=" | ">" | ">=" | IN.

SimpleExpression = ["+"|"-"] term {AddOperator term}.

AddOperator = "+" | "-" | OR.

term = factor {MulOperator factor}.

MulOperator = "*" |"/" | DIV | MOD | AND.

factor = number | string | set | designator [ActualParameters] |
        "(" expression ")" | NOT factor.
        
set = [qualident] "{" [element {"," element}] "}".

element = expression [".." expression].

ActualParameters = "(" [ExpList] ")" .

statement = [assignment | ProcedureCall |
            IfStatement | CaseStatement | WhileStatement |
            RepeatStatement | LoopStatement | ForStatement |
            WithStatement | EXIT | RETURN [expression] ].
            
assignment = designator ":=" expression.

ProcedureCall = designator [ActualParameters].

StatementSequence = statement {";" statement}.

IfStatement = IF expression THEN StatementSequence
              {ELSIF expression THEN StatementSequence}
              [ELSE StatementSequence] END.
              
CaseStatement = CASE expression OF case {"|" case}
                [ELSE StatementSequence] END.
                
case = [CaseLabelList ":" StatementSequence].

WhileStatement = WHILE expression DO StatementSequence END.

RepeatStatement = REPEAT StatementSequence UNTIL expression.

ForStatement = FOR ident ":=" expression TO expression
               [BY ConstExpression] DO StatementSequence END.
               
LoopStatement = LOOP StatementSequence END.

WithStatement = WITH designator DO StatementSequence END .

ProcedureDeclaration = ProcedureHeading ";" block ident.

ProcedureHeading = PROCEDURE ident [FormalParameters].

block = {declaration} [BEGIN StatementSequence] END.

declaration = CONST {ConstantDeclaration ";"} |
              TYPE {TypeDeclaration ";"} |
              VAR {VariableDeclaration ";"} |
              ProcedureDeclaration ";" | ModuleDeclaration ";".
              
FormalParameters = "(" [FPSection {";" FPSection}] ")" [":" qualident].

FPSection = [VAR] IdentList ":" FormalType.

FormalType = [ARRAY OF] qualident.

ModuleDeclaration = MODULE ident [priority] ";" [import] [export] block ident.

priority = "[" ConstExpression "]".

export = EXPORT [QUALIFIED] IdentList ";".

import = [FROM ident] IMPORT IdentList ";".

DefinitionModule = DEFINITION MODULE ident ";"

                   {import} {definition} END ident "." .
                   
definition = CONST {ConstantDeclaration ";"} |
             TYPE {ident ["=" type] ";"} |
             VAR {VariableDeclaration ";"} |
             ProcedureHeading ";".
             
ProgramModule = MODULE ident [priority] ";" {import} block ident ".".

CompilationUnit = DefinitionModule | [IMPLEMENTATION] ProgramModule.
\end{Verbatim}





\section{Oberon}

\input{./grammars/grammars13/gramOberon}
\doublePageEmpty

%%%%%%%%%%%%%%
% APPEND E: PUG Newsletters
%%%%%%%%%%%%%%

\begin{enumerate}[I.]

\item 1974

\begin{enumerate}[i.]

\item \texttt{Newsletter \#2, Page 1:} \textit{Historia de Pascal documentada.}

\item \texttt{Newsletter \#2, Page 6:} \textit{Wirth especifica Pascal 6000-3.4.}

\item \texttt{Newsletter \#2, Page 18:} \textit{Wirth especifica Pascal-P (Máquina-P, probablemente P1).}

\end{enumerate}

\item 1975

\begin{enumerate}[i.]

\item \texttt{Newsletter \#3 , Page 1:} \textit{Manual de usuario y comentarios publicados (Se establece como Primera Versión).}

\item \texttt{Newsletter \#3 , Page 4:} \textit{Historia de Pascal corregida}

\item \texttt{Newsletter \#3 , Page 10:} \textit{Pascal-P2.}

\end{enumerate}

\item 1976

\begin{enumerate}[i.]

\item \texttt{Newsletter \#4 , Page 40:} \textit{Per Brinch Hansen discute sobre la concurrencia en Pascal.}

\item \texttt{Newsletter \#4 , Page 81:} \textit{Pascal P4 publicado.}

\end{enumerate}

\item 1978

\begin{enumerate}[i.]

\item \texttt{Newsletter \#11 Page 64:} \textit{Discursión sobre ISO Standard Pascal.}

\item \texttt{Newsletter \#11 Page 70:} \textit{Pascal P4 notas de desarrollos ¿Es P4 estándar Pascal?).}

\item \texttt{Newsletter \#12 Page 7:} \textit{Palabras reservadas e identificadores en Frances e Inglés.}

\item \texttt{Newsletter \#12 Page 17:} \textit{Aplicaciones sobre el código fuente de Pascal.}

\item \texttt{Newsletter \#12 Page 33:} \textit{Méritos del análisis y diseño de Pascal.}

\item \texttt{Newsletter \#13 Page 13:} \textit{Debate sobre las derivaciones y omisiones del estándar Pascal por parte de UCSD Pascal.}

\item \texttt{Newsletter \#13 Page 34:} \textit{Pascal prettyprinter (Hueras).}

\item \texttt{Newsletter \#13 Page 45:} \textit{Pascal prettyprinter (Condict).}

\item \texttt{Newsletter \#13 Page 83:} \textit{Cartas de Wirth y otros sobre el borrador del estándar ISO.}

\item \texttt{Newsletter \#13 Page 84:} \textit{Carta sobre ISO Pascal (Wichmann).}

\item \texttt{Newsletter \#13 Page 86:} \textit{Anuncio del grupo de para el estándar ANSI.}

\item \texttt{Newsletter \#13 Page 92:} \textit{Interfaz Entrada/Salida descrita.}

\end{enumerate}

\item 1979

\begin{enumerate}[i.]

\item \texttt{Newsletter \#14 Page 5:} \textit{Working draft of BSI/ISO Pascal standard.}

\item \texttt{Newsletter \#15 Page 7:} \textit{Comentarios sobre ADA.}

\item \texttt{Newsletter \#14 Page 31:} \textit{Programa de traducción ID2ID.}

\item \texttt{Newsletter \#14 Page 35:} \textit{Formato de texto para programas.}

\item \texttt{Newsletter \#14 Page 62:} \textit{How to process scope in Pascal (A. Sale).}

\item \texttt{Newsletter \#14 Page 63:} \textit{Cooperación con Pascal-S.}

\item \texttt{Newsletter \#14 Page 90:} \textit{Progreso del estándar de Pascal.}

\item \texttt{Newsletter \#14 Page 99:} \textit{Pascal validation suite available.}

\item \texttt{Newsletter \#14 Page 102:} \textit{Modula-2.}

\item \texttt{Newsletter \#14 Page 112:} \textit{UCSD becomes commercial product.}

\end{enumerate}

\item 1980

\begin{enumerate}[i.]

\item \texttt{Newsletter \#17 Page 12:} \textit{Report on Ada.}

\item \texttt{Newsletter \#17 Page 18:} \textit{Pascal cross reference program.}

\item \texttt{Newsletter \#17 Page 29:} \textit{Procesador de macro para Pascal.}

\item \texttt{Newsletter \#17 Page 54:} \textit{Conformant array parameters proposed.}

\end{enumerate}

\end{enumerate}
\doublePageEmpty

%%%%%%%%%%%%%%
% APPEND F: GNU
%%%%%%%%%%%%%%

\include{./anexos/E/gnu}
\doublePageEmpty
\addcontentsline{toc}{section}{Notas}
\pagestyle{plain}
\theendnotes
\doublePageEmpty

%%%%%%%%%%%%%%
% APPEND G: Linux
%%%%%%%%%%%%%%

\include{./anexos/F/linux}
\doublePageEmpty
\addcontentsline{toc}{section}{Notas}
\pagestyle{plain}
\theendnotes
\doublePageEmpty

%%%%%%%%%%%%%%
% APPEND H: ASCII codes 
%%%%%%%%%%%%%%

\pagestyle{plain}
\chapter{Tabla ASCII $[0, (2^8 - 1)]$}

\begin{table}[h]
\begin{center}
{\footnotesize
  \begin{tabular}{|cccc|ccc|ccc|ccc|}
    \hline
    000\texttt{d} & 00\texttt{h} & \NUL & (nul) & 032\texttt{d} & 20\texttt{h} & \textvisiblespace & 064\texttt{d} & 40\texttt{h} & @ & 096\texttt{d} & 60\texttt{h} & \textquoteleft \\
    001\texttt{d} & 01\texttt{h} & \SOH & (soh) & 033\texttt{d} & 21\texttt{h} & ! & 065\texttt{d} & 41\texttt{h} & A & 097\texttt{d} & 61\texttt{h} & a \\
    002\texttt{d} & 02\texttt{h} & \STX & (stx) & 034\texttt{d} & 22\texttt{h} & " & 066\texttt{d} & 42\texttt{h} & B & 098\texttt{d} & 62\texttt{h} & b \\
    003\texttt{d} & 03\texttt{h} & \ETX & (etx) & 035\texttt{d} & 23\texttt{h} & \# & 067\texttt{d} & 43\texttt{h} & C & 099\texttt{d} & 63\texttt{h} & c \\
    004\texttt{d} & 04\texttt{h} & \EOT & (eot) & 036\texttt{d} & 24\texttt{h} & \$ & 068\texttt{d} & 44\texttt{h} & D & 100\texttt{d} & 64\texttt{h} & d \\
    005\texttt{d} & 05\texttt{h} & \ENQ & (enq) & 037\texttt{d} & 25\texttt{h} & \% & 069\texttt{d} & 45\texttt{h} & E & 101\texttt{d} & 65\texttt{h} & e \\
    006\texttt{d} & 06\texttt{h} & \ACK & (ack) & 038\texttt{d} & 26\texttt{h} & \& & 070\texttt{d} & 46\texttt{h} & F & 102\texttt{d} & 66\texttt{h} & f \\
    007\texttt{d} & 07\texttt{h} & \BEL & (bel) & 039\texttt{d} & 27\texttt{h} & \textquotesingle & 071\texttt{d} & 47\texttt{h} & G & 103\texttt{d} & 67\texttt{h} & g \\
    008\texttt{d} & 08\texttt{h} & \BS & (bs) & 040\texttt{d} & 28\texttt{h} & ( & 072\texttt{d} & 48\texttt{h} & H & 104\texttt{d} & 68\texttt{h} & h \\
    009\texttt{d} & 09\texttt{h} & ~ & (tab) & 041\texttt{d} & 29\texttt{h} & ) & 073\texttt{d} & 49\texttt{h} & I & 105\texttt{d} & 69\texttt{h} & i \\
    010\texttt{d} & 0A\texttt{h} & \LF & (lf) & 042\texttt{d} & 2A\texttt{h} & * & 074\texttt{d} & 4A\texttt{h} & J & 106\texttt{d} & 6A\texttt{h} & j \\
    011\texttt{d} & 0B\texttt{h} & \VT & (vt) & 043\texttt{d} & 2B\texttt{h} & + & 075\texttt{d} & 4B\texttt{h} & K & 107\texttt{d} & 6B\texttt{h} & k \\
    012\texttt{d} & 0C\texttt{h} & ~ & (np) & 044\texttt{d} & 2C\texttt{h} & \textquoteright & 076\texttt{d} & 4C\texttt{h} & L & 108\texttt{d} & 6C\texttt{h} & l \\
    013\texttt{d} & 0D\texttt{h} & \CR & (cr) & 045\texttt{d} & 2D\texttt{h} & - & 077\texttt{d} & 4D\texttt{h} & M & 109\texttt{d} & 6D\texttt{h} & m \\
    014\texttt{d} & 0E\texttt{h} & \SO & (so) & 046\texttt{d} & 2E\texttt{h} & . & 078\texttt{d} & 4E\texttt{h} & N & 110\texttt{d} & 6E\texttt{h} & n \\
    015\texttt{d} & 0F\texttt{h} & \SI & (si) & 047\texttt{d} & 2F\texttt{h} & / & 079\texttt{d} & 4F\texttt{h} & O & 111\texttt{d} & 6F\texttt{h} & o \\
    016\texttt{d} & 10\texttt{h} & \DLE & (dle) & 048\texttt{d} & 30\texttt{h} & 0 & 080\texttt{d} & 50\texttt{h} & P & 112\texttt{d} & 70\texttt{h} & p \\
    017\texttt{d} & 11\texttt{h} & \DCa & (dc1) & 049\texttt{d} & 31\texttt{h} & 1 & 081\texttt{d} & 51\texttt{h} & Q & 113\texttt{d} & 71\texttt{h} & q \\
    018\texttt{d} & 12\texttt{h} & \DCb & (dc2) & 050\texttt{d} & 32\texttt{h} & 2 & 082\texttt{d} & 52\texttt{h} & R & 114\texttt{d} & 72\texttt{h} & r \\
    019\texttt{d} & 13\texttt{h} & \DCc & (dc3) & 051\texttt{d} & 33\texttt{h} & 3 & 083\texttt{d} & 53\texttt{h} & S & 115\texttt{d} & 73\texttt{h} & s \\
    020\texttt{d} & 14\texttt{h} & \DCd & (dc4) & 052\texttt{d} & 34\texttt{h} & 4 & 084\texttt{d} & 54\texttt{h} & T & 116\texttt{d} & 74\texttt{h} & t \\
    021\texttt{d} & 15\texttt{h} & \NAK & (nak) & 053\texttt{d} & 35\texttt{h} & 5 & 085\texttt{d} & 55\texttt{h} & U & 117\texttt{d} & 75\texttt{h} & u \\
    022\texttt{d} & 16\texttt{h} & \SYN & (syn) & 054\texttt{d} & 36\texttt{h} & 6 & 086\texttt{d} & 56\texttt{h} & V & 118\texttt{d} & 76\texttt{h} & v \\
    023\texttt{d} & 17\texttt{h} & \ETB & (etb) & 055\texttt{d} & 37\texttt{h} & 7 & 087\texttt{d} & 57\texttt{h} & W & 119\texttt{d} & 77\texttt{h} & w \\
    024\texttt{d} & 18\texttt{h} & \CAN & (can) & 056\texttt{d} & 38\texttt{h} & 8 & 088\texttt{d} & 58\texttt{h} & X & 120\texttt{d} & 78\texttt{h} & x \\
    025\texttt{d} & 19\texttt{h} & \EM & (em) & 057\texttt{d} & 39\texttt{h} & 9 & 089\texttt{d} & 59\texttt{h} & Y & 121\texttt{d} & 79\texttt{h} & y \\
    026\texttt{d} & 1A\texttt{h} & ~ & (eof) & 058\texttt{d} & 3A\texttt{h} & : & 090\texttt{d} & 5A\texttt{h} & Z & 122\texttt{d} & 7A\texttt{h} & z \\
    027\texttt{d} & 1B\texttt{h} & \ESC & (esc) & 059\texttt{d} & 3B\texttt{h} & ; & 091\texttt{d} & 5B\texttt{h} & [ & 123\texttt{d} & 7B\texttt{h} & \char`\{ \\
    028\texttt{d} & 1C\texttt{h} & \FS & (fs) & 060\texttt{d} & 3C\texttt{h} & < & 092\texttt{d} & 5C\texttt{h} & \char`\\ & 124\texttt{d} & 7C\texttt{h} & | \\
    029\texttt{d} & 1D\texttt{h} & \GS & (gs) & 061\texttt{d} & 3D\texttt{h} & = & 093\texttt{d} & 5D\texttt{h} & ] & 125\texttt{d} & 7D\texttt{h} & \char`\} \\
    030\texttt{d} & 1E\texttt{h} & \RS & (rs) & 062\texttt{d} & 3E\texttt{h} & > & 094\texttt{d} & 5E\texttt{h} & \^{} & 126\texttt{d} & 7E\texttt{h} & \~{} \\
    031\texttt{d} & 1F\texttt{h} & \US & (us) & 063\texttt{d} & 3F\texttt{h} & ? & 095\texttt{d} & 5F\texttt{h} & \char`\_ & 127\texttt{d} & 7F\texttt{h} & \DEL \\
    \hline
  \end{tabular}
}
\caption{Tabla ASCII $[0, (2^7 - 1)]$}
\end{center}
\end{table}
 
\newpage

\begin{table}[h]
\begin{center}
{\footnotesize
  \begin{tabular}{|ccc|ccc|ccc|ccc|}
    \hline
    128\texttt{d} & 80\texttt{h} & \euro & 160\texttt{d} & A0\texttt{h} & \NBSP & 192\texttt{d} & C0\texttt{h} & \`{A} & 224\texttt{d} & E0\texttt{h} & \`{a} \\
    129\texttt{d} & 81\texttt{h} & ~ & 161\texttt{d} & A1\texttt{h} & !` & 193\texttt{d} & C1\texttt{h} & \'{A} & 225\texttt{d} & E1\texttt{h} & \'{a} \\
    130\texttt{d} & 82\texttt{h} & \quotesinglbase & 162\texttt{d} & A2\texttt{h} & \textcent & 194\texttt{d} & C2\texttt{h} & \^{A} & 226\texttt{d} & E2\texttt{h} & \^{a} \\
    131\texttt{d} & 83\texttt{h} & \textit{f} & 163\texttt{d} & A3\texttt{h} & \pounds & 195\texttt{d} & C3\texttt{h} & \~{A} & 227\texttt{d} & E3\texttt{h} & \~{a} \\
    132\texttt{d} & 84\texttt{h} & \quotedblbase & 164\texttt{d} & A4\texttt{h} & \textcurrency & 196\texttt{d} & C4\texttt{h} & \"{A} & 228\texttt{d} & E4\texttt{h} & \"{a} \\
    133\texttt{d} & 85\texttt{h} & \dots & 165\texttt{d} & A5\texttt{h} & \textyen & 197\texttt{d} & C5\texttt{h} & \AA & 229\texttt{d} & E5\texttt{h} & \aa \\
    134\texttt{d} & 86\texttt{h} & \dag & 166\texttt{d} & A6\texttt{h} & \textbrokenbar & 198\texttt{d} & C6\texttt{h} & \AE & 230\texttt{d} & E6\texttt{h} & \ae \\
    135\texttt{d} & 87\texttt{h} & \ddag & 167\texttt{d} & A7\texttt{h} & \S & 199\texttt{d} & C7\texttt{h} & \c{C} & 231\texttt{d} & E7\texttt{h} & \c{c} \\
    136\texttt{d} & 88\texttt{h} & \textasciicircum & 168\texttt{d} & A8\texttt{h} & \textasciidieresis & 200\texttt{d} & C8\texttt{h} & \`{E} & 232\texttt{d} & E8\texttt{h} & \`{e} \\
    137\texttt{d} & 89\texttt{h} & \textperthousand & 169\texttt{d} & A9\texttt{h} & \textrm{\small\textcopyright} & 201\texttt{d} & C9\texttt{h} & \'{E} & 233\texttt{d} & E9\texttt{h} & \'{e} \\
    138\texttt{d} & 8A\texttt{h} & \v{S} & 170\texttt{d} & AA\texttt{h} & \textordfeminine & 202\texttt{d} & CA\texttt{h} & \^{E} & 234\texttt{d} & EA\texttt{h} & \^{e} \\
    139\texttt{d} & 8B\texttt{h} & \guilsinglleft & 171\texttt{d} & AB\texttt{h} & \guillemotleft & 203\texttt{d} & CB\texttt{h} & \"{E} & 235\texttt{d} & EB\texttt{h} & \"{e} \\
    140\texttt{d} & 8C\texttt{h} & \OE & 172\texttt{d} & AC\texttt{h} & \textlnot & 204\texttt{d} & CC\texttt{h} & \`{I} & 236\texttt{d} & EC\texttt{h} & \`{i} \\
    141\texttt{d} & 8D\texttt{h} & ~ & 173\texttt{d} & AD\texttt{h} & \- & 205\texttt{d} & CD\texttt{h} & \'{I} & 237\texttt{d} & ED\texttt{h} & \'{i} \\
    142\texttt{d} & 8E\texttt{h} & \v{Z} & 174\texttt{d} & AE\texttt{h} & \textrm{\small\textregistered} & 206\texttt{d} & CE\texttt{h} & \^{I} & 238\texttt{d} & EE\texttt{h} & \^{i} \\
    143\texttt{d} & 8F\texttt{h} & ~ & 175\texttt{d} & AF\texttt{h} & \textasciimacron & 207\texttt{d} & CF\texttt{h} & \"{I} & 239\texttt{d} & EF\texttt{h} & \"{i} \\
    144\texttt{d} & 90\texttt{h} & ~ & 176\texttt{d} & B0\texttt{h} & \textdegree & 208\texttt{d} & D0\texttt{h} & \DH & 240\texttt{d} & F0\texttt{h} & \dh \\
    145\texttt{d} & 91\texttt{h} & ` & 177\texttt{d} & B1\texttt{h} & \textpm & 209\texttt{d} & D1\texttt{h} & \~{N} & 241\texttt{d} & F1\texttt{h} & \~{n} \\
    146\texttt{d} & 92\texttt{h} & ' & 178\texttt{d} & B2\texttt{h} & \texttwosuperior & 210\texttt{d} & D2\texttt{h} & \`{O} & 242\texttt{d} & F2\texttt{h} & \`{o} \\
    147\texttt{d} & 93\texttt{h} & `` & 179\texttt{d} & B3\texttt{h} & \textthreesuperior & 211\texttt{d} & D3\texttt{h} & \'{O} & 243\texttt{d} & F3\texttt{h} & \'{o} \\
    148\texttt{d} & 94\texttt{h} & '' & 180\texttt{d} & B4\texttt{h} & \textasciiacute & 212\texttt{d} & D4\texttt{h} & \^{O} & 244\texttt{d} & F4\texttt{h} & \^{o} \\
    149\texttt{d} & 95\texttt{h} & \textbullet & 181\texttt{d} & B5\texttt{h} & \textmu & 213\texttt{d} & D5\texttt{h} & \~{O} & 245\texttt{d} & F5\texttt{h} & \~{o} \\
    150\texttt{d} & 96\texttt{h} & -- & 182\texttt{d} & B6\texttt{h} & \P & 214\texttt{d} & D6\texttt{h} & \"{O} & 246\texttt{d} & F6\texttt{h} & \"{o} \\
    151\texttt{d} & 97\texttt{h} & --- & 183\texttt{d} & B7\texttt{h} & \textperiodcentered & 215\texttt{d} & D7\texttt{h} & \texttimes & 247\texttt{d} & F7\texttt{h} & \textdiv \\
    152\texttt{d} & 98\texttt{h} & \textasciitilde & 184\texttt{d} & B8\texttt{h} & \c{} & 216\texttt{d} & D8\texttt{h} & \O & 248\texttt{d} & F8\texttt{h} & \o \\
    153\texttt{d} & 99\texttt{h} & \texttrademark & 185\texttt{d} & B9\texttt{h} & \textonesuperior & 217\texttt{d} & D9\texttt{h} & \`{U} & 249\texttt{d} & F9\texttt{h} & \`{u} \\
    154\texttt{d} & 9A\texttt{h} & \v{s} & 186\texttt{d} & BA\texttt{h} & \textordmasculine & 218\texttt{d} & DA\texttt{h} & \'{U} & 250\texttt{d} & FA\texttt{h} & \'{u} \\
    155\texttt{d} & 9B\texttt{h} & \guilsinglright & 187\texttt{d} & BB\texttt{h} & \guillemotright & 219\texttt{d} & DB\texttt{h} & \^{U} & 251\texttt{d} & FB\texttt{h} & \^{u} \\
    156\texttt{d} & 9C\texttt{h} & \oe & 188\texttt{d} & BC\texttt{h} & \textonequarter & 220\texttt{d} & DC\texttt{h} & \"{U} & 252\texttt{d} & FC\texttt{h} & \"{u} \\
    157\texttt{d} & 9D\texttt{h} & ~ & 189\texttt{d} & BD\texttt{h} & \textonehalf & 221\texttt{d} & DD\texttt{h} & \'{Y} & 253\texttt{d} & FD\texttt{h} & \'{y} \\
    158\texttt{d} & 9E\texttt{h} & \v{z} & 190\texttt{d} & BE\texttt{h} & \textthreequarters & 222\texttt{d} & DE\texttt{h} & \TH & 254\texttt{d} & FE\texttt{h} & \th \\
    159\texttt{d} & 9F\texttt{h} & \"{Y} & 191\texttt{d} & BF\texttt{h} & ?` & 223\texttt{d} & DF\texttt{h} & \ss & 255\texttt{d} & FF\texttt{h} & \"{y} \\
    \hline
  \end{tabular}
}
\caption{Tabla ASCII $[2^7, (2^8 - 1)]$}
\end{center}
\end{table}
\doublePageEmpty

%%%%%%%%%%%%%%
% APPEND I: GPL version 2.0
%%%%%%%%%%%%%%

\include{anexos/H/gpl20}
\doublePageEmpty

%%%%%%%%%%%%%%
% General Bib
%%%%%%%%%%%%%%

\addstarredchapter{Bibliografía}
\renewcommand*{\bibname}{Bibliografía}
%\onecolumn
%\let\OLDthebibliography=\thebibliography
%\def\thebibliography#1{\OLDthebibliography{#1}
%    \addcontentsline{toc}{chapter}{\bibname}}
\nocite{*}
\bibliography{lib}\bibliographystyle{alpha}
\doublePageEmpty

%%%%%%%%%%%%%%
% Index of Book
%%%%%%%%%%%%%%


\addstarredchapter{Índice alfabético}

\def\malgroup#1{\par\medskip\textbf{\large$\sim$#1$\sim$\newline}\par\nopagebreak}



\begingroup
\def\thispagestyle#1{}
\printindex
\endgroup



\end{document} 
