\textbf{Ken Thomson}, nacio en Nueva Orleans (New Orleans, EEUU) el 4 de Febrero de 1943. Se trata de un prestigioso cientítico de la computación conocido historicamente como uno de los creadores del Sistema Opreativo UNIX. Thomson se gradua en 1966 como Ingeniero Electrónico y de Ciencias de la Computación por la Universidad de Berkley (California). Comenzó a trabajar para los Laboratorio Bell en el proyecto Multics bajo la suprevisión de Elwyn Berlekamp.


A comienzos de los años sesenta del siglo XX, Thomson en compañia de Ritchie continuaron en solitario (sin financiación) una nueva versión de Multics. La base de esta nueva implementación gira en torno a las utilidades Software creadas para el juego "Space Travel". En 1969 se publica la primera versión de este Sistema Operativo llamado UNICS (posteriormente UNIX).


Thomson también ha trabajado en el editor QED, que hace un potente uso de las expresiones regulares, base de posteriores lenguajes como Perl.


Otro hito de la historia de Ken Thomson es su participación en el desarrollo de UTF-8 junto a Rob Pike en 1992.


En el año 2000 Thomson deja los Laboratorios Bell para en 2006 trabajar como ingeniero distinguido en la empresa Google Inc.


Entre todos sus reconocimientos por su trabajo, destacamos los siguientes:

\begin{itemize}
\item En 1983, junto a Ritchie recibe el prestigios premio Turing.
\item En 1990, de nuevo con Denis Ritchie recibe el premio IEEE Richard W. Hamming Medal del instituto IEEE (Institute of Electrical and Electronics Engineers).
\item En 1997 entra a formar parte del Museo Historico de la Computación (Computer History Museum) con su compañero Ritchie.
\item En 1999 Thomson es elegido por IEEE para recibir el primer premio Tsutomu Kanai Award.
\end{itemize}