
\section{Aspectos de eficiencia}
\subsection{¿Qué es la eficiencia?}

Realmente si nos paramos a pensar en lo que nos rodea, y diremos que mucho del trabajo realizado (\textit{medida física}) no se hace de manera eficiente. Existen causas que intervienen directamente, internas y externas para los diferente agentes que generan trabajo\footnote{Determinadas por \textit{Ortega y Gasset} como el \textit{Yo y las Circunstancias}.}:

\begin{enumerate}
\item \textbf{Factores internos:} Determinados por las características propias del agente que está realizando el esfuerzo\footnote{Trataremos al género humano como un agente más.}. Estados del agente, el volumen de esfuerzo a realizar, estadísticos del agente, etc. Podemos decir que un agente no es igual a otro aunque estemos comparando máquinas diseñadas a través de un proceso industrial equitativo, no existen en el mundo dos máquinas iguales, debido a la intervención del superagente, el ser consciente, el ser humano\footnote{La perfección no existe, y se puede demostrar.}.
\item \textbf{Factores externos:} Determinados por el contexto en el que desempeña el esfuerzo.
\end{enumerate}


El esfuerzo está mayormente determinado por el sujeto (\textit{observador}) que trata el esfuerzo de una agente. En estas notas trataremos la eficiencia (\textit{esfuerzo algorítmico}).


Por último me gustaría resaltar el hecho de que ser eficiente no significa rendir al 100\%, ya que esto, provoca, sino está debidamente preparado el sistema, diremos que ningún sistema real lo está, deficiencias e incorrecciones en el funcionamiento del mismo\footnote{Hecho igualmente demostrable.}.
\subsection{Medidas asintóticas para al eficiencia de algoritmos}

\subsubsection{Notación: $\mathcal{O}$}
\defn Sea $f: \mathbf(N) \longrightarrow \mathbf(R)^+ \cup \lbrace0\rbrace$. El conjunto $\mathcal{O}(f(n))$, cumple:
\begin{equation}
\mathcal{O}(f(n)) = \lbrace g:\mathbf(N) \longrightarrow \mathbf(R)^+ \cup \lbrace 0 \rbrace \vert \exists  c\in \mathbf(R)^+, n_0 \in \mathbf(N) \diagdown: \forall n \geq n_0 \cdot g(n) \leq cf(n)\rbrace
\end{equation}

\subsubsection{Notación: $\Omega$}
\defn Sea $f: \mathbf(N) \longrightarrow \mathbf(R)^+ \cup \lbrace0\rbrace$. El conjunto $\Omega(f(n))$, cumple: 
\begin{equation}
\Omega(f(n)) = \lbrace g:\mathbf(N) \longrightarrow \mathbf(R)^+ \cup \lbrace 0 \rbrace \vert \exists  c\in \mathbf(R)^+, n_0 \in \mathbf(N) \diagdown: \forall n \geq n_0 \cdot g(n) \geq cf(n)\rbrace
\end{equation}

\subsubsection{Notación: $\Theta$}
\defn El conjunto $\Theta(f(n))$, llamado \textit{de orden exacto de} $f(n)$, cumple:
\begin{equation}
\Theta(f(n)) = \mathcal{O}(f(n)) \cap \Omega(f(n))
\end{equation}

\subsection{Propiedades de los órdenes}

Denotamos orden como: $\Phi$:

\begin{equation}
\Phi(f(n)) + \Phi(g(n)) = f\prime (n) + g\prime(n)
\end{equation}

\begin{equation}
\Phi(f(n)) \cdot \Phi(g(n)) = f\prime (n) \cdot g\prime(n) 
\end{equation}


\subsection{Principales ordenes de complejidad}
Se observan en la Figura (). El orden ideal para un algoritmo es claramente $n\log n$.

\begin{figure}[h]
\centering
\begin{subfigure}[A]{0.45\textwidth}
\centering
% GNUPLOT: LaTeX picture using PSTRICKS macros
% Define new PST objects, if not already defined
\ifx\PSTloaded\undefined
\def\PSTloaded{t}
\psset{arrowsize=.01 3.2 1.4 .3}
\psset{dotsize=.01}
\catcode`@=11

\newpsobject{PST@Border}{psline}{linewidth=.0015,linestyle=solid}
\newpsobject{PST@Axes}{psline}{linewidth=.0015,linestyle=dotted,dotsep=.004}
\newpsobject{PST@Solid}{psline}{linewidth=.0015,linestyle=solid}
\newpsobject{PST@Dashed}{psline}{linewidth=.0015,linestyle=dashed,dash=.01 .01}
\newpsobject{PST@Dotted}{psline}{linewidth=.0025,linestyle=dotted,dotsep=.008}
\newpsobject{PST@LongDash}{psline}{linewidth=.0015,linestyle=dashed,dash=.02 .01}
\newpsobject{PST@Diamond}{psdots}{linewidth=.001,linestyle=solid,dotstyle=square,dotangle=45}
\newpsobject{PST@Filldiamond}{psdots}{linewidth=.001,linestyle=solid,dotstyle=square*,dotangle=45}
\newpsobject{PST@Cross}{psdots}{linewidth=.001,linestyle=solid,dotstyle=+,dotangle=45}
\newpsobject{PST@Plus}{psdots}{linewidth=.001,linestyle=solid,dotstyle=+}
\newpsobject{PST@Square}{psdots}{linewidth=.001,linestyle=solid,dotstyle=square}
\newpsobject{PST@Circle}{psdots}{linewidth=.001,linestyle=solid,dotstyle=o}
\newpsobject{PST@Triangle}{psdots}{linewidth=.001,linestyle=solid,dotstyle=triangle}
\newpsobject{PST@Pentagon}{psdots}{linewidth=.001,linestyle=solid,dotstyle=pentagon}
\newpsobject{PST@Fillsquare}{psdots}{linewidth=.001,linestyle=solid,dotstyle=square*}
\newpsobject{PST@Fillcircle}{psdots}{linewidth=.001,linestyle=solid,dotstyle=*}
\newpsobject{PST@Filltriangle}{psdots}{linewidth=.001,linestyle=solid,dotstyle=triangle*}
\newpsobject{PST@Fillpentagon}{psdots}{linewidth=.001,linestyle=solid,dotstyle=pentagon*}
\newpsobject{PST@Arrow}{psline}{linewidth=.001,linestyle=solid}
\catcode`@=12

\fi
\psset{unit=5.0in,xunit=5.0in,yunit=3.0in}
\pspicture(0.000000,0.000000)(0.750000,0.750000)
\ifx\nofigs\undefined
\catcode`@=11

\PST@Border(0.1492,0.0840)
(0.1642,0.0840)

\PST@Border(0.6247,0.0840)
(0.6097,0.0840)

\rput[r](0.1332,0.0840){-6}
\PST@Border(0.1492,0.1897)
(0.1642,0.1897)

\PST@Border(0.6247,0.1897)
(0.6097,0.1897)

\rput[r](0.1332,0.1897){-4}
\PST@Border(0.1492,0.2953)
(0.1642,0.2953)

\PST@Border(0.6247,0.2953)
(0.6097,0.2953)

\rput[r](0.1332,0.2953){-2}
\PST@Border(0.1492,0.4010)
(0.1642,0.4010)

\PST@Border(0.6247,0.4010)
(0.6097,0.4010)

\rput[r](0.1332,0.4010){ 0}
\PST@Border(0.1492,0.5066)
(0.1642,0.5066)

\PST@Border(0.6247,0.5066)
(0.6097,0.5066)

\rput[r](0.1332,0.5066){ 2}
\PST@Border(0.1492,0.6123)
(0.1642,0.6123)

\PST@Border(0.6247,0.6123)
(0.6097,0.6123)

\rput[r](0.1332,0.6123){ 4}
\PST@Border(0.1492,0.7179)
(0.1642,0.7179)

\PST@Border(0.6247,0.7179)
(0.6097,0.7179)

\rput[r](0.1332,0.7179){ 6}
\PST@Border(0.1968,0.0840)
(0.1968,0.1040)

\PST@Border(0.1968,0.7179)
(0.1968,0.6979)

\rput(0.1968,0.0420){-4}
\PST@Border(0.2919,0.0840)
(0.2919,0.1040)

\PST@Border(0.2919,0.7179)
(0.2919,0.6979)

\rput(0.2919,0.0420){-2}
\PST@Border(0.3870,0.0840)
(0.3870,0.1040)

\PST@Border(0.3870,0.7179)
(0.3870,0.6979)

\rput(0.3870,0.0420){ 0}
\PST@Border(0.4821,0.0840)
(0.4821,0.1040)

\PST@Border(0.4821,0.7179)
(0.4821,0.6979)

\rput(0.4821,0.0420){ 2}
\PST@Border(0.5772,0.0840)
(0.5772,0.1040)

\PST@Border(0.5772,0.7179)
(0.5772,0.6979)

\rput(0.5772,0.0420){ 4}
\PST@Axes(0.1492,0.4010)
(0.6247,0.4010)

\PST@Axes(0.3870,0.0840)
(0.3870,0.7179)

\PST@Solid(0.1492,0.1368)
(0.1492,0.1368)
(0.1540,0.1422)
(0.1588,0.1475)
(0.1636,0.1528)
(0.1684,0.1582)
(0.1732,0.1635)
(0.1780,0.1688)
(0.1828,0.1742)
(0.1876,0.1795)
(0.1924,0.1848)
(0.1972,0.1902)
(0.2020,0.1955)
(0.2068,0.2009)
(0.2116,0.2062)
(0.2164,0.2115)
(0.2212,0.2169)
(0.2260,0.2222)
(0.2309,0.2275)
(0.2357,0.2329)
(0.2405,0.2382)
(0.2453,0.2435)
(0.2501,0.2489)
(0.2549,0.2542)
(0.2597,0.2595)
(0.2645,0.2649)
(0.2693,0.2702)
(0.2741,0.2756)
(0.2789,0.2809)
(0.2837,0.2862)
(0.2885,0.2916)
(0.2933,0.2969)
(0.2981,0.3022)
(0.3029,0.3076)
(0.3077,0.3129)
(0.3125,0.3182)
(0.3173,0.3236)
(0.3221,0.3289)
(0.3269,0.3343)
(0.3317,0.3396)
(0.3365,0.3449)
(0.3413,0.3503)
(0.3461,0.3556)
(0.3509,0.3609)
(0.3557,0.3663)
(0.3605,0.3716)
(0.3653,0.3769)
(0.3701,0.3823)
(0.3749,0.3876)
(0.3797,0.3929)
(0.3845,0.3983)
(0.3894,0.4036)
(0.3942,0.4090)
(0.3990,0.4143)
(0.4038,0.4196)
(0.4086,0.4250)
(0.4134,0.4303)
(0.4182,0.4356)
(0.4230,0.4410)
(0.4278,0.4463)
(0.4326,0.4516)
(0.4374,0.4570)
(0.4422,0.4623)
(0.4470,0.4676)
(0.4518,0.4730)
(0.4566,0.4783)
(0.4614,0.4837)
(0.4662,0.4890)
(0.4710,0.4943)
(0.4758,0.4997)
(0.4806,0.5050)
(0.4854,0.5103)
(0.4902,0.5157)
(0.4950,0.5210)
(0.4998,0.5263)
(0.5046,0.5317)
(0.5094,0.5370)
(0.5142,0.5424)
(0.5190,0.5477)
(0.5238,0.5530)
(0.5286,0.5584)
(0.5334,0.5637)
(0.5382,0.5690)
(0.5430,0.5744)
(0.5479,0.5797)
(0.5527,0.5850)
(0.5575,0.5904)
(0.5623,0.5957)
(0.5671,0.6010)
(0.5719,0.6064)
(0.5767,0.6117)
(0.5815,0.6171)
(0.5863,0.6224)
(0.5911,0.6277)
(0.5959,0.6331)
(0.6007,0.6384)
(0.6055,0.6437)
(0.6103,0.6491)
(0.6151,0.6544)
(0.6199,0.6597)
(0.6247,0.6651)

\catcode`@=12
\fi
\endpspicture


\caption{$f(x) = x$}

\end{subfigure}%
\quad
\begin{subfigure}[B]{0.45\textwidth}
\centering
% GNUPLOT: LaTeX picture using PSTRICKS macros
% Define new PST objects, if not already defined
\ifx\PSTloaded\undefined
\def\PSTloaded{t}
\psset{arrowsize=.01 3.2 1.4 .3}
\psset{dotsize=.01}
\catcode`@=11

\newpsobject{PST@Border}{psline}{linewidth=.0015,linestyle=solid}
\newpsobject{PST@Axes}{psline}{linewidth=.0015,linestyle=dotted,dotsep=.004}
\newpsobject{PST@Solid}{psline}{linewidth=.0015,linestyle=solid}
\newpsobject{PST@Dashed}{psline}{linewidth=.0015,linestyle=dashed,dash=.01 .01}
\newpsobject{PST@Dotted}{psline}{linewidth=.0025,linestyle=dotted,dotsep=.008}
\newpsobject{PST@LongDash}{psline}{linewidth=.0015,linestyle=dashed,dash=.02 .01}
\newpsobject{PST@Diamond}{psdots}{linewidth=.001,linestyle=solid,dotstyle=square,dotangle=45}
\newpsobject{PST@Filldiamond}{psdots}{linewidth=.001,linestyle=solid,dotstyle=square*,dotangle=45}
\newpsobject{PST@Cross}{psdots}{linewidth=.001,linestyle=solid,dotstyle=+,dotangle=45}
\newpsobject{PST@Plus}{psdots}{linewidth=.001,linestyle=solid,dotstyle=+}
\newpsobject{PST@Square}{psdots}{linewidth=.001,linestyle=solid,dotstyle=square}
\newpsobject{PST@Circle}{psdots}{linewidth=.001,linestyle=solid,dotstyle=o}
\newpsobject{PST@Triangle}{psdots}{linewidth=.001,linestyle=solid,dotstyle=triangle}
\newpsobject{PST@Pentagon}{psdots}{linewidth=.001,linestyle=solid,dotstyle=pentagon}
\newpsobject{PST@Fillsquare}{psdots}{linewidth=.001,linestyle=solid,dotstyle=square*}
\newpsobject{PST@Fillcircle}{psdots}{linewidth=.001,linestyle=solid,dotstyle=*}
\newpsobject{PST@Filltriangle}{psdots}{linewidth=.001,linestyle=solid,dotstyle=triangle*}
\newpsobject{PST@Fillpentagon}{psdots}{linewidth=.001,linestyle=solid,dotstyle=pentagon*}
\newpsobject{PST@Arrow}{psline}{linewidth=.001,linestyle=solid}
\catcode`@=12

\fi
\psset{unit=5.0in,xunit=5.0in,yunit=3.0in}
\pspicture(0.000000,0.000000)(0.750000,0.750000)
\ifx\nofigs\undefined
\catcode`@=11

\PST@Border(0.1572,0.0840)
(0.1722,0.0840)

\PST@Border(0.6327,0.0840)
(0.6177,0.0840)

\rput[r](0.1412,0.0840){ 0}
\PST@Border(0.1572,0.2108)
(0.1722,0.2108)

\PST@Border(0.6327,0.2108)
(0.6177,0.2108)

\rput[r](0.1412,0.2108){ 5}
\PST@Border(0.1572,0.3376)
(0.1722,0.3376)

\PST@Border(0.6327,0.3376)
(0.6177,0.3376)

\rput[r](0.1412,0.3376){ 10}
\PST@Border(0.1572,0.4643)
(0.1722,0.4643)

\PST@Border(0.6327,0.4643)
(0.6177,0.4643)

\rput[r](0.1412,0.4643){ 15}
\PST@Border(0.1572,0.5911)
(0.1722,0.5911)

\PST@Border(0.6327,0.5911)
(0.6177,0.5911)

\rput[r](0.1412,0.5911){ 20}
\PST@Border(0.1572,0.7179)
(0.1722,0.7179)

\PST@Border(0.6327,0.7179)
(0.6177,0.7179)

\rput[r](0.1412,0.7179){ 25}
\PST@Border(0.2048,0.0840)
(0.2048,0.1040)

\PST@Border(0.2048,0.7179)
(0.2048,0.6979)

\rput(0.2048,0.0420){-4}
\PST@Border(0.2999,0.0840)
(0.2999,0.1040)

\PST@Border(0.2999,0.7179)
(0.2999,0.6979)

\rput(0.2999,0.0420){-2}
\PST@Border(0.3950,0.0840)
(0.3950,0.1040)

\PST@Border(0.3950,0.7179)
(0.3950,0.6979)

\rput(0.3950,0.0420){ 0}
\PST@Border(0.4901,0.0840)
(0.4901,0.1040)

\PST@Border(0.4901,0.7179)
(0.4901,0.6979)

\rput(0.4901,0.0420){ 2}
\PST@Border(0.5852,0.0840)
(0.5852,0.1040)

\PST@Border(0.5852,0.7179)
(0.5852,0.6979)

\rput(0.5852,0.0420){ 4}
\PST@Axes(0.1572,0.0840)
(0.6327,0.0840)

\PST@Axes(0.3950,0.0840)
(0.3950,0.7179)

\PST@Solid(0.1572,0.7179)
(0.1572,0.7179)
(0.1620,0.6925)
(0.1668,0.6677)
(0.1716,0.6434)
(0.1764,0.6196)
(0.1812,0.5963)
(0.1860,0.5735)
(0.1908,0.5513)
(0.1956,0.5296)
(0.2004,0.5083)
(0.2052,0.4876)
(0.2100,0.4675)
(0.2148,0.4478)
(0.2196,0.4287)
(0.2244,0.4100)
(0.2292,0.3919)
(0.2340,0.3743)
(0.2389,0.3573)
(0.2437,0.3407)
(0.2485,0.3247)
(0.2533,0.3091)
(0.2581,0.2941)
(0.2629,0.2796)
(0.2677,0.2657)
(0.2725,0.2522)
(0.2773,0.2393)
(0.2821,0.2269)
(0.2869,0.2150)
(0.2917,0.2036)
(0.2965,0.1927)
(0.3013,0.1824)
(0.3061,0.1725)
(0.3109,0.1632)
(0.3157,0.1544)
(0.3205,0.1462)
(0.3253,0.1384)
(0.3301,0.1311)
(0.3349,0.1244)
(0.3397,0.1182)
(0.3445,0.1125)
(0.3493,0.1073)
(0.3541,0.1027)
(0.3589,0.0986)
(0.3637,0.0949)
(0.3685,0.0918)
(0.3733,0.0892)
(0.3781,0.0872)
(0.3829,0.0856)
(0.3877,0.0846)
(0.3925,0.0841)
(0.3974,0.0841)
(0.4022,0.0846)
(0.4070,0.0856)
(0.4118,0.0872)
(0.4166,0.0892)
(0.4214,0.0918)
(0.4262,0.0949)
(0.4310,0.0986)
(0.4358,0.1027)
(0.4406,0.1073)
(0.4454,0.1125)
(0.4502,0.1182)
(0.4550,0.1244)
(0.4598,0.1311)
(0.4646,0.1384)
(0.4694,0.1462)
(0.4742,0.1544)
(0.4790,0.1632)
(0.4838,0.1725)
(0.4886,0.1824)
(0.4934,0.1927)
(0.4982,0.2036)
(0.5030,0.2150)
(0.5078,0.2269)
(0.5126,0.2393)
(0.5174,0.2522)
(0.5222,0.2657)
(0.5270,0.2796)
(0.5318,0.2941)
(0.5366,0.3091)
(0.5414,0.3247)
(0.5462,0.3407)
(0.5510,0.3573)
(0.5559,0.3743)
(0.5607,0.3919)
(0.5655,0.4100)
(0.5703,0.4287)
(0.5751,0.4478)
(0.5799,0.4675)
(0.5847,0.4876)
(0.5895,0.5083)
(0.5943,0.5296)
(0.5991,0.5513)
(0.6039,0.5735)
(0.6087,0.5963)
(0.6135,0.6196)
(0.6183,0.6434)
(0.6231,0.6677)
(0.6279,0.6925)
(0.6327,0.7179)

\catcode`@=12
\fi
\endpspicture


\caption{$f(x) = x^2$}

\end{subfigure}%
\quad
\begin{subfigure}[C]{0.45\textwidth}
\centering
% GNUPLOT: LaTeX picture using PSTRICKS macros
% Define new PST objects, if not already defined
\ifx\PSTloaded\undefined
\def\PSTloaded{t}
\psset{arrowsize=.01 3.2 1.4 .3}
\psset{dotsize=.01}
\catcode`@=11

\newpsobject{PST@Border}{psline}{linewidth=.0015,linestyle=solid}
\newpsobject{PST@Axes}{psline}{linewidth=.0015,linestyle=dotted,dotsep=.004}
\newpsobject{PST@Solid}{psline}{linewidth=.0015,linestyle=solid}
\newpsobject{PST@Dashed}{psline}{linewidth=.0015,linestyle=dashed,dash=.01 .01}
\newpsobject{PST@Dotted}{psline}{linewidth=.0025,linestyle=dotted,dotsep=.008}
\newpsobject{PST@LongDash}{psline}{linewidth=.0015,linestyle=dashed,dash=.02 .01}
\newpsobject{PST@Diamond}{psdots}{linewidth=.001,linestyle=solid,dotstyle=square,dotangle=45}
\newpsobject{PST@Filldiamond}{psdots}{linewidth=.001,linestyle=solid,dotstyle=square*,dotangle=45}
\newpsobject{PST@Cross}{psdots}{linewidth=.001,linestyle=solid,dotstyle=+,dotangle=45}
\newpsobject{PST@Plus}{psdots}{linewidth=.001,linestyle=solid,dotstyle=+}
\newpsobject{PST@Square}{psdots}{linewidth=.001,linestyle=solid,dotstyle=square}
\newpsobject{PST@Circle}{psdots}{linewidth=.001,linestyle=solid,dotstyle=o}
\newpsobject{PST@Triangle}{psdots}{linewidth=.001,linestyle=solid,dotstyle=triangle}
\newpsobject{PST@Pentagon}{psdots}{linewidth=.001,linestyle=solid,dotstyle=pentagon}
\newpsobject{PST@Fillsquare}{psdots}{linewidth=.001,linestyle=solid,dotstyle=square*}
\newpsobject{PST@Fillcircle}{psdots}{linewidth=.001,linestyle=solid,dotstyle=*}
\newpsobject{PST@Filltriangle}{psdots}{linewidth=.001,linestyle=solid,dotstyle=triangle*}
\newpsobject{PST@Fillpentagon}{psdots}{linewidth=.001,linestyle=solid,dotstyle=pentagon*}
\newpsobject{PST@Arrow}{psline}{linewidth=.001,linestyle=solid}
\catcode`@=12

\fi
\psset{unit=5.0in,xunit=5.0in,yunit=3.0in}
\pspicture(0.000000,0.000000)(0.750000,0.750000)
\ifx\nofigs\undefined
\catcode`@=11

\PST@Border(0.1652,0.0840)
(0.1802,0.0840)

\PST@Border(0.6407,0.0840)
(0.6257,0.0840)

\rput[r](0.1492,0.0840){-150}
\PST@Border(0.1652,0.1897)
(0.1802,0.1897)

\PST@Border(0.6407,0.1897)
(0.6257,0.1897)

\rput[r](0.1492,0.1897){-100}
\PST@Border(0.1652,0.2953)
(0.1802,0.2953)

\PST@Border(0.6407,0.2953)
(0.6257,0.2953)

\rput[r](0.1492,0.2953){-50}
\PST@Border(0.1652,0.4010)
(0.1802,0.4010)

\PST@Border(0.6407,0.4010)
(0.6257,0.4010)

\rput[r](0.1492,0.4010){ 0}
\PST@Border(0.1652,0.5066)
(0.1802,0.5066)

\PST@Border(0.6407,0.5066)
(0.6257,0.5066)

\rput[r](0.1492,0.5066){ 50}
\PST@Border(0.1652,0.6123)
(0.1802,0.6123)

\PST@Border(0.6407,0.6123)
(0.6257,0.6123)

\rput[r](0.1492,0.6123){ 100}
\PST@Border(0.1652,0.7179)
(0.1802,0.7179)

\PST@Border(0.6407,0.7179)
(0.6257,0.7179)

\rput[r](0.1492,0.7179){ 150}
\PST@Border(0.2128,0.0840)
(0.2128,0.1040)

\PST@Border(0.2128,0.7179)
(0.2128,0.6979)

\rput(0.2128,0.0420){-4}
\PST@Border(0.3079,0.0840)
(0.3079,0.1040)

\PST@Border(0.3079,0.7179)
(0.3079,0.6979)

\rput(0.3079,0.0420){-2}
\PST@Border(0.4030,0.0840)
(0.4030,0.1040)

\PST@Border(0.4030,0.7179)
(0.4030,0.6979)

\rput(0.4030,0.0420){ 0}
\PST@Border(0.4981,0.0840)
(0.4981,0.1040)

\PST@Border(0.4981,0.7179)
(0.4981,0.6979)

\rput(0.4981,0.0420){ 2}
\PST@Border(0.5932,0.0840)
(0.5932,0.1040)

\PST@Border(0.5932,0.7179)
(0.5932,0.6979)

\rput(0.5932,0.0420){ 4}
\PST@Axes(0.1652,0.4010)
(0.6407,0.4010)

\PST@Axes(0.4030,0.0840)
(0.4030,0.7179)

\PST@Solid(0.1652,0.1368)
(0.1652,0.1368)
(0.1700,0.1525)
(0.1748,0.1676)
(0.1796,0.1820)
(0.1844,0.1958)
(0.1892,0.2091)
(0.1940,0.2217)
(0.1988,0.2338)
(0.2036,0.2453)
(0.2084,0.2563)
(0.2132,0.2667)
(0.2180,0.2767)
(0.2228,0.2861)
(0.2276,0.2951)
(0.2324,0.3035)
(0.2372,0.3115)
(0.2420,0.3191)
(0.2469,0.3262)
(0.2517,0.3329)
(0.2565,0.3392)
(0.2613,0.3450)
(0.2661,0.3505)
(0.2709,0.3557)
(0.2757,0.3604)
(0.2805,0.3648)
(0.2853,0.3689)
(0.2901,0.3727)
(0.2949,0.3761)
(0.2997,0.3793)
(0.3045,0.3822)
(0.3093,0.3848)
(0.3141,0.3872)
(0.3189,0.3893)
(0.3237,0.3912)
(0.3285,0.3928)
(0.3333,0.3943)
(0.3381,0.3956)
(0.3429,0.3967)
(0.3477,0.3976)
(0.3525,0.3984)
(0.3573,0.3991)
(0.3621,0.3996)
(0.3669,0.4000)
(0.3717,0.4004)
(0.3765,0.4006)
(0.3813,0.4008)
(0.3861,0.4009)
(0.3909,0.4009)
(0.3957,0.4009)
(0.4005,0.4009)
(0.4054,0.4010)
(0.4102,0.4010)
(0.4150,0.4010)
(0.4198,0.4010)
(0.4246,0.4011)
(0.4294,0.4013)
(0.4342,0.4015)
(0.4390,0.4019)
(0.4438,0.4023)
(0.4486,0.4028)
(0.4534,0.4035)
(0.4582,0.4043)
(0.4630,0.4052)
(0.4678,0.4063)
(0.4726,0.4076)
(0.4774,0.4091)
(0.4822,0.4107)
(0.4870,0.4126)
(0.4918,0.4147)
(0.4966,0.4171)
(0.5014,0.4197)
(0.5062,0.4226)
(0.5110,0.4258)
(0.5158,0.4292)
(0.5206,0.4330)
(0.5254,0.4371)
(0.5302,0.4415)
(0.5350,0.4462)
(0.5398,0.4514)
(0.5446,0.4569)
(0.5494,0.4627)
(0.5542,0.4690)
(0.5590,0.4757)
(0.5639,0.4828)
(0.5687,0.4904)
(0.5735,0.4984)
(0.5783,0.5068)
(0.5831,0.5158)
(0.5879,0.5252)
(0.5927,0.5352)
(0.5975,0.5456)
(0.6023,0.5566)
(0.6071,0.5681)
(0.6119,0.5802)
(0.6167,0.5928)
(0.6215,0.6061)
(0.6263,0.6199)
(0.6311,0.6343)
(0.6359,0.6494)
(0.6407,0.6651)

\catcode`@=12
\fi
\endpspicture


\caption{$f(x) = x^3$}

\end{subfigure}%
\quad
\begin{subfigure}[D]{0.45\textwidth}
\centering
% GNUPLOT: LaTeX picture using PSTRICKS macros
% Define new PST objects, if not already defined
\ifx\PSTloaded\undefined
\def\PSTloaded{t}
\psset{arrowsize=.01 3.2 1.4 .3}
\psset{dotsize=.01}
\catcode`@=11

\newpsobject{PST@Border}{psline}{linewidth=.0015,linestyle=solid}
\newpsobject{PST@Axes}{psline}{linewidth=.0015,linestyle=dotted,dotsep=.004}
\newpsobject{PST@Solid}{psline}{linewidth=.0015,linestyle=solid}
\newpsobject{PST@Dashed}{psline}{linewidth=.0015,linestyle=dashed,dash=.01 .01}
\newpsobject{PST@Dotted}{psline}{linewidth=.0025,linestyle=dotted,dotsep=.008}
\newpsobject{PST@LongDash}{psline}{linewidth=.0015,linestyle=dashed,dash=.02 .01}
\newpsobject{PST@Diamond}{psdots}{linewidth=.001,linestyle=solid,dotstyle=square,dotangle=45}
\newpsobject{PST@Filldiamond}{psdots}{linewidth=.001,linestyle=solid,dotstyle=square*,dotangle=45}
\newpsobject{PST@Cross}{psdots}{linewidth=.001,linestyle=solid,dotstyle=+,dotangle=45}
\newpsobject{PST@Plus}{psdots}{linewidth=.001,linestyle=solid,dotstyle=+}
\newpsobject{PST@Square}{psdots}{linewidth=.001,linestyle=solid,dotstyle=square}
\newpsobject{PST@Circle}{psdots}{linewidth=.001,linestyle=solid,dotstyle=o}
\newpsobject{PST@Triangle}{psdots}{linewidth=.001,linestyle=solid,dotstyle=triangle}
\newpsobject{PST@Pentagon}{psdots}{linewidth=.001,linestyle=solid,dotstyle=pentagon}
\newpsobject{PST@Fillsquare}{psdots}{linewidth=.001,linestyle=solid,dotstyle=square*}
\newpsobject{PST@Fillcircle}{psdots}{linewidth=.001,linestyle=solid,dotstyle=*}
\newpsobject{PST@Filltriangle}{psdots}{linewidth=.001,linestyle=solid,dotstyle=triangle*}
\newpsobject{PST@Fillpentagon}{psdots}{linewidth=.001,linestyle=solid,dotstyle=pentagon*}
\newpsobject{PST@Arrow}{psline}{linewidth=.001,linestyle=solid}
\catcode`@=12

\fi
\psset{unit=5.0in,xunit=5.0in,yunit=3.0in}
\pspicture(0.000000,0.000000)(0.750000,0.750000)
\ifx\nofigs\undefined
\catcode`@=11

\PST@Border(0.1652,0.0840)
(0.1802,0.0840)

\PST@Border(0.6407,0.0840)
(0.6257,0.0840)

\rput[r](0.1492,0.0840){-1.4}
\PST@Border(0.1652,0.1416)
(0.1802,0.1416)

\PST@Border(0.6407,0.1416)
(0.6257,0.1416)

\rput[r](0.1492,0.1416){-1.2}
\PST@Border(0.1652,0.1993)
(0.1802,0.1993)

\PST@Border(0.6407,0.1993)
(0.6257,0.1993)

\rput[r](0.1492,0.1993){-1}
\PST@Border(0.1652,0.2569)
(0.1802,0.2569)

\PST@Border(0.6407,0.2569)
(0.6257,0.2569)

\rput[r](0.1492,0.2569){-0.8}
\PST@Border(0.1652,0.3145)
(0.1802,0.3145)

\PST@Border(0.6407,0.3145)
(0.6257,0.3145)

\rput[r](0.1492,0.3145){-0.6}
\PST@Border(0.1652,0.3721)
(0.1802,0.3721)

\PST@Border(0.6407,0.3721)
(0.6257,0.3721)

\rput[r](0.1492,0.3721){-0.4}
\PST@Border(0.1652,0.4298)
(0.1802,0.4298)

\PST@Border(0.6407,0.4298)
(0.6257,0.4298)

\rput[r](0.1492,0.4298){-0.2}
\PST@Border(0.1652,0.4874)
(0.1802,0.4874)

\PST@Border(0.6407,0.4874)
(0.6257,0.4874)

\rput[r](0.1492,0.4874){ 0}
\PST@Border(0.1652,0.5450)
(0.1802,0.5450)

\PST@Border(0.6407,0.5450)
(0.6257,0.5450)

\rput[r](0.1492,0.5450){ 0.2}
\PST@Border(0.1652,0.6026)
(0.1802,0.6026)

\PST@Border(0.6407,0.6026)
(0.6257,0.6026)

\rput[r](0.1492,0.6026){ 0.4}
\PST@Border(0.1652,0.6603)
(0.1802,0.6603)

\PST@Border(0.6407,0.6603)
(0.6257,0.6603)

\rput[r](0.1492,0.6603){ 0.6}
\PST@Border(0.1652,0.7179)
(0.1802,0.7179)

\PST@Border(0.6407,0.7179)
(0.6257,0.7179)

\rput[r](0.1492,0.7179){ 0.8}
\PST@Border(0.2128,0.0840)
(0.2128,0.1040)

\PST@Border(0.2128,0.7179)
(0.2128,0.6979)

\rput(0.2128,0.0420){-4}
\PST@Border(0.3079,0.0840)
(0.3079,0.1040)

\PST@Border(0.3079,0.7179)
(0.3079,0.6979)

\rput(0.3079,0.0420){-2}
\PST@Border(0.4030,0.0840)
(0.4030,0.1040)

\PST@Border(0.4030,0.7179)
(0.4030,0.6979)

\rput(0.4030,0.0420){ 0}
\PST@Border(0.4981,0.0840)
(0.4981,0.1040)

\PST@Border(0.4981,0.7179)
(0.4981,0.6979)

\rput(0.4981,0.0420){ 2}
\PST@Border(0.5932,0.0840)
(0.5932,0.1040)

\PST@Border(0.5932,0.7179)
(0.5932,0.6979)

\rput(0.5932,0.0420){ 4}
\PST@Axes(0.1652,0.4874)
(0.6407,0.4874)

\PST@Axes(0.4030,0.0840)
(0.4030,0.7179)

\PST@Solid(0.4054,0.1138)
(0.4054,0.1138)
(0.4102,0.2513)
(0.4150,0.3152)
(0.4198,0.3573)
(0.4246,0.3887)
(0.4294,0.4138)
(0.4342,0.4347)
(0.4390,0.4526)
(0.4438,0.4683)
(0.4486,0.4822)
(0.4534,0.4948)
(0.4582,0.5061)
(0.4630,0.5166)
(0.4678,0.5262)
(0.4726,0.5351)
(0.4774,0.5435)
(0.4822,0.5513)
(0.4870,0.5587)
(0.4918,0.5656)
(0.4966,0.5722)
(0.5014,0.5785)
(0.5062,0.5844)
(0.5110,0.5901)
(0.5158,0.5956)
(0.5206,0.6008)
(0.5254,0.6058)
(0.5302,0.6106)
(0.5350,0.6152)
(0.5398,0.6197)
(0.5446,0.6240)
(0.5494,0.6282)
(0.5542,0.6322)
(0.5590,0.6361)
(0.5639,0.6399)
(0.5687,0.6436)
(0.5735,0.6472)
(0.5783,0.6507)
(0.5831,0.6540)
(0.5879,0.6573)
(0.5927,0.6605)
(0.5975,0.6637)
(0.6023,0.6667)
(0.6071,0.6697)
(0.6119,0.6726)
(0.6167,0.6755)
(0.6215,0.6782)
(0.6263,0.6810)
(0.6311,0.6836)
(0.6359,0.6862)
(0.6407,0.6888)

\catcode`@=12
\fi
\endpspicture


\caption{$f(x) = log(x)$}

\end{subfigure}%
\quad
\begin{subfigure}[E]{0.45\textwidth}
\centering
% GNUPLOT: LaTeX picture using PSTRICKS macros
% Define new PST objects, if not already defined
\ifx\PSTloaded\undefined
\def\PSTloaded{t}
\psset{arrowsize=.01 3.2 1.4 .3}
\psset{dotsize=.01}
\catcode`@=11

\newpsobject{PST@Border}{psline}{linewidth=.0015,linestyle=solid}
\newpsobject{PST@Axes}{psline}{linewidth=.0015,linestyle=dotted,dotsep=.004}
\newpsobject{PST@Solid}{psline}{linewidth=.0015,linestyle=solid}
\newpsobject{PST@Dashed}{psline}{linewidth=.0015,linestyle=dashed,dash=.01 .01}
\newpsobject{PST@Dotted}{psline}{linewidth=.0025,linestyle=dotted,dotsep=.008}
\newpsobject{PST@LongDash}{psline}{linewidth=.0015,linestyle=dashed,dash=.02 .01}
\newpsobject{PST@Diamond}{psdots}{linewidth=.001,linestyle=solid,dotstyle=square,dotangle=45}
\newpsobject{PST@Filldiamond}{psdots}{linewidth=.001,linestyle=solid,dotstyle=square*,dotangle=45}
\newpsobject{PST@Cross}{psdots}{linewidth=.001,linestyle=solid,dotstyle=+,dotangle=45}
\newpsobject{PST@Plus}{psdots}{linewidth=.001,linestyle=solid,dotstyle=+}
\newpsobject{PST@Square}{psdots}{linewidth=.001,linestyle=solid,dotstyle=square}
\newpsobject{PST@Circle}{psdots}{linewidth=.001,linestyle=solid,dotstyle=o}
\newpsobject{PST@Triangle}{psdots}{linewidth=.001,linestyle=solid,dotstyle=triangle}
\newpsobject{PST@Pentagon}{psdots}{linewidth=.001,linestyle=solid,dotstyle=pentagon}
\newpsobject{PST@Fillsquare}{psdots}{linewidth=.001,linestyle=solid,dotstyle=square*}
\newpsobject{PST@Fillcircle}{psdots}{linewidth=.001,linestyle=solid,dotstyle=*}
\newpsobject{PST@Filltriangle}{psdots}{linewidth=.001,linestyle=solid,dotstyle=triangle*}
\newpsobject{PST@Fillpentagon}{psdots}{linewidth=.001,linestyle=solid,dotstyle=pentagon*}
\newpsobject{PST@Arrow}{psline}{linewidth=.001,linestyle=solid}
\catcode`@=12

\fi
\psset{unit=5.0in,xunit=5.0in,yunit=3.0in}
\pspicture(0.000000,0.000000)(0.750000,0.750000)
\ifx\nofigs\undefined
\catcode`@=11

\PST@Border(0.1652,0.0840)
(0.1802,0.0840)

\PST@Border(0.6407,0.0840)
(0.6257,0.0840)

\rput[r](0.1492,0.0840){-0.5}
\PST@Border(0.1652,0.1632)
(0.1802,0.1632)

\PST@Border(0.6407,0.1632)
(0.6257,0.1632)

\rput[r](0.1492,0.1632){ 0}
\PST@Border(0.1652,0.2425)
(0.1802,0.2425)

\PST@Border(0.6407,0.2425)
(0.6257,0.2425)

\rput[r](0.1492,0.2425){ 0.5}
\PST@Border(0.1652,0.3217)
(0.1802,0.3217)

\PST@Border(0.6407,0.3217)
(0.6257,0.3217)

\rput[r](0.1492,0.3217){ 1}
\PST@Border(0.1652,0.4010)
(0.1802,0.4010)

\PST@Border(0.6407,0.4010)
(0.6257,0.4010)

\rput[r](0.1492,0.4010){ 1.5}
\PST@Border(0.1652,0.4802)
(0.1802,0.4802)

\PST@Border(0.6407,0.4802)
(0.6257,0.4802)

\rput[r](0.1492,0.4802){ 2}
\PST@Border(0.1652,0.5594)
(0.1802,0.5594)

\PST@Border(0.6407,0.5594)
(0.6257,0.5594)

\rput[r](0.1492,0.5594){ 2.5}
\PST@Border(0.1652,0.6387)
(0.1802,0.6387)

\PST@Border(0.6407,0.6387)
(0.6257,0.6387)

\rput[r](0.1492,0.6387){ 3}
\PST@Border(0.1652,0.7179)
(0.1802,0.7179)

\PST@Border(0.6407,0.7179)
(0.6257,0.7179)

\rput[r](0.1492,0.7179){ 3.5}
\PST@Border(0.2128,0.0840)
(0.2128,0.1040)

\PST@Border(0.2128,0.7179)
(0.2128,0.6979)

\rput(0.2128,0.0420){-4}
\PST@Border(0.3079,0.0840)
(0.3079,0.1040)

\PST@Border(0.3079,0.7179)
(0.3079,0.6979)

\rput(0.3079,0.0420){-2}
\PST@Border(0.4030,0.0840)
(0.4030,0.1040)

\PST@Border(0.4030,0.7179)
(0.4030,0.6979)

\rput(0.4030,0.0420){ 0}
\PST@Border(0.4981,0.0840)
(0.4981,0.1040)

\PST@Border(0.4981,0.7179)
(0.4981,0.6979)

\rput(0.4981,0.0420){ 2}
\PST@Border(0.5932,0.0840)
(0.5932,0.1040)

\PST@Border(0.5932,0.7179)
(0.5932,0.6979)

\rput(0.5932,0.0420){ 4}
\PST@Axes(0.1652,0.1632)
(0.6407,0.1632)

\PST@Axes(0.4030,0.0840)
(0.4030,0.7179)

\PST@Solid(0.4054,0.1529)
(0.4054,0.1529)
(0.4102,0.1436)
(0.4150,0.1393)
(0.4198,0.1379)
(0.4246,0.1386)
(0.4294,0.1408)
(0.4342,0.1442)
(0.4390,0.1488)
(0.4438,0.1542)
(0.4486,0.1605)
(0.4534,0.1675)
(0.4582,0.1752)
(0.4630,0.1835)
(0.4678,0.1923)
(0.4726,0.2017)
(0.4774,0.2115)
(0.4822,0.2218)
(0.4870,0.2325)
(0.4918,0.2437)
(0.4966,0.2551)
(0.5014,0.2670)
(0.5062,0.2792)
(0.5110,0.2917)
(0.5158,0.3045)
(0.5206,0.3176)
(0.5254,0.3310)
(0.5302,0.3446)
(0.5350,0.3586)
(0.5398,0.3727)
(0.5446,0.3872)
(0.5494,0.4018)
(0.5542,0.4167)
(0.5590,0.4318)
(0.5639,0.4471)
(0.5687,0.4627)
(0.5735,0.4784)
(0.5783,0.4943)
(0.5831,0.5104)
(0.5879,0.5267)
(0.5927,0.5432)
(0.5975,0.5599)
(0.6023,0.5767)
(0.6071,0.5937)
(0.6119,0.6109)
(0.6167,0.6282)
(0.6215,0.6457)
(0.6263,0.6633)
(0.6311,0.6811)
(0.6359,0.6990)
(0.6407,0.7171)

\catcode`@=12
\fi
\endpspicture


\caption{$f(x) = x \cdot log(x)$}

\end{subfigure}

\caption{Principales ordenes de Complejidad.}

\end{figure}

\subsection{¿Cómo calcular la eficiencia de un algoritmo?}
\paragraph*{Nota:} Nosotros trabajaremos con la notación $\mathcal{O}$.
\paragraph*{Reglas:} 

\begin{enumerate}
\item Asumiremos que las sentencias simples tienen complejidad constante $\equiv \mathcal{O}(1)$
\item Dada una situación de bifurcación $f(n) \vee g(n)$ del algoritmo se tomará siempre el camino más largo:
\ejem $g(n) \diagdown\ \forall n \Longrightarrow f(n) < g(n)$.
\end{enumerate}s active [1
\paragraph*{Nota:} Para determinar el orden de complejidad siempre elegiremos el peor de los casos, en virtud de que tengamos constancia de una cota máxima\footnote{El problema de la cota tiene una directa relación con los ordenes de complejidad antes comentados, puesto, que tanto, para $\mathcal{O}$, se establece una cota superior y para $\Omega$ una cota inferior.}. Simplifica a la hora de hacer los cálculos.

% \prog \texttt{hello.pas}
% 
% \lstinputlisting[language=Pascal]{./programms/programms41/Pascal/hello.pas}
% 
% \prog \texttt{roman.pas}
% 
% \lstinputlisting[language=Pascal]{./programms/programms41/Pascal/roman.pas}
% 
% \prog \texttt{basics.pas}
% 
% \lstinputlisting[language=Pascal]{./programms/programms41/Pascal/basics.pas}

