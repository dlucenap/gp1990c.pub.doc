%\pagestyle{plain}
\chapter{Linux}

\label{sec:mailLinus}

Parece ser que el punto de partida se fija el 25 de agosto de 1991, 20:57:08
GMT\index{25 de agosto de 1991, 20:57:08 GMT}, con el siguiente correo
electr\'onico\endnote{\input{./indice/entradas/mailTorvals.html}}:

\begin{verbatim}
From: torvalds@klaava.Helsinki.FI (Linus Benedict Torvalds)
Newsgroups: comp.os.minix
Subject: What would you like to see most in minix?
Summary: small poll for my new operating system
Message-ID:
Date: 25 Aug 91 20:57:08 GMT
Organization: University of Helsinki


Hello everybody out there using minix -

I'm doing a (free) operating system (just a hobby, won't be big and
professional like gnu) for 386(486) AT clones.  This has been brewing
since april, and is starting to get ready.  I'd like any feedback on
things people like/dislike in minix, as my OS resembles it somewhat
(same physical layout of the file-system (due to practical reasons)
among other things).

I've currently ported bash(1.08) and gcc(1.40), and things seem to work.
This implies that I'll get something practical within a few months, and
I'd like to know what features most people would want.  Any suggestions
are welcome, but I won't promise I'll implement them :-)

              Linus (torvalds@kruuna.helsinki.fi)

PS.  Yes - it's free of any minix code, and it has a multi-threaded fs.
It is NOT protable (uses 386 task switching etc), and it probably never
will support anything other than AT-harddisks, as that's all I have :-(.
\end{verbatim}

\section{Historia}
Linux nace a principio de los años noventa del siglo XX. Fue el resultado de 
enormes esfuerzos por parte de la cultura ``Hacker'' para tener la posibilidad 
de ejecutar un Sistema Operativo fiable y abierto como alternativa a los 
privativos y comerciales: MS-DOS de Microsoft y Mac OS de Apple.


La base de para Linux fue el código de Minix que era a su vez, un prototipo funcional de UNIX con alrededor de 2000 líneas 
de 
código y disponible para cualquier estudiante que tuviera acceso al manual 
``\textit{The Desing of Operating Systems}'' del profesor Tanenbaum.


Además y como hemos citado anteriormente, el Proyecto GNU proporcionaba un 
potente compilador GCC (GNU C Compiler) además de otras herramientas para y el 
desarrollo y una Licencia que daba al propio proyecto un ``marco legal''.


Se podría afirmar que Linux no era más que un hobby de un estudiante de segundo 
curso de Informátiva en la Universidad de Helsinki. 


Tras el correo escribo por Linus a la propia comunidad de Minix ver apartado 
(\ref{sec:mailLinus}). fue publicada la versión 0.0.1 en Septiembre de 1991 que 
a su vez sirvio como recurso publitario para una comunidad que empezaba a usar 
Internet y que era capaz de conectar grandes distancias en un tiempo muy corto. 
El trabajo inicialmente era una mejora de Minix, las versiones 0.10 y 0.11 
(sobretodo está última) comenzador a dar forma a una pieza de un Sistema 
Operativo, un nuevo núcleo para Sistemas GNU.


Dichas versiones incluían soporte para: VGA, EGA, controladoras Floppy y 
múltiple ``keymaps'' fruto del trabajo de la comunidad.


El trabajo intenso de los desarrolladores provocó que la siguiente versión a la 
0.12 fuera la 0.95. Linux fue duramente criticado por instituciones académicas 
dado que heredaba la arquitectura monolítica de Minix. Por ello y tras diversas 
discursiones se modifico la arquitectura del propio Kernel.


No debemos olvidar que gran parte de su éxito se debe al empuje comercial y la 
creación de importantes empresas en torno al propio Linux y otro Software GNU.


Debemos destarcar importantes empresas como Red Hat o Caldera. De igual manera, 
programadores y aficionados de todo el mundo comenzarón a trabajar es su 
propias 
versiones GNU/Linux por ejemplo: Ian Murdock con Debian o Peter Volkerding con 
Slackware.


Gracias a la participación del miles de desarrolladores, Linux es hoy día uno 
de 
los productos con capacidad de ejecutarse en múltiples dispositivos con 
arquitecturas muy distintas.


En Marzo de 1994 se publicó la versión 1.0 estable del Kernel.

%%%%%%%%%%
% Gráfico: Evolución de Linux
%%%%%%%%%%

\begin{landscape}
\pagestyle{empty}
\begin{figure}[h]
\begin{center}
\begin{pspicture}(19,15)%\psgrid
\psline[linecolor=black,linewidth=1pt]{->}(0,0.52)(19,0.52)
\psline[linecolor=black,linewidth=1pt]{-}(0,0.5)(0,15)

\rput(1.5,7){Linux 0.0.1}
\rput(1,0){1991}
\psline[linecolor=black,linewidth=0.8pt]{-}(1,0.3)(1,0.7)

%% to Slackware
\pscurve[linecolor=black,linewidth=1pt]{->}(1.4,7.4)(1.2,9.2)(2,9)
%% to Debian
\pscurve[linecolor=black,linewidth=1pt]{->}(1,7.4)(0.8,11.2)(2.2,11)
%% S.u.S.E
\pscurve[linecolor=black,linewidth=1pt]{->}(1,6.8)(2,2)(4,2)

\rput(3,11){Debian}
%% Evo
\psline[linecolor=black,linewidth=1pt]{-}(3.8,11)(7,11)
%% to Ubuntu
\pscurve[linecolor=black,linewidth=1pt]{->}(7,11)(9,12)(12.6,12)
%% to KNOPPIX
\pscurve[linecolor=black,linewidth=1pt]{->}(7,11)(9,10)(9.8,10)

\rput(3,9){Slackware}
%% to Gentoo
\pscurve[linestyle=dotted, linecolor=black,linewidth=1pt]{->}(3,8.6)(4,6)(8.2,6)
\rput(3,0){1993}
\psline[linecolor=black,linewidth=0.8pt]{-}(3,0.3)(3,0.7)
%% S.u.S.E
\pscurve[linecolor=black,linewidth=1pt]{->}(4,9)(4.8,9.2)(5,8.4)


\rput(5,8){S.u.S.E}
%% to SUSE
\psline[linecolor=black,linewidth=1pt]{->}(6,8)(12.2,8)
\rput(5,2){Red Hat}
%% Evo
\psline[linecolor=black,linewidth=1pt]{-}(6,2)(8,2)
%% to CentOS
\pscurve[linecolor=black,linewidth=1pt]{->}(8,2)(9,3)(12.2,3)
%% to Fedora
\pscurve[linecolor=black,linewidth=1pt]{->}(8,2)(9,1)(10.6,1)

\rput(5,0){1994}
\psline[linecolor=black,linewidth=0.8pt]{-}(5,0.3)(5,0.7)
%% Mandrake
\pscurve[linecolor=black,linewidth=1pt]{->}(5,2.4)(4.8,4.2)(6,4)


\rput(7,4){Mandrake}
%% to Mandriva
\psline[linecolor=black,linewidth=1pt]{->}(8,4)(14,4)
\rput(7,0){1998}
\psline[linecolor=black,linewidth=0.8pt]{-}(7,0.3)(7,0.7)

\rput(9,7){Arch}
\rput(9,6){Gentoo}
%% to Arch
\pscurve[linecolor=black,linewidth=1pt]{<->}(8.4,6.2)(7.2,7)(8.6,7)
\rput(9,0){2002}
\psline[linecolor=black,linewidth=0.8pt]{-}(9,0.3)(9,0.7)


\rput(11,10){KNOPPIX}
\rput(11.4,1){Fedora}
\rput(11,0){2003}
\psline[linecolor=black,linewidth=0.8pt]{-}(11,0.3)(11,0.7)

\rput(13.4,12){Ubuntu}
\rput(13,8){SUSE}
%% to OpenSUSE
\psline[linecolor=black,linewidth=1pt]{->}(13.8,8)(15.8,8)
\rput(13,3){CentOS}
\rput(13,0){2004}
\psline[linecolor=black,linewidth=0.8pt]{-}(13,0.3)(13,0.7)


\rput(15,4){Mandriva}
\rput(15,0){2005}
\psline[linecolor=black,linewidth=0.8pt]{-}(15,0.3)(15,0.7)

\rput(17,8){OpenSUSE}
\rput(17,0){2006}
\psline[linecolor=black,linewidth=0.8pt]{-}(17,0.3)(17,0.7)


%%% Leyenda

\pspolygon[fillstyle=solid,fillcolor=white](0.4,13.6)(0.4,15)(3.5,15)(3.5,13.6)

\psline[linecolor=black,linewidth=1pt]{->}(2.6,14.8)(3.4,14.8)
\rput(1.45,14.8){{\scriptsize{\textit{Evo. Directa:}}}}
\psline[linecolor=black,linewidth=1pt]{<->}(2.6,14.3)(3.4,14.3)
\rput(1.5,14.3){{\scriptsize{\textit{Evo. Concept:}}}}
\psline[linestyle=dotted, linecolor=black,linewidth=1pt]{->}(2.6,13.8)(3.4,13.8)
\rput(1.25,13.8){{\scriptsize{\textit{Influencia:}}}}
\end{pspicture}
\caption{Evolución de Linux y distintas distribuciones en el tiempo.}
\end{center}
\end{figure}
\end{landscape}

