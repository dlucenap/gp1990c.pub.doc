%\pagestyle{plain}
\chapter{GNU's Not UNIX!}

\section{UNIX}

%%%%%%%%%%
% Sec: UNIX
%%%%%%%%%%


Era el año 1965 cuando los Laboratorios
Bell\endnote{\href{http://www.alcatel-lucent.com/wps/portal/BellLabs}{http://www.alcatel-lucent.com/wps/portal/BellLabs}} de
AT\&T\endnote{\href{http://www.att.com/}{http://www.att.com/}}, el Instituto
Tecnológico
de Massachusetts\endnote{\href{http://web.mit.edu/}{http://web.mit.edu/}} y
General Electric\endnote{\href{http://www.ge.com/}{http://www.ge.com/}}
comenzarón a trabajar en el Sistema
Operativo
Multics\endnote{\href{http://www.multicians.org/}{http://www.multicians.org/}},
acrónimo de ``Multiplexed Information and Computing System''\endnote{``Sistema
de Computación e Información Multiplexada''}, diseñado
para funcionar en una máquina Mainframe modelo GE-645.


El proyecto era base de las ideas de construir un Sistema Operativo que
permitiese el acceso a distintos usuarios de modo simúltaneo, con el objetivo
principal de compartir los recusos de computación\endnote{En aquella época el
coste de computo era realmente elevado debido en gran parte al consumo
energético.}
Laboratorio Bell de AT\&T deciden abandonar el proyecto tras distintas versiones
fallidas de Multics.


Ken Thomson\endnote{\textbf{Ken Thomson}, nacio en Nueva Orleans (New Orleans, EEUU) el 4 de Febrero de 1943. Se trata de un prestigioso cientítico de la computación conocido historicamente como uno de los creadores del Sistema Opreativo UNIX. Thomson se gradua en 1966 como Ingeniero Electrónico y de Ciencias de la Computación por la Universidad de Berkley (California). Comenzó a trabajar para los Laboratorio Bell en el proyecto Multics bajo la suprevisión de Elwyn Berlekamp.


A comienzos de los años sesenta del siglo XX, Thomson en compañia de Ritchie continuaron en solitario (sin financiación) una nueva versión de Multics. La base de esta nueva implementación gira en torno a las utilidades Software creadas para el juego "Space Travel". En 1969 se publica la primera versión de este Sistema Operativo llamado UNICS (posteriormente UNIX).


Thomson también ha trabajado en el editor QED, que hace un potente uso de las expresiones regulares, base de posteriores lenguajes como Perl.


Otro hito de la historia de Ken Thomson es su participación en el desarrollo de UTF-8 junto a Rob Pike en 1992.


En el año 2000 Thomson deja los Laboratorios Bell para en 2006 trabajar como ingeniero distinguido en la empresa Google Inc.


Entre todos sus reconocimientos por su trabajo, destacamos los siguientes:

\begin{itemize}
\item En 1983, junto a Ritchie recibe el prestigios premio Turing.
\item En 1990, de nuevo con Denis Ritchie recibe el premio IEEE Richard W. Hamming Medal del instituto IEEE (Institute of Electrical and Electronics Engineers).
\item En 1997 entra a formar parte del Museo Historico de la Computación (Computer History Museum) con su compañero Ritchie.
\item En 1999 Thomson es elegido por IEEE para recibir el primer premio Tsutomu Kanai Award.
\end{itemize}}, programador
de los propios laboratorios, continuó trabajando sobre
la misma arquitectura original de Multics, la computadora
GE-6354, desarrollando
un juego espacial llamado ``Space Travel''\endnote{``Viaje Espacial''}. El
coste
de
cada partida se estimo en 75 dólares americanos de la época. El hecho del
redimiento tan desfavorable origino que de nuevo, Ken Thomson, con la ayuda de
Dennis Ritchie\endnote{\input{indice/entradas/dennisRitchie.html}},
reescribiese el juego para una computadora DEC PDP-7.

El trabajo invetido en esta nueva versión del juego, dio pie a la creación
de un nuevo Sistema Operativo. Los dos programadores junto Rudd
Canaday crearon
gran cantidad de Software de soporte. Entre otra utilidades crearon un nuevo
sistema de fichero distribuido y un potente intérprete de órdenes. 


El nuevo proyecto se denomino UNICS, acrónimo de ``Uniplexed Information and
Computing
System''\endnote{``Sistema de Computación e Información Uniplexada''}, que era
un juego de palabras sobre el citado MULTICS. Finalmente paso
a denominarse UNIX por influencia del hacker de MULTICS Brian
Kernighan\endnote{\input{indice/entradas/brianKernighan.html}}. 

El proyecto UNIX finalmente fue financiado por los Laboratorio Bell, debido a
que se trabajó para máquinas superiores de la
propia familia PDP\endnote{Programmed Data Processor (PDP) was a series of
minicomputers made and marketed by the Digital Equipment Corporation from 1957
to 1990. The name 'PDP' intentionally avoided the use of the term 'computer'
because, at the time of the first PDPs, computers had a reputation of being
large, complicated, and expensive machines, and the venture capitalists behind
Digital (especially Georges Doriot) would not support Digital's attempting to
build a "computer"; the word "minicomputer" had not yet been coined. So instead,
Digital used their existing line of logic modules to build a Programmable Data
Processor and aimed it at a market which could not afford the larger
computers.}. En concreto sobre el conjunto [PDP-11, PDP-20].


Corría el año 1970 y UNIX era una realidad. Destaca de las primeras versiones el
editor de textos 
\texttt{rundoff}\endnote{\href{http://www.gnu.org/software/groff/}{
http://www.gnu.org/software/groff/}}. Todas las versiones de UNIX escritas antes
de 1972 estaban codificadas en Lenguaje Ensamblador de las propias computadoras.
Este año se produjo un hito histórico para el Sistema UNIX, y es la reescritura
prácticamente completa del propio sistema a Lenguaje C\endnote{Decir que partes
de UNIX para aquel entonces y aun hoy en día se mantienen en código nativo de
cada procesador.}.


La protabilidad de UNIX era una realidad, y el propio AT\&T dotó a universidades
y empresas de es nuevo sistema. Destaca de las versiones de la matriz UNIX, la
distribución BSD (Berkley Software Distribution). El propio AT\&T creo un
departamentp para comercializar UNIX. Su desarrollo, en distintas versiones, se
descontinuó con la versión 7, en el año 1979. A principios de la década de los
80, AT\&T inició el proyecto Plan 9 del que ya era Software base el Sistema
X-Window del MIT. 


Le siguió un intento comercial sobre UNIX 7 denominada UNIX System III. Por
aquella época la dispersión de fabricantes era enorme. 


En 1993, la empresa Novell adquirió los laboratorios de UNIX de AT\&T. En ese
mismo año la versión BSD evolucionada en el tiempo, era constantemente demandada
debido a problemas de patentes.


En 1995 Novell creo su propia versión de UNIX llamada UNIX-Ware. 

%%%%%%%%%%
% Gráfico: Evolución de UNIX
%%%%%%%%%%

\begin{landscape}
\pagestyle{plain}
\begin{figure}[h]
\begin{center}
\begin{pspicture}(21,15)%\psgrid
\psline[linecolor=black,linewidth=1pt]{->}(0,0.52)(21,0.52)
\psline[linecolor=black,linewidth=1pt]{-}(0,0.5)(0,15)

\rput(1,7){Multics}
\rput(1,0){1965}
\psline[linecolor=black,linewidth=0.8pt]{-}(1,0.3)(1,0.7)

\psline[linecolor=black,linewidth=1pt]{<->}(1.8,7)(2.4,7)

\rput(3,7){Unics}
\rput(3,0){1969}
\psline[linecolor=black,linewidth=0.8pt]{-}(3,0.3)(3,0.7)

\pscurve[linecolor=black,linewidth=1pt]{->}(3,7.3)(2.8,8.2)(3.4,8)

\rput(4,8){Unix}
\rput(4,0){1971}
\psline[linecolor=black,linewidth=0.8pt]{-}(4,0.3)(4,0.7)

%% to BSD
\pscurve[linecolor=black,linewidth=1pt]{->}(4,7.6)(4.4,6.2)(3,5)(4.2,5)
%% to Xenix
\pscurve[linecolor=black,linewidth=1pt]{<->}(4.6,8)(7,9)(7,9.6)
%% to System III
\pscurve[linecolor=black,linewidth=1pt]{<->}(4,8.4)(4,11)(5.8,11)

\rput(5,5){BSD 1.0}
\rput(5,0){1978}
\psline[linecolor=black,linewidth=0.8pt]{-}(5,0.3)(5,0.7)

%% to Sun OS 1.0
\pscurve[linecolor=black,linewidth=1pt]{->}(5.4,5.4)(7,8)(7.8,8)
%% to AIX 1.0
\pscurve[linestyle=dotted, linecolor=black,linewidth=1pt]{->}(5.8,5.4)(7,6.2)(11.2,7)(10.2,8.6)
%% evo
\psline[linecolor=black,linewidth=1pt]{->}(6,5)(10,5)


\rput(7,10){Xenix 1.0}
\rput(7,11){System III}
\rput(7,0){1980}
\psline[linecolor=black,linewidth=0.8pt]{-}(7,0.3)(7,0.7)
%% Evo
\psline[linecolor=black,linewidth=1pt]{<-}(8.2,11)(11,11)
%% to Solaris
\pscurve[linecolor=black,linewidth=1pt]{->}(11,11)(13,12)(14.2,12)
%% to UnixWare 1.0
\psline[linecolor=black,linewidth=1pt]{->}(11,11)(15.6,11)

%% to HP/UX 1.0
\pscurve[linestyle=dotted, linecolor=black,linewidth=1pt]{->}(7,11.4)(7.2,13.2)(8.8,13)


%% to SCO Unix
\psline[linecolor=black,linewidth=1pt]{->}(8,10)(17.8,10)

\rput(9,8){Sun OS 1.0}
\rput(9,0){1982}
\psline[linecolor=black,linewidth=0.8pt]{-}(9,0.3)(9,0.7)


\rput(10,13){HP/UX 1.0}
\rput(10,9){AIX 1.0}
\rput(10,0){1985}
\psline[linecolor=black,linewidth=0.8pt]{-}(10,0.3)(10,0.7)


\rput(11,5){BSD 4.3}
\rput(11,0){1986}
\psline[linecolor=black,linewidth=0.8pt]{-}(11,0.3)(11,0.7)
%% Evo
\psline[linecolor=black,linewidth=1pt]{-}(12,5)(13,5)
%% to NetBSD
\pscurve[linecolor=black,linewidth=1pt]{->}(13,5)(15,6)(16,6)
%% to FreeBSD
\pscurve[linecolor=black,linewidth=1pt]{->}(13,5)(15,4)(16,4)

\rput(13,1){Minix 1.x}
\rput(13,0){1987}
\psline[linecolor=black,linewidth=0.8pt]{-}(13,0.3)(13,0.7)

%% Evo
\psline[linecolor=black,linewidth=1pt]{->}(14,1)(21,1)
%% to Linux 0.0.1
\pscurve[linestyle=dotted, linecolor=black,linewidth=1pt]{->}(13,1.4)(12.8,2.2)(13.8,2)


\rput(15,2){Linux 0.0.1}
%% Evo
\psline[linecolor=black,linewidth=1pt]{->}(16.4,2)(21,2)

\rput(15,12){Solaris}
\rput(15,0){1991}
\psline[linecolor=black,linewidth=0.8pt]{-}(15,0.3)(15,0.7)



\rput(17,4){FreeBSD}
%% Evo
\psline[linecolor=black,linewidth=1pt]{->}(18,4)(21,4)

\rput(17,6){NetBSD}
%% Evo
\psline[linecolor=black,linewidth=1pt]{->}(18,6)(21,6)
%% to OpenBSD
\pscurve[linecolor=black,linewidth=1pt]{<->}(17,6.4)(17.2,7.2)(19,7)


\rput(17,11){UnixWare 1.0}
\rput(17,0){1992}
\psline[linecolor=black,linewidth=0.8pt]{-}(17,0.3)(17,0.7)

\rput(19,10){SCO UNIX}
\rput(19,0){1993}
\psline[linecolor=black,linewidth=0.8pt]{-}(19,0.3)(19,0.7)

\rput(20,7){OpenBSD}

%%% Leyenda

\pspolygon[fillstyle=solid,fillcolor=white](0.4,13.6)(0.4,15)(3.5,15)(3.5,13.6)

\psline[linecolor=black,linewidth=1pt]{->}(2.6,14.8)(3.4,14.8)
\rput(1.45,14.8){{\scriptsize{\textit{Evo. Directa:}}}}
\psline[linecolor=black,linewidth=1pt]{<->}(2.6,14.3)(3.4,14.3)
\rput(1.5,14.3){{\scriptsize{\textit{Evo. Concept:}}}}
\psline[linestyle=dotted, linecolor=black,linewidth=1pt]{->}(2.6,13.8)(3.4,13.8)
\rput(1.25,13.8){{\scriptsize{\textit{Influencia:}}}}
\end{pspicture}
\caption{Evolución de UNIX y sus distintas versiones en el tiempo.}
\end{center}
\end{figure}
\end{landscape}



%%%%%%%%%%
% Sec: GNU
%%%%%%%%%%

\section{Proyecto GNU}

Lo cierto es que la historia que contamos nace en 1983 con la creaci\'on de
GNU\endnote{\href{http://www.gnu.org/}{http://www.gnu.org/}}
\cite{Stallman:1985:GM} por
parte de Richard Stallman\endnote{\input{indice/entradas/richardStallman.html}}.
Dos a\~{n}os despu\'es se crea la
asociaci\'on ``Free Software
Fundation''\endnote{\href{http://www.fsf.org/}{http://www.fsf.org/}}\index{Free
Software Fundation} con el
objetivo de promover el Software en torno a ciertas libertades, las cuales se
consolidan con
GPL\endnote{\href{http://www.gnu.org/copyleft/gpl.html}{
http://www.gnu.org/copyleft/gpl.html}}\index{GPL}, la licencia general y
p\'ublica de GNU.


Gran cantidad de Software se desarroll\'o entre la \'ultima etapa de la d\'ecada
de los ochenta y principios de los noventa del siglo XX. Era tal, dicha
cantidad, que se podr\'ia construir un clon de UNIX a falta de un n\'ucleo en
proyecto llamado
Hurd\endnote{\href{http://www.gnu.org/software/hurd/}{
http://www.gnu.org/software/hurd/}}. El problema surg\'ia de una mala
planificaci\'on y la
insistencia por crear un MiniKernel\index{MiniKernel}, que conllevaba a una
excesiva burocracia en la etapa de soluci\'on para cualquier posible fallo. Y es
que, por entonces, un estudiante finland\'es, Linus
Torvals\endnote{\input{indice/entradas/linusTorvals.html}}, primero
trabajando
con grandes m\'aquina y poco despu\'es con el 8386\index{8386} d\'onde a partir
de Minix\endnote{\href{http://www.minix3.org/}{http://www.minix3.org/}},
adapt\'andolo a sus necesidades escribe y extiende funcionalidades que
finalmente constituir\'ian un nuevo n\'ucleo muy discreto.

\subsection{Licencia GPL}

%%%%%%%%%%
% Gráfico: Evolución de GPL
%%%%%%%%%%

\begin{figure}[h]
\begin{center}
\begin{pspicture}(15,1.5)%\psgrid
\psline[linecolor=black,linewidth=1pt]{->}(0,0.5)(15,0.5)
\rput(1.5,1){Versión 1}

\psline[linecolor=black,linewidth=0.8pt]{-}(1.5,0.3)(1.5,0.7)

\rput(1.5,0){25 de Febrero de 1989}

\psline[linecolor=black,linewidth=1pt]{->}(2.6,1)(6.4,1)

\rput(7.5,1){Versión 2}

\psline[linecolor=black,linewidth=0.8pt]{-}(7.5,0.3)(7.5,0.7)

\rput(7.5,0){Junio de 1991}

\psline[linecolor=black,linewidth=1pt]{->}(8.6,1)(12.4,1)

\rput(13.5,1){Versión 3}

\psline[linecolor=black,linewidth=0.8pt]{-}(13.5,0.3)(13.5,0.7)

\rput(13.5,0){29 de Junio de 2007}

\end{pspicture}
\caption{Evolución de la Licencia GPL.}
\end{center}
\end{figure}




La General Public License (Licencia Pública de GNU) se trata de un marco legal
para desarrolladores y Software creada por Free Software Foundation (Fundación
de Software Libre).


Su primera versión se hizo pública el 25 de Febrero de 1989. Con este nuevo
``marco legal'' se pretendía consolidar el proyecto GNU que a su vez tenía su
esencia espiritual en la cultura Hacker de los años sesenta del siglo XX. Los
desarrolles que decidieran distribuir su Software bajo Copyright de GNU
permitian que se derivasen copias del mismo sin tener por ello una remuneración.
Al igual, se obligaba a que aquellos que mejorasen o ampliasen las
características de un programa con la licencia GPL publicasen dichas mejoras
para que la comunidad se siguiera enriqueciendo y creciendo.

La licencia GPL es compatible con otro tipo de licencias ``Libres'' partiendo de
la base de que estas últimas se basan en los postulados de la propia GPL.

De igual manera se han creado licencias basadas en GPL que añaden restriciones
para el uso y compartición del Software. Es la respuesta de la industria ante la
proliferación de la cultura GNU.

La licencia GPL ha tenido tres versiones oficiales hasta la fecha:

\begin{enumerate}[I.]
\item GPL versión 1: Como hemos citado anteriomente fue publicada el 25 de
Febrero de 1989 bajo las siguientes ideas:

\begin{enumerate}[i.]
\item Dado el problema que plantea el Software comercial con su política de
distribución de Software en formato ejecutable (binario), la licencia GPL
determina que los programas que se distribuyan con su Copyrights deben ir
acompañados del código fuente.

\item El segundo problema que intenta solucionar es el de la compatibilidad con
Software comercial, no tanto para oponerse a el, sino para proteger los
programas GPL. Por ello, la versiones modificadas de un Software GPL deben
acerse públicas. 
\end{enumerate}

\item GPL versión 2: Fue publicada en Junio de 1991 con los objetivos:

\begin{enumerate}[i.]
\item Hacer más restrictiva la publicación del Software, añadiendo la imperiosa
necesidad de acompañar a un fichero ejecutable de su código fuente.
 

\item De igual manera y basado en los problemas que presentaban las Bibliotecas
GNU con Software comercial, se creo una licencia específica para las misma;
\textit{Library
General Public License} (Licencia Pública General de Bibliotecas) bajo la misma
idea de que se deben abrir sus códigos fuente. 
\end{enumerate}

\item GPL versión 3: Tras un largo periodo de trabajo (el proyecto comienza en
2005) finalmente el 29 de Junio de 2007 se hizo pública la versión 3 de la
licencia GPL.

Entre sus novedades destacan:

\begin{enumerate}[i.]

\item Nuevas clausulas para formatos industriales restrictivos, como DRM
(Gestión Digital de Derechos).

\item Resolución de ambigüedades.

\item Adaptar la GPL a los marcos jurídicos de los distintos países.

\item Proteger a desarrolles de Software Libre frente a las patentes. 

\end{enumerate}

\end{enumerate}