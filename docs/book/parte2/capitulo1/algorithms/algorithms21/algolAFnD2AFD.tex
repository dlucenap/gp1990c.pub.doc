\begin{enumerate}[I.]

\item Entrada: AFnD $N$

\item Salida: AFD $N\prime$

\item Método:
{
\begin{enumerate}[i.]
\item Correspondencia entre Estados.

{\defn Los \'unicos estados $Q_{D}$ accesibles en AFD son subconjunto de $Q_{N}$
cerrados respecto a $\lambda$.}

\item Correspondencia del Estado Inicial.

{\defn El estado inicial de AFD es el resultado de calcular el cierre $\lambda$
del estado inicial del AFND.}

{\cor El \index{Cierre $\lambda$}Cierre $\lambda$, también llamado \index{$CLAUS_\lambda$}$CLAUS_\lambda$ para un estado, es el conjunto de estados alcanzables mediante cero o más transiciones.}

\item Cálculo de Transiciones.
{

\form $\delta_{D} (i,o) \backslash i \in Q_{D}$ para:

\begin{enumerate}[a.]
\item $i = \{p_{0}, p_{1}, p_{2}, \ldots, p_{n} \}$
\item Obtener: $\bigcup\limits^{n}_{i=0} \delta_{N}(p_{i},a) = [r_{0}, r_{1},
r_{2}, \ldots, r_{m} ]$
\item $\delta_{D} (i,o) = \bigcup\limits^{m}_{j=1}   CLAUS_{\lambda}(r_{j})$
\end{enumerate}
}

\item Correspondencia del Estado Final.

{\defn En el AFD un estado ser\'a final si contiene alg\'un estado final del
AFND.}

\end{enumerate}
}
\end{enumerate}

