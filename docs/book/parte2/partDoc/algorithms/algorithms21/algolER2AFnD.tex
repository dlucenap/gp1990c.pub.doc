\begin{enumerate}[I.]

\item Entrada: Expresión Regular $p \in \Sigma$

\item Salida: AFnD $M\prime$ que procesa $L(p)$

\item Método:
{
\begin{enumerate}[i.]
\item Definir la Expresión Regular $p \in \Sigma$.

\item Generar un AFnD para dicha Expresión Regular.

\item Aplicar el Algoritmo: Algoritmo: AFnD $\Rightarrow$ AFD (Ver Apartado \ref{algolAFnD2AFD})

\item Minimizar el Número de Estados:
{

\defn Para $\{p,q\} \in Q$ y $w \in \Sigma$, se dice que son Estados Equivalentes (no distinguibles):

\begin{equation}
w\longrightarrow \delta(p,w) = w\longrightarrow \delta(q,w) 
\end{equation}

{\cor La relación que se establece entre $p$ y $q$ es de Equivalencia.}
\form Para $\{p,q\} \in Q$:
\begin{enumerate}[i.]

\item Si $p$ es un estado final y $q$ no lo es, el par $\{p,q\}$ son distinguibles.

\item Si para dos estados $\{p,q\}$ se cumple:

\begin{equation}
\exists\ \delta\ t.q.\ \delta(p,w) = p_\lambda, \exists\ \delta\ t.q.\ \delta(q,w) = q_\beta \Rightarrow \ p_\lambda \neq q_\beta
\end{equation}

El par $\{p,q\}$ no son distinguibles.  

\end{enumerate}
}
\end{enumerate}
}
\end{enumerate}

