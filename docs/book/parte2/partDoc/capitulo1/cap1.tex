\label{chap:lexAnalycer}

\section{Introducción}

\defn La función de un \index{Analizador Léxico}Analizador Léxico es la de leer el/los archivo/s
de código fuente de un programa para relacionarlos con los Lexemas del
lenguaje, producir los Tokens e informar de errores a
nivel de lectura.

\defn Un \index{Lexema}Lexema es un conjunto de uno o más caracteres que
agrupan una determinada secuencia de caracteres del código fuente.

\paragraph*{Nota:} Los ejemplos se basan en Pascal ISO 7185:1990

\ejem `99'\ \textbar\ `3.14'\ \textbar\ `IF'\ \textbar\ `q'\ \textbar\ ``Hellow Programmer''

\defn Un Token es un identificador para un determinado Lexema.

\ejem Para el ejemplo anterior y en secuencia: INTEGER \textbar\ FLOAT \textbar\
WORD-SYMBOL \textbar\ CHAR \textbar\ STRING

\defn Un \index{Patrón}Patrón es una descripción formal de un Lexema (Ver Tabla \ref{tab:examplePatTokLex})

\begin{table}

\begin{center}

\begin{tabular}{|l|l|l|}\hline
\textbf{Lexema} & \textbf{Patrón} & \textbf{Token} \\
\hline
\hline
Identificador & $[A-Z, a-z, 0-9]^*$ & ``tokIdent'' \\ \hline
Entero & $[0-9]^*$ & ``tokInteger'' \\ \hline
Real & $[0-9].[0-9]^*$ & ``tokReal'' \\ \hline
\end{tabular}

\caption{Ejemplo de relación entre: Lexema, Patrón y Token.}\label{tab:examplePatTokLex}

\end{center}

\end{table}

{
\defn A la hora de trabajar con \index{Errores Léxicos}Errores Léxicos debemos tener en cuenta como queremos que sea de restrictivo nuestro Analizador. A grandes rasgos podemos establecer las siguientes categorías:

\begin{enumerate}[i.]

\item Analizadores ``Modo Pánico''

\item Analizadores con Recuperación

\end{enumerate}

La idea de esta división en la arquitectura radica en lo que conocemos como ``modo pánico'' es decir, ante una palabra del Lenguaje que o bien no pertenece al mismo o bien, genera ambigüedad.

El modelo mayormente aceptado es el primero Analizadores ``\index{Modo Pánico}Modo Pánico'', ante un error léxico, se termina el programa e informa al usuario/programador con el mayor detalle posible de dónde y por qué existe dicho error.

Por contra, la segunda categoría que hemos establecido: \index{Analizadores con Recuperación}Analizadores con Recuperación asume que ante el error léxico se debe actuar de manera lógica e intentar que el análisis no se detenga.

Es lógico entonces, dotar a este programa de cierta ``inteligencia sintáctica''.

La técnicas más comunes para recuperar el análisis son:

\begin{enumerate}[i.]

\item Borrar un carácter extraño.

\item Reemplazar/Sustituir caracteres.

\end{enumerate}

}

\begin{figure}[h]
\begin{center}
\begin{pspicture}(9,5)%\psgrid

\pspolygon[fillstyle=solid,fillcolor=white](0.5,3.7)(3.5,3.7)(3.5,4.4)(0.5,4.4)
\rput(2,4){{Código Fuente}}
\pspolygon[fillstyle=solid,fillcolor=white](5.6,3.7)(8.4,3.7)(8.4,4.4)(5.6,4.4)
\rput(7,4){{\tt gp1990la}}
\pspolygon[fillstyle=solid,fillcolor=white](5.2,0.7)(8.8,0.7)(8.8,1.4)(5.2,1.4)
\rput(7,1){{Tabla de Símbolos}}
\psline[linecolor=black,linewidth=1pt]{->}(3.6,4)(5.5,4)
\psline[linecolor=black,linewidth=1pt]{<->}(7,1.5)(7,3.6)
\end{pspicture}
\caption{Relación entre el Analizador Léxico y el Programa Fuente.}
\end{center}
\end{figure}

%%%%%%%%%%%%%%%%%%%%%%%%%
% Lenguajes
%%%%%%%%%%%%%%%%%%%%%%%%%

\section{Teoría de Lenguajes}

\subsection{Definiciones}\label{subsec:Lang}

\defn Decimos que un \textbf{\index{Alfabeto}Alfabeto} es un conjunto de elementos finito y no 
vació denotado como $\Sigma$.

{\cor Cada uno de estos elementos recibe el nombre de \textbf{\index{Símbolo}Símbolo}.}

\defn Existe una palabra común a todos los alfabetos que recibe el nombre de 
\textbf{\index{Palabra Vacía}Palabra Vacía}, denotada como: $\lambda$.

{\cor El conjunto de todas las palabras de un alfabeto (incluida la palabra 
vacía) se denota como: $\Sigma^*$}.

\ejem El Alfabeto $\Sigma = \{0,1\}$ es un conjunto de símbolos que a su vez 
forma parte de todas las palabras que utiliza cualquier sistema binario (por 
ejemplo un Disco Compacto).

\ejem El Alfabeto $\Sigma = \{a,b,c,\ldots,z\}$ es un conjunto de símbolos que 
a 
su vez forma parte de todas las palabras que utiliza la Lengua Castellana (que 
es un Lenguaje Humano).

\subsection{Palabras}

\defn Decimos que una \index{Palabra}Palabra es una concatenación de símbolos de un Alfabeto 
dado.

\begin{equation}
v \in \Sigma \Leftrightarrow v_i \in \Sigma
\end{equation}


\subsubsection{Operaciones}

\textbf{Nota:} Se trabajan con dos palabras: $u = [hola]$ y $v = [Mundo]$. La
palabra vacía se denotara con el símbolo $\lambda \Rightarrow |\lambda| = 0$

\begin{enumerate}[I.]

\item \index{Concatenación}Concatenación: Entendemos que para un alfabeto dado $\Sigma^*$ y dos 
palabras $u = \{u_1, u_2, \ldots, u_n\}$ y $v=\{v_1, v_2, \ldots, v_m\} \in 
\Sigma^*$ con $n,m \in \mathbb{N}$ por \textbf{concatenación} (denotado como 
$\circ$): {

\begin{equation}
u \circ v = \{u_1, u_2, \ldots, u_n, v_1, v_2, \ldots, v_m\} 
\end{equation}


\ejem Para las palabras $u\ \wedge\ v$ se tiene por
concatenación:

\begin{equation}
uv = [holaMundo]
\end{equation}

\paragraph*{Propiedades:}

\begin{enumerate}[i.]

%\item Dicha operación es cerrada: ?

\item Dicha operación es asociativa: Para $w = [azul]$ tenemos:

\begin{equation}
(u \circ v) \circ w \equiv  u \circ (v \circ w) = [holaMundoazul]
\end{equation}


\item Dicha operación tiene elemento neutro:

\begin{equation}
u  \circ \lambda \equiv u = [hola]
\end{equation}

\item Dicha operación no es conmutativa:

\begin{equation}
uv = [holaMundo] \neq vu = [Mundohola]
\end{equation}
 
\end{enumerate}

%}\item Binoide Libre: ?
%{

}
\item \index{Longitud}Longitud:{

\defn Definimos \textbf{longitud} de una cadena $u \in \Sigma^{*}$, al número de 
símbolos que la componen (incluyendo los símbolos repetidos). Se denota 
normalmente como: $|u|$

\begin{equation}
|u| = \lambda + |u_1| + |u_2| + \ldots |u_n|  
\end{equation}


\ejem Sea $u = [hola] \Rightarrow |u| = 4$

}\item \index{Potencia}Potencia: Entendemos que para un alfabeto dado $\Sigma^*$ y la palabras 
$u = \{u_1, u_2, \ldots, u_n\} \in \Sigma^*$ con $n \in \mathbb{N}$ por 
\textbf{potencia} (denotado como $u^n$): {

\begin{equation}
u^n = u^0 \circ u^1 \circ u^2 \circ \ldots \circ u^n 
\end{equation}

\ejem Para las palabras $u$ se tiene como:

\begin{equation}
u_0 = \lambda,\ u_1 = hola,\ u_2 = holahola,\ \ldots\ u_n = {CONCAT}_{i = 0}^{i
= n}{u_i}
\end{equation}

\paragraph*{Propiedades:}

\begin{enumerate}[i.]

\item La potencia unidad de una palabra equivale a esa misma palabra:

\begin{equation}
a^1 = a
\end{equation}

\ejem

\begin{equation}
u^1 = [hola]^1 \equiv u = [hola]
\end{equation}


\item El producto de distintas potencias de una palabra es igual a esa misma
palabra con potencia resultado de la suma de los índices:

\begin{equation}
a^3\cdot a^6 = a^9
\end{equation}

\ejem

\begin{equation} \label{eq:u^5}
u^2\cdot u^3 = [holahola]\cdot[holaholahola] \equiv u^5 =
[holaholaholaholahola]
\end{equation}

\item La potencia de una palabra sobre cero es igual a palabra vacía:

\begin{equation}
a^0 = \lambda
\end{equation}

\ejem

\begin{equation}
u^0 = [hola] = \lambda
\end{equation}

\item El tamaño de una palabra sobre un índice cualquiera es igual al índice por
el tamaño de la palabra original:

\begin{equation}
a^i\ = i \cdot |a|
\end{equation}

\ejem

\begin{equation}
u^5 = 5 \cdot Length(hola) = 20
\end{equation}
\paragraph*{Nota:} Ver ecuación: (\ref{eq:u^5})

\end{enumerate}

}\item \index{Reflexión}Reflexión: Entendemos que para un alfabeto dado $\Sigma^*$ y la palabras 
$u = \{u_1, u_2, \ldots, u_n\} \in \Sigma^*$ con $n \in \mathbb{N}$ por 
\textbf{reflexión} (denotado como $u^{-1}$): {

\begin{equation}
u^{-1} = \{u_n, \ldots u_2, u_1\}
\end{equation}

\ejem Para la palabra $u$ se tiene como:

\begin{equation}
u^{-1} \equiv \frac{1}{u} = [aloh]
\end{equation}

\paragraph*{Propiedades:}

\begin{enumerate}[i.]

\item El tamaño de una palabra coincide con el de su inversa:

\begin{equation}
Length(a) = Length(a^{-1})
\end{equation}

\ejem

\begin{equation}
Length(u) = Length(hola) = 4 \equiv Length(u^{-1}) = Length(aloh) = 4
\end{equation}

\end{enumerate}

}
\end{enumerate}


\subsection{Lenguajes}

\defn Un \textbf{\index{Lenguaje}Lenguaje} es un subconjunto de palabras de algún alfabeto dado:

\begin{equation}
L \subseteq \wp(\Sigma) \equiv L \subseteq \Sigma^*
\end{equation}

\defn Existe el \textbf{\index{Lenguaje Vacio}Lenguaje Vacio} denotado como: $\varepsilon = 
\{\varepsilon\}$.

\subsubsection{Operaciones}

\textbf{Nota:} Se trabaja con un alfabeto $\Sigma = \{a,b\}$ y con dos 
lenguajes:
$P = \{a,ab\}$ y $Q = \{a, b, bb\}\ \diagup\ P,Q\ \wp(\Sigma)$.

\begin{enumerate}[I.]
\item \index{Unión}Unión: Para los alfabetos $P$ y $Q$ con el símbolo $\cup$, siendo $r$ una
palabra de la unión, se tiene que $P\ \cup\ Q$ = $\{ r\ \arrowvert\ r\ \in\ P\
\vee\ Q\}$ {

\ejem

\begin{equation}
P \cup Q = \{a,ab,b,bb\}
\end{equation}

\paragraph*{Propiedades:} Al ser cada alfabeto un conjunto en sí, esta 
operación 
conserva todas las propiedades de la Unión.

\begin{enumerate}[i.]

\item Conmuntatividad:
{
\begin{equation}
P \cup Q \equiv Q \cup P = \{a,ab,b,bb\}
\end{equation}
}
\item Asociatividad:
{

\paragraph*{Nota:} Con $R= \{aaa,ba\}$.


\begin{equation}
(P \cup Q) \cup R = P \cup (Q \cup R) = \{a,aaa,ab,b,bb,ba\}
\end{equation}

}
\item Idempotencia
{
\begin{equation}
P \cup P \equiv P = \{a,ab\}
\end{equation}

}
\item Absorción:
{
\begin{equation}
P \cup W(P) \equiv W(P)
\end{equation}
}
\item Neutralidad:
{
\begin{equation}
P \cup \varepsilon \equiv P = \{a,ab\}
\end{equation}
}
\end{enumerate}

}
%\item Binoide Libre: ?
%{
%
%}

\item \index{Intersección}Intersección: Para los alfabetos $P$ y $Q$ con el símbolo $\cap$, siendo
$r$ una palabra de la intersección, se tiene que $P\ \cap\ Q$ = $\{ r\ \in P\
\wedge\ r\ \in\ Q\}$ {

\ejem

\begin{equation}
P \cap Q = \{a\}
\end{equation}

\input{./graphics/graphics21/exampleUNIONandINTERSECTIONofFLS}

}\item \index{Diferencia}Diferencia: Para los alfabetos $P$ y $Q$ con el símbolo $-$, siendo
$r$ una palabra de la diferencia, se tiene que $P\ \cap\ Q$ = $\{ r\ \in P\
\wedge\ r\ \notin\ Q\}$ {

\ejem

\begin{equation}
P - Q = \{ab\}
\end{equation}


}\item \index{Complemento}Complemento: Para el alfabeto $P$ se tiene por complemento de $P 
\Rightarrow \bar{P} = \Sigma^* - P$ {

\ejem

\begin{equation}
\bar{P} = \{b\}
\end{equation}

\begin{figure}[h]
\centering
\begin{subfigure}[A]{0.4\textwidth}
\centering
{
\begin{pspicture}(-1.8,-3)(1.2,3)%\psgrid

\psset{unit=2cm}

\pspolygon[fillstyle=solid,fillcolor=white](-1.5,-1.5)(1.5,-1.5)(1.5,1.5)(-1.5,1.5)
\pscircle[fillstyle=solid,fillcolor=white](0,0){1}

\rput(0,0){$\{ab\}$}
\rput(1.3,1.3){$\varepsilon$}
\rput(1.7,1.4){$L$}

\end{pspicture}

\caption{Diagrama de Venn para la operación: $P - Q$.}
}
\end{subfigure}%
\quad
\begin{subfigure}[B]{0.4\textwidth}
\centering
{
\begin{pspicture}(1.3,-3)(4.3,3)%\psgrid

\psset{unit=2cm}

\pspolygon[fillstyle=solid,fillcolor=white](0,-1.5)(3,-1.5)(3,1.5)(0,1.5)
\pscircle[fillstyle=solid,fillcolor=white](1.5,0){1}

\rput(1.5,0){$L - \{a,ab\}$}
\rput(2.8,1.3){$\varepsilon$}
\rput(3.2,1.4){$L$}

\end{pspicture}

\caption{Diagrama de Venn para la operación: $\bar{P}$.}
}
\end{subfigure}

\end{figure}


}\item \index{Concatenación}Concatenación: Para los conjuntos $P$ y $Q$, siendo $rs$ una palabra de
la concatenación, se tiene que $PQ$ = \{ $rs\ \arrowvert\ r\ \in\ P\ \wedge\ s\
\in\ Q$ \} {

\ejem

\begin{equation}
PQ = \{aa,ab,abb,ba,bab,bba,bbab\}
\end{equation}

}\item \index{Potencia}Potencia: Se tiene por Potencia de un alfabeto $L$: $L^n$ dónde $L$ es 
un 
alfabeto y $n \in \mathbb{N}$ que representa en número de concatenaciones: {

\begin{equation}
L^n = \{L_0, L_1, L_2, \ldots, L_n\}
\end{equation}


\ejem Para el alfabeto $P = \{a,ab\}$, se tiene en iteración:

\begin{equation}
P^2 = \{aa,aab,aba,abab\}
\end{equation}


}\item \index{Clausura}, también denominada: \index{Cerradura de Klenee}Cerradura de Klenee o \index{Cerradura Estrella}Cerradura Estrella: Se denomina Clausura del Lenguaje, dónde $L$ es 
un  alfabeto y $n \in \mathbb{N}$ a todas las uniones de las Potencias del Lenguaje L (incluida $L_0$): {

\begin{equation}
L^* = \bigcup_{i = 0}^{i = n}{L^i}
\end{equation}

\ejem Para el alfabeto $P = \{a,ab\}$, se tiene:

\begin{equation}
P_0 \cup P^1 \cup P^2 \dots P^n = \{\varepsilon\} \cup \{a,ab\} \cup \{aa,aab,aba,abab\} \dots
\end{equation}

}\item \index{Clausura Positiva}Clausura Positiva: Se denomina Clausura del Lenguaje, dónde $L$ es 
un  alfabeto y $n \in \mathbb{N}$ a todas las uniones de las Potencias del Lenguaje L (excepto $L_0$): {

\begin{equation}
L^+ = \bigcup_{i = 1}^{i = n}{L^i}
\end{equation}

\ejem Para el alfabeto $P = \{a,ab\}$, se tiene:

\begin{equation}
P^1 \cup P^2 \dots P^n = \{a,ab\} \cup \{aa,aab,aba,abab\} \dots
\end{equation}


\paragraph*{Propiedades:}

\begin{enumerate}[i.]

\item $L^* = L^+ \cup \{\lambda\}$
{
\begin{equation}
P^2 \cup \{\lambda\} \equiv \{aa,aab, aba, abab\} \cup \{\lambda\} = 
\{aa,aab,aba, abab\}
\end{equation}

\dhm Siendo $L^+= \{L_1, L_2, \ldots, L_n\}$ tenemos que:

\begin{equation}
L^+ \cup \{\lambda\} = \{L_0,L_1,L_2, \ldots, L_n\} = L^*.
\end{equation}
}
\item $L^+ = LL^* = L^*L$
{
\dhm Siendo $L^*= \{L_0,L_1,L_2,\ldots, L_n\}$ tenemos que:

\begin{equation}
LL^* = L\{L_0,L_1,L_2,\ldots, L_n\} = L\{L_1,L_2,\ldots, L_n\}
\end{equation}
}
\end{enumerate}

}\item \index{Reflexión}Reflexión: Se denota Reflexión del Lenguaje $L$ a $L^{-1}${

\begin{equation}
L^{-1} = \{p^{-1} : p \in L\} 
\end{equation}

\ejem Para el alfabeto $P$:

\begin{equation}
P^{-1} = \{ba,a\}
\end{equation}


}
\end{enumerate}

%%%%%%%%%%%%%%%%%%%%%%%%%

\section{Lenguajes Regulares}

\defn Decimos que un \textbf{Lenguaje es Regular} si contiene los elementos: 
$\varnothing$ , $\{ \lambda \}$ , $p \in \Sigma$ y está ligado a las  
operaciones: Unión ($\cup$), Concatenación ($\circ$) y Cerradura de Klenee 
($p^{*}$).

\paragraph*{Nota:} $\varnothing$ , $\{ \lambda \}$, $\{p\}$ , $p \in \Sigma$ 
son 
los denominados \textbf{Lenguajes Regulares Básicos}.

\ejem Sea el $\Sigma = \{ a, b, c\}$ tenemos:

\begin{equation}
U = \{a\} \circ \{b\}^* \cup \{c\}^* . 
\end{equation}

\begin{equation}
V = \{a\}^* \cup \{b\} \circ \{c\}. 
\end{equation}


\begin{enumerate}[i.]

\item \index{Unión}Unión:

\begin{equation}
U \cup V =  [\{a\} \circ \{b\}^* \cup \{c\}^* \cup \{a\}^* \cup \{b\} \circ 
\{c\}]
\end{equation}


\item \index{Concatenación}Concatenación:

\begin{equation}
U \circ V =  [\{a\} \circ \{b\}^* \cup \{c\}^* \circ \{a\}^* \cup \{b\} \circ 
\{c\}]
\end{equation}

\item \index{Cerradura de Klenee}Cerradura de Klenee: 

\begin{equation}
U^{*} =  [\{a\} \circ \{b\}^* \cup \{c\}^*]^*
\end{equation}


\end{enumerate}

{\cor Todo \index{Lenguaje Finito}Lenguaje Finito es un \index{Lenguajes Regular}Lenguajes Regular.}

\begin{table}
\begin{center}

\begin{tabular}{|c|c|}\hline
\textbf{Lenguaje Regular} & \textbf{Expresión Regular} \\ \hline
\hline
$\varnothing$ & $\varnothing$\\ \hline
$\{\lambda \}$ & $\lambda$\\ \hline
$\{a\}, a \in \Sigma$ & $a$\\ \hline
$U \cup V$ & $(U \cup V)$\\ \hline
$U \circ V$ & $(U)(V)$\\ \hline
$U^*$ & $(U)^*$\\ \hline
\end{tabular} 
\caption{Relación de operadores entre Lenguajes Regulares y Expresiones 
Regulares.}
 
\end{center}
\end{table}


\section{Expresiones Regulares}\label{sec:expresionesRegulares}
\defn Las \textbf{\index{Expresiones Regulares}Expresiones Regulares} se forman recursivamente:

\begin{enumerate}[i.]
\item Por medio de los símbolos: $\varnothing$ y $\varepsilon$
\item $e\ \in \Sigma$
\item Siendo $r$ y $s$ el resultado, por medio de las operaciones $r \cup s, rs,
r^*$
\end{enumerate}

{\cor Las Expresiones Regulares son conceptualmente y operativamente lo mismo 
que los antes descritos Lenguajes Regulares, la diferencia radica en que estas 
últimas tienen el objetivo de ser lenguajes más legibles, es decir, las 
Expresiones Regulares son una simplificación de los Lenguajes Regulares para 
mejorar el entendimiento entre el hombre y la máquina.}

\begin{equation}
ER(E) := LR(E)
\end{equation}

\begin{table}[h]
\begin{center}
\begin{tabular}{|c|l|}\hline
\textbf{Expresión} & \textbf{Significado} \\ \hline
\hline
`.' & Representa cualquier carácter excepto `\textbackslash n'. \\ \hline
`*' & $\varnothing$ o más copias de la expresión que le precede.\\ \hline
`[ ]' & Operador de conjunto de caracteres. También se usa para expresar rangos.\\ \hline
`\textasciicircum' & Operador para indicar el comienzo de la expresión.\\ \hline
`\$' & Indica el final de una expresión regular. \\ \hline
`\{ \}' & Indica el número de ocurrencias para el carácter que le precede. \\ \hline
`\textbackslash' & Operador de ``escapé'' para caracteres restringidos. \\ \hline
`+' & Se utiliza con rangos, indica 1 o más copias de la expresión que le precede. \\ \hline
`?' & Indica $\varnothing$ o una ocurrencia.\\ \hline
`\textbar' & Operador de disyunción. \\ \hline
``...'' & Maraca el texto entrecomillado como literal. \\ \hline
`\textfractionsolidus' & Maraca como literal los caracteres que le preceden.\\ \hline
`( )' & Operador para subconjuntos de expresiones regulares. \\ \hline
\end{tabular}
\caption{Operadores comunes para Expresiones Regulares en UNIX.}
\end{center}
\end{table}




% \subsection{Propiedades}
% 
% {\thm Sean r,s y t Expresiones Reguales sobre un mismo alfabeto O. Entonces:}
% 
% \begin{enumerate}
% 
% \item
% 
% \begin{equation}
% r \cup s = s \cup r
% \end{equation}
% 
% \item
% 
% \begin{equation}
% r \cup \varnothing = \varnothing \cup r
% \end{equation}
% 
% \item
% 
% \begin{equation}
% r \cup r = r
% \end{equation}
% 
% \item
% 
% \begin{equation}
% (r \cup s) \cup t = r \cup (s \cup t)
% \end{equation}
% 
% \item
% 
% \begin{equation}
% r\varepsilon = r = \varepsilon r
% \end{equation}
% 
% \item
% 
% \begin{equation}
% r \varnothing = \varnothing r = \varnothing
% \end{equation}
% 
% \item
% 
% \begin{equation}
% r(st) = (rs)t
% \end{equation}
% 
% \item
% 
% \begin{equation}
% (r \cup s)t = r(s \cup t); r(s \cup t) = rs \cup rt
% \end{equation}
% 
% \item
% 
% \begin{equation}
% r^* = r^* r^* = (\varepsilon \cup r)^* = r^* (\varepsilon \cup r) = 
%(\varepsilon
% \cup r) r^* = \varepsilon \cup r r^*
% \end{equation}
% 
% \item
% 
% \begin{equation}
% (r \cup s)^* = (r^* \cup s^*)* = (r^* s^*)^* = (r^* s)^* r^* = r^* (sr^*)^*
% \end{equation}
% 
% \item
% 
% \begin{equation}
% r(sr)^* = (rs)^* r
% \end{equation}
% 
% \item
% 
% \begin{equation}
% (r* s)^* = \varepsilon \cup (r \cup s)s^*
% \end{equation}
% 
% \item
% 
% \begin{equation}
% (rs^*)^* = \varepsilon \cup r(r \cup s)^*
% \end{equation}
% 
% \item
% 
% \begin{equation}
% s(r \cup \varepsilon)^* (r \cup \varepsilon) \cup s = rs^*
% \end{equation}
% 
% \item
% 
% \begin{equation}
% rr^* = r^* r
% \end{equation}
% 
% \end{enumerate}


%%%%%%%%%%%%%%%%%%%%%%%%%
% Autómatas
%%%%%%%%%%%%%%%%%%%%%%%%%

\section{Autómatas}\label{sec:aut}

\subsection{¿Qué es un Autómata?}

\defn Se conoce como \index{Autómata Finito}Autómata Finito a máquinas abstractas que procesan cadenas para un determinado lenguaje.

Dicha máquina se compone de una serie de partes (Ver Figura \ref{fig:generalAut}):

\begin{enumerate}[I.]

\item Cinta semi-infinita: Dividida a su vez en celdas donde se escribe la cadena de entrada.

\item \index{Unidad de Control}Unidad de Control (también llamada Cabeza Lectora): Que se encarga de procesar la citan.

\item El Autómata propiamente dicho que mantiene la lógica del lenguaje a través de una serie de estados (de aceptación y finales).

\end{enumerate}

\begin{figure}[h]
\begin{center}
%
\begin{pspicture}(0,0)(10,6)%\psgrid
\psline[linecolor=black,linewidth=1pt]{-}(3,2)(3,5)
\psline[linecolor=black,linewidth=1pt]{-}(3,2)(6,1)
\psline[linecolor=black,linewidth=1pt]{-}(6,1)(6,4)
\psline[linecolor=black,linewidth=1pt]{-}(3,5)(6,4)

\psline[linestyle=dotted,linecolor=black,linewidth=1pt]{-}(5,3)(5,6)
\psline[linestyle=dotted,linecolor=black,linewidth=1pt]{-}(5,3)(8,2)
\psline[linecolor=black,linewidth=1pt]{-}(8,2)(8,5)
\psline[linecolor=black,linewidth=1pt]{-}(5,6)(8,5)


\psline[linestyle=dotted,linecolor=black,linewidth=1pt]{-}(3,2)(5,3)
\psline[linecolor=black,linewidth=1pt]{-}(6,1)(8,2)

\psline[linecolor=black,linewidth=1pt]{-}(3,5)(5,6)
\psline[linecolor=black,linewidth=1pt]{-}(6,4)(8,5)

\end{pspicture}
\caption{Representación de un Autómata Finito.\label{fig:generealAut}}
\end{center}
\end{figure}


\defn Toda cadena necesariamente tiene una condición de parada, descrita como casilla vacía.

\defn Todo Autómata $M$ se expresa mediante una quíntupla: $M = \{\Sigma, Q, q_0, F, t)$ 

\begin{enumerate}[i.]

\item $\Sigma$: Es el Lenguaje de aceptación: $\Sigma \subseteq \Sigma^*$

\item $Q$: Es el conjunto de estados: $\{q_0, q_1, q_2, \ldots q_n\} \in Q$

\item $q_0$: Es el estado inicial.

\item $F$: Conjunto de estados finales o de aceptación.

\item $t$: Se trata de la función de transición. Dependiendo del tipo de Autómata procesará o no determinados caracteres.

\end{enumerate}

\paragraph*{Nota:} Dependiendo de la configuración de los estados internos del Autómata, diferenciamos tres tipos:

\begin{enumerate}[i.]

\item Autómatas Finitos Determinista: Transiciones del tipo: $\delta (q, a)$. Procesan la palabra $\lambda$.

\item Autómatas Finitos No Determinista: Transiciones del tipo: $\Delta (q, a)$. No procesan la palabra $\lambda$.

\end{enumerate}

\subsection{Representación}

La representación gráfica de un Autómata se realiza por medio de grafo dirigido y etiquetado o grafo etiquetado (Ver Sección \ref{sec:clasificacionGrafos}). 

\begin{figure}[h]
\centering
\begin{subfigure}[A]{0.3\textwidth}
\centering

\begin{pspicture}(0,0)(4,2)%\psgrid
\cput(2,1){\large $q$} 
\end{pspicture}

\caption{Estado o Nodo.}

\end{subfigure}%
\quad
\begin{subfigure}[B]{0.3\textwidth}
\centering
\begin{pspicture}(0,0)(4,2)\psgrid
\cput(2,1){\large $q_0$} 
\rput(1.45,1){\large $>$}
\end{pspicture}

\caption{Estado Inicial $q_0$.}

\end{subfigure}%
\quad
\begin{subfigure}[C]{0.3\textwidth}
\centering

\begin{pspicture}(0,0)(4,2)%\psgrid
\cput[doubleline=true](2,1){\large $q$} 
\end{pspicture}

\caption{Estado Final $q$.}

\end{subfigure}
\quad
\begin{subfigure}[D]{0.3\textwidth}
\centering

\begin{pspicture}(0,0)(4,2)%\psgrid
\cput(0.8,0.5){\large $q$} 
\cput(3.2,0.5){\large $p$}
\pscurve[linecolor=black,linewidth=1pt]{<->}(1,0.85)(2,1.2)(3,0.85)
\rput(2,1.45){\large $a$}
\end{pspicture}

\caption{Transición: $t = (q,a) = p$.}

\end{subfigure}

\caption{Representación de Autómatas.}

\end{figure}

\subsection{Autómata Finito Determinista}\label{sec:dfa}

Un \textbf{\index{Autómata Finito Determinista}Autómata Finito Determinista} (AFD) se trata de un Autómata: $M_{AFD} = \{\Sigma, Q, q_0, F, \delta\}$ dónde la función de transición de estados se expresa como\endnote{Usaremos en el texto para describir cualquier autómata la notación Backus-Naur Form (BNF).}: 

\begin{equation}
\delta: Q \times \Sigma \longrightarrow Q\ t.q.\ (q, a) \longmapsto \delta (q, a)
\end{equation}

{\cor Dado que $q_0$ es un estado de aceptación, la palabra $\lambda$ aun siendo vacía es también procesada. Con el objetivo omitir este reconocimiento tenemos los AFnD.}

\ejem Dada la siguiente gramática:

\begin{enumerate}[i.]

\item $\Sigma = \{a,b\}$

\item $Q = \{q_0,q_1,q_2\}$

\item $q_0$: Es el estado inicial.

\item $F = \{q_1,q_2\}$

\item $\delta = $ (Ver Figura \ref{fig:generalDFA})

\end{enumerate}

\begin{figure}[h]
\centering
\begin{subfigure}[A]{0.4\textwidth}
\centering

\begin{tabular}{c|c|c}
$\delta$ & $a$ & $b$\\ \hline
\hline
$q_0$ & $q_0$ & $q_1$ \\ \hline
$q_1$ & $q_1$ & $q_2$ \\ \hline
$q_2$ & $q_1$ & $q_1$ \\ \hline
\end{tabular} 

\caption{Relación de Transiciones $\delta$}

\end{subfigure}%
\quad
\begin{subfigure}[B]{0.4\textwidth}
\centering

\begin{pspicture}(0,0)(5,4)%\psgrid
\rput(0.48,3){\large $>$}
\cput(1,3){\large $q_0$} 
\pscurve[linecolor=black,linewidth=1pt]{->}(1.5,3.2)(2.5,3.55)(3.5,3.2)
\rput(1,4){\large $a$}

\pscurve[linecolor=black,linewidth=1pt]{->}(0.67,3.2)(1,3.7)(1.3,3.2)


\pscurve[linecolor=black,linewidth=1pt]{->}(1.5,3.2)(2.5,3.55)(3.5,3.2)
\rput(2.5,3.85){\large $b$}


\cput[doubleline=true](4,3){\large $q_1$}
\pscurve[linecolor=black,linewidth=1pt]{->}(3.63,3.25)(4,3.75)(4.35,3.25)
\rput(4,4){\large $a$}


\cput[doubleline=true](2.5,1){\large $q_2$}
%\pscurve[linecolor=black,linewidth=1pt]{->}(4.2,2.5)(3.5,1.4)(3.1,0.8)
\pscurve[linecolor=black,linewidth=1pt]{<-}(3,1.2)(4,2)(4,2.4)
\pscurve[linecolor=black,linewidth=1pt]{<-}(3.6,2.6)(2.8,2.1)(2.8,1.5)

\rput(3.8,1.3){\large $b$}

\rput(2.6,2.6){\large $a,b$}

\end{pspicture}

\caption{AFD para Tabla (a)}

\end{subfigure}


\caption{Ejemplo Autómata Finito Determinista.}\label{fig:generalDFA}

\end{figure}

\begin{figure}[h]
\centering
\begin{subfigure}[A]{0.4\textwidth}
\centering

\begin{pspicture}(0,0)(5,4)%\psgrid
\psline[linecolor=black,linewidth=0.8pt]{->}(1.38,2)(3.61,2)
\cput(1,2){\large $q_0$} 
\cput(4,2){\large $q_1$}
\pscurve[linecolor=black,linewidth=0.8pt]{->}(0.67,2.2)(1,2.7)(1.3,2.2)
\pscurve[linecolor=black,linewidth=0.8pt]{->}(3.67,2.2)(4,2.7)(4.3,2.2)

\pscurve[linecolor=black,linewidth=0.8pt]{->}(3.67,1.8)(4,1.3)(4.3,1.8)

\rput(2.5,2.3){\large $b$}

\rput(0.5,2){\large $>$}

\rput(1,3){\large $a$}

\rput(4,3){\large $a$}

\rput(4,1){\large $b$}

\end{pspicture}

\caption{AFD $a^*$.}

\end{subfigure}%
\quad
\begin{subfigure}[B]{0.4\textwidth}
\centering

\begin{pspicture}(0,0)(5,4)%\psgrid
\pscurve[linecolor=black,linewidth=0.8pt]{->}(2.13,2.25)(2.5,2.75)(2.85,2.25)
\cput[doubleline=true](2.5,2){\large $q_0$}
\rput(1.93,2){\large $>$}


\rput(2.5,3){\large $a$}

\end{pspicture}

\caption{AFD $a^*$ Simplificado.}

\end{subfigure}%
\quad
\begin{subfigure}[C]{0.4\textwidth}
\centering

\begin{pspicture}(0,0)(5,4)\psgrid
\end{pspicture}

\caption{AFD $a^+$.}

\end{subfigure}
\quad
\begin{subfigure}[D]{0.4\textwidth}
\centering

\begin{pspicture}(0,0)(5,4)%\psgrid
\psline[linecolor=black,linewidth=0.8pt]{->}(1.38,2)(3.55,2)
\cput(1,2){\large $q_0$} 
\cput[doubleline=true](4,2){\large $q_1$}
\pscurve[linecolor=black,linewidth=0.8pt]{->}(3.63,2.25)(4,2.75)(4.35,2.25)
\rput(2.5,2.3){\large $a$}

\rput(0.5,2){\large $>$}


\rput(4,3){\large $a$}

\end{pspicture}

\caption{AFD $a^+$ Simplificado.}

\end{subfigure}

\caption{Ejemplos AFD.}

\end{figure}

\subsection{Autómata Finito no Determinista}\label{sec:nda}

Un \textbf{\index{Autómata Finito no Determinista}Autómata Finito no Determinista} (AFnD) se trata de un Autómata: $M_{AFnD} = \{\Sigma, Q, q_0, F, \Delta\}$ dónde la función de transición entre estados se expresa como:

\begin{equation}
\Delta:  Q \times \Sigma \longrightarrow \wp (Q)\ t.q.\ (q, a) \longmapsto \Delta (q, a) = \{q_{i1} \}
\end{equation}

\ejem Dada la siguiente gramática:

\begin{enumerate}[i.]

\item $\Sigma = \{a,b\}$

\item $Q = \{q_0,q_1,q_2,q_3\}$

\item $q_0$: Es el estado inicial.

\item $F = \{q_1,q_3\}$

\item $\delta = $ (Ver Figura \ref{fig:generalNDA})


\end{enumerate}

\begin{figure}[h]
\centering
\begin{subfigure}[A]{0.4\textwidth}
\centering

\begin{tabular}{c|c|c}
$\Delta$ & $a$ & $b$\\ \hline
\hline
$q_0$ & $\{q_0,q_1,q_2\}$ & $\varnothing$ \\ \hline
$q_1$ & $\{q_1\}$ & $\{q_2\}$ \\ \hline
$q_2$ & $\varnothing$ & $\{q_1,q_2\}$ \\ \hline
$q_3$ & $\varnothing$ & $\{q_3\}$ \\ \hline
\end{tabular} 

\caption{Relación de Transiciones $\Delta$.}

\end{subfigure}%
\quad
\begin{subfigure}[B]{0.4\textwidth}
\centering

\begin{pspicture}(0,0)(5,4)\psgrid
\pscurve[linecolor=black,linewidth=0.8pt]{->}(2.13,2.25)(2.5,2.75)(2.85,2.25)
\cput[doubleline=true](2.5,2){\large $q_0$}
\rput(1.93,2){\large $>$}


\rput(2.5,3){\large $a$}

\end{pspicture}

\caption{AFD $a^*$ Simplificado.}

\end{subfigure}


\caption{Ejemplo Autómata Finito no Determinista.}\label{fig:generalNDA}

\end{figure}

\subsection{Algoritmo: AFnD $\Rightarrow$ AFD}\label{algolAFnD2AFD}

{
\thm Para todo AFnD $M$ = $(\Sigma, Q,q_0,F,\Delta)$ se puede obtener un AFD equivalente $M\prime$ tal que $L(M) \equiv L(M\prime)$.
}

\defn La diferencia entre un AFnD y un AFD radica en las función de transición $\delta$.

\prog Pseudocódigo del Algoritmo AFnD $\Rightarrow$ AFN:

\begin{enumerate}[I.]

\item Entrada: AFnD $N$

\item Salida: AFD $N\prime$

\item Método:
{
\begin{enumerate}[i.]
\item Correspondencia entre Estados.

{\defn Los \'unicos estados $Q_{D}$ accesibles en AFD son subconjunto de $Q_{N}$
cerrados respecto a $\lambda$.}

\item Correspondencia del Estado Inicial.

{\defn El estado inicial de AFD es el resultado de calcular el cierre $\lambda$
del estado inicial del AFND.}

{\cor El \index{Cierre $\lambda$}Cierre $\lambda$, también llamado \index{$CLAUS_\lambda$}$CLAUS_\lambda$ para un estado, es el conjunto de estados alcanzables mediante cero o más transiciones.}

\item Cálculo de Transiciones.
{

\form $\delta_{D} (i,o) \backslash i \in Q_{D}$ para:

\begin{enumerate}[a.]
\item $i = \{p_{0}, p_{1}, p_{2}, \ldots, p_{n} \}$
\item Obtener: $\bigcup\limits^{n}_{i=0} \delta_{N}(p_{i},a) = [r_{0}, r_{1},
r_{2}, \ldots, r_{m} ]$
\item $\delta_{D} (i,o) = \bigcup\limits^{m}_{j=1}   CLAUS_{\lambda}(r_{j})$
\end{enumerate}
}

\item Correspondencia del Estado Final.

{\defn En el AFD un estado ser\'a final si contiene alg\'un estado final del
AFND.}

\end{enumerate}
}
\end{enumerate}



\ejem Para el siguiente Automata:

\begin{enumerate}[i.]

\item $\Sigma = \{a\}$

\item $Q = \{q_0,q_1,q_2,q_3,q_4,q_5\}$

\item $q_0$: Es el estado inicial.

\item $F = \{q_5\}$

\item $\delta = $ (Ver Figura \ref{fig:exampleNDA})

\end{enumerate}

\begin{figure}[h]
\centering
\begin{subfigure}[A]{0.4\textwidth}
\centering

\begin{tabular}{l|l|l}
$\Delta$ & $\Delta (a)$ & $\Delta (\lambda)$ \\ \hline
\hline
$q_0$ & $\{q_1\}$ & $\{\varnothing\}$ \\ \hline
$q_1$ & $\{\varnothing\}$ & $\{q_2\}$ \\ \hline
$q_2$ & $\{\varnothing\}$ & $\{q_3,q_5\}$ \\ \hline
$q_3$ & $\{q_4\}$ & $\{\varnothing\}$ \\ \hline
$q_4$ & $\{\varnothing\}$ & $\{q_3,q_5\}$ \\ \hline
$q_5$ & $\{\varnothing\}$ & $\{\varnothing\}$\\ \hline
\end{tabular} 

\caption{Relación de Transiciones $\Delta$}

\end{subfigure}%
\quad
\begin{subfigure}[B]{0.4\textwidth}
\centering

\begin{pspicture}(0,-1)(10,5)%\psgrid
%\pscurve[linecolor=black,linewidth=0.8pt]{->}(0.63,3.25)(1,3.75)(1.35,3.25)
\cput(1,3){\large $q_0$}
\rput(0.48,3){\large $>$}
%\pscurve[linecolor=black,linewidth=1pt]{->}(0.67,3.2)(1,3.7)(1.3,3.2)
%\rput(1,4){\large $a$}

\pscurve[linecolor=black,linewidth=1pt]{->}(1.5,3.2)(2.5,3.55)(3.5,3.2)
\pscurve[linecolor=black,linewidth=1pt]{<-}(1.5,0.8)(2.5,0.45)(3.5,0.8)
%\rput(2.5,3.85){\large $b$}


\cput(4,3){\large $q_1$}

%\pscurve[linecolor=black,linewidth=0.8pt]{->}(3.63,3.25)(4,3.75)(4.35,3.25)
%\rput(4,4){\large $a$}


\cput(7,3){\large $q_2$}



\pscurve[linecolor=black,linewidth=1pt]{->}(4.6,3.2)(5.5,3.55)(6.5,3.2)
\pscurve[linecolor=black,linewidth=1pt]{<-}(4.6,0.8)(5.5,0.45)(6.5,0.8)


\pscurve[linecolor=black,linewidth=1pt]{->}(4.6,1.2)(5.5,1.55)(6.5,1.2)
%\rput(5.5,3.85){\large $b$}
%\pscurve[linecolor=black,linewidth=1pt]{<-}(4.6,2.8)(5.5,2.45)(6.5,2.8)
%\rput(5.5,2.15){\large $b$}
%\pscurve[linecolor=black,linewidth=1pt]{->}(6.67,3.2)(7,3.7)(7.3,3.2)


\cput[doubleline=true](1,1){\large $q_5$}

%\pscurve[linecolor=black,linewidth=0.8pt]{->}(0.63,0.75)(1,0.25)(1.35,0.75)
%\rput(1,0){\large $b$}


%\rput(7,4){\large $b$}

%\pscurve[linecolor=black,linewidth=1pt]{->}(0.6,2.7)(0.3,2)(0.6,1.4)
%\rput(0,2){\large $a$}

%\cput[doubleline=true](2.5,1){\large $q_2$}
%\pscurve[linecolor=black,linewidth=1pt]{->}(4.2,2.5)(3.5,1.4)(3.1,0.8)
%\pscurve[linecolor=black,linewidth=1pt]{<-}(3,1.2)(4,2)(4,2.5)
%\pscurve[linecolor=black,linewidth=1pt]{<-}(3.7,2.7)(2.8,2.1)(2.8,1.5)

%\rput(3.8,1.3){\large $b$}

%\rput(2.6,2.6){\large $a,b$}

\cput(7,1){\large $q_3$}

\cput(4,1){\large $q_4$}

\pscurve[linecolor=black,linewidth=1pt]{->}(7.4,2.7)(7.7,2)(7.4,1.4)
\rput(8,2){\large $\lambda$}

\rput(2.5,3.85){\large $a$}
\rput(2.5,0.15){\large $\lambda$}

\rput(5.5,0.15){\large $a$}

\rput(5.5,3.85){\large $\lambda$}

\rput(5.5,1.85){\large $\lambda$}

% \pscurve[linecolor=black,linewidth=1pt]{->}(7.6,3)(8.3,1.8)(7.6,-0.4)(5.5,-0.4)(2,-0.85)(0.45,-0.2)(0.6,0.6)
% 
 \rput(5.5,2.95){\large $\lambda$}

\pscurve[linecolor=black,linewidth=1pt]{->}(6.5,2.8)(6.1,2.6)(4.8,2.6)(4,1.45)
\end{pspicture}

\caption{AFnD para Tabla (a)}

\end{subfigure}


\caption{Autómata Finito no Determinista $a\cdot a^*$}\label{fig:exampleNDA}

\end{figure}

\begin{enumerate}[I.]

\item Correspondencia entre Estados (Cierre $\lambda$):\\
{
\begin{flushleft}
\begin{tabular}{l|l}
\textbf{$q_\lambda$} & \textbf{$CLAUS_\lambda$}\\ \hline
\hline
$q_0$ & $CLAUS_\lambda(q_0) = \{q_0\}$ \\ \hline
$q_1$ & $\{q_1,q_2,q_3,q_4\}$ \\ \hline
$q_2$ & $\{q_2, q_3, q_4\}$ \\ \hline
$q_3$ & $\{q_3\}$ \\ \hline
$q_4$ & $\{q_3, q_4, q_5\}$ \\ \hline
$q_5$ & $\{q_5\}$ \\ \hline
\end{tabular} 
\end{flushleft}
}
\item Correspondencia del Estado Inicial.

\begin{equation}
qN_0 = CLAUS_\lambda(q_0) = \{q_0\} \equiv q_D0 = \{q_1\}
\end{equation}


\item Cálculo de Transiciones.
{
\begin{figure}[h]
\centering
\begin{subfigure}[A]{0.4\textwidth}
\centering

\begin{tabular}{l|l|l}
$q_{Ni}$ & $\delta (a)$ & $\delta (\lambda)$ \\ \hline
\hline
$q_{N0}$ & $\{q_1\}$ & $\{\varnothing\}$ \\ \hline
$q_{N1}$ & $\{\varnothing\}$ & $\{q_2\}$ \\ \hline
$q_{N2}$ & $\{\varnothing\}$ & $\{q_3,q_5\}$\\ \hline
$q_{N3}$ & $\{q_4\}$ & $\{\varnothing\}$ \\ \hline
$q_{N4}$ & $\{\varnothing\}$ & $\{q_3,q_5\}$ \\ \hline
$q_{N5}$ & $\{\varnothing\}$ & $\{\varnothing\}$ \\ \hline
\end{tabular} 

\end{subfigure}%
\quad
\begin{subfigure}[B]{0.4\textwidth}
\centering

\begin{tabular}{l|l|l}
$q_{Di}$ & $\delta (a)$ & $CLAUS_\lambda$ \\ \hline
\hline
$q_{D0}$ & $ \{q_1\}$ & $\{q_1,q_2,q_3,q_5\}$ \\ \hline
$q_{D3}$ & $ \{q_4\}$ & $\{q_3,q_4,q_5\}$ \\ \hline
\end{tabular}
\end{subfigure}
\end{figure}

}
\item Correspondencia del Estado Final.

\begin{enumerate}[i.]

\item $q_0$ no contiene ningún estado final del AFnD.

\item $q_1$ contiene el estado $q_5$ del AFnD por lo que será estado final.

\item $q_2$ contiene el estado $q_5$ del AFnD por lo que será estado final.

\end{enumerate}

\end{enumerate}

\begin{figure}[h]
\centering
\begin{pspicture}(0,0)(8,4)%\psgrid

\cput(1,2){\large $q_0$}
\rput(0.48,2){\large $>$}
\pscurve[linecolor=black,linewidth=1pt]{->}(1.5,2.2)(2.5,2.55)(3.5,2.2)
\rput(1,1.2){$\{q_0\}$}

\rput(2.5,2.85){\large $a$}
\rput(5.5,2.85){\large $a$}

\cput[doubleline=true](4,2){\large $q_1$}
\rput(4,1.2){$\{q_1,q_2,q_3,q_5\}$}


\cput[doubleline=true](7,2){\large $q_2$}
\rput(7,1.2){$\{q_3,q_4,q_5\}$}
\pscurve[linecolor=black,linewidth=0.8pt]{->}(6.63,2.25)(7,2.75)(7.35,2.25)
\rput(7,3){\large $a$}


\pscurve[linecolor=black,linewidth=1pt]{->}(4.6,2.2)(5.5,2.55)(6.5,2.2)


\end{pspicture}

\caption{Autómata Finito Determinista a partir de AFnD $a\cdot a^*$}

\end{figure}

\begin{table}[h]

\begin{center}

\begin{tabular}{|l|l|}\hline
\textbf{Tipo} & \textbf{Criterio} \\ \hline
\hline
AFD & $u \in \Sigma^* : M$ terminar de procesar $u$ en un estado $q \in F$. \\ \hline
AFnD & $u \in \Sigma^* : M$ terminar de procesar $u$ de manera completa en un estado $q \in F$. \\ \hline

\end{tabular}

\caption{$L(M)$ para Autómatas.}
\end{center}

\end{table}

\subsection{Algoritmo: Expresión Regular $\Rightarrow$ AFnD}\label{algolER2AFnD}

\prog Pseudocódigo del Algoritmo Expresión Regular $\Rightarrow$ AFnD:

\begin{enumerate}[I.]

\item Entrada: Expresión Regular $p \in \Sigma$

\item Salida: AFnD $M\prime$ que procesa $L(p)$

\item Método:
{
\begin{enumerate}[i.]
\item Definir la Expresión Regular $p \in \Sigma$.

\item Generar un AFnD para dicha Expresión Regular.

\item Aplicar el Algoritmo: Algoritmo: AFnD $\Rightarrow$ AFD (Ver Apartado \ref{algolAFnD2AFD})

\item Minimizar el Número de Estados:
{

\defn Se tiene por Estados Equivalentes (no distinguibles)...

}

\end{enumerate}

}

\end{enumerate}

\ejem {Para el ejemplo de la Figura \ref{fig:exampleNDA}

\begin{enumerate}[I.]

\item \textit{idem} 

\item \textit{idem}

\item \textit{idem}

\item Minimizar el Número de Estados:

\begin{enumerate}[i.]

\item $q_0$ y $q_1$ no son distinguibles.

\item $q_1$ y $q_2$ no son distinguibles.

\item $q_0$ y $q_2$ si son distinguibles.

\begin{equation}
\delta(q_0,a) = \delta(q_2,a)
\end{equation}


\end{enumerate}


\end{enumerate}

}

\begin{figure}[h]
\centering
\begin{pspicture}(0,0)(5,4)%\psgrid
%\psline[linecolor=black,linewidth=1pt]{->}(1.5,2)(3.5,2)

\pscurve[linecolor=black,linewidth=1pt]{->}(1.5,2.2)(2.5,2.55)(3.5,2.2)
\rput(2.5,2.85){\large $a$}
\cput(1,2){\large $q_0$}

\cput[doubleline=true](4,2){\large $q_1$}
\pscurve[linecolor=black,linewidth=1pt]{->}(3.63,2.25)(4,2.75)(4.35,2.25)
%\rput(2.5,2.3){\large $a$}


\rput(0.48,2){\large $>$}


\rput(4,3){\large $a$}

\end{pspicture}

\caption{Autómata Finito Determinista Mínimo a partir de AFnD $a\cdot a^*\equiv a^+$}

\end{figure}

\begin{figure}[h]
\begin{center}
\begin{pspicture}(10,8)%\psgrid


\pscurve[linecolor=black,linewidth=1pt]{<-}(7,7)(8,6.6)(8,4.6)
\pspolygon[fillstyle=solid,fillcolor=white](3.2,7.4)(6.8,7.4)(6.8,6.7)(3.2,6.7)
\rput(5,7){Min. de Estados}


\pscurve[linecolor=black,linewidth=1pt]{<-}(8,3.4)(8,1.4)(7,1)
\pspolygon[fillstyle=solid,fillcolor=white](6,4.4)(10,4.4)(10,3.6)(6,3.6)
\rput(8,4){AFD}


\pspolygon[fillstyle=solid,fillcolor=white](3.2,1.4)(6.8,1.4)(6.8,0.7)(3.2,0.7)
\rput(5,1){AFnD}


\pscurve[linecolor=black,linewidth=1pt]{<-}(3,1)(1.8,1.4)(2,3.3)
\pspolygon[fillstyle=solid,fillcolor=white](0.2,4.6)(3.8,4.6)(3.8,3.4)(0.2,3.4)
\pspolygon[fillstyle=solid,fillcolor=white](0.4,4.4)(3.6,4.4)(3.6,3.6)(0.4,3.6)
\rput(2,4){Exp. Regular}

\pscurve[linecolor=black,linewidth=1pt]{->}(3,7)(2,6.6)(2,4.8)

%\rput(9.8,2){T. de Kleene I}

%\rput(0,2){T. de Kleene II}

\end{pspicture}
\caption{Ciclo de Thompson.}
\end{center}
\end{figure}


\section{El Lenguaje LEX}

\index{LEX}LEX\endnote{LEX (Flex en su implementación GNU) nos permite generar un Analizador Léxico (AFD) mediante Expresiones Regulares.} o Lenguaje de Especificación para Analizadores Léxicos, se trata de un lenguaje que relaciona Expresiones Regulares con acciones determinadas.

Para generar un analizador LEX se han de seguir los siguientes pasos:

\begin{enumerate}[i.]

\item Crear el programa fuente con las Expresiones Regulares \texttt{source.l}

\item Compilar el fichero \texttt{source.l} con LEX y generar (por defecto) \texttt{lex.yy.c}

\item Compilar \texttt{lex.yy.c} (con un compilador de Lenguaje C) y obtener el ejecutable.

\end{enumerate}

La estructura de un programa LEX es la que sigue:

\lstinputlisting[language=bash]{./programms/programms21/lexSyntax}

\begin{enumerate}[I.]

\item Sección de Definiciones: En ella se definen variables, constantes y los patrones necesarios para el resto del programa.

\item Sección de Reglas: Contiene el conjunto de reglas, definidas de la siguiente manera:

\begin{equation}
er_\lambda\ \ \ \{sentencias\}
\end{equation}

Donde:

\begin{enumerate}[i.]

\item $er$: Es la Expresión Regular

\item $sentencias$: Es el conjunto de acciones a ejecutar cuando se estable la relación entre patrón y lexema.

\end{enumerate}

\item Sección de Código C: Consiste en una serie de sentencias auxiliares en Lenguaje C que permiten una mayor flexibilidad al desarrollador/programador.

\end{enumerate}

\section{Código fuente: gp1990la.l}

\subsection{Expresiones Regulares}

El fichero fuente contiene las siguientes Expresiones Regulares:

\begin{enumerate}[I.]

\item {\verb#NQUOTE [^']#} $\Rightarrow$ Toda palabra que comience por el carácter  `{\verb#'#}'

\item {\verb#IDENTIFIER [a-zA-Z]([a-zA-Z0-9\-])*#} $\Rightarrow$: Toda palabra que comience por una letra del alfabeto (mayúscula o minúscula) y seguido contenga los elementos de conjuntos repetidos de $[0,\infty]$:{

\begin{enumerate}[i.]
\item Letras Minúsculas {\verb#[a-z]#} 

\item Letras Mayúsculas: {\verb#[A-Z]#} 

\item Dígitos: {\verb#[0-9]#} 

\end{enumerate}

}
\item {\verb#DIGITSECUENCE [0-9]+#} $\Rightarrow$ Toda palabra que contenga dígitos repetidos de $[1,\infty]$

\item {\verb#REALNUMBER	[0-9]+"."[0-9]+#} $\Rightarrow$ Toda palabra que comience por un dígito de $[1,\infty]$ seguida del carácter `{\verb#.#}'\ y contenga de $[1,\infty]$ digitos.

\end{enumerate}

%\lstinputlisting[language=bash]{./code/regp1990la.l}




\subsection{Tokens}

%A continuación se desglosa el conjunto de: $\{key,value\}$ para el analizador léxico \texttt{gp1990la}

\begin{figure}
\begin{flushleft}
\begin{verbatim}
----------------------------------------------------------------------------------
| Token              | Valor            || Token              | Valor            |
----------------------------------------------------------------------------------
| AND                | AND              || TYPE               | TYPE             |        
| ARRAY              | ARRAY            || UNTIL              | UNTIL            |
| CASE               | CASE             || VAR                | VAR              |
| CONST              | CONST            || WHILE              | WHILE            | 
| DIV                | DIV              || WITH               | WITH             |
| DOWNTO             | DOWNTO           || {IDENTIFIER}       | IDENTIFIER       | 
| ELSE               | ELSE             || ":="               | ASSIGNMENT       | 
| EXTERN or EXTERNAL | EXTERNAL         || '({NQUOTE}|'')+'   | CHARACTER_STRING |
| ELSE               | ELSE             || ":"                | COLON            |
| FOR                | FOR              || ","                | COMMA            |
| FORWARD            | FORWARD          || {DIGITSECUENCE}    | DIGSEQ           |
| FUNCTION           | FUNCTION         || "."                | DOT              |
| GOTO               | GOTO             || ".."               | DOTDOT           |
| IF                 | IF               || "="                | EQUAL            |
| IN                 | IN               || ">="               | GE               |
| LABEL              | LABEL            || ">"                | GT               |
| MOD                | MOD              || "["                | LBRAC            |
| NIL                | NIL              || "<="               | LE               |
| NOT                | NOT              || "("                | LPAREN           |
| OF                 | OF               || "<"                | LT               |
| OR                 | OR               || "-"                | MINUS            |
| OTHERWISE          | OTHERWISE        || "<>"               | NOTEQUAL         |
| PACKED             | PACKED           || "+"                | PLUS             |
| BEGIN              | BEGIN            || "]"                | RBRAC            |
| FILE               | FILE             || {REALNUMBER}       | REALNUMBER       |
| PROCEDURE          | PROCEDURE        || ")"                | RPAREN           | 
| PROGRAM            | PROGRAM          || ";"                | SEMICOLON        |
| RECORD             | RECORD           || "/"                | SLASH            |
| REPEAT             | REPEAT           || "*"                | STAR             |
| SET                | SET              || "**"               | STARSTAR         |
| THEN               | THEN             || "->" or "^"        | UPARROW          |
| TO                 | TO               ||                                       |
----------------------------------------------------------------------------------


\end{verbatim}              
\end{flushleft}
\caption{Conjunto de Tokens para \texttt{gp1990la}}
\end{figure}


% \begin{table}[h]
% \begin{center}
% 
% \begin{tabular}{|l|l|l|l|}\hline
% \textbf{Token} & \textbf{Valor} & \textbf{Token} & \textbf{Valor}\\ \hline
% \hline
% AND & AND & THEN & THEN\\ \hline
% 
% \end{tabular} 
% \caption{hola}
%  
% \end{center}
% \end{table}

%\lstinputlisting[language=bash]{./code/tokgp1990la.l}

%\nocite{cptt, book/compilers/es/cptt}