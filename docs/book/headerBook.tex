\documentclass[a4paper,12pt]{book}
%Paquetes
\usepackage[utf8x]{inputenc}
\usepackage[spanish,activeacute,es-lcroman,es-tabla]{babel}
%\usepackage[spanish, es-tabla]{babel}
\usepackage[spanish]{minitoc}
%\usepackage{mtcoff}
%\usepackage{anttor}
\usepackage{amsmath, amsthm, amssymb}
\usepackage{amsfonts}

\usepackage{yfonts}
\usepackage{wrapfig}

%\usepackage{lettrine}
\usepackage[T1]{fontenc}
%\usepackage{lmodern}

%texlive-fontsextra
\usepackage{dsfont}
\usepackage{graphs}
\usepackage{graphicx}
\usepackage{caption}
\usepackage{subcaption}

\usepackage{pstricks} % para color
\usepackage{pst-node} % para diagramas
\usepackage{pst-plot} % para representacion de datos
                      % funciones, etc
\usepackage{pst-text}
\usepackage{pst-tree}
\usepackage{pst-circ}
\usepackage{pst-poly}
\usepackage{pstricks-add}
%\usepackage{pst-coil}
\usepackage{pst-gantt}

\usepackage{hyperref}


\usepackage{graphicx}
\usepackage{textcomp}
\usepackage{color}
\usepackage{fancyvrb}
\usepackage{fancyhdr}
\usepackage{eso-pic,calc}                        
\usepackage{bibunits}
\usepackage{listings}
\usepackage{lscape}

\usepackage{ascii}

\usepackage{anysize}

\usepackage{endnotes}

\usepackage{enumerate}





\usepackage{ascii}
\usepackage{textcomp}
\usepackage{eurosym}
\usepackage{cclicenses}

\usepackage[spanish,intoc]{nomencl}
\makenomenclature

\usepackage{makeidx}

\usepackage{filecontents}


%\TPGrid[10mm,5mm]{26}{20} 

%\parindent=0pt\chapter*{}
%\parskip=0.5\baselineskip








\let\T=\text
\fancyhead[L]{\textbf{(\textit{preview} 1.0)}} 
%\fancyhead[RE]{}
%\fancyhead[RO,LE]{\thepage}

% Utilidades
%--------------------------------------------------------------------------
\newtheorem{thm}{Teorema}[section]
\newtheorem{cor}[thm]{Corolario}
\newtheorem{lem}[thm]{Lema}
\newtheorem{prop}[thm]{Proposici\'on}
\theoremstyle{definition}
\newtheorem{defn}[thm]{Definici\'on}
\newtheorem{form}[thm]{Formalidad}
\newtheorem{regl}[thm]{Reglas}
\newtheorem{ejem}[thm]{Ejemplo}
\newtheorem{prog}[thm]{Programa}
\newtheorem{algo}[thm]{Algoritmo}
\theoremstyle{remark}
\newtheorem{rem}[thm]{Observación}
\newtheorem{dhm}[thm]{Demostración}


\let\stdthebibliography\thebibliography
\renewcommand*{\thebibliography}{%
\let\section\subsection\stdthebibliography}

\renewcommand{\notesname}{Notas}


\newcommand{\dos}{\textsf{~dos/win~}}
\newcommand{\unix}{\UNIX}
% \pdfpagewidth 6\chapter{Formalismos}
% \pdfpageheight 9in
% \setlength\topmargin{0in}
% \setlength\headheight{0in}
% \setlength\headsep{0in}
% \setlength\textheight{7.7in}
% \setlength\textwidth{6.5in}
% \setlength\oddsidemargin{0in}
% \setlength\evensidemargin{0in}
% \setlength\headheight{25pt}
% \setlength\headsep{0.25in}

\marginsize{2cm}{2cm}{2cm}{2cm} 

% \makeatletter\renewcommand\theenumii{\@roman\c@enumiii}\makeatother
% \makeatletter\renewcommand\theenumii{\@roman\c@enumii}
% \renewcommand\labelenumii{\theenumii)}
% \makeatletter % for internal macros with @
% \renewcommand\theenumii{\@roman\c@enumii}
% \makeatother

\pagestyle{empty}
\frontmatter

\setcounter{secnumdepth}{5}
\setcounter{tocdepth}{5}

\setcounter{parttocdepth}{1}
\setcounter{minitocdepth}{1}
%\nomtcrule 

\renewcommand{\mtctitle}{Resumen:}
 
%\setlength{\mtcindent}{24pt}
%\renewcommand{\mtcfont}{\texttt}
%\renewcommand{\mtcSfont}{\small\bf}

%% for Greek Alphabet

\def\X#1{$#1$ &\tt\string#1}

\definecolor{gray}{rgb}{0.98,0.98,0.98}
\definecolor{black}{rgb}{0,0,0}

\lstset{showstringspaces=false,numbers=left,
        numberstyle=\footnotesize,backgroundcolor=\color{gray},
        rulesep=1pt, rulesepcolor=\color{black},frame=leftline,
        basicstyle=\ttfamily, mathescape=true}

        
%%% Commands@dlucenap
        
        
%%%%%%%%%%%%%%%%
% Command: Double Page with style Empty
%%%%%%%%%%%%%%%%

\global\def\doublePageEmpty{\newpage{\pagestyle{empty}\cleardoublepage}}        
        
        
%%%%%%%%%%%%%%%%
% Command: Enviroment for compile chapter
% #1 name of Chapter
% #2 is a Source of file
%%%%%%%%%%%%%%%%
   
\global\def\metaChapter#1#2
{
\doublePageEmpty

\pagestyle{fancy}
% 
% \renewcommand*{\bibname}{Bibliografía capitular}
% \bibliographyunit
% \doublePageEmpty
% \bibliographyunit[\chapter]
% \bibliographystyle*{alpha}
% \bibliography*{lib}

\chapter{#1}
\minitoc
\input{#2}

\doublePageEmpty
\addcontentsline{toc}{section}{Notas}

\pagestyle{plain}

\theendnotes

\doublePageEmpty
% \addcontentsline{toc}{section}{Bibliografía capitular}
% \putbib
% \bibliographyunit

}


%%%%%%%%%%%%%%%%
% Command: Enviroment for compile chapter
% #1 name of Chapter
% #2 is a Source of file
%%%%%%%%%%%%%%%%
   
\global\def\simpleChapter#1#2
{
\doublePageEmpty

\pagestyle{fancy}
% 
% \renewcommand*{\bibname}{Bibliografía capitular}
% \bibliographyunit
% \doublePageEmpty
% \bibliographyunit[\chapter]
% \bibliographystyle*{alpha}
% \bibliography*{lib}

\chapter{#1}
\minitoc
\input{#2}

\doublePageEmpty
% \addcontentsline{toc}{section}{Notas del capítulo}
% 
% \pagestyle{plain}
% 
% \theendnotes

% \doublePageEmpty
% \addcontentsline{toc}{section}{Bibliografía capitular}
% \putbib
% \bibliographyunit

}

%%%%%%%%%%%%%%%%
% Command: 
% #1 name of Chapter
% #2 is a Source of file
%%%%%%%%%%%%%%%%
   
\global\def\preChapter#1#2
{
\doublePageEmpty
\chapter*{#1}
\input{#2}
\doublePageEmpty
}

%%%%%%%%%%%%%%%%
% Command: 
% #1 name of word
%%%%%%%%%%%%%%%%
   
\global\def\toIndex#1
{#1\index{#1} }

% \begin{filecontents*}{myStyle.xdy}
% -----------------xindy style file-----------------------
% ;;; xindy style file for the VIBS book series
% 
% ;;; xindy style file
% (markup-locclass-list :open "\dotfill" :sep "")
% 
% (define-attributes (( "textbf" "default" )) )
% (markup-locref   :attr  "textbf"     :open "\textbf{\hyperpage{" :close "}}")
% (markup-locref   :attr  "textit"     :open "\textit{\hyperpage{" :close "}}")
% (markup-locref   :attr  "textttt"     :open "\textttt{\hyperpage{" :close "}}")
% (markup-locref   :attr  "texttsc"     :open "\texttsc{\hyperpage{" :close "}}")
% (markup-locref   :attr  "default"     :open "\hyperpage{" :close "}")
% 
% define-letter-groups
%   ("a" "b" "c" "d" "e" "f" "g" "h" "i" "j" "k" "l" "m"
%    "n" "o" "p" "q" "r" "s" "t" "u" "v" "w" "x" "y" "z"))
% 
% (markup-letter-group-list :sep "~n\indexspace")
% 
% ;; End
% 
% \end{filecontents*}

\makeatletter
\renewcommand\part{%
  \if@openright
    \cleardoublepage
  \else
    \clearpage
  \fi
  \thispagestyle{empty}%   % Original »plain« replaced by »emptyx
  \if@twocolumn
    \onecolumn
    \@tempswatrue
  \else
    \@tempswafalse
  \fi
  \null\vfil
  \secdef\@part\@spart}
\makeatother
