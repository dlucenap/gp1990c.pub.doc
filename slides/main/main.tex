\documentclass{beamer}
\usepackage[utf8x]{inputenc}
\usepackage[spanish,activeacute,es-lcroman,es-tabla]{babel}
%\usepackage[spanish, es-tabla]{babel}
\usepackage[spanish]{minitoc}
\usepackage{multicol} % indice en 2 columnas
\usepackage{amsmath, amsthm, amssymb}
\usepackage{amsfonts}

\usepackage{pstricks} % para color
\usepackage{pst-node} % para diagramas
\usepackage{pst-plot} % para representacion de datos
                      % funciones, etc
\usepackage{pst-text}


\usepackage{hyperref}

\newtheorem{thm}{Teorema}[section]
\newtheorem{cor}[thm]{Corolario}
\newtheorem{lem}[thm]{Lema}
\newtheorem{prop}[thm]{Proposici\'on}
\theoremstyle{definition}
\newtheorem{defn}[thm]{Definici\'on}
\newtheorem{form}[thm]{Formalidad}
\newtheorem{regl}[thm]{Reglas}
\newtheorem{ejem}[thm]{Ejemplo}
\newtheorem{prog}[thm]{Programa}
\newtheorem{algo}[thm]{Algoritmo}
\theoremstyle{remark}
\newtheorem{rem}[thm]{Observación}
\newtheorem{dhm}[thm]{Demostración}


\usefonttheme[stillsansseriflarge]{serif}
\setbeamerfont*{block title}{family=\sffamily,series=\bfseries,size=\Huge}

\usetheme{default}
\usecolortheme{beaver}
\useoutertheme{shadow}
\useinnertheme{rectangles}

\title[\texttt{gp1990c}]{\texttt{GNU Pascal 1990 Compiler}}
\author[Diego Antonio Lucena Pumar]
{
Universidad de Alcalá\\
Ingeniería Técnica en Informática de Gestión\\
Escuela Politécnica Superior\\
Dpto. de Automática\\
\texttt{\{diego.lucena.pumar\}@gmail.com}
}
\date{\today}

\begin{document}

\begin{frame}

\titlepage

\end{frame}

\begin{frame}
  \frametitle{Índice}
  \begin{multicols}{2}
  \tableofcontents
  \end{multicols}
\end{frame}

\section{Introducción}

\subsection{Motivación}

\begin{frame}
\begin{enumerate}[i.]

\item \textbf{Desarrollar el estándar ISO Pascal 
7185:1990 además de la construcción de un prototipo} para su parte léxica (basada en 
\index{Flex}Flex) y su
parte sintáctica (basada en \index{Bison}Bison).

\item \textbf{Síntesis y Lenguaje Matemático propio de la Teoría de Lenguajes de
Programación} así como su evolución e influencias históricas.

\end{enumerate} 
\end{frame}

\subsection{Pascal e ISO Pascal 7185}

\begin{frame}
\begin{enumerate}[i.] 
\item Crear un \textbf{lenguaje claro y natural orientado a la enseñanza} de los
fundamentos de la programación de computadores. Por ello se estructuran los
módulos como funciones y procedimientos.
\item Definir un lenguaje que \textbf{permita realizar programas lo más
eficientes
posibles}. El tipado de datos es explícito.
\end{enumerate}
\end{frame}

\subsection{Lenguajes Formales}

\begin{frame}

\textit{Un Lenguaje Formal se compone de un conjunto de signos finitos y unas
leyes para operar con ellos.}
\begin{enumerate}[i.]
\item Al conjunto de símbolos de un lenguaje se les denomina \textit{Alfabeto},
denotado como $\Sigma$.

\item Al conjunto de leyes que describen al lenguaje se les denomina
\textit{Sintaxis}.
\end{enumerate}

Se puenden definir a través de:
\begin{enumerate}[i.]
\item Mediante cadenas producidas por una gramática de Chomsky. 

\item Por medio de una Expresión Regular.

\item Por cadenas aceptadas por un Autómata.

\end{enumerate}
\end{frame}

\subsection{Diferencia al LH de los LF}

\begin{frame}


Dadas las siguientes palabras: 

\begin{equation}
\{Javier,compr\acute{o},una,casa\} 
\end{equation}

Se puede construir la frase:

\begin{equation}
Javier\ compr\acute{o}\ una\ casa 
\end{equation} 

que sintáctica y semánticamente es correcta, pero la oración:

\begin{equation}
Una\ casa\ compr\acute{o}\ Javier 
\end{equation}

es sintácticamente correcta pero no semánticamente.
\end{frame}

\section{Autómatas}

\begin{frame}
\textit{Se conoce como \index{Autómata Finito}Autómata Finito a máquinas abstractas que procesan cadenas para un determinado lenguaje.}

\begin{figure}[h]
\begin{center}
%
\begin{pspicture}(0,0)(13,5)%\psgrid
\psline[linecolor=black,linewidth=1pt]{-}(2,3)(5,3)
\psline[linecolor=black,linewidth=1pt]{-}(2,2)(5,2)
\psline[linestyle=dotted,linecolor=black,linewidth=1pt]{-}(5,3)(8,3)
\psline[linestyle=dotted,linecolor=black,linewidth=1pt]{-}(5,2)(8,2)

\psline[linecolor=black,linewidth=1pt]{-}(8,3)(11,3)
\psline[linecolor=black,linewidth=1pt]{-}(8,2)(11,2)

\psline[linecolor=black,linewidth=1pt]{-}(5,1)(5,4)
\psline[linecolor=black,linewidth=1pt]{-}(8,1)(8,4)
\psline[linecolor=black,linewidth=1pt]{-}(5,1)(8,1)
\psline[linecolor=black,linewidth=1pt]{-}(5,4)(8,4)

\psline[linecolor=black,linewidth=1pt]{-}(3,2)(3,3)
\psline[linecolor=black,linewidth=1pt]{-}(4,2)(4,3)

\psline[linestyle=dotted,linecolor=black,linewidth=1pt]{-}(6,2)(6,3)
\psline[linestyle=dotted,linecolor=black,linewidth=1pt]{-}(7,2)(7,3)

\psline[linecolor=black,linewidth=1pt]{-}(9,2)(9,3)
\psline[linecolor=black,linewidth=1pt]{-}(10,2)(10,3)

\psline[linecolor=black,linewidth=1pt]{-}(5.8,1.2)(5.8,3.3)
\psline[linecolor=black,linewidth=1pt]{-}(7.2,1.2)(7.2,3.3)
\psline[linecolor=black,linewidth=1pt]{-}(5.8,1.2)(7.2,1.2)
\psline[linecolor=black,linewidth=1pt]{-}(5.8,3.3)(7.2,3.3)

\rput(6.5,1.7){$q_\alpha$}

\rput(3.5,2.5){$a$}
\rput(4.5,2.5){$a$}

\rput(8.5,2.5){$a$}
\rput(9.5,2.5){$a$}

\rput(2,0.5){Cabeza Lectora}
\rput(1.5,3.5){Cadena de Entrada $u$}
\rput(6.5,4.5){Autómata $M$}

% \psline[linecolor=black,linewidth=1pt]{-}(3,2)(3,5)
% \psline[linecolor=black,linewidth=1pt]{-}(3,2)(6,1)
% \psline[linecolor=black,linewidth=1pt]{-}(6,1)(6,4)
% \psline[linecolor=black,linewidth=1pt]{-}(3,5)(6,4)
% 
% \psline[linestyle=dotted,linecolor=black,linewidth=1pt]{-}(5,3)(5,6)
% \psline[linestyle=dotted,linecolor=black,linewidth=1pt]{-}(5,3)(8,2)
% \psline[linecolor=black,linewidth=1pt]{-}(8,2)(8,5)
% \psline[linecolor=black,linewidth=1pt]{-}(5,6)(8,5)
% 
% 
% \psline[linestyle=dotted,linecolor=black,linewidth=1pt]{-}(3,2)(5,3)
% \psline[linecolor=black,linewidth=1pt]{-}(6,1)(8,2)
% 
% \psline[linecolor=black,linewidth=1pt]{-}(3,5)(5,6)
% \psline[linecolor=black,linewidth=1pt]{-}(6,4)(8,5)

\end{pspicture}
\caption{Representación de un Autómata Finito.\label{fig:generalAut}}
\end{center}
\end{figure}


\end{frame}

\subsection{Partes de un Autómata}

\begin{frame}

\begin{enumerate}[I.]

\item Cinta semi-infinita: Dividida a su vez en celdas donde se escribe la cadena de entrada.

\item \index{Unidad de Control}Unidad de Control (también llamada Cabeza Lectora): Que se encarga de procesar la citan.

\item El Autómata propiamente dicho que mantiene la lógica del lenguaje a través de una serie de estados (de aceptación y finales).

\end{enumerate}
\end{frame}

\subsection{Tipos}

\begin{frame}
Dependiendo de la configuración de los estados internos del Autómata, diferenciamos tres tipos:

\begin{enumerate}[i.]

\item Autómatas Finitos Determinista: Transiciones del tipo: $\delta (q, a)$. Procesan la palabra $\lambda$.

\item Autómatas Finitos No Determinista: Transiciones del tipo: $\Delta (q, a)$. No procesan la palabra $\lambda$.

\end{enumerate}

\end{frame}

\section{LEX}

\subsection{¿Qué es LEX?}

\begin{frame}
\textit{LEX o Lenguaje de Especificación para Analizadores Léxicos, se trata de un lenguaje que relaciona Expresiones Regulares con acciones determinadas.}
\end{frame}

\subsection{Apartados de Lex}

\begin{frame}
La estructura de un programa LEX es la que sigue:

\begin{enumerate}[I.]

\item Sección de Definiciones: En ella se definen variables, constantes y los patrones necesarios para el resto del programa.

\item Sección de Reglas: Contiene el conjunto de reglas, definidas de la siguiente manera:

\begin{equation}
er_\lambda\ \ \ \{sentencias\}
\end{equation}

\item Sección de Código C: Consiste en una serie de sentencias auxiliares en Lenguaje C que permiten una mayor flexibilidad al desarrollador/programador.

\end{enumerate}
\end{frame}

\section{Yacc}

\subsection{¿Qué es Yacc?}

\begin{frame}
\textit{Yacc se trata de un popular ``Front-End'' para construir compiladores a nivel sintáctico diseñado originalmente por S.C. Johnson en 1970.\\
El análisis realizado por Yacc es del tipo LALR. }
\end{frame}

\subsection{Apartados de Yacc}

\begin{frame}
\begin{enumerate}[i.]

\item Apartado de rutinas en C: Delimitada por los símbolos \{\% (apertura) \%\} (cierre) contiene las directivas del preprocesador además, de variables y definiciones necesarias para el resto del programa.

\item Apartado de Tokens: Establece los Tokens a utilizar en el programa. 

\item Sección de Reglas de Traducción: Se definen en el mismo, las acciones semánticas que se corresponde a su vez con instrucciones en Código C.

\item Apartado de Código en C: Se trata del conjunto de rutinas en C definidas por el desarrollador/programador. 

\end{enumerate}
\end{frame}

\begin{frame}
\begin{equation}
E\ \longrightarrow \ E\ +\ T\ |\ T
\end{equation}

Donde:

\begin{enumerate}[i.]

\item $E$: Es un símbolo No Terminal.

\item $T$: Es un símbolo Terminal.

\end{enumerate} 
\end{frame}


\section{Código}

\subsection{Compilador}
\begin{frame}
\texttt{pascal:	pascal.tab.o pascal.lex.o\\
\ \ \ (CC) -o gp90c pascal.tab.o pascal.lex.o (LDLIBS)\\
pascal.lex.o:	pascal.lex.c pascal.tab.h\\
\ \ \ (CC) -c pascal.lex.c\\
pascal.tab.o:	pascal.tab.c pascal.tab.h\\
\ \ \ (CC) -c pascal.tab.c\\ 
pascal.tab.c:	pascal.y\\
\ \ \ (BISON) -d pascal.y\\
pascal.lex.c:	pascal.l\\
\ \ \ (FLEX) pascal.l\\
\ \ \ mv  lex.yy.c pascal.lex.c
}

 
\end{frame}

\subsection{Mejoras}


\begin{frame}
\begin{enumerate}[i.]
\item 
\end{enumerate}

\end{frame}

\end{document}
